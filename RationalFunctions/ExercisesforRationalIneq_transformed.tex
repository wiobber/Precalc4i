\documentclass{ximera}
% \handouttrue
\begin{document}
	\author{Stitz-Zeager}
	\xmtitle{Exercises for Rational Inequalities(TRANSFORMED)}{}

\mfpicnumber{1} \opengraphsfile{ExercisesforRationalIneq} % mfpic settings added 


(Review of Solving Equations):\footnote{For more review, see Section \ref{AppRatExpEqus}.} In Exercises \ref{ratleqnexercisefirst} - \ref{ratleqnexerciselast},  solve the rational equation.  Be sure to check for extraneous solutions.



% Transformed Exercises with Solutions

\begin{question}
$\dfrac{x}{5x + 4} = 3$
\begin{solution}
$x = -\frac{6}{7}$
\end{solution}

\end{question}

\begin{question}
$\dfrac{3x - 1}{x^{2} + 1} = 1$

\begin{solution}
$x = 1, \; x = 2$
\end{solution}

\end{question}

\begin{question}
$\dfrac{1}{t + 3} + \dfrac{1}{t - 3} = \dfrac{t^{2} - 3}{t^{2} - 9}$
\begin{solution}
$t = -1$

\end{solution}

\end{question}

\begin{question}
$\dfrac{2t + 17}{t + 1} = t + 5$

\begin{solution}
$t = -6, \; x = 2$
\end{solution}

\end{question}

\begin{question}
$\dfrac{z^{2} - 2z + 1}{z^{3} + z^{2} - 2z} = 1$
\begin{solution}
No solution
\end{solution}

\end{question}

\begin{question}
$\dfrac{4z- z^3}{z^{2} - 9} = 4z$  

\begin{solution}
$z = 0, \; z = \pm 2\sqrt{2}$

\end{solution}

\end{question}

\begin{question}
$\dfrac{1}{x + 2} \geq 0$
\begin{solution}
$(-2, \infty)$
\end{solution}

\end{question}

\begin{question}
$\dfrac{5}{x + 2} \geq 1$
\begin{solution}
$(-2, 3]$

\end{solution}

\end{question}

\begin{question}
$\dfrac{x}{x^{2} - 1} <  0$

\begin{solution}
$(-\infty, -1) \cup (0, 1)$
\end{solution}

\end{question}

\begin{question}
$\dfrac{4t}{t^2+4} \geq 0$
\begin{solution}
$[0, \infty)$

\end{solution}

\end{question}

\begin{question}
$\dfrac{2t+6}{t^2+t-6} < 1$
\begin{solution}
$(-\infty, -3) \cup (-3,2) \cup (4, \infty)$
\end{solution}

\end{question}

\begin{question}
$\dfrac{5}{t-3} + 9 < \dfrac{20}{t+3}$

\begin{solution}
$\left(-3, -\frac{1}{3} \right) \cup (2,3)$

\end{solution}

\end{question}

\begin{question}
$\dfrac{6z+6}{2+z-z^2} \leq z+3$
\begin{solution}
$(-1,0] \cup (2, \infty)$
\end{solution}

\end{question}

\begin{question}
$\dfrac{6}{z-1} + 1 > \dfrac{1}{z+1}$
\begin{solution}
$(-\infty, -3) \cup (-2, -1) \cup (1, \infty)$

\end{solution}

\end{question}

\begin{question}
$\dfrac{3z - 1}{z^{2} + 1} \leq 1$

\begin{solution}
$(-\infty, 1] \cup [2, \infty)$
\end{solution}

\end{question}

\begin{question}
$(2x+17)(x+1)^{-1} > x + 5$
\begin{solution}
$(-\infty, -6) \cup (-1, 2)$

\end{solution}

\end{question}

\begin{question}
$(4x-x^3)(x^{2} - 9)^{-1} \geq 4x$
\begin{solution}
$(-\infty, -3) \cup \left[-2\sqrt{2}, 0\right] \cup \left[2\sqrt{2}, 3\right)$
\end{solution}

\end{question}

\begin{question}
$(x^{2} + 1)^{-1} < 0$ 

\begin{solution}
No solution

\end{solution}

\end{question}

\begin{question}
$(2t-8)(t+1)^{-1} \leq (t^2-8t)(t+1)^{-2}$ % $[-4, -1) \cup (-1,2]$
\begin{solution}
$[-4, -1) \cup (-1,2]$
\end{solution}

\end{question}

\begin{question}
$(t-3)(2t+7)(t^2+7t+6)^{-2} \geq (t^2+7t+6)^{-1}$ % $(-\infty, -6) \cup (-6, -3] \cup [9, \infty)$

\begin{solution}
$(-\infty, -6) \cup (-6, -3] \cup [9, \infty)$

\end{solution}

\end{question}

\begin{question}
$60z^{-2}+23z^{-1} \geq 7(z-4)^{-1}$
\begin{solution}
$[-3,0) \cup (0,4) \cup [5, \infty)$
\end{solution}

\end{question}

\begin{question}
$2z+6(z-1)^{-1} \geq 11 - 8(z+1)^{-1}$ 

\begin{solution}
$\left(-1,-\frac{1}{2}\right] \cup (1, \infty)$

\end{solution}

\end{question}

\begin{question}
Solve $f(x) \geq 0$.


% \input{ExercisesforRationalIneq_transformed_pic1.tex}
\begin{tikzpicture}
\begin{axis}[fplot, xmin=-7, xmax=7, ymin=-6, ymax=8]
  \addplot[fgraph, domain=-7:-0.45] {1 - 3/x};
  \addplot[fgraph, domain=0.45:7]  {1 - 3/x};
  \addplot[dashed, domain=-7:7, samples=2] {1};
  \addplot[dashed, samples=2] coordinates {(0,-6) (0,8)};
  \addplot[only marks, mark=*, mark size=2pt] coordinates {(3,0)};
  \node[flabel, label=below right :{$(3,0)$}] at (axis cs:3,0) {};
  \node at (rel axis cs:0.5,-0.12) {\scriptsize $y=f(x)$, asymptotes: $x=0$, $y=1$.};
\end{axis}
\end{tikzpicture}


\vfill
\begin{solution}
$f(x) \geq 0$ on $(-\infty, 0) \cup [3, \infty)$.
\end{solution}

\end{question}

\begin{question}
Solve $f(x) < 1$.


% \input{ExercisesforRationalIneq_transformed_pic2.tex}
\begin{tikzpicture}
\begin{axis}[fplot, xmin=-7, xmax=7, ymin=-6, ymax=8]
  \addplot[fgraph, domain=-7:-0.45] {1 - 3/x};
  \addplot[fgraph, domain=0.45:7]  {1 - 3/x};
  \addplot[dashed, domain=-7:7, samples=2] {1};
  \addplot[dashed, samples=2] coordinates {(0,-6) (0,8)};
  \addplot[only marks, mark=*, mark size=2pt] coordinates {(3,0)};
  \node[flabel, label=below right :{$(3,0)$}] at (axis cs:3,0) {};
  \node at (rel axis cs:0.5,-0.12) {\scriptsize $y=f(x)$, asymptotes: $x=0$, $y=1$.};
\end{axis}
\end{tikzpicture}



\begin{solution}
$f(x) < 1$ on $(0, \infty)$.

\end{solution}

\end{question}

\begin{question}
Solve $g(t) \geq  -1 $. 

% \input{ExercisesforRationalIneq_transformed_pic3.tex}
\begin{tikzpicture}
\begin{axis}[fplot, xmin=-2, xmax=6, ymin=-4, ymax=4]
  \addplot[fgraph, domain=-2:1.75] {1/(x-2)};
  \addplot[fgraph, domain=2.25:6]  {1/(x-2)};
  \addplot[dashed, samples=2] coordinates {(2,-4) (2,4)};
  \addplot[dashed, domain=-2:6, samples=2] {0};
  \addplot[only marks, mark=*, mark size=2pt] coordinates {(1,-1) (3,1)};
  \node[flabel, label=below left :{$(1,-1)$}]  at (axis cs:1,-1) {};
  \node[flabel, label=above right:{$(3,1)$}]   at (axis cs:3,1)  {};
  \node at (rel axis cs:0.5,-0.12) {\scriptsize $y=g(t)$, asymptotes: $t=2$, $y=0$.};
\end{axis}
\end{tikzpicture}


\vfill
\begin{solution}
$g(t) \geq -1$ on $(-\infty, 1] \cup (2, \infty)$.
\end{solution}

\end{question}

\begin{question}
Solve $-1 \leq g(t)  < 1$. 

% \input{ExercisesforRationalIneq_transformed_pic4.tex}
\begin{tikzpicture}
\begin{axis}[fplot, xmin=-2, xmax=6, ymin=-4, ymax=4]
  \addplot[fgraph, domain=-2:1.75] {1/(x-2)};
  \addplot[fgraph, domain=2.25:6]  {1/(x-2)};
  \addplot[dashed, samples=2] coordinates {(2,-4) (2,4)};
  \addplot[dashed, domain=-2:6, samples=2] {0};
  \addplot[only marks, mark=*, mark size=2pt] coordinates {(1,-1) (3,1)};
  \node[flabel, label=below left :{$(1,-1)$}]  at (axis cs:1,-1) {};
  \node[flabel, label=above right:{$(3,1)$}]   at (axis cs:3,1)  {};
  \node at (rel axis cs:0.5,-0.12) {\scriptsize $y=g(t)$, asymptotes: $t=2$, $y=0$.};
\end{axis}
\end{tikzpicture}


\begin{solution}
$-1 \leq g(t) < 1$ on $(-\infty, 1] \cup (3, \infty)$.

\end{solution}

\end{question}

\begin{question}
Solve $r(z) \leq 1$ 

% \input{ExercisesforRationalIneq_transformed_pic5.tex}
\begin{tikzpicture}
\begin{axis}[fplot, xmin=-5, xmax=5, ymin=-1, ymax=6]
  \addplot[fgraph, domain=-5:-0.42] {1/(x^2)};
  \addplot[fgraph, domain=0.42:5]   {1/(x^2)};
  \addplot[dashed, samples=2] coordinates {(0,-1) (0,6)};
  \addplot[dashed, domain=-5:5, samples=2] {0};
  \addplot[only marks, mark=*, mark size=2pt]                 coordinates {(-1,1)};
  \addplot[only marks, mark=o, mark size=2.25pt, line width=1pt] coordinates {(1,1)};
  \node[flabel, label=above left :{$(-1,1)$}]       at (axis cs:-1,1) {};
  \node[flabel, label=above right:{hole at $(1,1)$}] at (axis cs:1,1)  {};
  \node at (rel axis cs:0.5,-0.12) {\scriptsize $y = r(z)$, asymptotes: $z=0$, $y=0$.};
\end{axis}
\end{tikzpicture}



\vfill
\begin{solution}
$r(z) \leq 1$ on $(-\infty, -1] \cup (1, \infty)$.
\end{solution}

\end{question}

\begin{question}
Solve $r(z) > 0$.


% \input{ExercisesforRationalIneq_transformed_pic6.tex}
\begin{tikzpicture}
\begin{axis}[fplot, xmin=-5, xmax=5, ymin=-1, ymax=6]
  \addplot[fgraph, domain=-5:-0.42] {1/(x^2)};
  \addplot[fgraph, domain=0.42:5]   {1/(x^2)};
  \addplot[dashed, samples=2] coordinates {(0,-1) (0,6)};
  \addplot[dashed, domain=-5:5, samples=2] {0};
  \addplot[only marks, mark=*, mark size=2pt]                 coordinates {(-1,1)};
  \addplot[only marks, mark=o, mark size=2.25pt, line width=1pt] coordinates {(1,1)};
  \node[flabel, label=above left :{$(-1,1)$}]       at (axis cs:-1,1) {};
  \node[flabel, label=above right:{hole at $(1,1)$}] at (axis cs:1,1)  {};
  \node at (rel axis cs:0.5,-0.12) {\scriptsize $y = r(z)$, asymptotes: $z=0$, $y=0$.};
\end{axis}
\end{tikzpicture}



\begin{solution}
$r(z) > 0$ on $(-\infty, 0) \cup (0,1) \cup (1, \infty)$.


\end{solution}

\end{question}

\begin{question}
In Exercise \ref{newportaboycost} in Section \ref{GraphsofPolynomials},  the function $C(x) = .03x^{3} - 4.5x^{2} + 225x + 250$, for $x \geq 0$ was used to model the cost (in dollars) to produce $x$ PortaBoy game systems. Using this cost function, find the number of PortaBoys which should be produced to minimize the average cost $\overline{C}$.  Round your answer to the nearest number of systems.
\begin{solution}
The absolute minimum of $y=\overline{C}(x)$ occurs at $\approx (75.73, 59.57)$.  Since $x$ represents the number of game systems, we check $\overline{C}(75) \approx 59.58$ and $\overline{C}(76) \approx 59.57$.  Hence, to minimize the average cost, $76$ systems should be produced at an average cost of $\$59.57$ per system.
\end{solution}

\end{question}

\begin{question}
Suppose we are in the same situation as Example \ref{boxnotopfixedvolume}.  If the volume of the box is to be $500$ cubic centimeters, use a graphing utility to find the dimensions of the box which minimize the surface area.  What is the minimum surface area?  Round your answers to two decimal places.
\begin{solution}
The width (and depth) should be $10.00$ centimeters, the height should be $5.00$ centimeters.  The minimum surface area is $300.00$ square centimeters.
\end{solution}

\end{question}

\begin{question}
The box for the new Sasquatch-themed cereal, `Crypt-Os', is to have a volume of $140$ cubic inches.  For aesthetic reasons, the height of the box needs to be $1.62$ times the width of the base of the box.\footnote{1.62 is a crude approximation of the so-called `Golden Ratio' $\phi = \frac{1 + \sqrt{5}}{2}$.}  Find the dimensions of the box which will minimize the surface area of the box.  What is the minimum surface area?  Round your answers to two decimal places.
\begin{solution}
The width of the base of the box should be approximately $4.12$ inches, the height of the box should be approximately $ 6.67$ inches, and the depth of the base of the box should be approximately $5.09$ inches. The minimum surface area is approximately $164.91$ square inches.

\newpage
\end{solution}

\end{question}

\begin{question}
Sally is Skippy's neighbor from Exercise \ref{fixedperimetermaxareagarden} in Section \ref{QuadraticFunctions}.   Sally also wants to plant a vegetable garden along the side of her home.  She doesn't have any fencing, but wants to keep the size of the garden to 100 square feet.  What are the dimensions of the garden which will minimize the amount of fencing she needs to buy?  What is the minimum amount of fencing she needs to buy? Round your answers to the nearest foot. (Note:  Since one side of the garden will border the house, Sally doesn't need fencing along that side.)
\begin{solution}
The dimensions are  approximately  $7$ feet by $14$ feet.  Hence, the minimum amount of fencing required is approximately  $28$ feet.
\end{solution}

\end{question}

\begin{question}
Another Classic Problem: A can is made in the shape of a right circular cylinder and is to hold one pint. (For dry goods, one pint is equal to $33.6$ cubic inches.)\footnote{According to \href{http://dictionary.reference.com/browse/pint}{\underline{www.dictionary.com}}, there are different values given for this conversion. We use $33.6 \text{in}^{3}$ for this problem.}  

\begin{solution}
\end{solution}

\end{question}

\begin{question}
Find an expression for the volume $V$ of the can in terms of the height $h$ and the base radius $r$.
\begin{solution}
$V = \pi r^{2}h$
\end{solution}

\end{question}

\begin{question}
Find an expression for the surface area $S$ of the can in terms of the height $h$ and the base radius $r$.  (Hint: The top and bottom of the can are circles of radius $r$ and the side of the can is really just a rectangle that has been bent into a cylinder.)
\begin{solution}
$S = 2 \pi r^{2} + 2\pi r h$

\end{solution}

\end{question}

\begin{question}
Using the fact that $V = 33.6$, write $S$ as a function of $r$ and state its applied domain.
\begin{solution}
$S(r) = 2\pi r^{2} + \frac{67.2}{r}, \;$  Domain $r > 0$
\end{solution}

\end{question}

\begin{question}
Use your graphing calculator to find the dimensions of the can which has minimal surface area.
\begin{solution}
$r \approx 1.749\,$in. and $h \approx 3.498\,$in.
\end{solution}

\end{question}

\end{document}