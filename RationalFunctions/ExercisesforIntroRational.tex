\documentclass{ximera}

\begin{document}
	\author{Stitz-Zeager}
	\xmtitle{Exercises}
\mfpicnumber{1} \opengraphsfile{ExercisesforIntroRational} % mfpic settings added 


(Review of Long Division):\footnote{For more review, see Section \ref{polylongdiv}.}  

\begin{problem}\label{longpolydivreviewfirst}
Use polynomial long division to perform the indicated division.  Write the polynomial in the form $p(x) = d(x)q(x) + r(x)$.

$\left(4x^2+3x-1 \right) \div (x-3)$ 
\end{problem}

\begin{problem}
Use polynomial long division to perform the indicated division.  Write the polynomial in the form $p(x) = d(x)q(x) + r(x)$.

$\left(2x^3-x+1 \right) \div \left(x^{2} +x+1 \right)$ 
\end{problem}

\begin{problem}
Use polynomial long division to perform the indicated division.  Write the polynomial in the form $p(x) = d(x)q(x) + r(x)$.

$\left(5x^{4} - 3x^{3} + 2x^{2} - 1 \right) \div \left(x^{2} + 4 \right)$ 
\end{problem}

\begin{problem}
Use polynomial long division to perform the indicated division.  Write the polynomial in the form $p(x) = d(x)q(x) + r(x)$.

$\left(-x^{5} + 7x^{3} - x \right) \div \left(x^{3} - x^{2} + 1 \right)$ 
\end{problem} 

\begin{problem}
Use polynomial long division to perform the indicated division.  Write the polynomial in the form $p(x) = d(x)q(x) + r(x)$.

$\left(9x^{3} + 5 \right) \div \left(2x - 3 \right)$

\begin{solution}
$9x^{3} + 5 =(2x - 3) \left(\frac{9}{2}x^{2} + \frac{27}{4}x + \frac{81}{8} \right) + \frac{283}{8}$
\end{solution}
\end{problem} 

\begin{problem}\label{longpolydivreviewlast}
Use polynomial long division to perform the indicated division.  Write the polynomial in the form $p(x) = d(x)q(x) + r(x)$.

$\left(4x^2 - x - 23 \right) \div \left(x^{2} - 1 \right)$ 
\end{problem}  


\begin{problem}\label{rationaltransfirst}
Given the pair of functions $f$ and $F$, sketch the graph of $y=F(x)$ by starting with the graph of $y = f(x)$ and using Theorem \ref{linearlaurentlgraphs}.  Track at least two points and the asymptotes.  State the domain and range using interval notation.

$f(x) = \dfrac{1}{x}$,  $F(x) = \dfrac{1}{x-2}+1$

\begin{solution}
$F(x) = \dfrac{1}{x-2}+1$ \\
Domain: $(-\infty, 2) \cup (2, \infty)$ \\
Range: $(-\infty, 1) \cup (1, \infty)$ \\
Vertical asymptote:  $x = 2$\\
Horizontal asymptote:  $y = 1$ \\

\begin{tikzpicture}
  \begin{axis}[
    axis lines=middle,
    xmin=-5, xmax=5,
    ymin=-5, ymax=5,
    xtick={-4,-3,-2,-1,1,2,3,4},
    xticklabels={$-4$,$-3$,$-2$,$-1$,$1$,$2$,$3$,$4$},
    ytick={-4,-3,-2,-1,1,2,3,4},
    yticklabels={$-4$,$-3$,$-2$,$-1$,$1$,$2$,$3$,$4$},
    axis line style={->},
    width=10cm, height=10cm,
    clip=false
  ]
    % Axis labels
    \node at (axis cs:5,-0.5) {\scriptsize $x$};
    \node at (axis cs:0.5,5) {\scriptsize $y$};

    % Dashed asymptotes
    \addplot[dashed, domain=-4.75:4.75] {1};
    \addplot[dashed] coordinates {(2,-4.75) (2,4.75)};

    % Function pieces (avoiding asymptote at x=2)
    \addplot[domain=-5:1.8, samples=200, thick, ->] {1 + 1/(x-2)};
    \addplot[domain=2.3:5, samples=200, thick, ->] {1 + 1/(x-2)};

    % Important points
    \addplot[only marks, mark=*] coordinates {(1,0) (3,2)};
  \end{axis}
\end{tikzpicture}
\end{solution}
\end{problem}

\begin{problem}
Given the pair of functions $f$ and $F$, sketch the graph of $y=F(x)$ by starting with the graph of $y = f(x)$ and using Theorem \ref{linearlaurentlgraphs}.  Track at least two points and the asymptotes.  State the domain and range using interval notation.

$f(x) =\dfrac{1}{x}$, $F(x) = \dfrac{2x}{x+1}$
\end{problem}

\begin{problem}
Given the pair of functions $f$ and $F$, sketch the graph of $y=F(x)$ by starting with the graph of $y = f(x)$ and using Theorem \ref{linearlaurentlgraphs}.  Track at least two points and the asymptotes.  State the domain and range using interval notation.

$f(x) =x^{-1}$, $F(x)=4x(2x+1)^{-1}$

\begin{solution}
$F(x)=4x(2x+1)^{-1} = \dfrac{4x}{2x+1} = \dfrac{-1}{x+\frac{1}{2}}+2$ \\
Domain: $\left(-\infty, -\frac{1}{2} \right) \cup \left(-\frac{1}{2},  \infty \right)$ \\
Range: $(-\infty, 2) \cup (2, \infty)$ \\
Vertical asymptote:  $x  = -\frac{1}{2}$ \\
Horizontal asymptote: $y= 2$\\

\begin{tikzpicture}
  \begin{axis}[
    axis lines=middle,
    xmin=-5, xmax=5,
    ymin=-5, ymax=5,
    xtick={-4,-3,-2,-1,1,2,3,4},
    xticklabels={$-4$,$-3$,$-2$,$-1$,$1$,$2$,$3$,$4$},
    ytick={1,2,3,4},
    yticklabels={$1$,$2$,$3$,$4$},
    axis line style={->},
    width=10cm, height=10cm,
    clip=false
  ]
    % Axis labels
    \node at (axis cs:5,-0.5) {\scriptsize $x$};
    \node at (axis cs:0.5,5) {\scriptsize $y$};

    % Dashed asymptotes
    \addplot[dashed, domain=-4.75:4.75] {2};
    \addplot[dashed] coordinates {(-0.5,-4.75) (-0.5,4.75)};

    % Function pieces (avoiding vertical asymptote at x=-0.5)
    \addplot[domain=-5:-0.84, samples=200, thick, ->] {4*x/(2*x+1)};
    \addplot[domain=-0.35:5, samples=200, thick, ->] {4*x/(2*x+1)};

    % Points
    \addplot[only marks, mark=*] coordinates {(-1.5,3) (0.5,1)};
  \end{axis}
\end{tikzpicture}
\end{solution}
\end{problem} 

\begin{problem}\label{rationaltranslast}
Given the pair of functions $f$ and $F$, sketch the graph of $y=F(x)$ by starting with the graph of $y = f(x)$ and using Theorem \ref{linearlaurentlgraphs}.  Track at least two points and the asymptotes.  State the domain and range using interval notation.

$f(x) = x^{-2}$, $F(x)=-(x-1)^{-2}+3$ 
\end{problem}   

\begin{problem}\label{findformulafor1overxgraphfirst}
Find a formula in the form $F(x) = \dfrac{a}{x-h}+k$ for the function.
%$F(x) = \dfrac{1}{x+2}-1$

\begin{tikzpicture}
  \begin{axis}[
    axis lines=middle,
    xmin=-5, xmax=5,
    ymin=-5, ymax=5,
    xtick={-4,-3,-2,-1,1,2,3,4},
    xticklabels={$-4$,$-3$,$-2$,$-1$,$1$,$2$,$3$,$4$},
    ytick={-4,-3,-2,-1,1,2,3,4},
    yticklabels={$-4$,$-3$,$-2$,$-1$,$1$,$2$,$3$,$4$},
    axis line style={->},
    width=10cm, height=10cm,
    clip=false
  ]
    % Axis labels
    \node at (axis cs:5,-0.5) {\scriptsize $x$};
    \node at (axis cs:0.5,5) {\scriptsize $y$};

    % Dashed asymptotes
    \addplot[dashed, domain=-4.75:4.75] {-1};
    \addplot[dashed] coordinates {(-2,-4.75) (-2,4.75)};

    % Function pieces (avoiding asymptote at x=-2)
    \addplot[domain=-5:-2.25, samples=200, thick, ->] {1/(x+2) - 1};
    \addplot[domain=-1.83:5, samples=200, thick, ->] {1/(x+2) - 1};

    % Important points
    \addplot[only marks, mark=*] coordinates {(-1,0) (0,-0.5)};

    % Caption as a node below
    \node[below] at (rel axis cs:0.5,0) 
      {\scriptsize $x$-intercept $(-1,0)$, $y$-intercept $\left(0,-\tfrac{1}{2}\right)$};
  \end{axis}
\end{tikzpicture}
\end{problem}

\begin{problem}\label{findformulafor1overxgraphlast}
Find a formula in the form $F(x) = \dfrac{a}{x-h}+k$ for the function.
%$F(x) = -\dfrac{2}{x-1}+1$

\begin{mfpic}[15]{-5}{5}{-5}{5}
\axes
\tlabel[cc](5,-0.5){\scriptsize $x$}
\tlabel[cc](0.5,5){\scriptsize $y$}
%\tlabel[cc](-1.5, 0.5){\scriptsize $(-1,0)$}
%\tlabel[cc](-0.5,-1){\scriptsize $\left(0, \frac{1}{2} \right)$}
\xmarks{-4,-3,-2,-1,1,2,3,4}
\ymarks{-4,-3,-2, -1, 1,2,3,4}
\tlpointsep{4pt}
\scriptsize
\axislabels {x}{ {$-4 \hspace{7pt}$} -4, {$-3 \hspace{7pt}$} -3, {$-2 \hspace{7pt}$} -2, {$-1 \hspace{7pt}$} -1, {$1$} 1, {$2$} 2, {$3$} 3, {$4$} 4}
\axislabels {y}{{$-1$} -1,{$1$} 1, {$2$} 2, {$3$} 3, {$4$} 4, {$-2$} -2, {$-3$} -3, {$-4$} -4}
\dashed \polyline{(-4.75,1), (4.75,1)}
\dashed \polyline{(1,-4.75), (1,4.75)}
\penwd{1.25pt}
\arrow \reverse \arrow \function{-5,0.45,0.1}{1-2/(x-1)}
\arrow \reverse \arrow \function{1.35,5,0.1}{1-2/(x-1)}
\point[4pt]{(3,0), (0,3)}
\tcaption{ \scriptsize $x$-intercept $(3,0)$, $y$-intercept $\left(0, 3 \right)$}
\normalsize
\end{mfpic}
\end{problem}

\begin{problem}\label{findformulafor1overxsquaredgraphfirst}
Find a formula in the form $F(x) = \dfrac{a}{(x-h)^2}+k$ for the function.
%$F(x) = -\dfrac{4}{(x+2)^2}+4$

\begin{tikzpicture}
  \begin{axis}[
    axis lines=middle,
    xmin=-5, xmax=5,
    ymin=-5, ymax=5,
    xtick={-4,-2,1,2,3,4},
    xticklabels={$-4$,$-2$,$1$,$2$,$3$,$4$},
    ytick={-4,-3,-2,-1,1,2,3,4},
    yticklabels={$-4$,$-3$,$-2$,$-1$,$1$,$2$,$3$,$4$},
    axis line style={->},
    width=10cm, height=10cm,
    clip=false
  ]
    % Axis labels
    \node at (axis cs:5,-0.5) {\scriptsize $x$};
    \node at (axis cs:0.5,5) {\scriptsize $y$};

    % Dashed asymptotes
    \addplot[dashed, domain=-4.75:4.75] {4};
    \addplot[dashed] coordinates {(-2,-4.75) (-2,4.75)};

    % Function pieces (avoiding vertical asymptote at x=-2)
    \addplot[domain=-5:-2.7, samples=200, thick, ->] {4 - 4/( (x+2)^2 )};
    \addplot[domain=-1.3:5, samples=200, thick, ->] {4 - 4/( (x+2)^2 )};

    % Important points
    \addplot[only marks, mark=*] coordinates {(-3,0) (-1,0) (0,3)};

    % Caption below graph
    \node[below] at (rel axis cs:0.5,0) 
      {\scriptsize $x$-intercepts $(-3,0)$, $(-1,0)$, $y$-intercept $(0,3)$};
  \end{axis}
\end{tikzpicture}
\end{problem}

\begin{problem}\label{findformula1overxsquaredgraphlast}
Find a formula in the form $F(x) = \dfrac{a}{(x-h)^2}+k$ for the function.
%$F(x) = \dfrac{4}{(2x-1)^2}-4$

\begin{mfpic}[15]{-5}{5}{-5}{5}
\axes
\tlabel[cc](5,-0.5){\scriptsize $x$}
\tlabel[cc](0.5,5){\scriptsize $y$}
%\tlabel[cc](-1.5, 0.5){\scriptsize $(-1,0)$}
%\tlabel[cc](-0.5,-1){\scriptsize $\left(0, \frac{1}{2} \right)$}
\xmarks{-4,-3,-2,-1,1,2,3,4}
\ymarks{-4,-3,-2, -1, 1,2,3,4}
\tlpointsep{4pt}
\scriptsize
\axislabels {x}{ {$-4 \hspace{7pt}$} -4, {$-3 \hspace{7pt}$} -3, {$-2 \hspace{7pt}$} -2, {$-1 \hspace{7pt}$} -1, {$1$} 1, {$2$} 2, {$3$} 3, {$4$} 4}
\axislabels {y}{{$-1$} -1,{$1$} 1, {$2$} 2, {$3$} 3, {$4$} 4, {$-2$} -2, {$-4$} -4}
\dashed \polyline{(-4.75,-4), (4.75,-4)}
\dashed \polyline{(0.5,-4.75), (0.5,4.75)}
\penwd{1.25pt}
\arrow \reverse \arrow \function{-5, 0.16, 0.1}{4/((2*x-1)**2) - 4}
\arrow \reverse \arrow \function{0.84, 5, 0.1}{4/((2*x-1)**2) - 4}
\point[4pt]{(0,0), (1,0)}
\tcaption{\scriptsize $x$-intercepts $(0,0)$,  $(1,0)$, Vertical Asymptote:  $x = \frac{1}{2}$}
\normalsize
\end{mfpic}
\end{problem}

\begin{problem}\label{alltheasympfirst}
Consider the function $f(x) = \dfrac{x}{3x - 6}$
\begin{itemize}
\item State the domain.
\begin{solution}
$(-\infty, 2) \cup (2, \infty)$
\end{solution}
\item Identify any vertical asymptotes of the graph.
$\answer{x = 2}$
\item Identify any holes in the graph.
\begin{solution}
No holes in the graph
\end{solution}
\item Find the horizontal asymptote, if it exists.
$y = \answer{\frac{1}{3}}$
\item Find the slant asymptote, if it exists.
\begin{solution}
No slant asymptote
\end{solution}
\item Graph the function using a graphing utility and describe the behavior near the asymptotes.
\begin{solution}
\begin{center}
\geogebra{fmrfliu7sk}{800}{600}
\end{center}

Vertical asymptote:

$\ds{\lim_{x \rightarrow 2^{-}} f(x) = -\infty}$, $\ds{\lim_{x \rightarrow 2^{+}} f(x) = \infty}$ 

Horizontal asymptote:

$\ds{\lim_{x \rightarrow  -\infty} f(x) =}$ $\frac{1}{3}$
More specifically: as $x \rightarrow -\infty, f(x) \rightarrow \frac{1}{3}^{-}$

$\ds{\lim_{x \rightarrow  \infty} f(x) =}$ $\frac{1}{3}$
More specifically:  as $x \rightarrow \infty, f(x) \rightarrow \frac{1}{3}^{+}$
\end{solution}
\end{itemize}
\end{problem}

\begin{problem}
Consider the function $f(x) = \dfrac{3 + 7x}{5 - 2x}$
\begin{itemize}
\item State the domain.
\item Identify any vertical asymptotes of the graph.
\item Identify any holes in the graph.
\item Find the horizontal asymptote, if it exists.
\item Find the slant asymptote, if it exists.
\item Graph the function using a graphing utility and describe the behavior near the asymptotes.
\end{itemize}
\end{problem} 

\begin{problem}
Consider the function $f(x) = \dfrac{x}{x^{2} + x - 12}$
\begin{itemize}
\item State the domain.
\begin{solution}
$(-\infty, -4) \cup (-4, 3) \cup (3, \infty)$
\end{solution}
\item Identify any vertical asymptotes of the graph.
\begin{solution}
$x = -4, x = 3$
\end{solution}
\item Identify any holes in the graph.
\begin{solution}
No holes in the graph
\end{solution}
\item Find the horizontal asymptote, if it exists.
$y = \answer{0}$
\item Find the slant asymptote, if it exists.
\begin{solution}
No slant asymptote
\end{solution}
\item Graph the function using a graphing utility and describe the behavior near the asymptotes.
\begin{solution}
\begin{center}
\geogebra{7jjwv7emh9}{800}{600}
\end{center}

Vertical asymptotes:

$\ds{\lim_{x \rightarrow -4^{-}} f(x) =  -\infty}$ , $\ds{\lim_{x \rightarrow -4^{+}} f(x) =  \infty}$ 

$\ds{\lim_{x \rightarrow 3^{-}} f(x) =  -\infty}$ , $\ds{\lim_{x \rightarrow 3^{+}} f(x) =  \infty}$ 

Horizontal asymptote:

$\ds{\lim_{x \rightarrow - \infty} f(x) = 0}$
More specifically, as  $x \rightarrow -\infty, f(x) \rightarrow 0^{-}$

$\ds{\lim_{x \rightarrow  \infty} f(x) = 0}$
More specifically, as $x \rightarrow \infty, f(x) \rightarrow 0^{+}$
\end{solution}
\end{itemize}
\end{problem}

\begin{problem}
Consider the function $g(t) = \dfrac{t}{t^{2} + 1}$
\begin{itemize}
\item State the domain.
\item Identify any vertical asymptotes of the graph.
\item Identify any holes in the graph.
\item Find the horizontal asymptote, if it exists.
\item Find the slant asymptote, if it exists.
\item Graph the function using a graphing utility and describe the behavior near the asymptotes.
\end{itemize}
\end{problem} 

\begin{problem}
Consider the function $g(t) = \dfrac{t + 7}{(t + 3)^{2}}$
\begin{itemize}
\item State the domain.
\begin{solution}
$(-\infty, -3) \cup (-3, \infty)$
\end{solution}
\item Identify any vertical asymptotes of the graph.
$x=\answer{-3}$
\item Identify any holes in the graph.
\begin{solution}
No holes in the graph
\end{solution}
\item Find the horizontal asymptote, if it exists.
$y = \answer{0}$
\item Find the slant asymptote, if it exists.
\begin{solution}
No slant asymptote
\end{solution}
\item Graph the function using a graphing utility and describe the behavior near the asymptotes.
\begin{solution}
\begin{center}
\geogebra{ixiribr6xw}{800}{600}
\end{center}

Vertical asymptote:

$\ds{\lim_{t \rightarrow -3} g(t) =  \infty}$ 

Horizontal asymptote:

$\ds{\lim_{t \rightarrow - \infty} g(t) = 0}$
\footnote{This is hard to see on the calculator, but trust me, the graph is below the $t$-axis to the left of $t = -7$.}  More specifically, as $t \rightarrow -\infty, g(t) \rightarrow 0^{-}$

$\ds{\lim_{t \rightarrow \infty} g(t) = 0}$\\
More specifically, as $t \rightarrow \infty, g(t) \rightarrow 0^{+}$\\
\end{solution}
\end{itemize}
\end{problem}

\begin{problem}
Consider the function $g(t) = \dfrac{t^{3} + 1}{t^{2} - 1}$
\begin{itemize}
\item State the domain.
\item Identify any vertical asymptotes of the graph.
\item Identify any holes in the graph.
\item Find the horizontal asymptote, if it exists.
\item Find the slant asymptote, if it exists.
\item Graph the function using a graphing utility and describe the behavior near the asymptotes.
\end{itemize}
\end{problem} 

\begin{problem}
Consider the function $r(z) = \dfrac{4z}{z^2+4}$
\begin{itemize}
\item State the domain.
\begin{solution}
$(-\infty,\infty)$
\end{solution}
\item Identify any vertical asymptotes of the graph.
\begin{solution}
No vertical asymptotes
\end{solution}
\item Identify any holes in the graph.
\begin{solution}
No holes in the graph
\end{solution}
\item Find the horizontal asymptote, if it exists.
$y = \answer{0}$
\item Find the slant asymptote, if it exists.
\begin{solution}
No slant asymptote
\end{solution}
\item Graph the function using a graphing utility and describe the behavior near the asymptotes.
\begin{solution}
\begin{center}
\geogebra{rbhqpfjbk2}{800}{600}
\end{center}

Horizontal asymptote:

$\ds{\lim_{z \rightarrow - \infty} r(z) =0}$
More specifically, as $z \rightarrow -\infty, r(z) \rightarrow 0^{-}$

$\ds{\lim_{z \rightarrow  \infty} r(z) =0}$
More specifically, as $z \rightarrow \infty, r(z) \rightarrow 0^{+}$
\end{solution}
\end{itemize}
\end{problem}  

\begin{problem}
Consider the function $r(z) = \dfrac{4z}{z^2-4}$
\begin{itemize}
\item State the domain.
\item Identify any vertical asymptotes of the graph.
\item Identify any holes in the graph.
\item Find the horizontal asymptote, if it exists.
\item Find the slant asymptote, if it exists.
\item Graph the function using a graphing utility and describe the behavior near the asymptotes.
\end{itemize}
\end{problem}

\begin{problem}
Consider the function $r(z) = \dfrac{z^2-z-12}{z^2+z-6}$
\begin{itemize}
\item State the domain.
\begin{solution}
$(-\infty, -3) \cup (-3, 2) \cup (2, \infty)$
\end{solution}
\item Identify any vertical asymptotes of the graph.
$z = \answer{2}$
\item Identify any holes in the graph.

There is a hole in the graph at $\answer{(-3,\frac{7}{5}})$

\item Find the horizontal asymptote, if it exists.
$y = \answer{1}$
\item Find the slant asymptote, if it exists.
\begin{solution}
No slant asymptote
\end{solution}
\item Graph the function using a graphing utility and describe the behavior near the asymptotes.
\begin{solution}
\begin{center}
\geogebra{z5wue2e7dc}{800}{600}
\end{center}

Vertical asymptote:

$\ds{\lim_{z \rightarrow 2^{-}} r(z)=  \infty}$, $\ds{\lim_{z \rightarrow 2^{+}} r(z)=  -\infty}$

Horizontal asymptote:

$\ds{\lim_{z \rightarrow - \infty} r(z) =1}$
More specifically, as $z \rightarrow -\infty, r(z) \rightarrow 1^{+}$

$\ds{\lim_{z \rightarrow  \infty} r(z) =1}$
More specifically, as $z \rightarrow \infty, r(z) \rightarrow 1^{-}$
\end{solution}
\end{itemize}
\end{problem} 

\begin{problem}
Consider the function $f(x) = \dfrac{3x^2-5x-2}{x^2-9}$
\begin{itemize}
\item State the domain.
\item Identify any vertical asymptotes of the graph.
\item Identify any holes in the graph.
\item Find the horizontal asymptote, if it exists.
\item Find the slant asymptote, if it exists.
\item Graph the function using a graphing utility and describe the behavior near the asymptotes.
\end{itemize}
\end{problem} 

\begin{problem}
Consider the function $f(x) = \dfrac{x^3+2x^2+x}{x^2-x-2}$
\begin{itemize}
\item State the domain.
\begin{solution}
$(-\infty, -1) \cup (-1, 2) \cup (2, \infty)$
\end{solution}
\item Identify any vertical asymptotes of the graph.
$x = \answer{2}$
\item Identify any holes in the graph.

There is a hole in the graph at $\answer{(-1,0)}$

\item Find the horizontal asymptote, if it exists.
\begin{solution}
No horizontal asymptote
\end{solution}
\item Find the slant asymptote, if it exists.
$y = \answer{x+3}$
\item Graph the function using a graphing utility and describe the behavior near the asymptotes.
\begin{solution}
\begin{center}
\geogebra{wjqtjjrjvh}{800}{600}
\end{center}

Vertical asymptote:

$\ds{\lim_{x \rightarrow 2^{-}} f(x)=  -\infty}$, $\ds{\lim_{x \rightarrow 2^{+}} f(x)=  \infty}$

Slant asymptote:

$\ds{\lim_{x \rightarrow -\infty} f(x) = -\infty}$
As $x \rightarrow -\infty$, the graph is below $y=x+3$

$\ds{\lim_{x \rightarrow \infty} f(x) = \infty}$
As $x \rightarrow \infty$, the graph is above $y=x+3$
\end{solution}
\end{itemize}
\end{problem}  

\begin{problem}
Consider the function $f(x) = \dfrac{x^{3} - 3x + 1}{x^{2} + 1}$
\begin{itemize}
\item State the domain.
\item Identify any vertical asymptotes of the graph.
\item Identify any holes in the graph.
\item Find the horizontal asymptote, if it exists.
\item Find the slant asymptote, if it exists.
\item Graph the function using a graphing utility and describe the behavior near the asymptotes.
\end{itemize}
\end{problem} 

\begin{problem}
Consider the function $g(t) = \dfrac{2t^{2} + 5t - 3}{3t + 2}$
\begin{itemize}
\item State the domain.
\begin{solution}
$\left(-\infty, -\frac{2}{3}\right) \cup \left(-\frac{2}{3}, \infty\right)$
\end{solution}
\item Identify any vertical asymptotes of the graph.
$t = \answer{-\frac{2}{3}}$
\item Identify any holes in the graph.
\begin{solution}
No holes in the graph
\end{solution}
\item Find the horizontal asymptote, if it exists.
\begin{solution}
No horizontal asymptote
\end{solution}
\item Find the slant asymptote, if it exists.
\begin{solution}
$y = \frac{2}{3}t + \frac{11}{9}$
\end{solution}
\item Graph the function using a graphing utility and describe the behavior near the asymptotes.
\begin{solution}
\begin{center}
\geogebra{jzyukn8w8d}{800}{600}
\end{center}

Vertical asymptote:

$\ds{\lim_{t \rightarrow -\frac{2}{3}^{-}} g(t)=  \infty}$ , $\ds{\lim_{t \rightarrow -\frac{2}{3}^{+}} g(t)=  -\infty}$

Slant asymptote:

$\ds{\lim_{t \rightarrow -\infty} g(t) = -\infty}$
As $t \rightarrow  -\infty$, the graph is above $y = \frac{2}{3}t + \frac{11}{9}$ 

$\ds{\lim_{t \rightarrow \infty} g(t) = \infty}$
 As $t \rightarrow \infty$, the graph is below $y = \frac{2}{3}t + \frac{11}{9}$ 
\end{solution}
\end{itemize}
\end{problem}  

\begin{problem}
Consider the function $g(t) = \dfrac{-t^{3} + 4t}{t^{2} - 9}$
\begin{itemize}
\item State the domain.
\item Identify any vertical asymptotes of the graph.
\item Identify any holes in the graph.
\item Find the horizontal asymptote, if it exists.
\item Find the slant asymptote, if it exists.
\item Graph the function using a graphing utility and describe the behavior near the asymptotes.
\end{itemize}
\end{problem} 

\begin{problem}
Consider the function $g(t) = \dfrac{-5t^{4} - 3t^{3} + t^{2} - 10}{t^{3} - 3t^{2} + 3t - 1}$ 
\begin{itemize}
\item State the domain.
\begin{solution}
$(-\infty, 1) \cup (1, \infty)$
\end{solution}
\item Identify any vertical asymptotes of the graph.
$t = \answer{1}$
\item Identify any holes in the graph.
\begin{solution}
No holes in the graph
\end{solution}
\item Find the horizontal asymptote, if it exists.
\begin{solution}
No horizontal asymptote
\end{solution}
\item Find the slant asymptote, if it exists.
\begin{solution}
$y=-5t-18$
\end{solution}
\item Graph the function using a graphing utility and describe the behavior near the asymptotes.
\begin{solution}
\begin{center}
\geogebra{7uro9mqusv}{800}{600}
\end{center}

Vertical asymptote:

$\ds{\lim_{t \rightarrow 1^{-}} g(t)=  \infty}$, $\ds{\lim_{t \rightarrow 1^{+}} g(t)=  -\infty}$

Slant asymptote:

$\ds{\lim_{t \rightarrow -\infty} g(t) = \infty}$
As $t \rightarrow -\infty$, the graph is above $y=-5t-18$ 

 $\ds{\lim_{t \rightarrow \infty} g(t) = -\infty}$
As $t \rightarrow \infty$, the graph is below $y=-5t-18$ 
\end{solution}
\end{itemize}
\end{problem}

\begin{problem}
Consider the function $r(z) = \dfrac{z^3}{1-z}$
\begin{itemize}
\item State the domain.
\item Identify any vertical asymptotes of the graph.
\item Identify any holes in the graph.
\item Find the horizontal asymptote, if it exists.
\item Find the slant asymptote, if it exists.
\item Graph the function using a graphing utility and describe the behavior near the asymptotes.
\end{itemize}
\end{problem} 

\begin{problem}
Consider the function $r(z) = \dfrac{18-2z^2}{z^2-9}$
\begin{itemize}
\item State the domain.
\begin{solution}
$(-\infty, -3) \cup (-3,3) \cup (3, \infty)$
\end{solution}
\item Identify any vertical asymptotes of the graph.
\begin{solution}
No vertical asymptotes
\end{solution}
\item Identify any holes in the graph.
\begin{solution}
Holes in the graph at $(-3,-2)$ and $(3,-2)$
\end{solution}
\item Find the horizontal asymptote, if it exists.
$y=\answer{-2}$
\item Find the slant asymptote, if it exists.
\begin{solution}
No slant asymptote
\end{solution}
\item Graph the function using a graphing utility and describe the behavior near the asymptotes.
\begin{solution}
\begin{center}
\geogebra{ybxxoog6c4}{800}{600}
\end{center}

Horizontal asymptote:

$\ds{\lim_{z \rightarrow   - \infty} r(z)  = -2}$ 
$\ds{\lim_{z \rightarrow    \infty} r(z)  = -2}$
\end{solution}
\end{itemize}
\end{problem}

\begin{problem}\label{alltheasymplast}
Consider the function $r(z) = \dfrac{z^3-4z^2-4z-5}{z^2+z+1}$ 
\begin{itemize}
\item State the domain.
\item Identify any vertical asymptotes of the graph.
\item Identify any holes in the graph.
\item Find the horizontal asymptote, if it exists.
\item Find the slant asymptote, if it exists.
\item Graph the function using a graphing utility and describe the behavior near the asymptotes.
\end{itemize}
\end{problem}

\begin{problem}
The cost $C(p)$ in dollars to remove $p$\% of the invasive  Ippizuti fish species from Sasquatch Pond is: \[C(p) = \frac{1770p}{100 - p}, \quad 0 \leq p < 100 \]

\begin{enumerate}

\item Find and interpret $C(25)$ and $C(95)$.
\item What does the vertical asymptote at $x = 100$ mean within the context of the problem?
\item What percentage of the Ippizuti fish can you remove for  \$40000?

\end{enumerate}
\end{problem}

\begin{problem}
In the scenario of  Example \ref{averagevelocityrocketex}, $s(t) = -5t^2+100t$, $0 \leq t \leq 20$ gives the height of a model rocket above the Moon's surface, in feet,  $t$ seconds after liftoff.  For each of the times $t_{0}$ listed below, find and simplify a the formula for the average velocity $\overline{v}(t)$ between $t$ and $t_{0}$ (see Definition \ref{averagevelocitydefn}) and use $\overline{v}(t)$ to find and interpret the instantaneous velocity of the rocket at $t = t_{0}$ (See Example \ref{averagevelocityrocketex}).

\begin{enumerate}

\item  $t_{0} = 5$

\item $t_{0} = 9$

\item $t_{0} = 10$

\item  $t_{0} = 11$

\end{enumerate}
\end{problem}

\begin{problem}\label{squatchpop}
The population of Sasquatch in Portage County $t$ years after the year 1803 is modeled by the function \[P(t) = \frac{150t}{t + 15}.\] Find and interpret the horizontal asymptote of the graph of $y = P(t)$ and explain what it means.
\end{problem}

\begin{problem}
The cost in dollars, $C(x)$ to make $x$ dOpi media players is $C(x) = 100x+2000$, $x \geq 0$.  You may wish to review the concepts of fixed and variable costs introduced in  Example \ref{PortaBoyCost} in Section \ref{LinearFunctions}.

\begin{enumerate}

\item  Find a formula for the average cost $\overline{C}(x)$.

\item  Find and interpret $\overline{C}(1)$ and $\overline{C}(100)$.

\item  How many dOpis need to be produced so that the average cost per dOpi is $\$ 200$?

\item  Find and interpret $\ds{ \lim_{x \rightarrow 0^{+}} \overline{C}(x)}$.

\item  Interpret the behavior of $\overline{C}(x)$ as $x \rightarrow \infty$.

\end{enumerate}
\end{problem}

\begin{problem}\label{averagevariablecostexercise}
This exercise explores the relationships between fixed cost, variable cost, and average cost.  The reader is encouraged to revisit Example \ref{PortaBoyCost} in Section \ref{LinearFunctions} as needed.  Suppose the cost in dollars $C(x)$ to make $x$ items is given by $C(x) = mx + b$ where $m$ and $b$ are positive real numbers.

\begin{enumerate}

\item  Show the fixed cost (the money spent even if no items are made) is $b$.
\item  Show the variable cost (the increase in cost per item made) is $m$.
\item  Find a formula for the average cost when making $x$ items, $\overline{C}(x)$.
\item  Show $\overline{C}(x) > m$ for all $x>0$ and, moreover,   $\overline{C}(x)  \rightarrow m^{+}$ as $x \rightarrow \infty$.
\item  Interpret $\overline{C}(x)  \rightarrow m^{+}$ both geometrically and in terms of fixed, variable, and average costs.

\end{enumerate}
\end{problem}

\begin{problem}\label{slantyintaverageprofitexercise}
Suppose the price-demand function for a particular product is given by $p(x) = mx + b$  where $x$ is the number of items made and sold for $p(x)$ dollars.  Here,  $m<0$ and $b>0$.  If the cost (in dollars) to make $x$ of these products is also a linear  function $C(x)$, show that the graph of the average profit function $\overline{P}(x)$ has a slant asymptote with slope $m$ and interpret.
\end{problem}

\begin{problem}
In Exercise \ref{circuitexercisepoly} in Section \ref{GraphsofPolynomials}, we fit a few polynomial models to the following electric circuit data. The circuit was built with a variable resistor.  For each of the following resistance values (measured in kilo-ohms, $k \Omega$),  the corresponding power to the load (measured in milliwatts, $mW$) is given below.\footnote{The authors wish to thank Don Anthan and Ken White of Lakeland Community College for devising this problem and generating the accompanying data set.}

\begin{tabular}{|l|r|r|r|r|r|r|} \hline
Resistance: ($k \Omega$) & 1.012 & 2.199 & 3.275 & 4.676 & 6.805 & 9.975 \\ \hline
Power: ($mW$) & 1.063 & 1.496 & 1.610 & 1.613 & 1.505 & 1.314 \\ \hline
\end{tabular}

\noindent Using some fundamental laws of circuit analysis mixed with a healthy dose of algebra, we can derive the actual formula relating power $P(x)$ to resistance $x$:   \[P(x) = \frac{25x}{(x + 3.9)^2}, \quad x \geq 0.\]

\begin{enumerate}

\item Graph the data along with the function $y = P(x)$ using a graphing utility.

\item Use a graphing utility to approximate the maximum power that can be delivered to the load.  What is the corresponding resistance value?

\item Find and interpret the end behavior of $P(x)$ as $x \rightarrow \infty$.

\end{enumerate}
\end{problem}

\begin{problem}
Let $f(x) = \dfrac{ax^2-c}{x+3}$.  Find values  for $a$ and  $c$ so  the graph of $f$ has a hole  at $(-3, 12)$.
\end{problem}

\begin{problem}
Let $f(x) = \dfrac{ax^{n} -4}{2x^2+1}$.

\begin{enumerate}

\item  Find values for $a$ and $n$ so the graph of $y = f(x)$  has the horizontal asymptote $y = 3$.

\item  Find values for $a$ and $n$ so the graph of  $y=f(x)$ has the slant asymptote $y = 5x$.

\end{enumerate}
\end{problem}

\begin{problem}
Suppose $p$ is a polynomial function and $a$ is a real number.  Define $r(x)= \dfrac{p(x) - p(a)}{x-a}$.  Use the Factor Theorem, Theorem \ref{factorthm}, to prove the graph of $y = r(x)$ has a hole at $x =a$.
\end{problem}

\begin{problem}\label{laurentarcexercise}
For each function $f(x)$ listed below, compute the average rate of change over the indicated interval.\footnote{See Definition \ref{arc} in Section \ref{AverageRateofChange} for a review of this concept, as needed.}  What trends do you observe?  How do your answers manifest themselves graphically?  How do you results compare with those of Exercise \ref{monomialarcexercise} in Section \ref{GraphsofPolynomials}?

\vspace*{-0.2in}

\[ \begin{array}{|r||c|c|c|c|}  \hline

 f(x) &  [0.9, 1.1] & [0.99, 1.01] &[0.999, 1.001] & [0.9999, 1.0001]  \\ \hline
 x^{-1} &&&&   \\  \hline
 x^{-2} &&&&    \\  \hline
 x^{-3} &&&&   \\  \hline
 x^{-4} &&&&   \\  \hline
\end{array} \]
\end{problem}

\begin{problem}
\index{Learning Curve Equation} \index{Thurstone, Louis Leon} In his now famous 1919 dissertation \underline{The Learning Curve Equation}, Louis Leon Thurstone presents a rational function which models the number of words a person can type in four minutes as a function of the number of pages of practice one has completed.\footnote{This paper, which is now in the public domain and can be found  \href{http://bit.ly/2uNaUBa}{\underline{here}}, is from a bygone era when students at business schools took typing classes on manual typewriters.} Using his original notation and original language, we have $Y = \frac{L(X + P)}{(X + P) + R}$ where $L$ is the predicted practice limit in terms of speed units, $X$ is pages written, $Y$ is writing speed in terms of words in four minutes, $P$ is equivalent previous practice in terms of pages and $R$ is the rate of learning. In Figure 5 of the paper, he graphs a scatter plot and the curve $Y = \frac{216(X + 19)}{X + 148}$.  Discuss this equation with your classmates.  How would you update the notation?  Explain what the horizontal asymptote of the graph means.  You should take some time to look at the original paper. Skip over the computations you don't understand yet and try to get a sense of the time and place in which the study was conducted.
\end{problem}


\end{document}
