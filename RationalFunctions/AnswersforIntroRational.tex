\documentclass{ximera}

\begin{document}
	\author{Stitz-Zeager}
	\xmtitle{Answers}
\mfpicnumber{1} \opengraphsfile{ExercisesforIntroRational} % mfpic settings added 


\subsection{Answers}

\begin{enumerate}
\item $4x^2+3x-1 = (x-3)(4x+15) + 44$
\item $2x^3-x+1 = \left(x^2+x+1\right)(2x-2)+(-x+3)$
\item $5x^{4} - 3x^{3} + 2x^{2} - 1 = \left(x^{2} + 4 \right) \left(5x^{2} - 3x - 18 \right) + (12x + 71)$
\item $-x^{5} + 7x^{3} - x = \left(x^{3} - x^{2} + 1 \right) \left(-x^{2} - x + 6 \right) + \left(7x^{2} - 6 \right)$
\item $9x^{3} + 5 =(2x - 3) \left(\frac{9}{2}x^{2} + \frac{27}{4}x + \frac{81}{8} \right) + \frac{283}{8}$
\item $4x^2 - x - 23 = \left(x^{2} - 1 \right)(4) + (-x - 19)$
\setcounter{HW}{\value{enumi}}
\end{enumerate}


\begin{multicols}{2}
\begin{enumerate}
\setcounter{enumi}{\value{HW}}

\item $F(x) = \dfrac{1}{x-2}+1$ \\ [10pt]
Domain: $(-\infty, 2) \cup (2, \infty)$ \\
Range: $(-\infty, 1) \cup (1, \infty)$ \\
Vertical asymptote:  $x = 2$\\
Horizontal asymptote:  $y = 1$ \\

\begin{mfpic}[15]{-5}{5}{-5}{5}
\axes
\tlabel[cc](5,-0.5){\scriptsize $x$}
\tlabel[cc](0.5,5){\scriptsize $y$}
%\tlabel[cc](-1.5, 0.5){\scriptsize $(-1,0)$}
%\tlabel[cc](-0.5,-1){\scriptsize $\left(0, \frac{1}{2} \right)$}
\xmarks{-4,-3,-2,-1,1,2,3,4}
\ymarks{-4,-3,-2, -1, 1,2,3,4}
\tlpointsep{4pt}
\scriptsize
\axislabels {x}{ {$-4 \hspace{7pt}$} -4, {$-3 \hspace{7pt}$} -3, {$-2 \hspace{7pt}$} -2, {$-1 \hspace{7pt}$} -1, {$1$} 1, {$2$} 2, {$3$} 3, {$4$} 4}
\axislabels {y}{{$-1$} -1,{$1$} 1, {$2$} 2, {$3$} 3, {$4$} 4, {$-2$} -2, {$-3$} -3, {$-4$} -4}
\dashed \polyline{(-4.75,1), (4.75,1)}
\dashed \polyline{(2,-4.75), (2,4.75)}
\penwd{1.25pt}
\arrow \reverse \arrow \function{-5,1.8,0.1}{1+1/(x-2)}
\arrow \reverse \arrow \function{2.3,5,0.1}{1+1/(x-2)}
\point[4pt]{(1,0), (3,2)}
\normalsize
\end{mfpic}


\vfill

\columnbreak

\item $F(x) = \dfrac{2x}{x+1} = \dfrac{-2}{x+1}+2$\\ [10pt]
Domain: $(-\infty, -1) \cup (-1, \infty)$ \\
Range: $(-\infty, 2) \cup (2, \infty)$ \\
Vertical asymptote:  $x = -1$\\
Horizontal asymptote:  $y = 2$ \\

\begin{mfpic}[15]{-5}{5}{-5}{5}
\axes
\tlabel[cc](5,-0.5){\scriptsize $x$}
\tlabel[cc](0.5,5){\scriptsize $y$}
%\tlabel[cc](-1.5, 0.5){\scriptsize $(-1,0)$}
%\tlabel[cc](-0.5,-1){\scriptsize $\left(0, \frac{1}{2} \right)$}
\xmarks{-4,-3,-2,-1,1,2,3,4}
\ymarks{-4,-3,-2, -1, 1,2,3,4}
\tlpointsep{4pt}
\scriptsize
\axislabels {x}{ {$-4 \hspace{7pt}$} -4, {$-3 \hspace{7pt}$} -3, {$-2 \hspace{7pt}$} -2, {$-1 \hspace{7pt}$} -1, {$1$} 1, {$2$} 2, {$3$} 3, {$4$} 4}
\axislabels {y}{{$1$} 1, {$2$} 2, {$3$} 3, {$4$} 4}
\dashed \polyline{(-4.75,2), (4.75,2)}
\dashed \polyline{(-1,-4.75), (-1,4.75)}
\penwd{1.25pt}
\arrow \reverse \arrow \function{-5,-1.7,0.1}{2x/(x+1)}
\arrow \reverse \arrow \function{-0.7,5,0.1}{2x/(x+1)}
\point[4pt]{(0,0), (-2,4)}
\normalsize
\end{mfpic}

\setcounter{HW}{\value{enumi}}
\end{enumerate}
\end{multicols}



\begin{multicols}{2}
\begin{enumerate}
\setcounter{enumi}{\value{HW}}

\item $F(x)=4x(2x+1)^{-1} = \dfrac{4x}{2x+1} = \dfrac{-1}{x+\frac{1}{2}}+2$ \\ [10pt]
Domain: $\left(-\infty, -\frac{1}{2} \right) \cup \left(-\frac{1}{2},  \infty \right)$ \\
Range: $(-\infty, 2) \cup (2, \infty)$ \\
Vertical asymptote:  $x  = -\frac{1}{2}$ \\
Horizontal asymptote: $y= 2$\\ 

\begin{mfpic}[15]{-5}{5}{-5}{5}
\axes
\tlabel[cc](5,-0.5){\scriptsize $x$}
\tlabel[cc](0.5,5){\scriptsize $y$}
%\tlabel[cc](-1.5, 0.5){\scriptsize $(-1,0)$}
%\tlabel[cc](-0.5,-1){\scriptsize $\left(0, \frac{1}{2} \right)$}
\xmarks{-4,-3,-2,-1,1,2,3,4}
\ymarks{-4,-3,-2, -1, 1,2,3,4}
\tlpointsep{4pt}
\scriptsize
\axislabels {x}{ {$-4 \hspace{7pt}$} -4, {$-3 \hspace{7pt}$} -3, {$-2 \hspace{7pt}$} -2, {$-1 \hspace{7pt}$} -1, {$1$} 1, {$2$} 2, {$3$} 3, {$4$} 4}
\axislabels {y}{{$1$} 1, {$2$} 2, {$3$} 3, {$4$} 4}
\dashed \polyline{(-4.75,2), (4.75,2)}
\dashed \polyline{(-0.5,-4.75), (-0.5,4.75)}
\penwd{1.25pt}
\arrow \reverse \arrow \function{-5,-0.84,0.1}{4*x/(2*x+1)}
\arrow \reverse \arrow \function{-0.35,5,0.1}{4*x/(2*x+1)}
\point[4pt]{(-1.5,3), (0.5,1)}
\normalsize
\end{mfpic}


\vfill

\columnbreak

\item $F(x)=-(x-1)^{-2}+3 = \dfrac{-1}{(x-1)^2} + 3$\\[10pt]
Domain: $(-\infty, 1) \cup (1, \infty)$ \\
Range: $(-\infty, 3) \cup (3, \infty)$ \\
Vertical asymptote:  $x = 1$\\
Horizontal asymptote:  $y = 3$ \\

\begin{mfpic}[15]{-5}{5}{-5}{5}
\axes
\tlabel[cc](5,-0.5){\scriptsize $x$}
\tlabel[cc](0.5,5){\scriptsize $y$}
%\tlabel[cc](-1.5, 0.5){\scriptsize $(-1,0)$}
%\tlabel[cc](-0.5,-1){\scriptsize $\left(0, \frac{1}{2} \right)$}
\xmarks{-4,-3,-2,-1,1,2,3,4}
\ymarks{-4,-3,-2, -1, 1,2,3,4}
\tlpointsep{4pt}
\scriptsize
\axislabels {x}{ {$-4 \hspace{7pt}$} -4, {$-3 \hspace{7pt}$} -3, {$-2 \hspace{7pt}$} -2, {$-1 \hspace{7pt}$} -1,  {$2$} 2, {$3$} 3, {$4$} 4}
\axislabels {y}{{$1$} 1, {$2$} 2, {$3$} 3, {$4$} 4, {$-1$} -1, {$-2$} -2, {$-3$} -3, {$-4$} -4}
\dashed \polyline{(-4.75,3), (4.75,3)}
\dashed \polyline{(1,-4.75), (1,4.75)}
\penwd{1.25pt}
\arrow \reverse \arrow \function{-5,0.64,0.1}{3-1/((x-1)**2)}
\arrow \reverse \arrow \function{1.36,5,0.1}{3-1/((x-1)**2)}
\point[4pt]{(0,2), (2,2)}
\normalsize
\end{mfpic}

\setcounter{HW}{\value{enumi}}
\end{enumerate}
\end{multicols}

\begin{multicols}{2}
\begin{enumerate}
\setcounter{enumi}{\value{HW}}

\item $F(x) = \dfrac{1}{x+2}-1$

\item  $F(x) = \dfrac{-2}{x-1}+1$

\setcounter{HW}{\value{enumi}}
\end{enumerate}
\end{multicols}


\begin{multicols}{2}

\begin{enumerate}

\setcounter{enumi}{\value{HW}}

\item  $F(x) = \dfrac{-4}{(x+2)^2}+4$  \vphantom{$F(x) = \dfrac{1}{\left(x-\frac{1}{2}\right)^2}-4$}

\item $F(x) = \dfrac{1}{\left(x-\frac{1}{2}\right)^2}-4$

\setcounter{HW}{\value{enumi}}
\end{enumerate}
\end{multicols}



\begin{multicols}{2}
\begin{enumerate}
\setcounter{enumi}{\value{HW}}

\item $f(x) = \dfrac{x}{3x - 6}$ \vphantom{$\dfrac{7x}{7x}$}\\
Domain: $(-\infty, 2) \cup (2, \infty)$\\
Vertical asymptote: $x = 2$\\
$\ds{\lim_{x \rightarrow 2^{-}} f(x) = -\infty}$, $\ds{\lim_{x \rightarrow 2^{+}} f(x) = \infty}$ \\
No holes in the graph\\
Horizontal asymptote: $y = \frac{1}{3}$ \\
$\ds{\lim_{x \rightarrow  -\infty} f(x) =}$ $\frac{1}{3}$\\
More specifically: as $x \rightarrow -\infty, f(x) \rightarrow \frac{1}{3}^{-}$\\
$\ds{\lim_{x \rightarrow  \infty} f(x) =}$ $\frac{1}{3}$\\
More specifically:  as $x \rightarrow \infty, f(x) \rightarrow \frac{1}{3}^{+}$\\

\vfill

\columnbreak

\item $f(x) = \dfrac{3 + 7x}{5 - 2x}$\\
Domain: $(-\infty, \frac{5}{2}) \cup (\frac{5}{2}, \infty)$\\
Vertical asymptote: $x = \frac{5}{2}$\\
$\ds{\lim_{x \rightarrow \frac{5}{2}^{-}} f(x) = \infty}$, $\ds{\lim_{x \rightarrow \frac{5}{2}^{+}} f(x) = -\infty}$ \\
No holes in the graph\\
Horizontal asymptote: $y = -\frac{7}{2}$ \\
$\ds{\lim_{x \rightarrow - \infty} f(x) =}$ $-\frac{7}{2}$\\
More specifically: as $x \rightarrow -\infty, f(x) \rightarrow -\frac{7}{2}^{+}$\\
$\ds{\lim_{x \rightarrow  \infty} f(x) =}$ $-\frac{7}{2}$\\
More specifically: as  $x \rightarrow \infty, f(x) \rightarrow -\frac{7}{2}^{-}$\\

\setcounter{HW}{\value{enumi}}
\end{enumerate}
\end{multicols}

\pagebreak 

\begin{multicols}{2}
\begin{enumerate}
\setcounter{enumi}{\value{HW}}

\item $f(x) = \dfrac{x}{x^{2} + x - 12} = \dfrac{x}{(x + 4)(x - 3)}$\\
Domain: $(-\infty, -4) \cup (-4, 3) \cup (3, \infty)$\\
Vertical asymptotes: $x = -4, x = 3$\\
$\ds{\lim_{x \rightarrow -4^{-}} f(x) =  -\infty}$ , $\ds{\lim_{x \rightarrow -4^{+}} f(x) =  \infty}$ \\
$\ds{\lim_{x \rightarrow 3^{-}} f(x) =  -\infty}$ , $\ds{\lim_{x \rightarrow 3^{+}} f(x) =  \infty}$ \\
No holes in the graph\\
Horizontal asymptote: $y = 0$ \\
$\ds{\lim_{x \rightarrow - \infty} f(x) = 0}$\\
More specifically, as  $x \rightarrow -\infty, f(x) \rightarrow 0^{-}$\\
$\ds{\lim_{x \rightarrow  \infty} f(x) = 0}$\\
More specifically, as $x \rightarrow \infty, f(x) \rightarrow 0^{+}$\\



\columnbreak


\item $g(t) = \dfrac{t}{t^{2} + 1}$\\
Domain: $(-\infty, \infty)$\\
No vertical asymptotes\\
No holes in the graph\\
Horizontal asymptote: $y = 0$ \\
$\ds{\lim_{t \rightarrow  - \infty} g(t) = 0}$\\
More specifically, as $t \rightarrow -\infty, g(t) \rightarrow 0^{-}$\\
$\ds{\lim_{t \rightarrow   \infty} g(t) = 0}$\\
More specifically, as $t \rightarrow \infty, g(t) \rightarrow 0^{+}$\\

\setcounter{HW}{\value{enumi}}
\end{enumerate}
\end{multicols}

\begin{multicols}{2}
\begin{enumerate}
\setcounter{enumi}{\value{HW}}

\item $g(t) = \dfrac{t + 7}{(t + 3)^{2}}$ \vphantom{$\dfrac{t^{3}}{t^{2}}$}\\
Domain: $(-\infty, -3) \cup (-3, \infty)$\\
Vertical asymptote: $t = -3$\\
$\ds{\lim_{t \rightarrow -3} g(t) =  \infty}$ \\
No holes in the graph\\
Horizontal asymptote: $y = 0$ \\
$\ds{\lim_{t \rightarrow - \infty} g(t) = 0}$\\
\footnote{This is hard to see on the calculator, but trust me, the graph is below the $t$-axis to the left of $t = -7$.}More specifically, as $t \rightarrow -\infty, g(t) \rightarrow 0^{-}$\\
$\ds{\lim_{t \rightarrow \infty} g(t) = 0}$\\
More specifically, as $t \rightarrow \infty, g(t) \rightarrow 0^{+}$\\

\vfill

\columnbreak

\item $g(t) = \dfrac{t^{3} + 1}{t^{2} - 1} = \dfrac{t^{2} - t+ 1}{t-1}$\\
Domain: $(-\infty, -1) \cup (-1, 1) \cup (1, \infty)$\\
Vertical asymptote: $t = 1$\\
$\ds{\lim_{t \rightarrow 1^{-}} g(t) =  -\infty}$, $\ds{\lim_{t \rightarrow 1^{+}} g(t) =  \infty}$ \\
Hole at $(-1, -\frac{3}{2})$\\
Slant asymptote: $y=t$  \\
$\ds{\lim_{t \rightarrow -\infty} g(t) = -\infty}$\\
As $t \rightarrow -\infty$, the graph is below $y=t$\\
$\ds{\lim_{t \rightarrow \infty} g(t) = \infty}$\\
As $t \rightarrow \infty$, the graph is above $y=t$\\

\setcounter{HW}{\value{enumi}}
\end{enumerate}
\end{multicols}

\begin{multicols}{2}
\begin{enumerate}
\setcounter{enumi}{\value{HW}}

\item $r(z) = \dfrac{4z}{z^{2} + 4}$\\
Domain: $(-\infty,  \infty)$\\
No vertical asymptotes \\
No holes in the graph\\
Horizontal asymptote: $y = 0$ \\
$\ds{\lim_{z \rightarrow - \infty} r(z) =0}$\\
More specifically, as $z \rightarrow -\infty, r(z) \rightarrow 0^{-}$\\
$\ds{\lim_{z \rightarrow  \infty} r(z) =0}$\\
More specifically, as $z \rightarrow \infty, r(z) \rightarrow 0^{+}$\\


\vfill

\columnbreak

\item $r(z) = \dfrac{4z}{z^{2} -4} = \dfrac{4z}{(z + 2)(z - 2)}$\\
Domain: $(-\infty, -2) \cup (-2, 2) \cup (2, \infty)$\\
Vertical asymptotes: $z = -2, z = 2$\\
$\ds{\lim_{z \rightarrow -2^{-}} r(z)=  -\infty}$, $\ds{\lim_{z \rightarrow -2^{+}} r(z)=  \infty}$ \\
$\ds{\lim_{z \rightarrow 2^{-}} r(z)=  -\infty}$, $\ds{\lim_{z \rightarrow 2^{+}}  r(z)=  \infty}$  \\
No holes in the graph\\
Horizontal asymptote: $y = 0$ \\
$\ds{\lim_{z \rightarrow -\infty} r(z) =0}$\\
More specifically, as $z \rightarrow -\infty, r(z) \rightarrow 0^{-}$\\
$\ds{\lim_{z \rightarrow \infty} r(z) =0}$\\
More specifically, as $z \rightarrow \infty, r(z) \rightarrow 0^{+}$\\

\setcounter{HW}{\value{enumi}}
\end{enumerate}
\end{multicols}

\begin{multicols}{2}
\begin{enumerate}
\setcounter{enumi}{\value{HW}}

\item $r(z) = \dfrac{z^2-z-12}{z^{2} +z - 6} = \dfrac{z-4}{z - 2}$\\
Domain: $(-\infty, -3) \cup (-3, 2) \cup (2, \infty)$\\
Vertical asymptote: $z = 2$\\
$\ds{\lim_{z \rightarrow 2^{-}} r(z)=  \infty}$, $\ds{\lim_{z \rightarrow 2^{+}} r(z)=  -\infty}$ \\
Hole at $\left(-3, \frac{7}{5} \right)$ \\
Horizontal asymptote: $y = 1$ \\
$\ds{\lim_{z \rightarrow - \infty} r(z) =1}$\\
More specifically, as $z \rightarrow -\infty, r(z) \rightarrow 1^{+}$\\
$\ds{\lim_{z \rightarrow  \infty} r(z) =1}$\\
More specifically, as $z \rightarrow \infty, r(z) \rightarrow 1^{-}$\\


\vfill

\columnbreak

\item $f(x) = \dfrac{3x^2-5x-2}{x^{2} -9} = \dfrac{(3x+1)(x-2)}{(x + 3)(x - 3)}$\\
Domain: $(-\infty, -3) \cup (-3, 3) \cup (3, \infty)$\\
Vertical asymptotes: $x = -3, x = 3$\\
$\ds{\lim_{x \rightarrow -3^{-}} f(x)=  \infty}$, $\ds{\lim_{x \rightarrow -3^{+}} f(x)=  -\infty}$ \\
$\ds{\lim_{x \rightarrow 3^{-}} f(x)=  -\infty}$, $\ds{\lim_{x \rightarrow 3^{+}} f(x)=  \infty}$ \\
No holes in the graph\\
Horizontal asymptote: $y = 3$ \\
$\ds{\lim_{x \rightarrow   -\infty} f(x)  =3}$\\
More specifically, as $x \rightarrow -\infty, f(x) \rightarrow 3^{+}$\\
$\ds{\lim_{x \rightarrow \infty} f(x)  =3}$\\
More specifically, as $x \rightarrow \infty, f(x) \rightarrow 3^{-}$\\

\setcounter{HW}{\value{enumi}}
\end{enumerate}
\end{multicols}

\begin{multicols}{2}
\begin{enumerate}
\setcounter{enumi}{\value{HW}}


\item $f(x) = \dfrac{x^3+2x^2+x}{x^{2} -x-2} = \dfrac{x(x+1)}{x - 2}$\\
Domain: $(-\infty, -1) \cup (-1, 2) \cup (2, \infty)$\\
Vertical asymptote: $x = 2$\\
$\ds{\lim_{x \rightarrow 2^{-}} f(x)=  -\infty}$, $\ds{\lim_{x \rightarrow 2^{+}} f(x)=  \infty}$ \\
Hole at $(-1,0)$ \\
Slant asymptote: $y=x+3$ \\
$\ds{\lim_{x \rightarrow -\infty} f(x) = -\infty}$\\
As $x \rightarrow -\infty$, the graph is below $y=x+3$\\
$\ds{\lim_{x \rightarrow \infty} f(x) = \infty}$\\
As $x \rightarrow \infty$, the graph is above $y=x+3$\\

\vfill

\columnbreak

\item $f(x) = \dfrac{x^3-3x+1}{x^2+1}$\\
Domain: $(-\infty, \infty)$\\
No vertical asymptotes \\
No holes in the graph \\
Slant asymptote: $y=x$ \\
$\ds{\lim_{x \rightarrow -\infty} f(x) = -\infty}$\\
As $x \rightarrow -\infty$, the graph is above $y=x$ \\
$\ds{\lim_{x \rightarrow \infty} f(x) = \infty}$\\
As $x \rightarrow \infty$, the graph is below $y=x$  \\


\setcounter{HW}{\value{enumi}}
\end{enumerate}
\end{multicols}


\begin{multicols}{2}
\begin{enumerate}
\setcounter{enumi}{\value{HW}}

\item $g(t) = \dfrac{2t^{2} + 5t - 3}{3t + 2}$\\
Domain: $\left(-\infty, -\frac{2}{3}\right) \cup \left(-\frac{2}{3}, \infty\right)$\\
Vertical asymptote: $t = -\frac{2}{3}$\\
$\ds{\lim_{t \rightarrow -\frac{2}{3}^{-}} g(t)=  \infty}$ , $\ds{\lim_{t \rightarrow -\frac{2}{3}^{+}} g(t)=  -\infty}$ \\
No holes in the graph \\
Slant asymptote:  $y = \frac{2}{3}t + \frac{11}{9}$ \\
$\ds{\lim_{t \rightarrow -\infty} g(t) = -\infty}$\\
As $t \rightarrow  -\infty$, the graph is above \small $y = \frac{2}{3}t + \frac{11}{9}$ \normalsize\\
$\ds{\lim_{t \rightarrow \infty} g(t) = \infty}$\\
 As $t \rightarrow \infty$, the graph is below \small $y = \frac{2}{3}t + \frac{11}{9}$ \normalsize \\

\vfill

\columnbreak

\item $g(t) = \dfrac{-t^{3} + 4t}{t^{2} - 9} = \dfrac{-t^{3} + 4t}{(t-3)(t+3)} $\\
Domain: $(-\infty, -3) \cup (-3, 3) \cup (3, \infty)$\\
Vertical asymptotes: $t = -3$, $t=3$\\
$\ds{\lim_{t \rightarrow -3^{-}} g(t)=  \infty}$, $\ds{\lim_{t \rightarrow -3^{+}} g(t)=  -\infty}$ \\
$\ds{\lim_{t \rightarrow 3^{-}} g(t)=  \infty}$, $\ds{\lim_{t \rightarrow 3^{+}} g(t)=  -\infty}$ \\
No holes in the graph \\
Slant asymptote: $y=-t$ \\
$\ds{\lim_{t \rightarrow -\infty} g(t) = \infty}$\\
As $t \rightarrow -\infty$, the graph is above $y=-t$\\
$\ds{\lim_{t \rightarrow \infty} g(t) = -\infty}$\\
As $t \rightarrow \infty$, the graph is below $y=-t$\\


\setcounter{HW}{\value{enumi}}
\end{enumerate}
\end{multicols}

\begin{multicols}{2}
\begin{enumerate}
\setcounter{enumi}{\value{HW}}

\item \small $g(t) = \dfrac{-5t^{4} - 3t^{3} + t^{2} - 10}{t^{3} - 3t^{2} + 3t - 1} \\ \phantom{g(t)} = \dfrac{-5t^{4} - 3t^{3} + t^{2} - 10}{(t-1)^3} $ \normalsize \\
Domain: $(-\infty, 1) \cup (1, \infty)$\\
Vertical asymptotes: $t = 1$\\
$\ds{\lim_{t \rightarrow 1^{-}} g(t)=  \infty}$, $\ds{\lim_{t \rightarrow 1^{+}} g(t)=  -\infty}$ \\
No holes in the graph \\
Slant asymptote: $y=-5t-18$ \\
$\ds{\lim_{t \rightarrow -\infty} g(t) = \infty}$\\
 \small  As $t \rightarrow -\infty$, the graph is above $y=-5t-18$ \normalsize\\
 $\ds{\lim_{t \rightarrow \infty} g(t) = -\infty}$\\
 \small  As $t \rightarrow \infty$, the graph is below $y=-5t-18$ \normalsize \\

 \vfill

 \columnbreak

\item $r(z) = \dfrac{z^3}{1-z}$\\
Domain: $(-\infty, 1) \cup (1, \infty)$\\
Vertical asymptote: $z=1$\\
$\ds{\lim_{z \rightarrow 1^{-}} r(z) =  \infty}$ \\
$\ds{\lim_{z \rightarrow 1^{+}} r(z) =  -\infty}$ \\
No holes in the graph \\
No horizontal or slant asymptote \\
$\ds{\lim_{z \rightarrow  -  \infty} r(z)  = -\infty}$\\
$\ds{\lim_{z \rightarrow   \infty} r(z)  = -\infty}$\\
\setcounter{HW}{\value{enumi}}
\end{enumerate}
\end{multicols}

\begin{multicols}{2}
\begin{enumerate}
\setcounter{enumi}{\value{HW}}

\item $r(z) = \dfrac{18-2z^2}{z^2-9} = -2$\\
Domain: $(-\infty, -3) \cup (-3,3) \cup (3, \infty)$\\
No vertical asymptotes \\
Holes in the graph at $(-3,-2)$ and $(3,-2)$ \\
Horizontal asymptote $y = -2$ \\
$\ds{\lim_{z \rightarrow   - \infty} r(z)  = -2}$ \\
$\ds{\lim_{z \rightarrow    \infty} r(z)  = -2}$ \\

\vfill
\columnbreak

\item $r(z) = \dfrac{z^3-4z^2-4z-5}{z^2+z+1} = z-5$\\
Domain: $(-\infty, \infty)$\\
No vertical asymptotes \\
No holes in the graph \\
Slant asymptote:  $y = z-5$ \\
$\ds{\lim_{z \rightarrow -\infty} r(z) = -\infty}$\\
$\ds{\lim_{z \rightarrow \infty} r(z) = \infty}$\\
$r(z) = z-5$ everywhere. \\
\setcounter{HW}{\value{enumi}}
\end{enumerate}
\end{multicols}

\begin{enumerate}
\setcounter{enumi}{\value{HW}}


\item \begin{enumerate}

\item $C(25) = 590$ means it costs \$590 to remove 25\% of the fish and and $C(95)= 33630$ means it would cost \$33630 to remove 95\% of the fish from the pond.
\item The vertical asymptote at $x = 100$ means that as we try to remove 100\% of the fish from the pond, the cost increases without bound; i.e., it's impossible to remove all of the fish.
\item For \$40000 you could remove about 95.76\% of the fish.

\end{enumerate}

\item  \begin{enumerate}

\item  $\overline{v}(t) = \frac{s(t) - s(5)}{t - 5} = \frac{-5t^2+100t-375}{t-5} = -5t+75$, $t \neq 5$.  The instantaneous velocity of the rocket when $t_{0} = 5$ is $-5(5)+75 = 50$ meaning it is traveling $50$ feet per second upwards.

\item  $\overline{v}(t) = \frac{s(t) - s(9)}{t - 9} = \frac{-5t^2+100t-495}{t-9} = -5t+55$, $t \neq 9$.  The instantaneous velocity of the rocket when $t_{0} = 9$ is $-5(9)+55 = 10$, so the rocket has slowed to $10$ feet per second (but still heading up.)

\item $\overline{v}(t) = \frac{s(t) - s(10)}{t - 10} = \frac{-5t^2+100t-495}{t-10} = -5t+50$, $t \neq 10$.  The instantaneous velocity of the rocket when $t_{0} = 10$ is $-5(10)+50 = 0$, so the rocket has momentarily stopped!  In Example \ref{ARCRocketExample}, we learned the rocket reaches its maximum height when $t = 10$ seconds, which means the rocket must change direction from heading up to coming back down, so it makes sense that for this instant, its velocity is $0$.

\item  $\overline{v}(t) = \frac{s(t) - s(11)}{t - 11} = \frac{-5t^2+100t-495}{t-11} = -5t+45$, $t \neq 11$.  The instantaneous velocity of the rocket when $t_{0} = 11$ is $-5(11)+45 = -10$ meaning the rocket has, indeed, changed direction and is heading downwards at a rate of $10$ feet per second.  (Note the symmetry here between this answer and our answer when $t=9$.)

\end{enumerate}

\item The horizontal asymptote of the graph of $P(t) = \frac{150t}{t + 15}$ is $y = 150$ and it means that the model predicts the population of Sasquatch in Portage County will never exceed 150.

\item  \begin{enumerate}

\item $\overline{C}(x) = \frac{100x+2000}{x} = 100 + \frac{2000}{x}$, $x > 0$.

\item  $\overline{C}(1) = 2100$ and $\overline{C}(100) = 120$. When just $1$ dOpi is produced, the cost per dOpi is $\$2100$, but when $100$ dOpis are produced, the cost per dOpi is $\$120$.

\item  $\overline{C}(x) = 200$ when $x = 20$.  So to get the cost per dOpi to $\$200$, $20$ dOpis need to be produced.

\item  We find $\ds{\lim_{x \rightarrow 0^{+}} \overline{C}(x) = \infty}$.  This means that as fewer and fewer dOpis are produced, the cost per dOpi becomes unbounded.  In this situation, there is a fixed cost of $\$2000$ ($C(0) = 2000$), we are trying to spread that $\$2000$ over fewer and fewer dOpis.

\item   As $x \rightarrow \infty$,  $\overline{C}(x) \rightarrow 100^{+}$.  This means that as more and more dOpis are produced, the cost per dOpi approaches $\$100$, but is always a little more than $\$100$.  Since $\$100$ is the variable cost per dOpi ($C(x) = \underline{100}x+2000$), it means that no matter how many dOpis are produced, the average cost per dOpi will always be a bit higher than the variable cost to produce a dOpi.  As before, we can attribute this to the $\$2000$ fixed cost, which factors into the average cost per dOpi no matter how many dOpis are produced.

\end{enumerate}

\item  \begin{enumerate}

\item  The cost to make $0$ items is $C(0) = m(0)+b = b$.  Hence,  so the fixed costs are $b$.
\item $C(x) = mx+b$ is a linear function with slope $m>0$.  Hence, the cost increases at a rate of $m$ dollars per item made.  Hence, the variable cost is $m$.
\item  $\overline{C}(x) = \frac{C(x)}{x} = \frac{mx+b}{x} = m + \frac{b}{x}$ for $x > 0$.
\item  Since $b>0$,  $\overline{C}(x)  = m + \frac{b}{x} > m$ for $x > 0$. As $x \rightarrow \infty$, $\frac{b}{x} \rightarrow 0$ so $\overline{C}(x)  = m + \frac{b}{x} \rightarrow m$.
\item Geometrically, the graph of $y = \overline{C}(x)$ has a horizontal asymptote $y = m$, the variable cost.  In terms of costs, as more items are produced, the affect of the fixed cost on the average cost, $\frac{b}{x}$ falls away so that the average cost per item approaches the variable cost to make each item.

\end{enumerate}

\item   If $p(x) = mx + b$ and $C(x)$ is linear, say $C(x) = rx+s$, then we can compute the the profit function (in general) as: $P(x) = xp(x) - C(x) = x(mx+b) - (rx+s)$ which simplifies to $P(x) = mx^2 + (b-r)x -s$.  Hence, the average profit $\overline{P}(x) = \frac{P(x)}{x} = \frac{mx^2 + (b-r)x -s}{x} = mx + (b-r) - \frac{s}{x}$.  We see that as $x \rightarrow \infty$, $\frac{s}{x} \rightarrow 0$ so $\overline{P}(x) \approx mx  + (b-r)$.  Hence, $y = mx + (b-r)$ is the slant asymptote  to $y = \overline{P}(x)$.  This means that as more items are sold, the average profit is decreasing at approximately the same rate as the price function is decreasing, $m$ dollars per item.  That is, to sell one additional item, we drop the price $p(x)$ by $m$ dollars which results in a drop in the average profit by approximately $m$ dollars.

\pagebreak


\item \begin{enumerate}

\item $~$

\includegraphics[height=2in]{./IntroRationalGraphics/MaxPowerRegression.jpg}

\item The maximum power is approximately $1.603 \; mW$ which corresponds to $3.9 \; k\Omega$.

\item As $x \rightarrow \infty, \; P(x) \rightarrow 0^{+}$ which means as the resistance increases without bound, the power diminishes to zero.

\end{enumerate}

\item  $a = -2$ and $c = -18$ so $f(x) = \dfrac{-2x^2+18}{x+3}$.

\item  \begin{multicols}{2}

 \begin{enumerate}

\item   $a=6$ and $n=2$ so $f(x) = \dfrac{6x^{2} -4}{2x^2+1}$

\item  $a=10$ and $n = 3$ so $f(x) = \dfrac{10x^{3} -4}{2x^2+1}$ .

\end{enumerate}
\end{multicols}


\item  If we define $f(x) = p(x) - p(a)$ then $f$ is a polynomial function with $f(a) = p(a) - p(a) = 0$.  The Factor Theorem guarantees $(x-a)$ is a factor of $f(x)$, that is, $f(x) = p(x) - p(a) = (x-a)q(x)$ for some polynomial $q(x)$. Hence, $r(x) = \frac{p(x)-p(a)}{x-a} = \frac{(x-a)q(x)}{x-a} = q(x)$ so the graph of $y = r(x)$ is the same as the graph of the polynomial $y = q(x)$ except for a hole when $x = a$.

\item  The slope of the curves near $x=1$ matches the exponent on $x$.  This exactly what we saw in  Exercise \ref{monomialarcexercise} in Section \ref{GraphsofPolynomials}.

\[ \begin{array}{|r||c|c|c|c|}  \hline

 f(x) &  [0.9, 1.1] & [0.99, 1.01] &[0.999, 1.001] & [0.9999, 1.0001]  \\ \hline
 x^{-1} & -1.0101 & -1.0001 & \approx -1 & \approx -1  \\  \hline
 x^{-2}& -2.0406& -2.0004 & \approx -2 & \approx -2   \\  \hline
 x^{-3} &-3.1021 & -3.0010 & \approx -3 & \approx -3   \\  \hline
 x^{-4} &-4.2057 & -4.0020 &\approx -4 & \approx -4   \\   \hline


\end{array} \]

\end{enumerate} 

\end{document}
