\documentclass{ximera}

\begin{document}
	\author{Stitz-Zeager}
	\xmtitle{TITLE}
\mfpicnumber{1} \opengraphsfile{ExercisesforMatMethods} % mfpic settings added 


\label{ExercisesforMatMethods}

In Exercises \ref{findmatinversefirst} - \ref{findmatinverselast}, find the inverse of the matrix or state that the matrix is not invertible.

\begin{multicols}{2}
\begin{enumerate}

\item $A = \left[ \begin{array}{rr} 1 & 2 \\ 3 & 4 \end{array} \right]$ \label{findmatinversefirst}
\item $B = \left[ \begin{array}{rr} 12 & -7 \\ -5 & 3 \end{array} \right]$ \label{matrixB}

\setcounter{HW}{\value{enumi}}
\end{enumerate}
\end{multicols}

\begin{multicols}{2}
\begin{enumerate}
\setcounter{enumi}{\value{HW}}

\item $C = \left[ \begin{array}{rr} 6 & 15 \\ 14 & 35 \end{array} \right]$
\item $D = \left[ \begin{array}{rr} 2 & -1 \\ 16 & -9 \end{array} \right]$ \label{matrixD}

\setcounter{HW}{\value{enumi}}
\end{enumerate}
\end{multicols}

\begin{multicols}{2}
\begin{enumerate}
\setcounter{enumi}{\value{HW}}

\item $E = \left[ \begin{array}{rrr} 3 & 0 & 4 \\ 2 & -1 & 3 \\ -3 & 2 & -5 \end{array} \right]$ \label{matrixE}
\item $F = \left[ \begin{array}{rrr} 4 & \hphantom{-}6 & -3 \\ 3 & 4 & -3 \\ 1 & 2 & 6 \end{array} \right]$

\setcounter{HW}{\value{enumi}}
\end{enumerate}
\end{multicols}

\begin{multicols}{2}
\begin{enumerate}
\setcounter{enumi}{\value{HW}}

\item $G = \left[ \begin{array}{rrr} 1 & \hphantom{1}2 & 3 \\ 2 & 3 & 11 \\ 3 & 4 & 19 \end{array} \right]$
\item $H = \left[ \begin{array}{rrrr} 1 & 0 & -3 & \hphantom{-}0 \\ 2 & -2 & 8 & 7 \\ -5 & 0 & 16 & 0 \\ 1 & 0 & 4 & 1 \end{array} \right]$ \label{findmatinverselast}

\setcounter{HW}{\value{enumi}}
\end{enumerate}
\end{multicols}


In Exercises \ref{2by2inversefirst} - \ref{2by2inverselast}, use one matrix inverse to solve the following systems of linear equations.

\begin{multicols}{3}
\begin{enumerate}
\setcounter{enumi}{\value{HW}}

\item $\left\{ \begin{array}{rcr}   3x + 7y & = & 26 \\ 5x + 12y & = & 39  \end{array} \right.$ \label{2by2inversefirst}
\item $\left\{ \begin{array}{rcr}   3x + 7y & = &  0 \\ 5x + 12y & = & -1  \end{array} \right.$
\item $\left\{ \begin{array}{rcr}   3x + 7y & = & -7 \\ 5x + 12y & = &  5  \end{array} \right.$ \label{2by2inverselast}


\setcounter{HW}{\value{enumi}}
\end{enumerate}
\end{multicols}


In Exercises \ref{3by3inversefirst} - \ref{3by3inverselast}, use the inverse of $E$ from Exercise \ref{matrixE} above to solve the following systems of linear equations.

\begin{multicols}{3}
\begin{enumerate}
\setcounter{enumi}{\value{HW}}

\item $\left\{ \begin{array}{rcr}   3x + 4z & = & 1 \\ 2x - y + 3z & = & 0 \\ \!-3x + 2y - 5z & = & 0  \end{array} \right.$ \label{3by3inversefirst}
\item $\left\{ \begin{array}{rcr}   3x + 4z & = & 0 \\ 2x - y + 3z & = & 1 \\ \!-3x + 2y - 5z & = & 0  \end{array} \right.$
\item $\left\{ \begin{array}{rcr}   3x + 4z & = & 0 \\ 2x - y + 3z & = & 0 \\ \!-3x + 2y - 5z & = & 1  \end{array} \right.$ \label{3by3inverselast}

\setcounter{HW}{\value{enumi}}
\end{enumerate}
\end{multicols}

\begin{enumerate}
\setcounter{enumi}{\value{HW}}

\item  This exercise is a continuation of Example \ref{rotationmatrixex} in Section \ref{MatArithmetic} and gives another application of matrix inverses.  Recall that given the position matrix $P$ for a point in the plane, the matrix $RP$ corresponds to a point rotated $45^{\circ}$ counterclockwise from $P$ where
 
\[R = \left[ \begin{array}{rr} \frac{\sqrt{2}}{2} & -\frac{\sqrt{2}}{2} \\[3pt] \frac{\sqrt{2}}{2} & \frac{\sqrt{2}}{2} \\ \end{array} \right]\]
 
\begin{enumerate}

\item  Find $R^{-1}$.
\item  If $RP$ rotates a point counterclockwise $45^{\circ}$, what should $R^{-1}P$ do?  Check your answer by finding $R^{-1}P$ for various points on the coordinate axes and the lines $y=\pm x$.
\item  Find $R^{-1}P$ where $P$ corresponds to a generic point $P(x,y)$. Verify that this takes points on the curve $y=\frac{2}{x}$ to points on the curve $x^2-y^2=4$.

\end{enumerate}

\item \label{SasquatchDiet} A Sasquatch's diet consists of three primary foods:  Ippizuti Fish, Misty Mushrooms, and Sun Berries.  Each serving of Ippizuti Fish is 500 calories, contains 40 grams of protein, and has no Vitamin X.  Each serving of Misty Mushrooms is 50 calories, contains 1 gram of protein, and 5 milligrams of Vitamin X.  Finally, each serving of Sun Berries is 80 calories, contains no protein, but has 15 milligrams of Vitamin X.\footnote{Misty Mushrooms and Sun Berries are the only known fictional sources of Vitamin X.}

\begin{enumerate}

\item  If an adult male Sasquatch requires 3200 calories, 130 grams of protein, and 275 milligrams of Vitamin X daily, use a matrix inverse to find how many servings each of Ippizuti Fish, Misty Mushrooms, and Sun Berries he needs to eat each day.

\item  An adult female Sasquatch requires 3100 calories, 120 grams of protein, and 300 milligrams of Vitamin X daily. Use the matrix inverse you found in part (a) to find how many servings each of Ippizuti Fish, Misty Mushrooms, and Sun Berries she needs to eat each day.

\item  An adolescent Sasquatch requires 5000 calories, 400 grams of protein daily, but no Vitamin X daily.\footnote{Vitamin X is needed to sustain Sasquatch longevity only.}   Use the matrix inverse you found in part (a) to find how many servings each of Ippizuti Fish, Misty Mushrooms, and Sun Berries she needs to eat each day.

\end{enumerate}


\item Matrices can be used in cryptography.  Suppose we wish to encode the message `BIGFOOT LIVES'.  We start by assigning a number to each letter of the alphabet, say $A=1$, $B=2$ and so on.  We reserve $0$ to act as a space.  Hence, our message `BIGFOOT LIVES' corresponds to the string of numbers `2, 9, 7, 6, 15, 15, 20, 0, 12, 9, 22, 5, 19.' To encode this message, we use an invertible matrix.  Any invertible matrix will do, but for this exercise, we choose

\[ A = \left[ \begin{array}{rrr} 2 & -3 & 5 \\ 3 & 1 &-2 \\ -7 & 1 & -1 \end{array} \right] \]

Since $A$ is  $3 \times 3$ matrix, we encode our message string into a matrix $M$ with $3$ rows.  To do this, we take the first three numbers, 2 9 7, and make them our first column, the next three numbers, 6 15 15, and make them our second column, and so on.  We put $0$'s to round out the matrix.


\[ M = \left[  \begin{array}{rrrrr} 2 & 6 & 20 & 9 & 19 \\ 9 & 15 & 0 & 22 & 0 \\ 7 & 15 & 12 & 5 & 0 \end{array} \right] \]

To encode the message, we find the product $AM$

\[AM =  \left[ \begin{array}{rrr} 2 & -3 & 5 \\ 3 & 1 &-2 \\ -7 & 1 & -1 \end{array} \right]\left[  \begin{array}{rrrrr} 2 & 6 & 20 & 9 & 19 \\ 9 & 15 & 0 & 22 & 0 \\ 7 & 15 & 12 & 5 & 0 \end{array} \right] = \left[  \begin{array}{rrrrr} 12 & 42 & 100 & -23 & 38 \\ 1 & 3 & 36 & 39 & 57 \\ -12 & -42 & -152 & -46 & -133 \end{array} \right]\]

So our coded message is `12, 1, $-12$, 42, 3, $-42$, 100, 36, $-152$, $-23$, 39, $-46$, 38, 57, $-133$.'  To decode this message, we start with this string of numbers, construct a message matrix as we did earlier (we should get the matrix $AM$ again) and then multiply by $A^{-1}$.

\begin{enumerate}

\item  Find $A^{-1}$.

\item  Use $A^{-1}$ to decode the message and check this method actually works.

\item  Decode the message `14, 37, $-76$, 128, 21, $-151$, 31, 65, $-140$'

\item  Choose another invertible matrix and encode and decode your own messages.

\end{enumerate}

\item Using the matrices $A$ from Exercise \ref{findmatinversefirst}, $B$ from Exercise \ref{matrixB} and $D$ from Exercise \ref{matrixD}, show $AB = D$ and  $D^{-1} = B^{-1}A^{-1}$.  That is, show that $(AB)^{-1} = B^{-1}A^{-1}$. 

\item Let $M$ and $N$ be invertible $n \times n$ matrices.  Show that $(MN)^{-1} = N^{-1}M^{-1}$ and compare your work to Exercise \ref{fcircginverse} in Section \ref{InverseFunctions}.


\end{enumerate}

\newpage

\subsection{Answers}

\begin{multicols}{2} 
\begin{enumerate}


\item $A^{-1} = \left[ \begin{array}{rr} -2 & 1 \\[3pt] \frac{3}{2} & -\frac{1}{2} \end{array} \right]$
\item $B^{-1} = \left[ \begin{array}{rr} 3 & 7 \\ 5 & 12 \end{array} \right]$

\setcounter{HW}{\value{enumi}}
\end{enumerate}
\end{multicols}

\begin{multicols}{2} 
\begin{enumerate}
\setcounter{enumi}{\value{HW}}

\item $C \vphantom{\left[ \begin{array}{rr} \frac{9}{2} & -\frac{1}{2} \\ 8 & -1 \end{array} \right]}$ is not invertible
\item $D^{-1} = \left[ \begin{array}{rr} \frac{9}{2} & -\frac{1}{2} \\ 8 & -1 \end{array} \right]$

\setcounter{HW}{\value{enumi}}
\end{enumerate}
\end{multicols}

\begin{multicols}{2} 
\begin{enumerate}
\setcounter{enumi}{\value{HW}}

\item $E^{-1} = \left[ \begin{array}{rrr} -1 & 8 & 4 \\ 1 & -3 & -1 \\ 1 & -6 & -3 \end{array} \right] \vphantom{\left[ \begin{array}{rrr} -\frac{5}{2} & \frac{7}{2} & \frac{1}{2} \\[3pt] \frac{7}{4} & -\frac{9}{4} & -\frac{1}{4} \\[3pt] -\frac{1}{6} & \frac{1}{6} & \frac{1}{6} \end{array} \right]}$
\item $F^{-1} = \left[ \begin{array}{rrr} -\frac{5}{2} & \frac{7}{2} & \frac{1}{2} \\[3pt] \frac{7}{4} & -\frac{9}{4} & -\frac{1}{4} \\[3pt] -\frac{1}{6} & \frac{1}{6} & \frac{1}{6} \end{array} \right]$

\setcounter{HW}{\value{enumi}}
\end{enumerate}
\end{multicols}

\begin{multicols}{2} 
\begin{enumerate}
\setcounter{enumi}{\value{HW}}

\item $G \vphantom{\left[ \begin{array}{rrrr} 16 & 0 & 3 & 0 \\[3pt] -90 & -\frac{1}{2} & -\frac{35}{2} & \frac{7}{2} \\[3pt] 5 & 0 & 1 & 0 \\[3pt] -36 & 0 & -7 & \hphantom{-}1 \end{array} \right]}$ is not invertible
\item $H^{-1} = \left[ \begin{array}{rrrr} 16 & 0 & 3 & 0 \\[3pt] -90 & -\frac{1}{2} & -\frac{35}{2} & \frac{7}{2} \\[3pt] 5 & 0 & 1 & 0 \\[3pt] -36 & 0 & -7 & \hphantom{-}1 \end{array} \right]$

\setcounter{HW}{\value{enumi}}
\end{enumerate}
\end{multicols}

The coefficient matrix is $B^{-1}$ from Exercise \ref{matrixB} above so the inverse we need is $(B^{-1})^{-1} = B$. 

\begin{enumerate}
\setcounter{enumi}{\value{HW}}

\item $\left[ \begin{array}{rr} 12 & -7 \\ -5 & 3 \end{array} \right] \left[ \begin{array}{r} 26 \\ 39 \end{array} \right] = \left[ \begin{array}{r} 39 \\ -13 \end{array} \right] \;$ So $x = 39$ and $y = -13$.
\item $\left[ \begin{array}{rr} 12 & -7 \\ -5 & 3 \end{array} \right] \left[ \begin{array}{r} 0 \\ -1 \end{array} \right] = \left[ \begin{array}{r} 7 \\ -3 \end{array} \right] \;$ So $x = 7$ and $y = -3$.
\item $\left[ \begin{array}{rr} 12 & -7 \\ -5 & 3 \end{array} \right] \left[ \begin{array}{r} -7 \\ 5 \end{array} \right] = \left[ \begin{array}{r} -119 \\ 50 \end{array} \right] \;$ So $x = -119$ and $y = 50$.

\setcounter{HW}{\value{enumi}}
\end{enumerate}

The coefficient matrix is $E = \left[ \begin{array}{rrr} 3 & 0 & 4 \\ 2 & -1 & 3 \\ -3 & 2 & -5 \end{array} \right]$ from Exercise \ref{matrixE}, so $E^{-1} = \left[ \begin{array}{rrr} -1 & 8 & 4 \\ 1 & -3 & -1 \\ 1 & -6 & -3 \end{array} \right]$ 

\begin{enumerate}
\setcounter{enumi}{\value{HW}}

\item $\left[ \begin{array}{rrr} -1 & 8 & 4 \\ 1 & -3 & -1 \\ 1 & -6 & -3 \end{array} \right] \left[ \begin{array}{r} 1 \\ 0 \\ 0 \end{array} \right] = \left[ \begin{array}{r} -1 \\ 1 \\ 1 \end{array} \right] \;$ So $x = -1$, $y = 1$ and $z = 1$.
\item $\left[ \begin{array}{rrr} -1 & 8 & 4 \\ 1 & -3 & -1 \\ 1 & -6 & -3 \end{array} \right] \left[ \begin{array}{r} 0 \\ 1 \\ 0 \end{array} \right] = \left[ \begin{array}{r} 8 \\ -3 \\ -6 \end{array} \right] \;$ So $x = 8$, $y = -3$ and $z = -6$.
\item $\left[ \begin{array}{rrr} -1 & 8 & 4 \\ 1 & -3 & -1 \\ 1 & -6 & -3 \end{array} \right] \left[ \begin{array}{r} 0 \\ 0 \\ 1 \end{array} \right] = \left[ \begin{array}{r} 4 \\ -1 \\ -3 \end{array} \right] \;$ So $x = 4$, $y = -1$ and $z = -3$.

\setcounter{HW}{\value{enumi}}
\end{enumerate}

\begin{enumerate}
\setcounter{enumi}{\value{HW}}

\addtocounter{enumi}{1}

\item  \begin{enumerate} \item  The adult male Sasquatch needs:  3 servings of Ippizuti Fish, 10 servings of Misty Mushrooms, and 15 servings of Sun Berries daily.

\item  The adult female Sasquatch needs:  3 servings of Ippizuti Fish and 20 servings of Sun Berries daily.  (No Misty Mushrooms are needed!)

\item  The adolescent Sasquatch requires 10 servings of Ippizuti Fish daily.  (No Misty Mushrooms or Sun Berries are needed!)

\end{enumerate}

\item  \begin{enumerate}

\item  $A^{-1} = \left[ \begin{array}{rrr} 1 & 2 & 1 \\ 17 & 33 & 19 \\ 10 & 19 & 11 \end{array} \right] $

\item  $ \left[ \begin{array}{rrr} 1 & 2 & 1 \\ 17 & 33 & 19 \\ 10 & 19 & 11 \end{array} \right] \left[  \begin{array}{rrrrr} 12 & 42 & 100 & -23 & 38 \\ 1 & 3 & 36 & 39 & 57 \\ -12 & -42 & -152 & -46 & -133 \end{array} \right] =  \left[  \begin{array}{rrrrr} 2 & 6 & 20 & 9 & 19 \\ 9 & 15 & 0 & 22 & 0 \\ 7 & 15 & 12 & 5 & 0 \end{array} \right] \quad \checkmark$

\item  `LOGS RULE'

\end{enumerate}

\end{enumerate}

\end{document}
