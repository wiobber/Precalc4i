\documentclass{ximera}

\begin{document}
	\author{Stitz-Zeager}
	\xmtitle{Exercises for Mat Arithmetic}{}

\mfpicnumber{1} \opengraphsfile{ExercisesforMatArithmetic} % mfpic settings added 


\label{ExercisesforMatArithmetic}

For each pair of matrices $A$ and $B$ in  Exercises \ref{easymatarithfirst} - \ref{easymatarithlast}, find the following, if defined

\begin{multicols}{3}
\begin{itemize}
\item  $3A$

\item $-B$

\item $A^2$

\end{itemize}
\end{multicols}


\begin{multicols}{3}
\begin{itemize}
\item  $A-2B$

\item $AB$

\item $BA$

\end{itemize}
\end{multicols}

\begin{multicols}{2} 
\begin{enumerate}

\item  $A = \left[ \begin{array}{rr} 2 & -3 \\ 1 & 4 \end{array} \right]$, $B=\left[ \begin{array}{rr} 5 & -2 \\ 4 & 8 \end{array} \right]$ \label{easymatarithfirst}

\item  $A = \left[ \begin{array}{rr} -1 & 5 \\ -3 & 6 \end{array} \right]$, $B=\left[ \begin{array}{rr} 2 & 10 \\ -7 & 1 \end{array} \right]$

\setcounter{HW}{\value{enumi}}
\end{enumerate}
\end{multicols}

\begin{multicols}{2} 
\begin{enumerate}
\setcounter{enumi}{\value{HW}}

\item  $A = \left[ \begin{array}{rr} -1 & 3 \\ 5 & 2 \end{array} \right]$, $B=\left[ \begin{array}{rrr} 7 & 0 & 8 \\ -3 & 1 & 4 \end{array} \right]$

\item  $A = \left[ \begin{array}{rr} 2 & 4 \\ 6 & 8 \end{array} \right]$, $B=\left[ \begin{array}{rrr} -1 & 3 & -5 \\ 7 & -9 & 11 \end{array} \right]$

\setcounter{HW}{\value{enumi}}
\end{enumerate}
\end{multicols}

\begin{multicols}{2} 
\begin{enumerate}
\setcounter{enumi}{\value{HW}}



\item  $A = \left[ \begin{array}{r} 7 \\ 8 \\ 9 \end{array} \right]$, $B=\left[ \begin{array}{rrr} 1 & 2 & 3 \end{array} \right]$

\item  $A = \left[ \begin{array}{rr} 1 & -2 \\ -3 & 4 \\ 5 & -6 \end{array} \right]$, $B=\left[ \begin{array}{rrr} -5 & 1 & 8 \end{array} \right]$


\setcounter{HW}{\value{enumi}}
\end{enumerate}
\end{multicols}

\begin{enumerate}
\setcounter{enumi}{\value{HW}}

\item  $ A = \left[ \begin{array}{rrr} 2 & -3 & 5 \\ 3 & 1 &-2 \\ -7 & 1 & -1 \end{array} \right]$, $B= \left[ \begin{array}{rrr} 1 & 2 & 1 \\ 17 & 33 & 19 \\ 10 & 19 & 11 \end{array} \right]$ \label{easymatarithlast}

\setcounter{HW}{\value{enumi}}
\end{enumerate}


In Exercises \ref{matarithfirst} - \ref{matarithlast}, use the matrices \[A = \left[ \begin{array}{rr} 1 & 2 \\ 3 & 4 \end{array} \right] \;\;\; B = \left[ \begin{array}{rr} 0 & -3 \\ -5 & 2 \end{array} \right] \;\;\; C = \left[ \begin{array}{rrr} 10 & -\frac{11}{2} & 0 \\ \frac{3}{5} & 5 & 9 \end{array} \right]\] \[ D = \left[ \begin{array}{rr} 7 & -13 \\ -\frac{4}{3} & 0 \\ 6 & 8 \end{array} \right] \;\;\; E = \left[ \begin{array}{rrr} 1 & \hphantom{-}2 & 3 \\ 0 & 4 & -9 \\ 0 & 0 & -5 \end{array} \right] \] to compute the following or state that the indicated operation is undefined.

\begin{multicols}{3} 
\begin{enumerate}
\setcounter{enumi}{\value{HW}}

\item $7B - 4A$  \label{matarithfirst}
\item $AB$
\item $BA$

\setcounter{HW}{\value{enumi}}
\end{enumerate}
\end{multicols}

\begin{multicols}{3} 
\begin{enumerate}
\setcounter{enumi}{\value{HW}}

\item $E + D$
\item $ED$
\item $CD + 2I_{2}A$

\setcounter{HW}{\value{enumi}}
\end{enumerate}
\end{multicols}

\begin{multicols}{3} 
\begin{enumerate}
\setcounter{enumi}{\value{HW}}

\item  $A - 4I_{2}$

\item  $A^2 - B^2$

\item  $(A+B)(A-B)$

\setcounter{HW}{\value{enumi}}
\end{enumerate}
\end{multicols}

\begin{multicols}{3} 
\begin{enumerate}
\setcounter{enumi}{\value{HW}}

\item  $A^2-5A-2I_{2}$

\item  $E^2 + 5E-36I_{3}$

\item $EDC$

\setcounter{HW}{\value{enumi}}
\end{enumerate}
\end{multicols}

\begin{multicols}{3} 
\begin{enumerate}
\setcounter{enumi}{\value{HW}}

\item $CDE$
\item $ABCEDI_{2}$ \label{matarithlast}


\setcounter{HW}{\value{enumi}}
\end{enumerate}
\end{multicols}

\begin{enumerate}
\setcounter{enumi}{\value{HW}}

\item Let $A = \left[ \begin{array}{rrr} a & b & c \\ d & e & f \end{array} \right] \;\;\; E_{\mbox{\tiny$1$}} = \left[ \begin{array}{rr} 0 & 1 \\ 1 & 0 \end{array} \right] \;\;\; E_{\mbox{\tiny$2$}} = \left[ \begin{array}{rr} 5 & 0 \\ 0 & 1 \end{array} \right] \;\;\; E_{\mbox{\tiny$3$}} = \left[ \begin{array}{rr} 1 & -2 \\ 0 & 1 \end{array} \right]$ 

\smallskip

 Compute $E_{\mbox{\tiny$1$}}A$, $\; E_{\mbox{\tiny$2$}}A$ and $E_{\mbox{\tiny$3$}}A$.  What effect did each of the $E_{i}$ matrices have on the rows of $A$?  Create $E_{\mbox{\tiny$4$}}$ so that its effect on $A$ is to multiply the bottom row by $-6$.  How would you extend this idea to matrices with more than two rows?
\setcounter{HW}{\value{enumi}}
\end{enumerate}

\phantomsection
\label{Markovchain} 

In Exercises \ref{MCfirst} - \ref{MClast}, consider the following scenario. In the small village of Pedimaxus in the country of Sasquatchia, all 150 residents get one of the two local newspapers.  Market research has shown that in any given week, 90\% of those who subscribe to the Pedimaxus Tribune want to keep getting it, but 10\% want to switch to the Sasquatchia Picayune.  Of those who receive the Picayune, 80\% want to continue with it and 20\% want switch to the Tribune.  We can express this situation using matrices.  Specifically, let $X$ be the `state matrix' given by \[X = \left[ \begin{array}{r} T \\ P \end{array} \right]\] where $T$ is the number of people who get the Tribune and $P$ is the number of people who get the Picayune in a given week.  Let $Q$ be the `transition matrix' given by \[Q = \left[ \begin{array}{rr} 0.90 & 0.20 \\ 0.10 & 0.80 \end{array} \right]\] such that $QX$ will be the state matrix for the next week. 

\begin{enumerate}
\setcounter{enumi}{\value{HW}}

\item \label{MCfirst} Let's assume that when Pedimaxus was founded, all 150 residents got the Tribune.  (Let's call this Week 0.) This would mean \[X = \left[ \begin{array}{r} 150 \\ 0 \end{array} \right]\] Since 10\% of that 150 want to switch to the Picayune, we should have that for Week 1, 135 people get the Tribune and 15 people get the Picayune.  Show that $QX$ in this situation is indeed \[QX = \left[ \begin{array}{r} 135 \\ 15 \end{array} \right]\]

\item Assuming that the percentages stay the same, we can get to the subscription numbers for Week 2 by computing $Q^{2}X$. How many people get each paper in Week 2?

\item Explain why the transition matrix does what we want it to do.

\item If the conditions do not change from week to week, then $Q$ remains the same and we have what's known as a \index{stochastic process} \index{Markov Chain} {\bf Stochastic Process}\footnote{More specifically, we have a Markov Chain, which is a special type of stochastic process.} because Week $n$'s numbers are found by computing $Q^{n}X$.  Choose a few values of $n$ and, with the help of your classmates and calculator, find out how many people get each paper for that week.  You should start to see a pattern as $n \rightarrow \infty$.

\item If you didn't see the pattern, we'll help you out.  Let \[X_{s} = \left[ \begin{array}{r} 100 \\ 50 \end{array} \right].\]  Show that $QX_{s} = X_{s}$  This is called the {\bf steady state} \index{steady state} because the number of people who get each paper didn't change for the next week.  Show that $Q^{n}X \rightarrow X_{s}$ as $n \rightarrow \infty$. 

\item Now let \[S = \left[ \begin{array}{rr} \frac{2}{3} & \frac{2}{3} \\ [3pt] \frac{1}{3} & \frac{1}{3} \end{array} \right]\]  Show that $Q^{n} \rightarrow S$ as $n \rightarrow \infty$.  

\item \label{MClast} Show that $SY = X_{s}$ for any matrix $Y$ of the form \[Y = \left[ \begin{array}{r} y \\ 150 - y \end{array} \right]\] This means that no matter how the distribution starts in Pedimaxus, if $Q$ is applied often enough, we always end up with 100 people getting the Tribune and 50 people getting the Picayune.

\setcounter{HW}{\value{enumi}}
\end{enumerate}

\begin{enumerate}
\setcounter{enumi}{\value{HW}}

\item Let $z = a + bi$ and $w = c + di$ be arbitrary complex numbers.  Associate $z$ and $w$ with the matrices \[Z = \left[ \begin{array}{rr} a & b \\ -b & a \end{array} \right] \;\; \mbox{and} \;\; W = \left[ \begin{array}{rr} c & d \\ -d & c \end{array} \right]\]  Show that complex number addition, subtraction and multiplication are mirrored by the associated \emph{matrix} arithmetic.  That is, show that $Z + W$, $Z - W$ and $ZW$ produce matrices which can be associated with the complex numbers $z + w$, $z - w$ and $zw$, respectively.

\setcounter{HW}{\value{enumi}}
\end{enumerate}

\begin{enumerate}
\setcounter{enumi}{\value{HW}}

\item Let \[A = \left[ \begin{array}{rr} 1 & 2 \\ 3 & 4 \end{array} \right] \; \mbox{and} \; B = \left[ \begin{array}{rr} 0 & -3 \\ -5 & 2 \end{array} \right]\]  Compare $(A + B)^{2}$ to $A^{2} + 2AB + B^{2}$.  Discuss with your classmates what constraints must be placed on two arbitrary matrices $A$ and $B$ so that both $(A + B)^{2}$ and $A^{2} + 2AB + B^{2}$ exist.  When will $(A + B)^{2} = A^{2} + 2AB + B^{2}$?  In general, what is the correct formula for $(A + B)^{2}$?
\setcounter{HW}{\value{enumi}}
\end{enumerate}

\phantomsection
\label{triangularmatrices}

In Exercises \ref{triangexfirst} - \ref{triangexlast}, consider the following definitions. A square matrix is said to be an \index{upper triangular matrix}\index{matrix ! upper triangular}{\bf upper triangular matrix} if all of its entries below the main diagonal are zero and it is said to be a {\bf lower triangular matrix}\index{lower triangular matrix}\index{matrix ! lower triangular} if all of its entries above the main diagonal are zero. For example, \[E = \left[ \begin{array}{rrr} 1 & \hphantom{-}2 & 3 \\ 0 & 4 & -9 \\ 0 & 0 & -5 \end{array} \right]\] from Exercises \ref{matarithfirst} - \ref{matarithlast} above is an upper triangular matrix whereas \[F = \left[ \begin{array}{rr} 1 & 0 \\ 3 & 0 \end{array} \right]\] is a lower triangular matrix.  (Zeros are allowed on the main diagonal.)  Discuss the following questions with your classmates.

\begin{enumerate}
\setcounter{enumi}{\value{HW}}

\item Give an example of a matrix which is neither upper triangular nor lower triangular. \label{triangexfirst} 
\item Is the product of two $n \times n$ upper triangular matrices always upper triangular?
\item Is the product of two $n \times n$ lower triangular matrices always lower triangular?
\item Given the matrix \[A = \left[ \begin{array}{rr} 1 & 2 \\ 3 & 4 \end{array} \right]\] write $A$ as $LU$ where $L$ is a lower triangular matrix and $U$ is an upper triangular matrix?
\item Are there any matrices which are simultaneously upper and lower triangular? \label{triangexlast}

\setcounter{HW}{\value{enumi}}
\end{enumerate}


\newpage

\subsection{Answers}

\begin{enumerate}

\item For  $A = \left[ \begin{array}{rr} 2 & -3 \\ 1 & 4 \end{array} \right]$ and $B=\left[ \begin{array}{rr} 5 & -2 \\ 4 & 8 \end{array} \right]$ 

\begin{multicols}{3}
\begin{itemize}
\item  $3A = \left[ \begin{array}{rr} 6 & -9 \\ 3 & 12 \end{array} \right]$

\item $-B = \left[ \begin{array}{rr} -5 & 2 \\ -4 & -8 \end{array} \right]$

\item $A^2 = \left[ \begin{array}{rr} 1 & -18 \\ 6 & 13 \end{array} \right]$

\end{itemize}
\end{multicols}


\begin{multicols}{3}
\begin{itemize}
\item  $A-2B = \left[ \begin{array}{rr} -8 & 1 \\ -7 & -12 \end{array} \right]$

\item $AB = \left[ \begin{array}{rr} -2 & -28 \\ 21 & 30 \end{array} \right]$

\item $BA = \left[ \begin{array}{rr} 8 & -23 \\ 16 & 20 \end{array} \right]$

\end{itemize}
\end{multicols}



\item For  $A = \left[ \begin{array}{rr} -1 & 5 \\ -3 & 6 \end{array} \right]$ and $B=\left[ \begin{array}{rr} 2 & 10 \\ -7 & 1 \end{array} \right]$

\begin{multicols}{3}
\begin{itemize}
\item  $3A = \left[ \begin{array}{rr} -3 & 15 \\ -9 & 18 \end{array} \right]$

\item $-B = \left[ \begin{array}{rr} -2 & -10 \\ 7 & -1 \end{array} \right]$

\item $A^2 = \left[ \begin{array}{rr} -14 & 25 \\ -15 & 21 \end{array} \right]$

\end{itemize}
\end{multicols}


\begin{multicols}{3}
\begin{itemize}
\item  $A-2B = \left[ \begin{array}{rr} -5 & -15 \\ 11 & 4 \end{array} \right]$

\item $AB = \left[ \begin{array}{rr} -37 & -5 \\ -48 & -24 \end{array} \right]$

\item $BA = \left[ \begin{array}{rr} -32 & 70 \\ 4 & -29 \end{array} \right]$

\end{itemize}
\end{multicols}

\item For  $A = \left[ \begin{array}{rr} -1 & 3 \\ 5 & 2 \end{array} \right]$ and
 $B=\left[ \begin{array}{rrr} 7 & 0 & 8 \\ -3 & 1 & 4 \end{array} \right]$
 
\begin{multicols}{3}
\begin{itemize}
\item  $3A = \left[ \begin{array}{rr} -3 & 9 \\ 15 & 6\end{array} \right]$

\item $-B = \left[ \begin{array}{rrr} -7 & 0 & -8 \\ 3 & -1 & -4 \end{array} \right]$

\item $A^2 = \left[ \begin{array}{rr} 16 & 3 \\ 5 & 19 \end{array} \right]$

\end{itemize}
\end{multicols}


\begin{multicols}{3}
\begin{itemize}
\item  $A-2B$ is not defined

\item $AB = \left[ \begin{array}{rrr} -16 & 3 & 4 \\ 29 & 2 & 48 \end{array} \right]$

\item $BA$ is not defined

\end{itemize}
\end{multicols}

\item For  $A = \left[ \begin{array}{rr} 2 & 4 \\ 6 & 8 \end{array} \right]$ and $B=\left[ \begin{array}{rrr} -1 & 3 & -5 \\ 7 & -9 & 11 \end{array} \right]$

\begin{multicols}{3}
\begin{itemize}
\item  $3A = \left[ \begin{array}{rr} 6 & 12 \\ 18 & 24 \end{array} \right]$

\item $-B = \left[ \begin{array}{rrr} 1 & -3 & 5 \\ -7 & 9 & -11 \end{array} \right]$

\item $A^2 = \left[ \begin{array}{rr} 28 & 40 \\ 60 & 88 \end{array} \right]$

\end{itemize}
\end{multicols}


\begin{multicols}{3}
\begin{itemize}
\item  $A-2B$ is not defined

\item $AB = \left[ \begin{array}{rrr} 26 & -30 & 34 \\ 50 & -54 & 58 \end{array} \right]$

\item $BA$ is not defined

\end{itemize}
\end{multicols}

\pagebreak

\item For $A = \left[ \begin{array}{r} 7 \\ 8 \\ 9 \end{array} \right]$ and $B=\left[ \begin{array}{rrr} 1 & 2 & 3 \end{array} \right]$

\begin{multicols}{2}
\begin{itemize}
\item  $3A = \left[ \begin{array}{r} 21 \\ 24 \\ 27 \end{array} \right]$

\item $-B = \left[ \begin{array}{rrr} -1 & -2 & -3 \end{array} \right] \vphantom{\left[ \begin{array}{r} 21 \\ 24 \\ 27 \end{array} \right]}$

\end{itemize}
\end{multicols}


\begin{multicols}{2}
\begin{itemize}

\item $A^2$ is not defined

\item  $A-2B$ is not defined

\end{itemize}
\end{multicols}

\begin{multicols}{2}
\begin{itemize}

\item $AB = \left[ \begin{array}{rrr} 7 & 14 & 21 \\ 8 & 16 & 24 \\ 9 & 18 & 27 \end{array} \right]$

\item $BA = [50] \vphantom{\left[ \begin{array}{rrr} 7 & 14 & 21 \\ 8 & 16 & 24 \\ 9 & 18 & 27 \end{array} \right]}$

\end{itemize}
\end{multicols}




\item For $A = \left[ \begin{array}{rr} 1 & -2 \\ -3 & 4 \\ 5 & -6 \end{array} \right]$ and $B=\left[ \begin{array}{rrr} -5 & 1 & 8 \end{array} \right]$

\begin{multicols}{2}
\begin{itemize}
\item  $3A = \left[ \begin{array}{rr} 3 & -6 \\ -9 & 12 \\ 15 & -18 \end{array} \right]$

\item $-B = \left[ \begin{array}{rrr} 5 & -1 & -8 \end{array} \right] \vphantom{\left[ \begin{array}{rr} 3 & -6 \\ -9 & 12 \\ 15 & -18 \end{array} \right]}$

\end{itemize}
\end{multicols}


\begin{multicols}{2}
\begin{itemize}

\item $A^2$ is not defined

\item  $A-2B$ is not defined

\end{itemize}
\end{multicols}

\begin{multicols}{2}
\begin{itemize}

\item $AB$ is not defined

\item $BA = \left[ \begin{array}{rr} 32 & -34 \end{array} \right]$

\end{itemize}
\end{multicols}

\item For  $ A = \left[ \begin{array}{rrr} 2 & -3 & 5 \\ 3 & 1 &-2 \\ -7 & 1 & -1 \end{array} \right]$ and $B= \left[ \begin{array}{rrr} 1 & 2 & 1 \\ 17 & 33 & 19 \\ 10 & 19 & 11 \end{array} \right]$ 

\begin{multicols}{2}
\begin{itemize}
\item  $3A = \left[ \begin{array}{rrr} 6 & -9 & 15 \\ 9 & 3 &-6 \\ -21 & 3 & -3 \end{array} \right]$

\item $-B = \left[ \begin{array}{rrr} -1 & -2 & -1 \\ -17 & -33 & -19 \\ -10 & -19 & -11 \end{array} \right]$

\end{itemize}
\end{multicols}


\begin{multicols}{2}
\begin{itemize}

\item  $A^2 = \left[ \begin{array}{rrr} -40 & -4 & 11 \\ 23 & -10 & 15 \\ -4 & 21 & -36 \end{array} \right]$

\item $A-2B = \left[ \begin{array}{rrr} 0 & -7 & 3 \\ -31 & -65 & -40 \\ -27 & -37 & -23 \end{array} \right]$


\end{itemize}
\end{multicols}

\begin{multicols}{2}
\begin{itemize}

\item $AB =  \left[ \begin{array}{rrr} 1 & 0 & 0 \\ 0 & 1 & 0 \\ 0 & 0 & 1 \end{array} \right]$

\item $BA = \left[ \begin{array}{rrr} 1 & 0 & 0 \\ 0 & 1 & 0 \\ 0 & 0 & 1 \end{array} \right]$


\end{itemize}
\end{multicols}

	
\setcounter{HW}{\value{enumi}}
\end{enumerate}

\begin{multicols}{2} 
\begin{enumerate}
\setcounter{enumi}{\value{HW}}

\item $7B - 4A = \left[ \begin{array}{rr} -4 & -29 \\ -47 & -2 \end{array} \right]$
\item $AB = \left[ \begin{array}{rr} -10 & 1 \\ -20 & -1 \end{array} \right]$


\setcounter{HW}{\value{enumi}}
\end{enumerate}
\end{multicols}

\begin{multicols}{2} 
\begin{enumerate}
\setcounter{enumi}{\value{HW}}

\item $BA = \left[ \begin{array}{rr} -9 & -12 \\ 1 & -2 \end{array} \right]$
\item $E + D \vphantom{\left[ \begin{array}{rr} -9 & -12 \\ 1 & -2 \end{array} \right]}$ is undefined


\setcounter{HW}{\value{enumi}}
\end{enumerate}
\end{multicols}

\begin{multicols}{2} 
\begin{enumerate}
\setcounter{enumi}{\value{HW}}

\item $ED = \left[ \begin{array}{rr} \frac{67}{3}& 11 \\[3pt] -\frac{178}{3} & -72 \\ -30 & -40 \end{array} \right]$
\item $CD + 2I_{2}A = \left[ \begin{array}{rr} \frac{238}{3} & -126 \\[3pt] \frac{863}{15} & \frac{361}{5} \end{array} \right] \vphantom{\left[ \begin{array}{rr} \frac{67}{3}& 11 \\[3pt] -\frac{178}{3} & -72 \\ -30 & -40 \end{array} \right]}$


\setcounter{HW}{\value{enumi}}
\end{enumerate}
\end{multicols}

\begin{multicols}{2} 
\begin{enumerate}
\setcounter{enumi}{\value{HW}}

\item  $A - 4I_{2} = \left[ \begin{array}{rr} -3 & 2 \\ 3 & 0 \end{array} \right]$

\item  $A^2 - B^2 = \left[ \begin{array}{rr} -8 & 16 \\ 25 & 3 \end{array} \right]$


\setcounter{HW}{\value{enumi}}
\end{enumerate}
\end{multicols}

\begin{multicols}{2} 
\begin{enumerate}
\setcounter{enumi}{\value{HW}}

\item  $(A+B)(A-B) = \left[ \begin{array}{rr} -7 & 3 \\ 46 & 2 \end{array} \right]$

\item  $A^2-5A-2I_{2} = \left[ \begin{array}{rr} 0 & 0 \\ 0 & 0 \end{array} \right]$


\setcounter{HW}{\value{enumi}}
\end{enumerate}
\end{multicols}

\begin{multicols}{2} 
\begin{enumerate}
\setcounter{enumi}{\value{HW}}

\item  $E^2 + 5E-36I_{3} = \left[ \begin{array}{rrr} -30 & 20 & -15 \\ 0 & 0 & -36 \\ 0 & 0 & -36 \end{array} \right] \vphantom{\left[ \begin{array}{rrr} \frac{3449}{15} & -\frac{407}{6} & 99 \\[3pt] -\frac{9548}{15} & -\frac{101}{3} & -648 \\ -324 & -35 & -360 \end{array} \right]}$

\item $EDC = \left[ \begin{array}{rrr} \frac{3449}{15} & -\frac{407}{6} & 99 \\[3pt] -\frac{9548}{15} & -\frac{101}{3} & -648 \\ -324 & -35 & -360 \end{array} \right]$


\setcounter{HW}{\value{enumi}}
\end{enumerate}
\end{multicols}

\begin{multicols}{2} 
\begin{enumerate}
\setcounter{enumi}{\value{HW}}

\item $CDE \vphantom{\left[ \begin{array}{rr} -\frac{90749}{15} & -\frac{28867}{5} \\[3pt] -\frac{156601}{15} & -\frac{47033}{5} \end{array} \right]}$ is undefined

\item $ABCEDI_{2} = \left[ \begin{array}{rr} -\frac{90749}{15} & -\frac{28867}{5} \\[3pt] -\frac{156601}{15} & -\frac{47033}{5} \end{array} \right]$


\setcounter{HW}{\value{enumi}}
\end{enumerate}
\end{multicols}

\begin{enumerate}
\setcounter{enumi}{\value{HW}}

\item $E_{\mbox{\tiny$1$}}A = \left[ \begin{array}{rrr} d & e & f \\ a & b & c\end{array} \right]\;\;$ $E_{\mbox{\tiny$1$}}$ interchanged $R1$ and $R2$ of $A$.\\
$E_{\mbox{\tiny$2$}}A = \left[ \begin{array}{rrr} 5a & 5b & 5c \\ d & e & f \end{array} \right]\;\;$ $E_{\mbox{\tiny$2$}}$ multiplied $R1$ of $A$ by 5.\\
$E_{\mbox{\tiny$3$}}A = \left[ \begin{array}{rrr} a - 2d & b - 2e & c - 2f \\ d & e & f \end{array} \right]\;\;$ $E_{\mbox{\tiny$3$}}$ replaced $R1$ in $A$ with $R1 - 2R2$.\\
$E_{\mbox{\tiny$4$}} = \left[ \begin{array}{rr} 1 & 0 \\ 0 & -6 \end{array} \right]\;\;$
\end{enumerate}





\end{document}
