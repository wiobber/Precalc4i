\documentclass{ximera}

\begin{document}
	\author{Stitz-Zeager}
	\xmtitle{Exercises for Parabolas}{}

\mfpicnumber{1} \opengraphsfile{ExercisesforParabolas} % mfpic settings added 


\label{ExercisesforParabolas}

\begin{question}
In Exercises \ref{parabolasketchfirst} - \ref{parabolasketchlast},  graph of the given equations in the $xy$-plane.  Find the vertex, focus and directrix.  Include the endpoints of the latus rectum in your sketch.

\begin{problem}\label{parabolasketchfirst}
    $(x - 3)^{2} = -16y$
\end{problem}

\begin{problem}
    $\left(x + \frac{7}{3}\right)^{2} = 2\left(y + \frac{5}{2}\right)$
\end{problem}

\begin{problem}\label{paranotfcnone} 
    $(y - 2)^{2} = -12(x + 3)$ 
\end{problem}

\begin{problem}\label{paranotfcntwo} 
    $(y + 4)^{2} = 4x$
\end{problem}

\begin{problem}
     $(x-1)^2 = 4(y+3)$
\end{problem}

\begin{problem}
    $(x+2)^2 = -20(y-5)$
\end{problem}

\begin{problem}\label{paranotfcnthree} 
    $(y-4)^2 = 18(x-2)$
\end{problem}

\begin{problem}\label{paranotfcnfour} \label{parabolasketchlast}
    $\left(y+ \frac{3}{2}\right)^2 = -7 \left(x+ \frac{9}{2}\right)$
\end{problem}

\end{question}


\begin{question}
    
In Exercises \ref{stdfrmparabolafirst} - \ref{stdfrmparabolalast}, put the equation into standard form.  Find the vertex, focus and directrix.\footnote{\ldots assuming the equation were graphed in the $xy$-plane.}

\begin{problem}\label{paranotfcnfive}\label{stdfrmparabolafirst}
    $y^{2} - 10y - 27x + 133 = 0$ 
\end{problem}

\begin{problem}
    $25x^{2} + 20x + 5y - 1 = 0$
\end{problem}

\begin{problem}
    $x^2 + 2x - 8y + 49 = 0$
\end{problem}

\begin{problem}\label{paranotfcnsix} 
    $2y^2 + 4y +x - 8 = 0$
\end{problem}

\begin{problem}
    $x^2-10x+12y+1=0$
\end{problem}

\begin{problem}\label{stdfrmparabolalast}\label{paranotfcnseven}
     $3y^2-27y+4x+\frac{211}{4} = 0$
\end{problem}

\end{question}



\begin{problem}
For each of the equations given in Exercises \ref{parabolasketchfirst} - \ref{stdfrmparabolalast} that do \textbf{not} describe $y$ as a function of $x$, find two or more explicit functions of $x$ represented by each of the equations.  (See Example \ref{horizontalparabolaex}.)
\end{problem}

\begin{question}

 In Exercises \ref{buildparafromgraphfirst} - \ref{buildparafromgraphlast}, find an equation for the parabola whose graph is given.


\begin{multicols}{2}
\begin{enumerate}
\setcounter{enumi}{\value{HW}}

\item $~$ \label{buildparafromgraphfirst}

\begin{mfpic}[13]{-5}{5}{-5}{5}
\axes
\tlabel[cc](5,-0.5){\scriptsize $x$}
\tlabel[cc](0.5,5){\scriptsize $y$}
\tlabel[cc](1, 1){\scriptsize $(0,2)$}
\tlabel[cc](-2.25,-3.75){\scriptsize $(-2,-6)$}
\xmarks{-4,-3,-2,-1,1,2,3,4}
\ymarks{-4,-3,-2, -1, 1,2,3,4}
\tlpointsep{4pt}
\scriptsize
\axislabels {x}{ {$-4 \hspace{7pt}$} -4, {$-3 \hspace{7pt}$} -3, {$-2 \hspace{7pt}$} -2, {$-1 \hspace{7pt}$} -1, {$1$} 1, {$2$} 2, {$3$} 3, {$4$} 4}
\axislabels {y}{ 1, {$4$} 2, {$6$} 3, {$8$} 4, {$-4$} -2, {$-6$} -3}
\penwd{1.25pt}
\arrow \reverse \arrow \function{-4.8,0.8,0.1}{(x+2)**2-3}
\point[4pt]{(-2,-3), (0,1)}
\normalsize
\end{mfpic} 

\vfill

\columnbreak

\item $~$

\begin{mfpic}[13]{-5}{5}{-5}{5}
\axes
\tlabel[cc](5,-0.5){\scriptsize $x$}
\tlabel[cc](0.5,5){\scriptsize $y$}
\tlabel[cc](-1.25, 4.25){\scriptsize $(0,4)$}
\tlabel[cc](3,0.75){\scriptsize $(\sqrt{2},0)$}
\tlabel[cc](-3.25,0.75){\scriptsize $(-\sqrt{2},0)$}
\xmarks{-3,,-1,1,3}
\ymarks{-4,-3,-2, -1, 1,2,3,4}
\tlpointsep{4pt}
\scriptsize
\axislabels {x}{ {$-2 \hspace{7pt}$} -3, {$-1 \hspace{7pt}$} -1, {$1$} 1, {$2$} 3}
\axislabels {y}{{$-1$} -1,{$1$} 1, {$2$} 2, {$3$} 3,  {$-2$} -2, {$-3$} -3, {$-4$} -4}
\penwd{1.25pt}
\arrow \reverse \arrow \function{-3,3,0.1}{4-(x**2)}
\point[4pt]{(0,4), (-2,0), (2,0)}
\normalsize
\end{mfpic} 

\setcounter{HW}{\value{enumi}}
\end{enumerate}
\end{multicols}


\pagebreak

\begin{multicols}{2}
\begin{enumerate}
\setcounter{enumi}{\value{HW}}


\item $~$   

\begin{mfpic}[13]{-6}{5}{-3}{7}
\axes
\tlabel[cc](5,-0.5){\scriptsize $x$}
\tlabel[cc](0.5,7){\scriptsize $y$}
\tlabel[cc](-5.5, 2){\scriptsize $(-4,2)$}
\tlabel[cc](1, 4.75){\scriptsize $(0,4)$}
\tlabel[cc](1, -1){\scriptsize $(0, 0)$}
%\tlabel[cc](-0.5,-1){\scriptsize $\left(0, \frac{1}{2} \right)$}
\xmarks{-4,-3,-2,-1,1,2,3,4}
\ymarks{-2 step 1 until 6}
\tlpointsep{4pt}
\scriptsize
\axislabels {x}{ {$-4 \hspace{7pt}$} -4, {$-3 \hspace{7pt}$} -3, {$-2 \hspace{7pt}$} -2, {$-1 \hspace{7pt}$} -1,  {$4$} 4}
\axislabels {y}{{$1$} 1, {$2$} 2, {$3$} 3,  {$5$} 5,{$6$} 6, {$-1$} -1, {$-2$} -2}
\penwd{1.25pt}
\arrow  \function{-4,5,0.1}{2-sqrt(x+4)}
\arrow  \function{-4,5,0.1}{2+sqrt(x+4)}
\point[4pt]{(-4,2), (0,0), (0,4)}
%\tcaption{ \scriptsize $x$,$y$-intercept $(0,0)$}
\normalsize
\end{mfpic} 

\vfill

\columnbreak

\item $~$ \label{buildparafromgraphlast} 

\begin{mfpic}[13]{-5}{5}{-5}{5}
\axes
\tlabel[cc](5,-0.5){\scriptsize $x$}
\tlabel[cc](0.5,5){\scriptsize $y$}
\tlabel[cc](1.25,2){\scriptsize $(0,4)$}
\tlabel[cc](1.5,-2){\scriptsize $(0,-4)$}
\tlabel[cc](2,-0.75){\scriptsize $(1,0)$}
%\tlabel[cc](-1.5, 0.5){\scriptsize $(-1,0)$}
%\tlabel[cc](-0.5,-1){\scriptsize $\left(0, \frac{1}{2} \right)$}
\xmarks{-4,-3,-2,-1,1,2,3,4}
\ymarks{-4 step 1 until 4}
\tlpointsep{4pt}
\scriptsize
\axislabels {x}{ {$-4 \hspace{7pt}$} -4, {$-3 \hspace{7pt}$} -3, {$-2 \hspace{7pt}$} -2, {$-1 \hspace{7pt}$} -1,   {$4$} 4}
\axislabels {y}{{$2$} 1, {$4$} 2, {$6$} 3, {$8$} 4, {$-2$} -1, {$-4$} -2, {$-6$} -3, {$-8$} -4}
\penwd{1.25pt}
\arrow \reverse \function{-5,1,0.1}{2*sqrt(1-x)}
\arrow \reverse \function{-5,1,0.1}{-2*sqrt(1-x)}
\point[4pt]{(1,0), (0,2), (0,-2)}
%\tcaption{ \scriptsize $x$-intercept $(1,0)$, $y$-intercept $(0,2)$}
\normalsize
\end{mfpic} 


\setcounter{HW}{\value{enumi}}
\end{enumerate}
\end{multicols}

\end{question}


\begin{question}
    
In Exercises \ref{buildparafirst} - \ref{buildparalast}, find an equation for the parabola which fits the given criteria.

\begin{problem}\label{buildparafirst}
    Vertex $(7, 0)$, focus $(0, 0)$.

\begin{solution}
    $y^{2} = -28(x - 7)$
\end{solution}
    
\end{problem}

\begin{problem}
    Focus $(10, 1)$, directrix $x = 5$.

\begin{solution}
    $(y - 1)^{2} = 10\left(x - \frac{15}{2} \right)$
\end{solution}
    
\end{problem}

\begin{problem}
     Vertex $(-8, -9)$; $(0, 0)$ and $(-16, 0)$ are points on the curve.

\begin{solution}
    $(x + 8)^{2} = \frac{64}{9}(y + 9)$
\end{solution}
     
\end{problem}

\begin{problem}\label{buildparalast}
    The endpoints of latus rectum are $(-2, -7)$ and $(4, -7)$.

\begin{solution}
     $(x - 1)^{2} = 6\left(y + \frac{17}{2}\right)$ or $(x - 1)^{2} = -6\left(y + \frac{11}{2}\right)$
\end{solution}
    
\end{problem}

\end{question}



\begin{problem}
  The mirror in Carl's flashlight is a paraboloid of revolution.  If the mirror is 5 centimeters in diameter and 2.5 centimeters deep, where should the light bulb be placed so it is at the focus of the mirror?

\begin{solution}
    The bulb should be placed $0.625$ centimeters above the vertex of the mirror.\footnote{As verified by Carl himself!}
\end{solution}
  
\end{problem}

\begin{problem}
    
A parabolic Wi-Fi antenna is constructed by taking a flat sheet of metal and bending it into a parabolic shape.\footnote{This shape is called a `parabolic cylinder.'}  If the cross section of the antenna is a parabola which is 45 centimeters wide and 25 centimeters deep, where should the receiver be placed to maximize reception?

\begin{solution}
    The receiver be placed $5.0625$ centimeters from the vertex of the cross section of the antenna.
\end{solution}

\end{problem}

\begin{problem}\label{parabolaarch}
    A parabolic arch is constructed which is 6 feet wide at the base and 9 feet tall in the middle. Find the height of the arch exactly 1 foot in from the base of the arch. 

\begin{solution}
     The arch can be modeled by $x^2=-(y-9)$ or $y=9-x^2$.  One foot in from the base of the arch corresponds to either $x = \pm 2$, so the height is $y=9-(\pm 2)^2=5$ feet.
\end{solution}
    
\end{problem}   

\begin{problem}
    A popular novelty item is the `mirage bowl.'  Follow this  \href{http://spie.org/etop/2007/etop07methodsV.pdf}{\underline{link}} to see another startling application of the reflective property of the parabola.
\end{problem}

\begin{problem}
    With the help of your classmates, research spinning liquid mirrors.  To get you started,  \href{http://www.astro.ubc.ca/LMT/lzt/}{\underline{here}}.
\end{problem}


\end{document}
