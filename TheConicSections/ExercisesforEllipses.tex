\documentclass{ximera}

\begin{document}
	\author{Stitz-Zeager}
	\xmtitle{Exercises}
\mfpicnumber{1} \opengraphsfile{ExercisesforEllipses} % mfpic settings added 


\label{ExercisesforEllipses}

In Exercises \ref{graphellipseexfirst} - \ref{graphellipseexlast},  graph the ellipse in the $xy$-plane.  Find the center, the lines which contain the major and minor axes, the vertices, the endpoints of the minor axis, the foci and the eccentricity.

\begin{multicols}{2}
\begin{enumerate}

\item $\dfrac{x^{2}}{169} + \dfrac{y^{2}}{25} = 1$  \label{graphellipseexfirst} \label{oddellipseone}
\item $\dfrac{x^2}{9} + \dfrac{y^2}{25} = 1$



\setcounter{HW}{\value{enumi}}
\end{enumerate}
\end{multicols}

\begin{multicols}{2}
\begin{enumerate}
\setcounter{enumi}{\value{HW}}

\item $\dfrac{(x - 2)^{2}}{4} + \dfrac{(y + 3)^{2}}{9} = 1$   \label{oddellipsethree}
\item $\dfrac{(x + 5)^{2}}{16} + \dfrac{(y - 4)^{2}}{1} = 1$


\setcounter{HW}{\value{enumi}}
\end{enumerate}
\end{multicols}

\begin{multicols}{2}
\begin{enumerate}
\setcounter{enumi}{\value{HW}}

\item $\dfrac{(x - 1)^{2}}{10} + \dfrac{(y - 3)^{2}}{11} = 1$   \label{oddellipsefive}
\item $\dfrac{(x-1)^2}{9}+\dfrac{(y+3)^2}{4} = 1$


\setcounter{HW}{\value{enumi}}
\end{enumerate}
\end{multicols}

\begin{multicols}{2}
\begin{enumerate}
\setcounter{enumi}{\value{HW}}

\item $\dfrac{(x+2)^2}{16}+\dfrac{(y-5)^2}{20} = 1$   \label{oddellipseseven}
\item $\dfrac{(x-4)^2}{8} + \dfrac{(y-2)^2}{18} = 1$ \label{graphellipseexlast}

\setcounter{HW}{\value{enumi}}
\end{enumerate}
\end{multicols}

In Exercises \ref{stdformellipseexfirst} - \ref{stdformellipseexlast}, put the equation in standard form.   Find the center, the lines which contain the major and minor axes, the vertices, the endpoints of the minor axis, the foci and the eccentricity.\footnote{\ldots assuming the equation were graphed in the $xy$-plane.}

\begin{multicols}{2}
\begin{enumerate}
\setcounter{enumi}{\value{HW}}

\item $9x^2+25y^2-54x-50y-119=0$  \label{stdformellipseexfirst}   \label{oddellipsenine}
\item $12x^{2} + 3y^{2} - 30y + 39 = 0$

\setcounter{HW}{\value{enumi}}
\end{enumerate}
\end{multicols}

\begin{multicols}{2}
\begin{enumerate}
\setcounter{enumi}{\value{HW}}

\item $5x^{2} + 18y^{2} - 30x + 72y + 27 = 0$   \label{oddellipseeleven}
\item $x^2 - 2x + 2y^2 - 12y + 3 = 0$


\setcounter{HW}{\value{enumi}}
\end{enumerate}
\end{multicols}

\begin{multicols}{2}
\begin{enumerate}
\setcounter{enumi}{\value{HW}}

\item $9x^2 + 4y^2 - 4y - 8 = 0$   \label{oddellipsethirteen}
\item $6x^2+5y^2-24x+20y+14=0$  \label{stdformellipseexlast}

\setcounter{HW}{\value{enumi}}
\end{enumerate}
\end{multicols}

\begin{enumerate}
\setcounter{enumi}{\value{HW}}

\item For each of the odd numbered equations given in Exercises \ref{oddellipseone} - \ref{oddellipsethirteen}, find two or more explicit functions of $x$ represented by each of the equations.  (See Example \ref{horizontalparabolaex} in Section \ref{Parabolas}.)

\setcounter{HW}{\value{enumi}}
\end{enumerate}

In Exercises \ref{semiellipsefunctionfirst} - \ref{semiellipsefunctionlast}, graph each function by recognizing it as a semi ellipse.

\begin{multicols}{2}
\begin{enumerate}
\setcounter{enumi}{\value{HW}}

\item   $f(x) = \sqrt{16-4x^2}$ \label{semiellipsefunctionfirst}
\item   $g(x) = -\frac{1}{2} \sqrt{6x-x^2}$

\setcounter{HW}{\value{enumi}}
\end{enumerate}
\end{multicols}

\begin{multicols}{2}
\begin{enumerate}
\setcounter{enumi}{\value{HW}}

\item  $f(x) = -2\sqrt{3-2x-x^2}$
\item  $g(x) = -2 + 2\sqrt{9-x^2}$ \label{semiellipsefunctionlast}

\setcounter{HW}{\value{enumi}}
\end{enumerate}
\end{multicols}

\pagebreak

In Exercises \ref{buildellipsefromgraphfirst} - \ref{buildellipsefromgraphlast}, find an equation for the ellipse or semi ellipse whose graph is given.

\begin{multicols}{2}
\begin{enumerate}
\setcounter{enumi}{\value{HW}}

\item $~$ \label{buildellipsefromgraphfirst}  % $\dfrac{(x-1)^2}{9} + \dfrac{y^2}{16} = 1$

\begin{mfpic}[13]{-4}{6}{-5}{5}
\axes
\tlabel[cc](6,-0.5){\scriptsize $x$}
\tlabel[cc](0.5,5){\scriptsize $y$}
\tlabel[cc](1, 4.5){\scriptsize $(1,4)$}
\tlabel[cc](1.25, -4.75){\scriptsize $(1,-4)$}
\tlabel[cc](-3, 0.75){\scriptsize $(-2,0)$}
\tlabel[cc](5, 0.75){\scriptsize $(4,0)$}
\xmarks{-3 step 1 until 5}
\ymarks{-4 step 1 until 4}
\tlpointsep{4pt}
\scriptsize
\axislabels {x}{ {$-3 \hspace{7pt}$} -3,  {$-1 \hspace{7pt}$} -1, {$1$} 1, {$2$} 2, {$3$} 3, {$5$} 5}
\axislabels {y}{ {$-4$} -4, {$-2$} -2, {$-1$} -1, {$1$} 1, {$2$} 2,  {$4$} 4  }
\penwd{1.25pt}
\ellipse{(1,0), 3,4}
\point[4pt]{(-2,0), (4,0), (1,4), (1,-4)}
\normalsize
\end{mfpic} 

\vfill

\columnbreak

\item $~$  % $\dfrac{(x-4)^2}{16} + \dfrac{(y-4)^2}{9} = 1$

\begin{mfpic}[13]{-1}{9}{-1}{9}
\axes
\tlabel[cc](9,-0.5){\scriptsize $x$}
\tlabel[cc](0.5,9){\scriptsize $y$}
\tlabel[cc](1, 4){\scriptsize $(0,4)$}
\tlabel[cc](7, 4){\scriptsize $(8,4)$}
\tlabel[cc](4, 7.75){\scriptsize $(4,7)$}
\tlabel[cc](4, 0.25){\scriptsize $(4,1)$}
\xmarks{1 step 1 until 8}
\ymarks{1 step 1 until 8}
\tlpointsep{4pt}
\scriptsize
\axislabels {x}{{$1$} 1, {$2$} 2, {$3$} 3, {$4$} 4, {$5$} 5, {$6$} 6, {$7$} 7, {$8$} 8}
\axislabels {y}{{$1$} 1, {$2$} 2, {$3$} 3, {$4$} 4, {$5$} 5, {$6$} 6, {$7$} 7, {$8$} 8}
\penwd{1.25pt}
\ellipse{(4,4), 4, 3}
\point[4pt]{(4,1), (0,4), (4,7), (8,4)}
\normalsize
\end{mfpic} 

\setcounter{HW}{\value{enumi}}
\end{enumerate}
\end{multicols}


\begin{multicols}{2}
\begin{enumerate}
\setcounter{enumi}{\value{HW}}


\item $~$   % $y = 3 - \frac{3}{4} \sqrt{16-x^2}$

\begin{mfpic}[13]{-5}{5}{-1}{6}
\axes
\tlabel[cc](5,-0.5){\scriptsize $x$}
\tlabel[cc](0.5,6){\scriptsize $y$}
\tlabel[cc](-4, 3.5){\scriptsize $(-4,3)$}
\tlabel[cc](4, 3.5){\scriptsize $(4,3)$}
\tlabel[cc](0.75, -0.75){\scriptsize $(0, 0)$}
%\tlabel[cc](-0.5,-1){\scriptsize $\left(0, \frac{1}{2} \right)$}
\xmarks{-4,-3,-2,-1,1,2,3,4}
\ymarks{1 step 1 until 5}
\tlpointsep{4pt}
\scriptsize
\axislabels {x}{ {$-4 \hspace{7pt}$} -4, {$-3 \hspace{7pt}$} -3, {$-2 \hspace{7pt}$} -2, {$-1 \hspace{7pt}$} -1,  {$4$} 4,  {$3$} 3,  {$2$} 2}
\axislabels {y}{{$1$} 1, {$2$} 2, {$3$} 3,  {$4$} 4,  {$5$} 5}
\penwd{1.25pt}
\function{-4,4,0.1}{3-0.75*sqrt(16-(x**2))}
\point[4pt]{(-4,3), (0,0), (4,3)}
%\tcaption{ \scriptsize $x$,$y$-intercept $(0,0)$}
\normalsize
\end{mfpic} 

\vfill

\columnbreak

\item $~$ \label{buildellipsefromgraphlast}   % $y = 2 \sqrt{8x-x^2-12}$

\begin{mfpic}[13]{-1}{9}{-1}{6}
\axes
\tlabel[cc](9,-0.5){\scriptsize $x$}
\tlabel[cc](0.5,6){\scriptsize $y$}
\tlabel[cc](6, -0.75){\scriptsize $(6,0)$}
\tlabel[cc](4, 4.5){\scriptsize $(4,4)$}
\tlabel[cc](2, -0.75){\scriptsize $(2, 0)$}
%\tlabel[cc](-0.5,-1){\scriptsize $\left(0, \frac{1}{2} \right)$}
\xmarks{1 step 1 until 8}
\ymarks{1 step 1 until 5}
\tlpointsep{4pt}
\scriptsize
\axislabels {x}{{$1$} 1,   {$3$} 3,  {$4$} 4,  {$5$} 5,  {$8$} 8, {$7$} 7 }
\axislabels {y}{{$1$} 1, {$2$} 2, {$3$} 3,  {$4$} 4,  {$5$} 5}
\penwd{1.25pt}
\function{2,6,0.1}{2*sqrt(8*x-12-(x**2))}
\point[4pt]{(6,0), (2,0), (4,4)}
%\tcaption{ \scriptsize $x$,$y$-intercept $(0,0)$}
\normalsize
\end{mfpic} 


\setcounter{HW}{\value{enumi}}
\end{enumerate}
\end{multicols}


In Exercises \ref{buildellipsefirst} - \ref{buildellipselast},  find the standard form of the equation of the ellipse which has the given properties.

\begin{enumerate}
\setcounter{enumi}{\value{HW}}

\item Center $(3, 7)$, Vertex $(3, 2)$, Focus $(3, 3)$  \label{buildellipsefirst}
\item Foci $(0, \pm 5)$, Vertices $(0, \pm 8)$.
\item Foci $(\pm 3, 0)$, length of the Minor Axis $10$
\item Vertices $(3,2)$, $(13,2)$; Endpoints of the Minor Axis $(8,4)$, $(8,0)$
\item Center $(5,2)$, Vertex $(0,2)$, eccentricity $\frac{1}{2}$
\item All points on the ellipse are in Quadrant IV except $(0, -9)$ and $(8, 0)$.  (One might also say that the ellipse is ``tangent to the axes'' at those two points.)  \label{buildellipselast}

\setcounter{HW}{\value{enumi}}
\end{enumerate}

\begin{enumerate}
\setcounter{enumi}{\value{HW}}

\item  Repeat Example \ref{whisgalleryex} for a whispering gallery 200 feet wide and 75 feet tall.

\item \label{ellipsearchex} An elliptical arch is constructed which is 6 feet wide at the base and 9 feet tall in the middle. Find the height of the arch exactly 1 foot in from the base of the arch. Compare your result with your answer to Exercise \ref{parabolaarch} in Section \ref{Parabolas}.

\item The Earth's orbit around the sun is an ellipse with the sun at one focus and eccentricity $e \approx 0.0167$.  The length of the semimajor axis (that is, half of the major axis) is defined to be $1$ astronomical unit (AU).  The vertices of the elliptical orbit are given special names: `aphelion' is the vertex farthest from the sun, and  `perihelion' is the vertex closest to the sun.  Find the distance in AU between the sun and aphelion and the distance in AU between the sun and perihelion.

\item  \label{MercuryOrbitGraph}  This exercise is a follow-up to Example \ref{MecuryOribitEx}.  Find the equation of the ellipse which models the orbit of Mercury.  Graph the ellipse using a graphing utility, and comment on the `roundness' of the orbit.


\item Some famous examples of whispering galleries include \href{http://www.stpauls.co.uk/}{\underline{St. Paul's Cathedral}} in London, England, \href{http://www.aoc.gov/cc/capitol/nat_stat_hall.cfm}{\underline{National Statuary Hall}} in Washington, D.C.,  and \href{http://www.cincymuseum.org/}{\underline{The Cincinnati Museum Center}}. With the help of your classmates, research these whispering galleries.  How does the whispering effect compare and contrast with the scenario in Example  \ref{whisgalleryex}?

\item With the help of your classmates, research ``extracorporeal shock-wave lithotripsy''.  It uses the reflective property of the ellipsoid to dissolve kidney stones.

\end{enumerate}

\newpage

\subsection{Answers}

\begin{enumerate}

\item \begin{multicols}{2} \raggedcolumns
$\dfrac{x^{2}}{169} + \dfrac{y^{2}}{25} = 1$

Center $(0, 0)$\\
Major axis along $y = 0$\\
Minor axis along $x = 0$\\
Vertices $(13, 0), \, (-13, 0)$\\
Endpoints of Minor Axis $(0,-5)$, $(0,5)$ \\
Foci $(12, 0), \, (-12, 0)$\\
$e = \frac{12}{13}$\\

\begin{mfpic}[7][10]{-14}{14}{-6}{6}
\axes
\tlabel(14,-0.5){\scriptsize $x$}
\tlabel(0.5,6){\scriptsize $y$}
\xmarks{-13 step 1 until 13}
\ymarks{-5 step 1 until 5}
\plotsymbol[4pt]{Asterisk}{(-12,0), (12,0)}
\plotsymbol[4pt]{Cross}{(0,0)}
\point[4pt]{(-13,0), (13,0), (0,5), (0,-5)}
\tlpointsep{4pt}
\tiny
\axislabels {x}{{$-13 \hspace{6pt}$} -13, {$-1 \hspace{6pt}$} -1, {$1$} 1, {$13$} 13}
\axislabels {y}{{$-5$} -5, {$-4$} -4, {$-3$} -3, {$-2$} -2, {$-1$} -1, {$1$} 1, {$2$} 2, {$3$} 3, {$4$} 4, {$5$} 5}
\normalsize
\penwd{1.25pt}
\ellipse{(0,0),13,5}
\end{mfpic} 

\end{multicols}

\item \begin{multicols}{2} \raggedcolumns
$\dfrac{x^{2}}{9} + \dfrac{y^{2}}{25} = 1$

Center $(0, 0)$\\
Major axis along $x = 0$\\
Minor axis along $y = 0$\\
Vertices $(0,5), \, (0,-5)$\\
Endpoints of Minor Axis $(-3,0)$, $(3,0)$ \\
Foci $(0,-4), \, (0,4)$\\
$e = \frac{4}{5}$\\

\begin{mfpic}[10]{-4}{4}{-6}{6}
\axes
\tlabel(4,-0.5){\scriptsize $x$}
\tlabel(0.5,6){\scriptsize $y$}
\xmarks{-3 step 1 until 3}
\ymarks{-5 step 1 until 5}
\plotsymbol[4pt]{Asterisk}{(0,-4), (0,4)}
\plotsymbol[4pt]{Cross}{(0,0)}
\point[4pt]{(-3,0), (3,0), (0,5), (0,-5)}
\tlpointsep{4pt}
\tiny
\axislabels {x}{{$-3 \hspace{6pt}$} -3, {$-2 \hspace{6pt}$} -2, {$-1 \hspace{6pt}$} -1,{$1$} 1, {$2$} 2, {$3$} 3}
\axislabels {y}{{$-5$} -5, {$-4$} -4, {$-3$} -3, {$-2$} -2, {$-1$} -1, {$1$} 1, {$2$} 2, {$3$} 3, {$4$} 4, {$5$} 5}
\normalsize
\penwd{1.25pt}
\ellipse{(0,0),3,5}
\end{mfpic} 

\end{multicols}

\item \begin{multicols}{2} \raggedcolumns
$\dfrac{(x - 2)^{2}}{4} + \dfrac{(y + 3)^{2}}{9} = 1$

Center $(2, -3)$\\
Major axis along $x = 2$\\
Minor axis along $y = -3$\\
Vertices $(2, 0), \, (2, -6)$\\
Endpoints of Minor Axis $(0,-3)$, $(4,-3)$\\
Foci $(2, -3 + \sqrt{5}), \, (2, -3 - \sqrt{5})$\\
$e = \frac{\sqrt{5}}{3}$\\

\begin{mfpic}[15]{-1}{5}{-7}{1}
\axes
\tlabel(5,-0.5){\scriptsize $x$}
\tlabel(0.5,1){\scriptsize $y$}
\xmarks{1 step 1 until 4}
\ymarks{-6 step 1 until 0}
\plotsymbol[4pt]{Asterisk}{(2, -0.7639), (2,-5.23606)}
\plotsymbol[4pt]{Cross}{(2,-3)}
\point[4pt]{(2,0), (2,-6), (0,-3), (4,-3)}
\tlpointsep{4pt}
\scriptsize
\axislabels {x}{{$1$} 1, {$2$} 2, {$3$} 3, {$4$} 4}
\axislabels {y}{{$-6$} -6, {$-5$} -5, {$-4$} -4, {$-3$} -3, {$-2$} -2, {$-1$} -1}
\normalsize
\penwd{1.25pt}
\ellipse{(2,-3),2,3}
\end{mfpic} 

\end{multicols}

\item \begin{multicols}{2} \raggedcolumns
$\dfrac{(x + 5)^{2}}{16} + \dfrac{(y - 4)^{2}}{1} = 1$

Center $(-5, 4)$\\
Major axis along $y = 4$\\
Minor axis along $x = -5$\\
Vertices $(-9, 4), \, (-1, 4)$\\
Endpoints of Minor Axis $(-5,3)$, $(-5,5)$\\
Foci $(-5 + \sqrt{15}, 4), \, (-5 - \sqrt{15}, 4)$\\
$e = \frac{\sqrt{15}}{4}$\\

\begin{mfpic}[16]{-10}{1}{-.5}{6}
\axes
\tlabel(1,-0.5){\scriptsize $x$}
\tlabel(0.5,6){\scriptsize $y$}
\xmarks{-9 step 1 until -1}
\ymarks{1 step 1 until 5}
\plotsymbol[4pt]{Asterisk}{(-8.873,4), (-1.127,4)}
\plotsymbol[4pt]{Cross}{(-5,4)}
\point[4pt]{(-9,4), (-1,4), (-5,5), (-5,3)}
\tlpointsep{4pt}
\scriptsize
\axislabels {x}{{$-9 \hspace{7pt}$} -9, {$-8 \hspace{7pt}$} -8, {$-7 \hspace{7pt}$} -7, {$-6 \hspace{7pt}$} -6, {$-5 \hspace{7pt}$} -5, {$-4 \hspace{7pt}$} -4, {$-3 \hspace{7pt}$} -3, {$-2 \hspace{7pt}$} -2, {$-1 \hspace{7pt}$} -1}
\axislabels {y}{{$1$} 1, {$2$} 2, {$3$} 3, {$4$} 4, {$5$} 5}
\normalsize
\penwd{1pt}
\ellipse{(-5,4),4,1}
\end{mfpic} 

\end{multicols}

\item \begin{multicols}{2} \raggedcolumns
$\dfrac{(x - 1)^{2}}{10} + \dfrac{(y - 3)^{2}}{11} = 1$

Center $(1, 3)$\\
Major axis along $x = 1$\\
Minor axis along $y = 3$\\
Vertices $(1, 3 + \sqrt{11}), \, (1, 3 - \sqrt{11})$\\
Endpoints of the Minor Axis \\ $(1-\sqrt{10}, 3), \, (1+\sqrt{10}, 3)$\\
Foci $(1, 2), \, (1, 4)$\\
$e = \frac{\sqrt{11}}{11}$\\

\begin{mfpic}[18]{-3}{5}{-1}{7}
\axes
\tlabel(5,-0.25){\scriptsize $x$}
\tlabel(0.25,7){\scriptsize $y$}
\xmarks{-2 step 1 until 4}
\ymarks{1 step 1 until 6}
\plotsymbol[4pt]{Asterisk}{(1,2), (1,4)}
\plotsymbol[4pt]{Cross}{(1,3)}
\point[4pt]{(1,6.3166), (1,-0.3166), (-2.1623, 3), (4.1623,3)}
\tlpointsep{4pt}
\scriptsize
\axislabels {x}{{$-2 \hspace{7pt}$} -2, {$-1 \hspace{7pt}$} -1, {$1$} 1, {$2$} 2, {$3$} 3, {$4$} 4}
\axislabels {y}{{$1$} 1, {$2$} 2, {$3$} 3, {$4$} 4, {$5$} 5, {$6$} 6}
\normalsize
\penwd{1.25pt}
\ellipse{(1,3),3.1623,3.3166}
\end{mfpic} 

\end{multicols}

\item \begin{multicols}{2} \raggedcolumns
$\dfrac{(x-1)^2}{9}+\dfrac{(y+3)^2}{4} = 1$

Center $(1, -3)$\\
Major axis along $y = -3$\\
Minor axis along $x = 1$\\
Vertices $(4, -3), \, (-2, -3)$\\
Endpoints of the Minor Axis $(1,-1)$, $(1,-5)$\\
Foci $(1+\sqrt{5}, -3), \, (1-\sqrt{5}, -3)$\\
$e = \frac{\sqrt{5}}{3}$\\

\begin{mfpic}[18]{-3}{5}{-6}{1}
\axes
\tlabel(5,-0.25){\scriptsize $x$}
\tlabel(0.25,1){\scriptsize $y$}
\xmarks{-2 step 1 until 4}
\ymarks{-5 step 1 until -1}
\plotsymbol[4pt]{Asterisk}{(3.2361,-3), (-1.2361,-3)}
\plotsymbol[4pt]{Cross}{(1,-3)}
\point[4pt]{(4,-3), (-2,-3), (1,-1), (1,-5)}
\tlpointsep{4pt}
\scriptsize
\axislabels {x}{{$-2 \hspace{7pt}$} -2, {$-1 \hspace{7pt}$} -1, {$1$} 1, {$2$} 2, {$3$} 3, {$4$} 4}
\axislabels {y}{{$-5$} -5, {$-4$} -4, {$-3$} -3, {$-2$} -2, {$-1$} -1}
\normalsize
\penwd{1.25pt}
\ellipse{(1,-3),3,2}
\end{mfpic} 

\end{multicols}



\item \begin{multicols}{2} \raggedcolumns
$\dfrac{(x+2)^2}{16}+\dfrac{(y-5)^2}{20} = 1$

Center $(-2, 5)$\\
Major axis along $x = -2$\\
Minor axis along $y = 5$\\
Vertices $(-2, 5 + 2\sqrt{5}), \, (-2, 5 - 2\sqrt{5})$\\
Endpoints of the Minor Axis $(-6,5)$, $(2,5)$
Foci $(-2, 7), \, (-2, 3)$\\
$e = \frac{\sqrt{5}}{5}$\\

\begin{mfpic}[18]{-7}{3}{-1}{11}
\axes
\tlabel(3,-0.25){\scriptsize $x$}
\tlabel(0.25,11){\scriptsize $y$}
\xmarks{-6 step 1 until 2}
\ymarks{1 step 1 until 11}
\plotsymbol[4pt]{Asterisk}{(-2,7), (-2,3)}
\plotsymbol[4pt]{Cross}{(-2,5)}
\point[4pt]{(-2,9.4721), (-2,0.5279), (-6,5), (2,5)}
\tlpointsep{4pt}
\scriptsize
\axislabels {x}{{$-6 \hspace{7pt}$} -6,{$-5 \hspace{7pt}$} -5,{$-4 \hspace{7pt}$} -4,{$-3 \hspace{7pt}$} -3,{$-2 \hspace{7pt}$} -2, {$-1 \hspace{7pt}$} -1, {$1$} 1, {$2$} 2}
\axislabels {y}{{$1$} 1, {$2$} 2, {$3$} 3, {$4$} 4, {$5$} 5, {$6$} 6, {$7$} 7, {$8$} 8, {$9$} 9, {$10$} 10 }
\normalsize
\penwd{1.25pt}
\ellipse{(-2,5),4,4.4721}
\end{mfpic} 

\end{multicols}


\pagebreak


\item \begin{multicols}{2} \raggedcolumns
$\dfrac{(x-4)^2}{8}+\dfrac{(y-2)^2}{18} = 1$

Center $(4, 2)$\\
Major axis along $x = 4$\\
Minor axis along $y = 2$\\
Vertices $(4, 2 + 3\sqrt{2}), \, (4, 2 - 3\sqrt{2})$\\
Endpoints of the Minor Axis \\
$(4-2\sqrt{2},2)$, $(4+2\sqrt{2},2)$\\
Foci $(4, 2+\sqrt{10}), \, (4, 2-\sqrt{10})$\\
$e = \frac{\sqrt{5}}{3}$\\

\begin{mfpic}[18]{-1}{8}{-4}{8}
\axes
\tlabel(8,-0.25){\scriptsize $x$}
\tlabel(0.25,8){\scriptsize $y$}
\xmarks{1 step 1 until 7}
\ymarks{-3 step 1 until 7}
\plotsymbol[4pt]{Asterisk}{(4,5.1623), (4,-1.1623)}
\plotsymbol[4pt]{Cross}{(4,2)}
\point[4pt]{(4,6.2426), (4,-2.2426), (1.1716, 2), (6.828,2)}
\tlpointsep{4pt}
\scriptsize
\axislabels {x}{{$1$} 1, {$2$} 2, {$3$} 3, {$4$} 4, {$5$} 5, {$6$} 6, {$7$} 7}
\axislabels {y}{{$-3$} -3,{$-2$} -2,{$-1$} -1,{$1$} 1, {$2$} 2, {$3$} 3, {$4$} 4, {$5$} 5, {$6$} 6, {$7$} 7}
\normalsize
\penwd{1.25pt}
\ellipse{(4,2),2.8284,4.2426}
\end{mfpic} 

\end{multicols}

\setcounter{HW}{\value{enumi}}
\end{enumerate}

\begin{multicols}{2}
\begin{enumerate}
\setcounter{enumi}{\value{HW}}

\item  $\dfrac{(x-3)^2}{25} + \dfrac{\left(y-1\right)^2}{9} = 1$\\
Center  $\left(3, 1 \right)$\\
Major Axis along $y=1$\\
Minor Axis along $x=3$\\
Vertices  $\left( 8, 1   \right)$, $(-2, 1)$\\
Endpoints of Minor Axis $\left(3,4\right)$, $\left(3,-2\right)$\\
Foci $\left(7,1 \right)$, $\left(-1, 1\right)$\\
$e = \frac{4}{5}$


\vfill

\columnbreak


\item $\dfrac{x^{2}}{3} + \dfrac{(y - 5)^{2}}{12} = 1$\\
Center $(0, 5)$\\
Major axis along $x = 0$\\
Minor axis along $y = 5$\\
Vertices $(0, 5 - 2\sqrt{3}), (0, 5 + 2\sqrt{3})$\\
Endpoints of Minor Axis $(-\sqrt{3},5)$, $(\sqrt{3},5)$\\
Foci $(0, 2), (0, 8)$\\
$e = \frac{\sqrt{3}}{2}$


\setcounter{HW}{\value{enumi}}
\end{enumerate}
\end{multicols}

\begin{multicols}{2}
\begin{enumerate}
\setcounter{enumi}{\value{HW}}

\item $\dfrac{(x - 3)^{2}}{18} + \dfrac{(y + 2)^{2}}{5} = 1$\\
Center $(3, -2)$\\
Major axis along $y = -2$\\
Minor axis along $x = 3$\\
Vertices $(3 - 3\sqrt{2}, -2), (3 + 3\sqrt{2}, -2)$\\
Endpoints of Minor Axis $(3,-2+\sqrt{5})$, $(3,-2-\sqrt{5})$\\
Foci $(3 - \sqrt{13}, -2), (3 + \sqrt{13}, -2)$\\
$e = \frac{\sqrt{26}}{6}$


\vfill

\columnbreak

\item $\dfrac{(x - 1)^{2}}{16} + \dfrac{(y - 3)^{2}}{8} = 1$\\
Center  $(1,3)$ \\
Major Axis along $y=3$\\
Minor Axis along $x=1$\\
Vertices $(5, 3)$, $(-3,3)$\\
Endpoints of Minor Axis $(1,3+2\sqrt{2})$, $(1,3-2\sqrt{2})$\\
Foci $(1 + 2 \sqrt{2}, 3)$, $(1-2 \sqrt{2},3)$\\
$e = \frac{\sqrt{2}}{2}$


\setcounter{HW}{\value{enumi}}
\end{enumerate}
\end{multicols}

\pagebreak

\begin{multicols}{2}
\begin{enumerate}
\setcounter{enumi}{\value{HW}}

\item  $\dfrac{x^2}{1} + \dfrac{4\left(y-\frac{1}{2}\right)^2}{9} = 1$\\
Center  $\left(0, \frac{1}{2} \right)$\\
Major Axis along $x=0$ (the $y$-axis)\\
Minor Axis along $y=\frac{1}{2}$\\
Vertices  $\left( 0, 2   \right)$, $(0, -1)$\\
Endpoints of Minor Axis $\left(-1, \frac{1}{2} \right)$, $\left(1, \frac{1}{2} \right)$\\
Foci $\left(0, \frac{1+\sqrt{5}}{2}\right)$, $\left(0, \frac{1-\sqrt{5}}{2}\right)$\\
$e = \frac{\sqrt{5}}{3}$

\vfill

\columnbreak


\item  $\dfrac{(x-2)^2}{5} + \dfrac{\left(y+2\right)^2}{6} = 1$\\
Center  $\left(2, -2 \right)$\\
Major Axis along $x=2$\\
Minor Axis along $y=-2$\\
Vertices  $\left( 2, -2+\sqrt{6}   \right)$, $(2, -2-\sqrt{6})$\\
Endpoints of Minor Axis $\left(2-\sqrt{5},-2 \right)$, $\left(2+\sqrt{5},-2\right)$\\
Foci $\left(2,-1 \right)$, $\left(2, -3\right)$\\
$e = \frac{\sqrt{6}}{6}$



\setcounter{HW}{\value{enumi}}
\end{enumerate}
\end{multicols}



\begin{enumerate}
\setcounter{enumi}{\value{HW}}

\item $~$
For number \ref{oddellipseone}:

\begin{itemize}

\item  $f(x) = \frac{5}{13} \sqrt{169-x^2}$ represents the upper half of the ellipse.

\item  $g(x) =  -\frac{5}{13} \sqrt{169-x^2}$   represents the lower half of the ellipse.

\end{itemize}


For number \ref{oddellipsethree}:

\begin{itemize}

\item  $f(x) = -3 + \frac{3}{2} \sqrt{4x-x^2}$ represents the upper half of the ellipse.

\item  $g(x) =   -3 - \frac{3}{2} \sqrt{4x-x^2}$   represents the lower half of the ellipse.

\end{itemize}


For number \ref{oddellipsefive}:

\begin{itemize}

\item  $f(x) = 3 + \frac{1}{10} \sqrt{990+220x-110x^2}$ represents the upper half of the ellipse.

\item  $g(x) =   3 - \frac{1}{10} \sqrt{990+220x-110x^2}$   represents the lower half of the ellipse.

\end{itemize}


For number \ref{oddellipseseven}:

\begin{itemize}

\item  $f(x) = 5 + \frac{1}{2} \sqrt{60-20x-5x^2}$ represents the upper half of the ellipse.

\item  $g(x) =  5 - \frac{1}{2} \sqrt{60-20x-5x^2}$  represents the lower half of the ellipse.

\end{itemize}


For number \ref{oddellipsenine}:

\begin{itemize}

\item  $f(x) = 1 + \frac{3}{5} \sqrt{16+6x-x^2}$ represents the upper half of the ellipse.

\item  $g(x) =  1 - \frac{3}{5} \sqrt{16+6x-x^2} $   represents the lower half of the ellipse.

\end{itemize}


For number \ref{oddellipseeleven}:

\begin{itemize}

\item  $f(x) = -2 + \frac{1}{6} \sqrt{90+60x-10x^2}$ represents the upper half of the ellipse.

\item  $g(x) =  -2 -  \frac{1}{6} \sqrt{90+60x-10x^2}  $   represents the lower half of the ellipse.

\end{itemize}

\newpage


For number \ref{oddellipsethirteen}:

\begin{itemize}

\item  $f(x) = \frac{1}{2} + \frac{3}{2} \sqrt{1-x^2} $ represents the upper half of the ellipse.

\item  $g(x) =  \frac{1}{2} - \frac{3}{2} \sqrt{1-x^2} $   represents the lower half of the ellipse.

\end{itemize}


\setcounter{HW}{\value{enumi}}
\end{enumerate}

\begin{multicols}{2}
\begin{enumerate}
\setcounter{enumi}{\value{HW}}

\item $f(x) = \sqrt{16-4x^2}$

\begin{mfpic}[15]{-4}{4}{-1}{5}
\axes
\tlabel[cc](4,-0.5){\scriptsize $x$}
\tlabel[cc](0.5,5){\scriptsize $y$}
\tlabel[cc](-2, -0.5){\scriptsize $(-2,0)$}
\tlabel[cc](2, -0.5){\scriptsize $(2, 0)$}
\tlabel[cc](-1, 4.5){\scriptsize $(0,4)$}
\xmarks{-3 step 1 until 3}
\ymarks{0 step 1 until 4}
\tlpointsep{4pt}
\scriptsize
%\axislabels {x}{ {$-3 \hspace{7pt}$} -3,  {$1$} 1,  {$3$} 3}
\axislabels {y}{ {$1$} 1, {$2$} 2, {$3$} 3}
\penwd{1.25pt}
\function{-2,2,0.1}{sqrt(16-4*(x**2))}
\point[4pt]{(-2,0), (0,4), (2,0)}
\normalsize
\end{mfpic} 

\vfill

\columnbreak

\item $g(x) = -\frac{1}{2} \sqrt{6x-x^2}$

\begin{mfpic}[15]{-1}{7}{-4.5}{1.5}
\axes
\tlabel[cc](7,-0.5){\scriptsize $x$}
\tlabel[cc](0.5,1.5){\scriptsize $y$}
\tlabel[cc](3, -2.25){\scriptsize $\left(3,-\frac{3}{2} \right)$}
\tlabel[cc](-0.5, 0.75){\scriptsize $(0,0)$}
\tlabel[cc](6, 0.5){\scriptsize $(6,0)$}
\xmarks{1 step 1 until 6}
\ymarks{-4 step 1 until -1}
\tlpointsep{4pt}
\scriptsize
\axislabels {x}{{$1$} 1, {$2$} 2, {$3$} 3, {$4$} 4, {$5$} 5}
\axislabels {y}{{$-1$} -1, {$-2$} -2, {$-3$} -3, {$-4$} -4}
\penwd{1.25pt}
\function{0, 6, 0.1}{-0.5*sqrt(6*x-(x**2))}
\point[4pt]{(0,0), (3,-1.5), (6,0)}
\normalsize
\end{mfpic} 

\setcounter{HW}{\value{enumi}}
\end{enumerate}
\end{multicols}



\begin{multicols}{2}
\begin{enumerate}
\setcounter{enumi}{\value{HW}}

\item  $f(x) = -2\sqrt{3-2x-x^2}$

\begin{mfpic}[20]{-4}{4}{-6}{2}
\axes
\tlabel[cc](4,-0.5){\scriptsize $x$}
\tlabel[cc](0.5,2){\scriptsize $y$}
\tlabel[cc](-3, 0.5){\scriptsize $(-3,0)$}
\tlabel[cc](-1.25, -4.5){\scriptsize $(-1,-4)$}
\tlabel[cc](1, 0.5){\scriptsize $(1,0)$}
\xmarks{-3 step 1 until 3}
\ymarks{-5 step 1 until 1}
\tlpointsep{4pt}
\scriptsize
\axislabels {x}{ {$-1 \hspace{7pt}$} -1, {$-2 \hspace{7pt}$} -2, {$2$} 2,{$3$} 3}
\axislabels {y}{ {$1$} 1, {$-2$} -2,{$-3$} -3,  {$-1$} -1,  {$-5$} -5}
\penwd{1.25pt}
\function{-3,1,0.1}{-2*sqrt(3-2*x-(x**2))}
\point[4pt]{(-3,0), (-1,-4), (1,0)}
\normalsize
\end{mfpic} 



\item  $g(x) = -2 + 2\sqrt{9-x^2}$

\begin{mfpic}[20]{-4}{4}{-3}{5}
\axes
\tlabel[cc](4,-0.5){\scriptsize $x$}
\tlabel[cc](0.5,5){\scriptsize $y$}
\tlabel[cc](1,4.25){\scriptsize $(0,4)$}
\tlabel[cc](-3, -2.5){\scriptsize $(-3,-2)$}
\tlabel[cc](3, -2.5){\scriptsize $(3,-2)$}
\xmarks{-3 step 1 until 3}
\ymarks{-2 step 1 until 4}
\tlpointsep{4pt}
\scriptsize
\axislabels {x}{ {$-1 \hspace{7pt}$} -1, {$-2 \hspace{7pt}$} -2, {$-3 \hspace{7pt}$} -3, {$2$} 2, {$3$} 3, {$1$} 1}
\axislabels {y}{ {$2$} 2,{$3$} 3,{$1$} 1, {$-1$} -1, {$-2$} -2}
\penwd{1.25pt}
\function{-3,3,0.1}{-2+2*sqrt(9-(x**2))}
\point[4pt]{(0,4), (-3,-2), (3,-2)}
\normalsize
\end{mfpic} 

\setcounter{HW}{\value{enumi}}
\end{enumerate}
\end{multicols}


\begin{multicols}{2}
\begin{enumerate}
\setcounter{enumi}{\value{HW}}

\item  $\dfrac{(x-1)^2}{9} + \dfrac{y^2}{16} = 1$

\item $\dfrac{(x-4)^2}{16} + \dfrac{(y-4)^2}{9} = 1$

\setcounter{HW}{\value{enumi}}
\end{enumerate}
\end{multicols}

\begin{multicols}{2}
\begin{enumerate}
\setcounter{enumi}{\value{HW}}

\item $y = 3 - \frac{3}{4} \sqrt{16-x^2}$

\item  $y = 2 \sqrt{8x-x^2-12}$ \vphantom{ $y = 3 - \frac{3}{4} \sqrt{16-x^2}$}

\setcounter{HW}{\value{enumi}}
\end{enumerate}
\end{multicols}

\begin{multicols}{2}
\begin{enumerate}
\setcounter{enumi}{\value{HW}}

\item $\dfrac{(x - 3)^{2}}{9} + \dfrac{(y - 7)^{2}}{25} = 1$
\item $\dfrac{x^{2}}{39} + \dfrac{y^{2}}{64} = 1$


\setcounter{HW}{\value{enumi}}
\end{enumerate}
\end{multicols}



\begin{multicols}{2}
\begin{enumerate}
\setcounter{enumi}{\value{HW}}

\item $\dfrac{x^{2}}{34} + \dfrac{y^{2}}{25} = 1$
\item $\dfrac{(x - 8)^{2}}{25} + \dfrac{(y - 2)^{2}}{4} = 1$

\setcounter{HW}{\value{enumi}}
\end{enumerate}
\end{multicols}

\begin{multicols}{2}
\begin{enumerate}
\setcounter{enumi}{\value{HW}}

\item $\dfrac{(x-5)^{2}}{25} + \dfrac{4(y-2)^{2}}{75} = 1$
\item $\dfrac{(x - 8)^{2}}{64} + \dfrac{(y + 9)^{2}}{81} = 1$

\setcounter{HW}{\value{enumi}}
\end{enumerate}
\end{multicols}

\begin{enumerate}
\setcounter{enumi}{\value{HW}}

\item  Jamie and Jason should stand $100-25\sqrt{7} \approx 33.86$ feet from opposite ends of the gallery.

\item  The arch can be modeled by the upper half of $\frac{x^2}{9} + \frac{y^2}{81} = 1$.  One foot in from the base of the arch corresponds to either $x = \pm 2$.  Plugging in $x = \pm 2$ gives $y = \pm 3\sqrt{5}$ and since $y$ represents a height, we choose $y=3\sqrt{5} \approx 6.71$ feet. 

\item Distance from the sun to aphelion $\approx 1.0167$ AU.\\
Distance from the sun to perihelion $\approx 0.9833$ AU.

\item  $\dfrac{(x-7.4)^2}{1296} + \dfrac{y^2}{1241.24} = 1$.  Graphing this equation\footnote{\ldots using the `Zoom Square' setting \ldots} reveals a very `round' orbit.

\end{enumerate}


\end{document}
