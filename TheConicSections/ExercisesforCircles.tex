\documentclass{ximera}

\begin{document}
	\author{Stitz-Zeager}
	\xmtitle{Exercises for Circles}{}

\mfpicnumber{1} \opengraphsfile{ExercisesforCircles} % mfpic settings added 


\label{ExercisesforCircles}

In Exercises \ref{circleeqnfirst} - \ref{circleeqnlast}, graph the circle in the $xy$-plane.  Find the center and radius.

\begin{multicols}{2}
\begin{enumerate}

\item $(x + 1)^{2} + (y + 5)^{2} = 100$ \label{circleeqnfirst} \label{oddcircleone}
\item $(x-4)^2+(y+2)^2 = 9$

\setcounter{HW}{\value{enumi}}
\end{enumerate}
\end{multicols}

\begin{multicols}{2}
\begin{enumerate}
\setcounter{enumi}{\value{HW}}

\item $\left(x + 3\right)^{2} + \left(y - \frac{7}{13}\right)^{2} = \frac{1}{4}$ \label{oddcirclethree}

\item $(x - 5)^{2} + (y + 9)^{2} = (\ln(8))^{2}$


\setcounter{HW}{\value{enumi}}
\end{enumerate}
\end{multicols}

\begin{multicols}{2}
\begin{enumerate}
\setcounter{enumi}{\value{HW}}


\item $(x  + e)^{2} + \left(y - \sqrt{2} \right)^{2} = \pi^{2}$  \label{oddcirclefive}

\item $\left(x - \pi \right)^{2} + \left(y -  e^{2}\right)^{2} = 91^{\frac{2}{3}}$ \label{circleeqnlast}

\setcounter{HW}{\value{enumi}}
\end{enumerate}
\end{multicols}


In Exercises \ref{ctscirclefirst} - \ref{ctscirclelast}, complete the square in order to put the equation into standard form.  Identify the center and the radius or explain why the equation does not represent a circle.\footnote{\ldots assuming the equation were graphed in the $xy$-plane.}


\begin{multicols}{2}
\begin{enumerate}
\setcounter{enumi}{\value{HW}}

\item $x^{2} - 4x + y^{2} + 10y = -25$  \label{ctscirclefirst}  \label{oddcircleseven}
\item $-2x^{2} - 36x - 2y^{2} - 112 = 0$

\setcounter{HW}{\value{enumi}}
\end{enumerate}
\end{multicols}

\begin{multicols}{2}
\begin{enumerate}
\setcounter{enumi}{\value{HW}}


\item $3x^2+3y^2+24x-30y -3 =0$  \label{oddcirclenine}
\item $x^2+y^2+5x-y-1=0$

\setcounter{HW}{\value{enumi}}
\end{enumerate}
\end{multicols}

\begin{multicols}{2}
\begin{enumerate}
\setcounter{enumi}{\value{HW}}


\item $x^{2} + x + y^{2} - \frac{6}{5}y = 1$  \label{oddcircleeleven}
\item $4x^{2} + 4y^{2} - 24y + 36 = 0$ \label{ctscirclelast}

\setcounter{HW}{\value{enumi}}
\end{enumerate}
\end{multicols}

\begin{enumerate}
\setcounter{enumi}{\value{HW}}

\item For each of the odd numbered equations given in Exercises \ref{oddcircleone} - \ref{oddcircleeleven}, find two or more explicit functions of $x$ represented by each of the equations.  (See Example \ref{horizontalparabolaex} in Section \ref{Parabolas}.)

\setcounter{HW}{\value{enumi}}
\end{enumerate}


In Exercises \ref{semicirclefunctionfirst} - \ref{semicirclefunctionlast}, graph each function by recognizing it as a semicircle.

\begin{multicols}{2}
\begin{enumerate}
\setcounter{enumi}{\value{HW}}

\item   $f(x) = \sqrt{4-x^2}$ \label{semicirclefunctionfirst}
\item   $g(x) = -\sqrt{6x-x^2}$

\setcounter{HW}{\value{enumi}}
\end{enumerate}
\end{multicols}

\begin{multicols}{2}
\begin{enumerate}
\setcounter{enumi}{\value{HW}}

\item  $f(x) = -\sqrt{3-2x-x^2}$
\item  $g(x) = -2 + \sqrt{9-x^2}$ \label{semicirclefunctionlast}

\setcounter{HW}{\value{enumi}}
\end{enumerate}
\end{multicols}

In Exercises \ref{buildcirclefromgraphfirst} - \ref{buildcirclefromgraphlast}, find an equation for the circle or semicircle whose graph is given.

\begin{multicols}{2}
\begin{enumerate}
\setcounter{enumi}{\value{HW}}

\item $~$ \label{buildcirclefromgraphfirst}

\begin{mfpic}[13]{-4}{6}{-5}{5}
\axes
\tlabel[cc](6,-0.5){\scriptsize $x$}
\tlabel[cc](0.5,5){\scriptsize $y$}
\tlabel[cc](1, 3.5){\scriptsize $(1,3)$}
\tlabel[cc](1.25, -3.75){\scriptsize $(1,-3)$}
\tlabel[cc](-3, 0.75){\scriptsize $(-2,0)$}
\tlabel[cc](5, 0.75){\scriptsize $(4,0)$}
\xmarks{-3 step 1 until 5}
\ymarks{-4 step 1 until 4}
\tlpointsep{4pt}
\scriptsize
\axislabels {x}{ {$-3 \hspace{7pt}$} -3,  {$-1 \hspace{7pt}$} -1, {$1$} 1, {$2$} 2, {$3$} 3, {$5$} 5}
\axislabels {y}{ {$-4$} -4, {$-2$} -2, {$-1$} -1, {$1$} 1, {$2$} 2,  {$4$} 4  }
\penwd{1.25pt}
\circle{(1,0), 3}
\point[4pt]{(-2,0), (4,0), (1,3), (1,-3)}
\normalsize
\end{mfpic} 

\vfill

\columnbreak

\item $~$

\begin{mfpic}[13]{-1}{9}{-1}{9}
\axes
\tlabel[cc](9,-0.5){\scriptsize $x$}
\tlabel[cc](0.5,9){\scriptsize $y$}
\tlabel[cc](1, 4){\scriptsize $(0,4)$}
\tlabel[cc](7, 4){\scriptsize $(8,4)$}
\tlabel[cc](4, 8.75){\scriptsize $(4,8)$}
\tlabel[cc](4, 0.75){\scriptsize $(4,0)$}
\xmarks{1 step 1 until 8}
\ymarks{1 step 1 until 8}
\tlpointsep{4pt}
\scriptsize
\axislabels {x}{{$1$} 1, {$2$} 2, {$3$} 3, {$4$} 4, {$5$} 5, {$6$} 6, {$7$} 7, {$8$} 8}
\axislabels {y}{{$1$} 1, {$2$} 2, {$3$} 3, {$4$} 4, {$5$} 5, {$6$} 6, {$7$} 7, {$8$} 8}
\penwd{1.25pt}
\circle{(4,4), 4}
\point[4pt]{(4,0), (0,4), (4,8), (8,4)}
\normalsize
\end{mfpic} 

\setcounter{HW}{\value{enumi}}
\end{enumerate}
\end{multicols}



\begin{multicols}{2}
\begin{enumerate}
\setcounter{enumi}{\value{HW}}


\item $~$   

\begin{mfpic}[13]{-5}{5}{-1}{6}
\axes
\tlabel[cc](5,-0.5){\scriptsize $x$}
\tlabel[cc](0.5,6){\scriptsize $y$}
\tlabel[cc](-4, 4.5){\scriptsize $(-4,4)$}
\tlabel[cc](4, 4.5){\scriptsize $(4,4)$}
\tlabel[cc](0.75, -0.75){\scriptsize $(0, 0)$}
%\tlabel[cc](-0.5,-1){\scriptsize $\left(0, \frac{1}{2} \right)$}
\xmarks{-4,-3,-2,-1,1,2,3,4}
\ymarks{1 step 1 until 5}
\tlpointsep{4pt}
\scriptsize
\axislabels {x}{ {$-4 \hspace{7pt}$} -4, {$-3 \hspace{7pt}$} -3, {$-2 \hspace{7pt}$} -2, {$-1 \hspace{7pt}$} -1,  {$4$} 4,  {$3$} 3,  {$2$} 2}
\axislabels {y}{{$1$} 1, {$2$} 2, {$3$} 3,  {$4$} 4,  {$5$} 5}
\penwd{1.25pt}
\function{-4,4,0.1}{4-sqrt(16-(x**2))}
\point[4pt]{(-4,4), (0,0), (4,4)}
%\tcaption{ \scriptsize $x$,$y$-intercept $(0,0)$}
\normalsize
\end{mfpic} 

\vfill

\columnbreak

\item $~$ \label{buildcirclefromgraphlast} 

\begin{mfpic}[13]{-1}{9}{-1}{6}
\axes
\tlabel[cc](9,-0.5){\scriptsize $x$}
\tlabel[cc](0.5,6){\scriptsize $y$}
\tlabel[cc](8, -0.75){\scriptsize $(8,0)$}
\tlabel[cc](4, 4.5){\scriptsize $(4,4)$}
\tlabel[cc](-0.75, -0.75){\scriptsize $(0, 0)$}
%\tlabel[cc](-0.5,-1){\scriptsize $\left(0, \frac{1}{2} \right)$}
\xmarks{1 step 1 until 8}
\ymarks{1 step 1 until 5}
\tlpointsep{4pt}
\scriptsize
\axislabels {x}{{$1$} 1,  {$2$} 2,  {$3$} 3,  {$4$} 4,  {$5$} 5,  {$6$} 6, {$7$} 7 }
\axislabels {y}{{$1$} 1, {$2$} 2, {$3$} 3,  {$4$} 4,  {$5$} 5}
\penwd{1.25pt}
\function{0,8,0.1}{sqrt(8*x-(x**2))}
\point[4pt]{(8,0), (0,0), (4,4)}
%\tcaption{ \scriptsize $x$,$y$-intercept $(0,0)$}
\normalsize
\end{mfpic} 


\setcounter{HW}{\value{enumi}}
\end{enumerate}
\end{multicols}


In Exercises \ref{buildcirclefirst} - \ref{buildcirclelast}, find the standard equation of the circle which satisfies the given criteria.

\begin{multicols}{2}
\begin{enumerate}
\setcounter{enumi}{\value{HW}}

\item center $(3, 5)$,  passes through $(-1, -2)$ \label{buildcirclefirst}

\item  center $(3, 6)$,  passes through  $(-1, 4)$

\setcounter{HW}{\value{enumi}}
\end{enumerate}
\end{multicols}

\begin{multicols}{2}
\begin{enumerate}
\setcounter{enumi}{\value{HW}}

\item  endpoints of a diameter: $(3,6)$ and $(-1,4)$

\item endpoints of a diameter:  $\left( \frac{1}{2}, 4\right)$, $\left(\frac{3}{2}, -1\right)$  \label{buildcirclelast}

\setcounter{HW}{\value{enumi}}
\end{enumerate}
\end{multicols}


\begin{enumerate}
\setcounter{enumi}{\value{HW}}

\item The Giant Wheel at Cedar Point is a circle with diameter 128 feet which sits on an 8 foot tall platform making its overall height is 136 feet.\footnote{Source: \href{http://www.cedarpoint.com/public/park/rides/tranquil/giant_wheel.cfm}{\underline{Cedar Point's webpage}}.}  Find an equation for the wheel assuming that its center lies on the $y$-axis and that the ground is the $x$-axis.
\label{giantwheelcircle}

\item Verify that the following points lie on the Unit Circle:

 $(\pm 1, 0)$, $(0, \pm 1)$, $\left(\pm \frac{\sqrt{2}}{2}, \pm \frac{\sqrt{2}}{2}\right)$, $\left(\pm \frac{1}{2}, \pm \frac{\sqrt{3}}{2}\right)$ and  $\left(\pm \frac{\sqrt{3}}{2}, \pm \frac{1}{2}\right)$


\item \label{circletransunitcircleexercise} Discuss with your classmates how to obtain the alternate standard equation of a circle, Equation \ref{standardcirclealternate}, from the equation of the Unit Circle, $x^2+y^2=1$ using the transformations discussed in Section \ref{Transformations}.  (Thus every circle is just a few transformations away from the Unit Circle.)

\item Find a one-to-one function whose graph is half of a circle. 

HINT:  Think piecewise \ldots

\end{enumerate}

\newpage

\subsection{Answers}

\begin{multicols}{2}
\begin{enumerate}


\item Center $(-1, -5)$, radius $10$ \\

\begin{mfpic}[6]{-12}{10}{-16}{6}
\axes
\plotsymbol[4pt]{Cross}{(-1,-5)}
\xmarks{-11,-1,9}
\ymarks{-15,-5,5}
\tlabel(10,-0.5){\scriptsize $x$}
\tlabel(0.5,6){\scriptsize $y$}
\tlabel(0.5,-5.25){\tiny $-5$}
\tlpointsep{4pt}
\tiny
\axislabels {x}{{$-11 \hspace{6pt}$} -11, {$-1 \hspace{6pt}$} -1, {$9$} 9}
\axislabels {y}{{$-15$} -15, {$5$} 5}
\normalsize
\penwd{1.25pt}
\circle{(-1,-5),10}
\end{mfpic}

\vfill

\columnbreak

\item  Center $(4,-2)$, radius $3$ \\
 
\begin{mfpic}[15.5]{-1}{8}{-6}{2}
\axes
\plotsymbol[4pt]{Cross}{(4,-2)}
\xmarks{1,4,7}
\ymarks{-5,-2,1}
\tlabel(8,-0.5){\scriptsize $x$}
\tlabel(0.5,2){\scriptsize $y$}
\tlpointsep{4pt}
\tiny
\axislabels {x}{{$1$} 1,{$4$} 4,{$7$} 7}
\axislabels {y}{{$-5$} -5, {$-2$} -2, {$1$} 1 }
\normalsize
\penwd{1.25pt}
\circle{(4,-2),3}
\end{mfpic}

\setcounter{HW}{\value{enumi}}
\end{enumerate}
\end{multicols}

\begin{multicols}{2}
\begin{enumerate}
\setcounter{enumi}{\value{HW}}

\item Center $\left(-3, \frac{7}{13}\right)$, radius $\frac{1}{2}$ \\

\begin{mfpic}[35]{-4}{1}{-0.75}{2}
\axes
\plotsymbol[4pt]{Cross}{(-3,0.53846)}
\xmarks{-3.5,-3,-2.5}
\ymarks{0.03846, 0.53836, 1.03846}
\tlabel(1,-0.25){\scriptsize $x$}
\tlabel(0.25,2){\scriptsize $y$}
\tlpointsep{4pt}
\tiny
\axislabels {x}{{$-\frac{7}{2} \hspace{6pt}$} -3.5, {$-3 \hspace{6pt}$} -3, {$-\frac{5}{2} \hspace{6pt}$} -2.5}
\axislabels {y}{{$\frac{1}{26}$} 0.03846, {$\frac{7}{13}$} 0.53846, {$\frac{27}{26}$} 1.03846}
\normalsize
\penwd{1.25pt}
\circle{(-3,0.53846),0.5}
\end{mfpic}

\vfill

\columnbreak

\item Center $(5, -9)$, radius $\ln(8)$ \\

\begin{mfpic}[10]{-1}{8}{-12}{1}
\axes
\plotsymbol[4pt]{Cross}{(5,-9)}
\xmarks{2.92055, 5, 7.07944}
\ymarks{-11.07944, -9, -6.92055}
\tlabel(8,0.5){\scriptsize $x$}
\tlabel(0.5,1){\scriptsize $y$}
\tlpointsep{4pt}
\tiny
\axislabels {x}{{$5 - \ln(8)$} 2.92055, {$5$} 5, {$5 + \ln(8)$} 7.07944}
\axislabels {y}{{$-9 - \ln(8)$} -11.07944, {$-9$} -9, {$-9 + \ln(8)$} -6.92055}
\normalsize
\penwd{1.25pt}
\circle{(5, -9),2.0794}
\end{mfpic}

\setcounter{HW}{\value{enumi}}
\end{enumerate}
\end{multicols}

\begin{multicols}{2}
\begin{enumerate}
\setcounter{enumi}{\value{HW}}

\item Center $\left(-e, \sqrt{2}\right)$, radius $\pi$ \\
 

\begin{mfpic}[10]{-7}{3}{-3}{6}
\axes
\plotsymbol[4pt]{Cross}{(-2.71828, 1.41421)}
\xmarks{-5.85987, -2.71828, 0.42331}
\ymarks{-1.72738,1.41421, 4.55581}
\tlabel(3,0.5){\scriptsize $x$}
\tlabel(0.5,6){\scriptsize $y$}
\tlpointsep{4pt}
\tiny
\axislabels {x}{{$-e-\pi$} -6.85987, {$-e$} -2.71828, {$-e+\pi$} 1.42331}
\tlabel(0.5,-2.22738){$\sqrt{2}-\pi$}
\tlabel(0.5,1.41421){$\sqrt{2}$}
\tlabel(0.5,4.55581){$\sqrt{2}+\pi$}
\normalsize
\penwd{1.25pt}
\circle{(-2.71828, 1.41421),3.14159}
\end{mfpic}

\vfill

\columnbreak

\item Center $(\pi, e^{2})$, radius $\sqrt[3]{91}$ \\

\begin{mfpic}[10]{-2}{8.25}{-0.25}{13}
\axes
\plotsymbol[4pt]{Cross}{(3.14159,7.389)}
\xmarks{-1.3563, 3.14159, 7.6395}
\ymarks{2.8911, 7.389, 11.88699}
\tlabel(8.25,0.5){\scriptsize $x$}
\tlabel(0.25,13){\scriptsize $y$}
\tlpointsep{4pt}
\tiny
\axislabels {x}{{$\pi - \sqrt[3]{91}$} -1.3563, {$\pi$} 3.14159, {$\pi + \sqrt[3]{91}$} 7.6395}
\axislabels {y}{{$e^{2} - \sqrt[3]{91}$} 2.8911, {$e^{2}$} 7.389, {$e^{2} + \sqrt[3]{91}$} 11.88699}
\normalsize
\penwd{1.25pt}
\circle{(3.14159,7.389),4.4979}
\end{mfpic}

\setcounter{HW}{\value{enumi}}
\end{enumerate}
\end{multicols}

\begin{multicols}{2}
\begin{enumerate}
\setcounter{enumi}{\value{HW}}

\item $(x - 2)^{2} + (y + 5)^{2} = 4$\\
Center $(2, -5)$, radius $r = 2$

\item $(x + 9)^{2} + y^{2} = 25$\\
Center $(-9, 0)$, radius $r = 5$

\setcounter{HW}{\value{enumi}}
\end{enumerate}
\end{multicols}

\begin{multicols}{2}
\begin{enumerate}
\setcounter{enumi}{\value{HW}}

\item $(x+4)^2 + (y-5)^2 = 42$ \\
Center $(-4,5)$, radius $r = \sqrt{42}$

\item $\left(x + \frac{5}{2}\right)^2 + \left(y - \frac{1}{2}\right)^2 = \frac{30}{4}$ \\
Center $\left( -\frac{5}{2}, \frac{1}{2}\right)$, radius $r = \frac{\sqrt{30}}{2}$

\setcounter{HW}{\value{enumi}}
\end{enumerate}
\end{multicols}

\begin{multicols}{2}
\begin{enumerate}
\setcounter{enumi}{\value{HW}}

\item $\left(x + \frac{1}{2}\right)^{2} + \left(y - \frac{3}{5}\right)^{2} = \frac{161}{100}$\\
Center $\left(-\frac{1}{2}, \frac{3}{5}\right)$, radius $r = \frac{\sqrt{161}}{10}$

\item $x^{2} + (y - 3)^{2} = 0$\\
This is not a circle.

\setcounter{HW}{\value{enumi}}
\end{enumerate}
\end{multicols}

\begin{enumerate}
\setcounter{enumi}{\value{HW}}

\item $~$


For number \ref{oddcircleone}:

\begin{itemize}

\item  $f(x) = -5 + \sqrt{99-2x-x^2}$ represents the upper semicircle.

\item  $g(x) = -5 - \sqrt{99-2x-x^2}$ represents the lower semicircle.

\end{itemize}

For number \ref{oddcirclethree}:

\begin{itemize}

\item  $f(x) = \frac{7}{13} + \frac{1}{2} \sqrt{-4x^2-24x-35}$ represents the upper semicircle.

\item  $g(x) = \frac{7}{13} - \frac{1}{2} \sqrt{-4x^2-24x-35}$ represents the lower semicircle.

\end{itemize}

For number \ref{oddcirclefive}:

\begin{itemize}

\item  $f(x) = \sqrt{2} + \sqrt{\pi^2-e^2-2ex-x^2}$ represents the upper semicircle.

\item  $g(x) = \sqrt{2} - \sqrt{\pi^2-e^2-2ex-x^2}$  represents the lower semicircle.

\end{itemize}


For number \ref{oddcircleseven}:

\begin{itemize}

\item  $f(x) = -5 + \sqrt{4x-x^2}$ represents the upper semicircle.

\item  $g(x) =  -5 - \sqrt{4x-x^2}$   represents the lower semicircle.

\end{itemize}

For number \ref{oddcirclenine}:

\begin{itemize}

\item  $f(x) = 5 + \sqrt{26-8x-x^2}$ represents the upper semicircle.

\item  $g(x) =  5 - \sqrt{26-8x-x^2}$   represents the lower semicircle.

\end{itemize}

For number \ref{oddcircleeleven}:

\begin{itemize}

\item  $f(x) = \frac{3}{5} + \frac{1}{5} \sqrt{34-25x-25x^2}$ represents the upper semicircle.

\item  $g(x) =  \frac{3}{5} - \frac{1}{5} \sqrt{34-25x-25x^2}$   represents the lower semicircle.

\end{itemize}


\setcounter{HW}{\value{enumi}}
\end{enumerate}

\begin{multicols}{2}
\begin{enumerate}
\setcounter{enumi}{\value{HW}}

\item $f(x) = \sqrt{4-x^2}$

\begin{mfpic}[15]{-5}{5}{-1}{5}
\axes
\tlabel[cc](5,-0.5){\scriptsize $x$}
\tlabel[cc](0.5,5){\scriptsize $y$}
\tlabel[cc](-4, -0.5){\scriptsize $(-2,0)$}
\tlabel[cc](4, -0.5){\scriptsize $(2, 0)$}
\tlabel[cc](-1, 4.5){\scriptsize $(0,2)$}
\xmarks{-4 step 2 until 4}
\ymarks{0 step 2 until 4}
\tlpointsep{4pt}
\scriptsize
\axislabels {x}{ {$-1 \hspace{7pt}$} -2,  {$1$} 2}
\axislabels {y}{ {$1$} 2 }
\penwd{1.25pt}
\function{-4,4,0.1}{sqrt(16-(x**2))}
\point[4pt]{(-4,0), (0,4), (4,0)}
\normalsize
\end{mfpic} 

\vfill

\columnbreak

\item $g(x) = -\sqrt{6x-x^2}$

\begin{mfpic}[15]{-1}{7}{-4.5}{1.5}
\axes
\tlabel[cc](7,-0.5){\scriptsize $x$}
\tlabel[cc](0.5,1.5){\scriptsize $y$}
\tlabel[cc](3, -3.5){\scriptsize $(3,-3)$}
\tlabel[cc](-0.75, 0.75){\scriptsize $(0,0)$}
\tlabel[cc](6, 0.75){\scriptsize $(6,0)$}
\xmarks{1 step 1 until 6}
\ymarks{-4 step 1 until -1}
\tlpointsep{4pt}
\scriptsize
\axislabels {x}{{$1$} 1, {$2$} 2, {$3$} 3, {$4$} 4, {$5$} 5}
\axislabels {y}{{$-1$} -1, {$-2$} -2, {$-3$} -3, {$-4$} -4}
\penwd{1.25pt}
\function{0, 6, 0.1}{-sqrt(6*x-(x**2))}
\point[4pt]{(0,0), (3,-3), (6,0)}
\normalsize
\end{mfpic} 

\setcounter{HW}{\value{enumi}}
\end{enumerate}
\end{multicols}



\begin{multicols}{2}
\begin{enumerate}
\setcounter{enumi}{\value{HW}}

\item  $f(x) = -\sqrt{3-2x-x^2}$

\begin{mfpic}[20]{-4}{4}{-4}{4}
\axes
\tlabel[cc](4,-0.5){\scriptsize $x$}
\tlabel[cc](0.5,4){\scriptsize $y$}
\tlabel[cc](-3, 0.5){\scriptsize $(-3,0)$}
\tlabel[cc](-1.25, -2.5){\scriptsize $(-1,-2)$}
\tlabel[cc](1, 0.5){\scriptsize $(1,0)$}
\xmarks{-3 step 1 until 3}
\ymarks{-3 step 1 until 3}
\tlpointsep{4pt}
\scriptsize
\axislabels {x}{ {$-1 \hspace{7pt}$} -1, {$-2 \hspace{7pt}$} -2, {$2$} 2,{$3$} 3}
\axislabels {y}{ {$1$} 1, {$2$} 2,{$3$} 3,  {$-1$} -1}
\penwd{1.25pt}
\function{-3,1,0.1}{-sqrt(3-2*x-(x**2))}
\point[4pt]{(-3,0), (-1,-2), (1,0)}
\normalsize
\end{mfpic} 



\item  $g(x) = -2 + \sqrt{9-x^2}$

\begin{mfpic}[20]{-4}{4}{-4}{4}
\axes
\tlabel[cc](4,-0.5){\scriptsize $x$}
\tlabel[cc](0.5,4){\scriptsize $y$}
\tlabel[cc](1,1.25){\scriptsize $(0,1)$}
\tlabel[cc](-3, -2.5){\scriptsize $(-3,-2)$}
\tlabel[cc](3, -2.5){\scriptsize $(3,-2)$}
\xmarks{-3 step 1 until 3}
\ymarks{-3 step 1 until 3}
\tlpointsep{4pt}
\scriptsize
\axislabels {x}{ {$-1 \hspace{7pt}$} -1, {$-2 \hspace{7pt}$} -2, {$-3 \hspace{7pt}$} -3, {$2$} 2, {$3$} 3, {$1$} 1}
\axislabels {y}{ {$2$} 2,{$3$} 3, {$-1$} -1, {$-2$} -2, {$-3$} -3}
\penwd{1.25pt}
\function{-3,3,0.1}{-2+sqrt(9-(x**2))}
\point[4pt]{(0,1), (-3,-2), (3,-2)}
\normalsize
\end{mfpic} 

\setcounter{HW}{\value{enumi}}
\end{enumerate}
\end{multicols}


\begin{multicols}{2}
\begin{enumerate}
\setcounter{enumi}{\value{HW}}

\item  $(x-1)^2+y^2=9$

\item  $(x-4)^2+(y-4)^2=16$

\setcounter{HW}{\value{enumi}}
\end{enumerate}
\end{multicols}

\begin{multicols}{2}
\begin{enumerate}
\setcounter{enumi}{\value{HW}}

\item  $y = 4-\sqrt{16-x^2}$

\item $y = \sqrt{8x-x^2}$

\setcounter{HW}{\value{enumi}}
\end{enumerate}
\end{multicols}




\begin{multicols}{2}
\begin{enumerate}
\setcounter{enumi}{\value{HW}}

\item $(x - 3)^{2} + (y - 5)^{2} = 65$

\item  $(x-3)^2+(y-6)^2 = 20$

\setcounter{HW}{\value{enumi}}
\end{enumerate}
\end{multicols}

\begin{multicols}{2}
\begin{enumerate}
\setcounter{enumi}{\value{HW}}

\item  $(x-1)^2 + (y-5)^2 = 5$

\item $(x-1)^2 + \left(y - \frac{3}{2}\right)^2 = \frac{13}{2}$

\setcounter{HW}{\value{enumi}}
\end{enumerate}
\end{multicols}

\begin{enumerate}
\setcounter{enumi}{\value{HW}}

\item $x^{2} + (y - 72)^{2} = 4096$

\end{enumerate}

\end{document}
