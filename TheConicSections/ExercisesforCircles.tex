\documentclass{ximera}

\begin{document}
	\author{Stitz-Zeager}
	\xmtitle{Exercises for Circles}{}

\mfpicnumber{1} \opengraphsfile{ExercisesforCircles} % mfpic settings added 


\label{ExercisesforCircles}

\begin{question}
In Exercises \ref{circleeqnfirst} - \ref{circleeqnlast}, graph the circle in the $xy$-plane.  Find the center and radius.

\begin{problem}\label{circleeqnfirst} \label{oddcircleone}
$(x + 1)^{2} + (y + 5)^{2} = 100$   

\end{problem}

\begin{problem}
$(x-4)^2+(y+2)^2 = 9$ 
\end{problem}

\begin{problem}\label{oddcirclethree}
$\left(x + 3\right)^{2} + \left(y - \frac{7}{13}\right)^{2} = \frac{1}{4}$ 
\end{problem}
 
\begin{problem}
$(x - 5)^{2} + (y + 9)^{2} = (\ln(8))^{2}$  
\end{problem}

\begin{problem}\label{oddcirclefive}
$(x  + e)^{2} + \left(y - \sqrt{2} \right)^{2} = \pi^{2}$ 
\end{problem}

\begin{problem}\label{circleeqnlast}
$\left(x - \pi \right)^{2} + \left(y -  e^{2}\right)^{2} = 91^{\frac{2}{3}}$
\end{problem}
\end{question}

\begin{question}
In Exercises \ref{ctscirclefirst} - \ref{ctscirclelast}, complete the square in order to put the equation into standard form.  Identify the center and the radius or explain why the equation does not represent a circle.\footnote{\ldots assuming the equation were graphed in the $xy$-plane.}

\begin{problem}\label{ctscirclefirst}\label{oddcircleseven}
$x^{2} - 4x + y^{2} + 10y = -25$

\begin{solution}
     $(x - 2)^{2} + (y + 5)^{2} = 4$\\
Center $(2, -5)$, radius $r = 2$
\end{solution}
\end{problem}   

\begin{problem}
$-2x^{2} - 36x - 2y^{2} - 112 = 0$

\begin{solution}
    $(x + 9)^{2} + y^{2} = 25$\\
Center $(-9, 0)$, radius $r = 5$
\end{solution}
\end{problem}

\begin{problem}\label{oddcirclenine}
$3x^2+3y^2+24x-30y -3 =0$

\begin{solution}
    $(x+4)^2 + (y-5)^2 = 42$ \\
Center $(-4,5)$, radius $r = \sqrt{42}$
\end{solution}
\end{problem} 
   
\begin{problem}
$x^2+y^2+5x-y-1=0$

\begin{solution}
    $\left(x + \frac{5}{2}\right)^2 + \left(y - \frac{1}{2}\right)^2 = \frac{30}{4}$ \\
Center $\left( -\frac{5}{2}, \frac{1}{2}\right)$, radius $r = \frac{\sqrt{30}}{2}$
\end{solution}
\end{problem} 

\begin{problem}\label{oddcircleeleven}
$x^{2} + x + y^{2} - \frac{6}{5}y = 1$

\begin{solution}
    $\left(x + \frac{1}{2}\right)^{2} + \left(y - \frac{3}{5}\right)^{2} = \frac{161}{100}$\\
Center $\left(-\frac{1}{2}, \frac{3}{5}\right)$, radius $r = \frac{\sqrt{161}}{10}$
\end{solution}
\end{problem} 

\begin{problem}\label{ctscirclelast}
$4x^{2} + 4y^{2} - 24y + 36 = 0$ 

\begin{solution}
    $x^{2} + (y - 3)^{2} = 0$\\
This is not a circle.
\end{solution}
\end{problem} 
\end{question}

\begin{problem}
For each of the odd numbered equations given in Exercises \ref{oddcircleone} - \ref{oddcircleeleven}, find two or more explicit functions of $x$ represented by each of the equations.  (See Example \ref{horizontalparabolaex} in Section \ref{Parabolas}.) 
\end{problem}

\begin{question}
In Exercises \ref{semicirclefunctionfirst} - \ref{semicirclefunctionlast}, graph each function by recognizing it as a semicircle.

\end{question}

\begin{problem}\label{semicirclefunctionfirst}
$f(x) = \sqrt{4-x^2}$ 
\end{problem} 

\begin{problem}
$g(x) = -\sqrt{6x-x^2}$
\end{problem}  

\begin{problem}
$f(x) = -\sqrt{3-2x-x^2}$
\end{problem}

\begin{problem}\label{semicirclefunctionlast}
$g(x) = -2 + \sqrt{9-x^2}$ 
\end{problem}

\begin{question}
In Exercises \ref{buildcirclefromgraphfirst} - \ref{buildcirclefromgraphlast}, find an equation for the circle or semicircle whose graph is given.

\begin{problem}\label{buildcirclefromgraphfirst}

\begin{tikzpicture}
  \begin{axis}[
      axis lines=middle,
      xmin=-4, xmax=6,
      ymin=-5, ymax=5,
      width=11cm,
      height=8cm,
      xlabel={$x$},
      ylabel={$y$},
      xtick={-3,-1,1,2,3,5},
      ytick={-4,-2,-1,1,2,4},
      ticklabel style={font=\scriptsize},
      label style={font=\scriptsize},
      enlargelimits=false,
      axis equal,
    ]

    % --- Circle: (x - 1)^2 + y^2 = 9 ---
    \addplot[
      thick,
      domain=0:360,
      samples=200,
    ] ({1 + 3*cos(x)}, {3*sin(x)});

    % --- Points on the circle ---
    \addplot[only marks, mark=*] coordinates {(-2,0) (4,0) (1,3) (1,-3)};

    % --- Labels for points ---
    \node[font=\scriptsize, anchor=south west] at (axis cs:1,3) {$(1,3)$};
    \node[font=\scriptsize, anchor=north west] at (axis cs:1,-3) {$(1,-3)$};
    \node[font=\scriptsize, anchor=east] at (axis cs:-2,0) {$(-2,0)$};
    \node[font=\scriptsize, anchor=west] at (axis cs:4,0) {$(4,0)$};

  \end{axis}
\end{tikzpicture}

\end{problem}

\begin{problem}
\begin{tikzpicture}
  \begin{axis}[
      axis lines=middle,
      xmin=-1, xmax=9,
      ymin=-1, ymax=9,
      width=10cm,
      height=10cm,
      xlabel={$x$},
      ylabel={$y$},
      xtick={1,2,3,4,5,6,7,8},
      ytick={1,2,3,4,5,6,7,8},
      ticklabel style={font=\scriptsize},
      label style={font=\scriptsize},
      axis equal,
    ]

    % --- Circle centered at (4,4) with radius 4 ---
    \addplot[
      thick,
      domain=0:360,
      samples=200,
    ] ({4 + 4*cos(x)}, {4 + 4*sin(x)});

    % --- Points on the circle ---
    \addplot[only marks, mark=*] coordinates {(4,0) (0,4) (4,8) (8,4)};

    % --- Labels for the points ---
    \node[font=\scriptsize, anchor=south east] at (axis cs:0,4) {$(0,4)$};
    \node[font=\scriptsize, anchor=south west] at (axis cs:8,4) {$(8,4)$};
    \node[font=\scriptsize, anchor=south] at (axis cs:4,8) {$(4,8)$};
    \node[font=\scriptsize, anchor=north] at (axis cs:4,0) {$(4,0)$};

  \end{axis}
\end{tikzpicture}
   
\end{problem}

\begin{problem}
\begin{tikzpicture}
  \begin{axis}[
      axis lines=middle,
      xmin=-5, xmax=5,
      ymin=-1, ymax=6,
      width=10cm,
      height=6cm,
      xlabel={$x$},
      ylabel={$y$},
      xtick={-4,-3,-2,-1,1,2,3,4},
      ytick={1,2,3,4,5},
      ticklabel style={font=\scriptsize},
      label style={font=\scriptsize},
      axis equal,
    ]

    % --- Upper semicircle: y = 4 - sqrt(16 - x^2) ---
    \addplot[
      thick,
      domain=-4:4,
      samples=200,
    ] {4 - sqrt(16 - x^2)};

    % --- Points on the curve ---
    \addplot[only marks, mark=*] coordinates {(-4,4) (0,0) (4,4)};

    % --- Labels for the points ---
    \node[font=\scriptsize, anchor=south east] at (axis cs:-4,4) {($-4,4$)};
    \node[font=\scriptsize, anchor=south west] at (axis cs:4,4) {$(4,4)$};
    \node[font=\scriptsize, anchor=north east] at (axis cs:0,0) {$(0,0)$};

  \end{axis}
\end{tikzpicture}
\end{problem} 

\begin{problem}\label{buildcirclefromgraphlast} 
\begin{tikzpicture}
  \begin{axis}[
      axis lines=middle,
      xmin=-1, xmax=9,
      ymin=-1, ymax=6,
      width=10cm,
      height=6cm,
      xlabel={$x$},
      ylabel={$y$},
      xtick={1,2,3,4,5,6,7,8},
      ytick={1,2,3,4,5},
      ticklabel style={font=\scriptsize},
      label style={font=\scriptsize},
      axis equal,
    ]

    % --- Curve: y = sqrt(8x - x^2) ---
    \addplot[
      thick,
      domain=0:8,
      samples=200,
    ] {sqrt(8*x - x^2)};

    % --- Points on the curve ---
    \addplot[only marks, mark=*] coordinates {(0,0) (4,4) (8,0)};

    % --- Labels for points ---
    \node[font=\scriptsize, anchor=north east] at (axis cs:0,0) {$(0,0)$};
    \node[font=\scriptsize, anchor=south] at (axis cs:4,4) {$(4,4)$};
    \node[font=\scriptsize, anchor=north west] at (axis cs:8,0) {$(8,0)$};

  \end{axis}
\end{tikzpicture}
\end{problem}
\end{question}

\begin{question}
In Exercises \ref{buildcirclefirst} - \ref{buildcirclelast}, find the standard equation of the circle which satisfies the given criteria.   

\begin{problem}\label{buildcirclefirst}
center $(3, 5)$,  passes through $(-1, -2)$ 

\begin{solution}
    $(x - 3)^{2} + (y - 5)^{2} = 65$
\end{solution}
\end{problem}

\begin{problem}
center $(3, 6)$,  passes through  $(-1, 4)$

\begin{solution}
    $(x-3)^2+(y-6)^2 = 20$
\end{solution}
\end{problem}  

\begin{problem}
endpoints of a diameter: $(3,6)$ and $(-1,4)$

\begin{solution}
    $(x-1)^2 + (y-5)^2 = 5$
\end{solution}
\end{problem} 

\begin{problem}\label{buildcirclelast}
endpoints of a diameter:  $\left( \frac{1}{2}, 4\right)$, $\left(\frac{3}{2}, -1\right)$ 
\end{problem}

\begin{solution}
    $(x-1)^2 + \left(y - \frac{3}{2}\right)^2 = \frac{13}{2}$
\end{solution}
\end{question}

\begin{problem}\label{giantwheelcircle}
The Giant Wheel at Cedar Point is a circle with diameter 128 feet which sits on an 8 foot tall platform making its overall height is 136 feet.\footnote{Source: \href{http://www.cedarpoint.com/public/park/rides/tranquil/giant_wheel.cfm}{\underline{Cedar Point's webpage}}.}  Find an equation for the wheel assuming that its center lies on the $y$-axis and that the ground is the $x$-axis.

\begin{solution}
    $x^{2} + (y - 72)^{2} = 4096$
\end{solution}
\end{problem}

\begin{problem}
Verify that the following points lie on the Unit Circle:

 $(\pm 1, 0)$, $(0, \pm 1)$, $\left(\pm \frac{\sqrt{2}}{2}, \pm \frac{\sqrt{2}}{2}\right)$, $\left(\pm \frac{1}{2}, \pm \frac{\sqrt{3}}{2}\right)$ and  $\left(\pm \frac{\sqrt{3}}{2}, \pm \frac{1}{2}\right)$
\end{problem}

\begin{problem}\label{circletransunitcircleexercise}
Discuss with your classmates how to obtain the alternate standard equation of a circle, Equation \ref{standardcirclealternate}, from the equation of the Unit Circle, $x^2+y^2=1$ using the transformations discussed in Section \ref{Transformations}.  (Thus every circle is just a few transformations away from the Unit Circle.)
\end{problem}

\begin{problem}
Find a one-to-one function whose graph is half of a circle. 

\begin{hint}
Think piecewise \ldots   
\end{hint}
\end{problem}


\end{document}
