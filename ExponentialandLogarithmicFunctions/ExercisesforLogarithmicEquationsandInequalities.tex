\documentclass{ximera}

\begin{document}
	\author{Stitz-Zeager}
	\xmtitle{Exercises for Logarithmic Equations and Inequalities}{}

\mfpicnumber{1} \opengraphsfile{ExercisesforLogarithmicEquationsandInequalities} % mfpic settings added 


\label{ExercisesforLogarithmicEquationsandInequalities}

\begin{question}

In Exercises \ref{solvelogeqexfirst} - \ref{solvelogeqexlast}, solve the equation analytically.

\begin{problem}\label{solvelogeqexfirst}
$\log(3x-1) = \log(4-x)$

$x = \answer{\frac{5}{4}}$
\end{problem}

\begin{problem}
$\log_{2}\left(x^{3}\right) = \log_{2}(x)$
\end{problem}

\begin{problem}
$\ln\left(8-t^2\right)=\ln(2-t)$ 

\begin{selectAll}
    \choice{$t=-3$}
    \choice[correct]{$t=-2$}
    \choice{$t=1$}
    \choice{$t=2$}
    \choice{$t=3$}
    \choice{$t=6$}
  \end{selectAll}
\end{problem}

\begin{problem}
$\log_{5}\left(18-t^2\right) = \log_{5}(6-t)$
\end{problem}


\begin{problem}
$\log_{3}(7-2x) = 2$

$x=\answer{-1}$
\end{problem}

\begin{problem}
$\log_{\frac{1}{2}} (2x-1) = -3$
\end{problem}

\begin{problem}
$\ln\left(t^2-99\right) = 0$

\begin{selectAll}
    \choice{$t=-33$}
    \choice[correct]{$t=-10$}
    \choice{$t=-9$}
    \choice{$t=9$}
    \choice[correct]{$t=10$}
    \choice{$t=33$}
  \end{selectAll}
\end{problem}


\begin{problem}
$\log(t^2-3t) = 1$
\end{problem}

\begin{problem}
$\log_{125} \left(\dfrac{3x-2}{2x+3}\right)=\dfrac{1}{3}$

$x = \answer{-\frac{17}{7}}$
\end{problem}

\begin{problem}\label{sixfourRichterequ}
$\log\left(\dfrac{x}{10^{-3}}\right) = 4.7$
\end{problem}

\begin{problem}\label{sixfourpHequ}
$-\log(x) = 5.4$

\begin{solution}
$x = 10^{-5.4}$
\end{solution}
\end{problem}

\begin{problem}\label{sixfourdecibelequ}
$10\log\left(\dfrac{x}{10^{-12}}\right) = 150$ 
\end{problem}

\begin{problem}
$6-3\log_{5}(2t)=0$

$t=\answer{\frac{25}{2}}$
\end{problem}

\begin{problem}
$3\ln(t)-2=1-\ln(t)$
\end{problem}

\begin{problem}
$\log_{3}(t - 4) + \log_{3}(t + 4) = 2$

\begin{selectAll}
    \choice{$t=-\sqrt{7}$}
    \choice{$t=-5$}
    \choice[correct]{$t=5$}
    \choice{$t=\sqrt{7}$}
  \end{selectAll}
\end{problem}

\begin{problem}
$\log_{5}(2t + 1) + \log_{5}(t + 2) = 1$
\end{problem}

\begin{problem}
$\log_{169}(3x + 7) - \log_{169}(5x - 9) = \dfrac{1}{2}$

$x=\answer{2}$
\end{problem}

\begin{problem}
$\ln(x+1) - \ln(x) = 3$
\end{problem}

\begin{problem}
$2\log_{7}(t) = \log_{7}(2) + \log_{7}(t+12)$

$t=\answer{6}$
\end{problem}

\begin{problem}
$\log(t) - \log(2) = \log(t+8)  - \log(t+2)$
\end{problem}

\begin{problem}
$\log_{3}(x) = \log_{\frac{1}{3}}(x) + 8$

$x = \answer{81}$
\end{problem}

\begin{problem}
$\ln(\ln(x)) = 3$
\end{problem}


\end{question}



%\begin{multicols}{2}
\begin{enumerate}
\setcounter{enumi}{\value{HW}}

\item $\left(\log(t)\right)^2=2\log(t)+15$

\item $\ln(t^{2}) = (\ln(t))^{2}$ \label{solvelogeqexlast}

\setcounter{HW}{\value{enumi}}
\end{enumerate}
%\end{multicols}


In Exercises \ref{solvelogineqexfirst} - \ref{solvelogineqexlast}, solve the inequality analytically.

%\begin{multicols}{2}
\begin{enumerate}
\setcounter{enumi}{\value{HW}}

\item $\dfrac{1 - \ln(t)}{t^{2}} < 0$ \label{solvelogineqexfirst}
\item $t\ln(t) - t > 0$ \phantom{$\dfrac{1 - \ln(x)}{x^{2}} < 0$}  


\setcounter{HW}{\value{enumi}}
\end{enumerate}
%\end{multicols}

%\begin{multicols}{2}
\begin{enumerate}
\setcounter{enumi}{\value{HW}}

\item $10\log\left(\dfrac{x}{10^{-12}}\right) \geq 90$ \label{sixfourdecibelineq} 
\item $5.6 \leq \log\left(\dfrac{x}{10^{-3}}\right) \leq 7.1$ \label{sixfourRichterineq}


\setcounter{HW}{\value{enumi}}
\end{enumerate}
%\end{multicols}

%\begin{multicols}{2}
\begin{enumerate}
\setcounter{enumi}{\value{HW}}


\item $2.3 < -\log(x) < 5.4$ \label{sixfourpHineq} 

\item $\ln(t^{2}) \leq (\ln(t))^{2}$ \label{solvelogineqexlast} 

\setcounter{HW}{\value{enumi}}
\end{enumerate}
%\end{multicols}

\pagebreak

In Exercises \ref{logeqcalcexfirst} - \ref{logeqcalcexlast}, use a graphing utility to help you solve the equation or  inequality.

%\begin{multicols}{2}
\begin{enumerate}
\setcounter{enumi}{\value{HW}}

\item $\ln(t) = e^{-t}$ \label{logeqcalcexfirst} 
\item $\ln(x) = \sqrt[4]{x}$ 

\setcounter{HW}{\value{enumi}}
\end{enumerate}
%\end{multicols}

%\begin{multicols}{2}
\begin{enumerate}
\setcounter{enumi}{\value{HW}}

\item $\ln(t^{2} + 1) \geq 5$
\item $\ln(-2x^{3} - x^{2} + 13x - 6) < 0$ \label{logeqcalcexlast} 

\setcounter{HW}{\value{enumi}}
\end{enumerate}
%\end{multicols}


In Exercises \ref{domaincomplicatedlogfirst} - \ref{domaincomplicatedloglast},  find the domain of the function.

%\begin{multicols}{2} 
\begin{enumerate}
\setcounter{enumi}{\value{HW}}

\item \label{domaincomplicatedlogfirst}  $r(x) =   \dfrac{x}{1 - \ln(x)}$  %(-\infty, e) \cup (e, \infty)$

\item   $R(x) = \dfrac{x \ln(x)}{1 - \ln(x)}$   % $(0,e) \cup (e, \infty)$

\setcounter{HW}{\value{enumi}}
\end{enumerate}
%\end{multicols}


%\begin{multicols}{2} 
\begin{enumerate}
\setcounter{enumi}{\value{HW}}

\item     $s(t) = \sqrt{2 - \log(t)}$  \vphantom{$c(t) =  (2 \ln(t) -1)^{\frac{2}{3}}$} %$(0, 100]$
\item     $c(t) =  (2 \ln(t) -1)^{\frac{2}{3}}$  %$(0, \infty)$

\setcounter{HW}{\value{enumi}}
\end{enumerate}
%\end{multicols}

%\begin{multicols}{2} 
\begin{enumerate}
\setcounter{enumi}{\value{HW}}
  
\item     $\ell(t) = \ln( \ln(t))$  \vphantom{$L(x) = \log\left( \dfrac{x \ln(x)}{1 - \ln(x)} \right)$} %$(1, \infty)$    

\item  \label{domaincomplicatedloglast}    $L(x) = \log\left( \dfrac{x \ln(x)}{1 - \ln(x)} \right)$  %$(1,e)$ 


\setcounter{HW}{\value{enumi}}
\end{enumerate}
%\end{multicols}


\begin{enumerate}
\setcounter{enumi}{\value{HW}}

\item \label{onetooneexpexercise} Since $f(x) = e^{x}$ is a strictly increasing function, if $a < b$ then $e^{a} < e^{b}$.  Use this fact to solve the inequality $\ln(2x + 1) < 3$ without a sign diagram. Use this technique to solve the inequalities in Exercises \ref{sixfourdecibelineq} - \ref{sixfourpHineq}. (Compare this to Exercise  \ref{onetoonelogexercise} in Section \ref{ExponentialEquationsandInequalities}.)

\item Solve $\ln(3 - y) - \ln(y) = 2x + \ln(5)$ for $y$.

\item In Example \ref{logfracinverse} we found the inverse of $f(x) = \dfrac{\log(x)}{1-\log(x)}$ to be $f^{-1}(x) = 10^{\frac{x}{x+1}}$.

\begin{enumerate}

\item Algebraically check our answer by verifying  $\left(f^{-1} \circ f\right)(x) = x$ for all $x$ in the domain of $f$ and that $\left(f \circ f^{-1}\right)(x) = x$ for all $x$ in the domain of $f^{-1}$.

\item Find the range of $f$ by finding the domain of $f^{-1}$.

\item Let $g(x) = \dfrac{x}{1 - x}$ and $h(x) = \log(x)$.  Show that $f = g \circ h$ and $(g \circ h)^{-1} = h^{-1} \circ g^{-1}$.\\


NOTE:  We know this is true in general by Exercise \ref{fcircginverse} in Section \ref{InverseFunctions}, but it's nice to see a specific example of the property.

\end{enumerate}

\item \label{inversehyptangent} Let $f(x) = \dfrac{1}{2}\ln\left(\dfrac{1 + x}{1 - x}\right)$.  Compute $f^{-1}(x)$ and find its domain and range.

\item Explain the equation in Exercise \ref{sixfourRichterequ} and the inequality in Exercise \ref{sixfourRichterineq} above in terms of the Richter scale for earthquake magnitude.  (See Exercise \ref{Richterexercise} in Section \ref{ExponentialFunctions}.)

\item Explain the equation in Exercise \ref{sixfourdecibelequ} and the inequality in Exercise \ref{sixfourdecibelineq} above in terms of sound intensity level as measured in decibels.  (See Exercise \ref{decibelexercise} in Section \ref{ExponentialFunctions}.)

\item Explain the equation in Exercise \ref{sixfourpHequ} and the inequality in Exercise \ref{sixfourpHineq} above in terms of the pH of a solution.  (See Exercise \ref{pHexercise} in Section \ref{ExponentialFunctions}.)

\item \label{powerloggrowthex} \begin{enumerate} \item\label{numericalinvestigationlimitlnxoverx} With the help of your classmates, numerically and graphically investigate $\ds{\lim_{x \rightarrow \infty}}$ $\frac{\ln(x)}{x^{p}}$ for various real number powers, $p > 0$.

 \item\label{numericalinvestigationlimitlnxtimesx} With the help of your classmates, numerically and graphically investigate $\ds{\lim_{x \rightarrow 0^{+}}}$ $x^{p} \, \ln(x) $ for various real number powers, $p > 0$.

\item  What do \ref{numericalinvestigationlimitlnxoverx} and \ref{numericalinvestigationlimitlnxtimesx} suggest about the relative growth rates of powers of $x$ and $\ln(x)$?

\end{enumerate}  

\setcounter{HW}{\value{enumi}}
\end{enumerate}

In Exercises \ref{logcurvesketchfirst}  - \ref{logcurvesketchlast} a function $f$ along with its derivatives $f'$ and $f''$ are given.

\begin{itemize}

\item  Find the domain of $f$.

\item  Find the $x$- and $y$-intercepts of the graph of each function, if any.

\item  Use limits to determine the end behavior and behavior at the endpoints of the domain.

\item  Use $f'$ to determine the open intervals over which $f$ is increasing or decreasing.

\item Determine the local extrema, if any.

\item  Use $f''$ to determine the open intervals over which the graph of $f$  is concave up or concave down.

\item  Determine the inflection points of the graph, if any.

\end{itemize}

\begin{enumerate}
\setcounter{enumi}{\value{HW}}

\item\label{logcurvesketchfirst}  $f(x) = \ln(x) - \ln(5-x)$,  $f'(x) = \dfrac{1}{x} + \dfrac{1}{5-x}$, $f''(x) = \dfrac{1}{(5-x)^2} - \dfrac{1}{x^2}$

\smallskip

\item\label{logcurvesketchlast}  $f(x) = \dfrac{\ln(x)}{x}$, $f'(x) = \dfrac{1 - \ln(x)}{x^2}$, $f''(x) = \dfrac{2 \ln(x) - 3}{x^3}$.


\setcounter{HW}{\value{enumi}}
\end{enumerate}






\end{document}
