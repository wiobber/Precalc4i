\documentclass{ximera}

\begin{document}
	\author{Stitz-Zeager}
	\xmtitle{Exercises for Exponential Equations and Inequalities}{}

\mfpicnumber{1} \opengraphsfile{ExercisesforExponentialEquationsandInequalities} % mfpic settings added 


\label{ExercisesforExponentialEquationsandInequalities}

\begin{question}
In Exercises \ref{expeqnfirst} - \ref{expeqnlast}, solve the equation analytically.

\begin{problem}\label{expeqnfirst}
$2^{4x} = 8$

$x = \answer{\frac{3}{4}}$
\end{problem}

\begin{problem}
$3^{(x - 1)} = 27$   

$x = \answer{4}$
\end{problem} 

\begin{problem}
$5^{2x-1} = 125$   

$x=\answer{2}$
\end{problem}  

\begin{problem}
$4^{2t} = \frac{1}{2}$     

$t = \answer{-\frac{1}{4}}$
\end{problem}

\begin{problem}
$8^{t} = \frac{1}{128}$

$t = \answer{-\frac{7}{3}}$
\end{problem}

\begin{problem}
$2^{(t^{3} - t)} = 1$   

\begin{selectAll}
    \choice{$t=-2$}
    \choice[correct]{$t=-1$}
    \choice[correct]{$t=0$}
    \choice[correct]{$t=1$}
    \choice{$t=2$}
    \choice{$t=3$}
    \choice{$t=8$}
  \end{selectAll}
\end{problem}

\begin{problem}
$3^{7x} = 81^{4-2x}$    

$x = \answer{\frac{16}{15}}$
\end{problem} 

\begin{problem}
$9 \cdot 3^{7x} = \left(\frac{1}{9}\right)^{2x}$ 

$x=\answer{-\frac{2}{11}}$
\end{problem} 

\begin{problem}
$3^{2x} = 5$ 

\begin{solution}
$x = \dfrac{\ln(5)}{2\ln(3)}$
\end{solution}
\end{problem}   

\begin{problem}
$5^{-t} = 2$   

\begin{solution}
    $t = -\dfrac{\ln(2)}{\ln(5)}$
\end{solution}
\end{problem}

\begin{problem}
$5^{t} = -2$     

\begin{solution}
    No solution.
\end{solution}
\end{problem}

\begin{problem}
$3^{(t - 1)} = 29$  

\begin{solution}
$t = \dfrac{\ln(29) + \ln(3)}{\ln(3)}$
\end{solution}
\end{problem}

\begin{problem}
$(1.005)^{12x} = 3$  

\begin{solution}
    $x = \dfrac{\ln(3)}{12\ln(1.005)}$
\end{solution}
\end{problem}

\begin{problem}
$e^{-5730k} = \frac{1}{2}$  

\begin{solution}
    $k = \dfrac{\ln\left(\frac{1}{2}\right)}{-5730} = \dfrac{\ln(2)}{5730} $
\end{solution}
\end{problem}

\begin{problem}
$2000e^{0.1t} = 4000$  

\begin{solution}
$t=\dfrac{\ln(2)}{0.1} = 10\ln(2)$
\end{solution}
\end{problem}

\begin{problem}
$500\left(1-e^{2t}\right) = 250$  

\begin{solution}
     $t=\frac{1}{2}\ln\left(\frac{1}{2}\right) = -\frac{1}{2}\ln(2)$
\end{solution}
\end{problem} 

\begin{problem}
$70 + 90e^{-0.1t} = 75$   

\begin{solution}
    $t = \dfrac{\ln\left(\frac{1}{18}\right)}{-0.1} =10 \ln(18)$
\end{solution}
\end{problem} 

\begin{problem}
$30-6e^{-0.1t}=20$

\begin{solution}
$t=-10\ln\left(\frac{5}{3}\right) = 10\ln\left(\frac{3}{5}\right)$
\end{solution}
\end{problem}  

\begin{problem}
$\dfrac{100e^{x}}{e^{x}+2}=50$ 

    $x=\answer{\ln(2)}$

\end{problem} 

\begin{problem}
$\dfrac{5000}{1+2e^{-3t}}=2500$ 

\begin{solution}
     $t=\frac{1}{3}\ln(2)$
\end{solution}
\end{problem} 

\begin{problem}
$\dfrac{150}{1 + 29e^{-0.8t}} = 75$

\begin{solution}
$t = \dfrac{\ln\left(\frac{1}{29}\right)}{-0.8} = \dfrac{5}{4}\ln(29)$
\end{solution}
\end{problem}   

\begin{problem}
$25\left(\frac{4}{5}\right)^{x} = 10$

\begin{solution}
    $x = \dfrac{\ln\left(\frac{2}{5}\right)}{\ln\left(\frac{4}{5}\right)} = \dfrac{\ln(2)-\ln(5)}{\ln(4) - \ln(5)}$
\end{solution}
\end{problem}      

\begin{problem}
$e^{2x} = 2e^{x}$ 

    $x = \answer{\ln(2)}$

\end{problem}  

\begin{problem}
$7e^{2t} = 28e^{-6t}$

\begin{solution}
$t = -\frac{1}{8} \ln\left(\frac{1}{4} \right) = \frac{1}{4}\ln(2)$ 
\end{solution}
\end{problem}
 
\begin{problem}
$3^{(x - 1)} = 2^{x}$

\begin{solution}
    $x = \dfrac{\ln(3)}{\ln(3) - \ln(2)}$
\end{solution}
\end{problem}

\begin{problem}
$3^{(x - 1)} = \left(\frac{1}{2}\right)^{(x + 5)}$

\begin{solution}
     $x = \dfrac{\ln(3) + 5\ln\left(\frac{1}{2}\right)}{\ln(3) - \ln\left(\frac{1}{2}\right)} = \dfrac{\ln(3)-5\ln(2)}{\ln(3)+\ln(2)}$
\end{solution}
\end{problem}

\begin{problem}
$7^{3+7x} = 3^{4-2x}$ 

\begin{solution}
$x = \dfrac{4 \ln(3) - 3 \ln(7)}{7 \ln(7) + 2 \ln(3)}$
\end{solution}
\end{problem}  

\begin{problem}
$e^{2t} - 3e^{t}-10=0$ 

    $t=\answer{\ln(5)}$

\end{problem}  

\begin{problem}
$e^{2t} = e^{t}+6$ 

    $t=\answer{\ln(3)}$

%Ans $t=\ln(2)$
\end{problem} 

\begin{problem}
$4^{t} + 2^{t} = 12$ 

\begin{solution}
$x=\dfrac{\ln(3)}{\ln(2)}$
\end{solution}
\end{problem}  

\begin{problem}
$e^{x}-3e^{-x}=2$

$x=\answer{\ln(3)}$
\end{problem}

\begin{problem}
$e^{x}+15e^{-x}=8$ %Ans $x=\ln(2)$, $\ln(5)$
\end{problem} 

\begin{problem}\label{expeqnlast}
$3^{x}+25\cdot3^{-x}=10$ 

\begin{solution}
$x=\dfrac{\ln(5)}{\ln(3)}$
\end{solution}
\end{problem} 
\end{question}

\begin{question}
In Exercises \ref{expineqfirst} - \ref{expineqlast}, solve the inequality analytically.

\begin{problem}\label{expineqfirst}
$e^{x} > 53$
\end{problem}   

\begin{problem}
$1000\left(1.005\right)^{12t} \geq 3000$
\end{problem} 

\begin{problem}
$2^{(x^{3} - x)} < 1$

\begin{solution}
$(-\infty, -1) \cup (0, 1)$
\end{solution}
\end{problem}  

\begin{problem}
$25\left(\dfrac{4}{5}\right)^{x} \geq 10$
\end{problem}   

\begin{problem}
$\dfrac{150}{1 + 29e^{-0.8t}} \leq 130$
\end{problem} 

\begin{problem}
$70 + 90e^{-0.1t} \leq 75$

\begin{solution}
$\left[\dfrac{\ln\left(\frac{1}{18}\right)}{-0.1}, \infty\right) = [10\ln(18), \infty)$
\end{solution}
\end{problem}  

\begin{problem}
$e^{-x} - xe^{-x} \geq 0$
\end{problem}   

\begin{problem}\label{expineqlast}
$(1-e^{t}) t^{-1} \leq 0$
\end{problem}  

\end{question}

\begin{question}
In Exercises \ref{calcexpineqfirst} - \ref{calcexpineqlast},  use a graphing utility to help you solve the equation or  inequality.

\begin{problem}\label{calcexpineqfirst} 
$2^{x} = x^2$

\begin{solution}
$x \approx -0.76666, \, x = 2, \, x = 4$
\end{solution}
\end{problem}  

\begin{problem}
$e^{t} = \ln(t) + 5$
\end{problem}

\begin{problem}
$e^{\sqrt{x}} = x + 1$ 
\end{problem} 

\begin{problem}
$e^{-2t}-te^{-t} \leq 0$

\begin{solution}
$\approx [0.567, \infty)$
\end{solution}
\end{problem} 

\begin{problem}
$3^{(x - 1)} < 2^{x}$
\end{problem}

\begin{problem}\label{calcexpineqlast}
$e^{t} < t^{3} - t$
\end{problem}  

\end{question}

\begin{question}
In Exercises \ref{domaincomplicatedexpfirst} - \ref{domaincomplicatedexplast},  find the domain of the function.

\begin{problem}\label{domaincomplicatedexpfirst}
$T(x) = \dfrac{e^{x} - e^{-x}}{e^{x} + e^{-x}}$

\begin{solution}
$(-\infty, \infty)$ 
\end{solution}
\end{problem}      

\begin{problem}
$C(x) = \dfrac{e^{x}  + e^{-x}}{e^{x}  - e^{-x}}$
\end{problem}   

\begin{problem}
$s(t) = \sqrt{e^{2t} - 3}$

\begin{solution}
$(-\infty, \infty)$ 
\end{solution}
\end{problem}      

\begin{problem}
$c(t) = \sqrt[3]{e^{2t} - 3}$
\end{problem}      

\begin{problem}
$L(x) = \log\left( 3 - e^{x} \right)$
\end{problem}     

\begin{problem}\label{domaincomplicatedexplast}
$\ell(x) = \ln\left( \dfrac{e^{2x}}{e^{x}-2} \right)$ 
\end{problem} 

    
\end{question}

\begin{problem}\label{onetoonelogexercise} 
Since $f(x) = \ln(x)$ is a strictly increasing function, if $0 < a < b$ then $\ln(a) < \ln(b)$.  Use this fact to solve the inequality $e^{(3x - 1)} > 6$ without a sign diagram. Use this technique to solve the inequalities in Exercises \ref{expineqfirst} - \ref{expineqlast}. (NOTE:  Isolate the exponential function first!)
\end{problem}

\begin{problem}\label{hyperbolicsine} 
Compute the inverse of $f(x) = \dfrac{e^{x} - e^{-x}}{2}$.  State the domain and range of both $f$ and $f^{-1}$. 
\end{problem}

\begin{problem}\label{checkingexpfracinverse} 
In Example \ref{expfracinverse}, we found that the inverse of $f(x) = \dfrac{5e^{x}}{e^{x}+1}$ was $f^{-1}(x) = \ln\left(\dfrac{x}{5-x}\right)$ but we left a few loose ends for you to tie up.  

\begin{enumerate}

\item Algebraically check our answer by verifying: $\left(f^{-1} \circ f\right)(x) = x$ for all $x$ in the domain of $f$ and that $\left(f \circ f^{-1}\right)(x) = x$ for all $x$ in the domain of $f^{-1}$.

\item Find the range of $f$ by finding the domain of $f^{-1}$.

\item With help of a graphing utility, graph $y = f(x)$,  $y = f^{-1}(x)$ and $y = x$ on the same set of axes.  How does this help to verify our answer?

\item Let $g(x) = \dfrac{5x}{x+1}$ and $h(x) = e^{x}$.  Show that $f = g \circ h$ and that $(g \circ h)^{-1} = h^{-1} \circ g^{-1}$. 

NOTE:  We know this is true in general by Exercise \ref{fcircginverse} in Section \ref{InverseFunctions}, but it's nice to see a specific example of the property.

\end{enumerate}
\end{problem}

\begin{problem}\label{powerexponentialgrowthex} 

\begin{enumerate} \item\label{numericalinvestigationlimitxpoverex} With the help of your classmates, numerically and graphically investigate $\ds{\lim_{x \rightarrow \infty}}$ $\frac{x^{p}}{e^{x}}$ for various real number powers, $p$.

\item  What does part \ref{numericalinvestigationlimitxpoverex} suggest about the relative growth rates of powers of $x$ as opposed to $e^{x}$?

\item\label{numericalxpoverexinequ}  For each power $p$ you investigated in part \ref{numericalinvestigationlimitxpoverex}, solve the inequality:  $\frac{x^{p}}{e^{x}} < \frac{1}{x}$. 

\item  Use your results from part \ref{numericalxpoverexinequ} to show that for each real number $p$ you investigated in part \ref{numericalinvestigationlimitxpoverex}, there is a real number $M$ so that if $x > M$, $0 < \frac{x^{p}}{e^{x}} < \frac{1}{x}$.  


Since  $\ds{\lim_{x \rightarrow \infty}}$ $\frac{1}{x} = 0$,  what do you conclude about $\ds{\lim_{x \rightarrow \infty}}$  $\frac{x^{p}}{e^{x}}$? 

 (This Exercise foreshadows the celebrated \index{Squeeze Theorem}\index{Theorem ! Squeeze}\textbf{Squeeze Theorem}, Theorem \ref{squeezeth}  which we'll formally introduce in Section \ref{Sequences}.)

\end{enumerate}  
\end{problem}

\begin{question}
In Exercises \ref{exponentialcurvesketchfirst}  - \ref{exponentialcurvesketchlast} a function $f$ along with its derivatives $f'$ and $f''$ are given.

\begin{itemize}

\item  Find the $x$- and $y$-intercepts of the graph of each function, if any.

\item  Use limits to determine the end behavior.

\item  Use $f'$ to determine the open intervals over which $f$ is increasing or decreasing.

\item Determine the local extrema, if any.

\item  Use $f''$ to determine the open intervals over which the graph of $f$  is concave up or concave down.

\item  Determine the inflection points of the graph, if any.

\end{itemize}

\begin{problem}\label{exponentialcurvesketchfirst} 
$f(x) = \dfrac{5}{1 + e^{-x}}$,  $f'(x) = \dfrac{5 e^{-x}}{\left(1 + e^{-x} \right)^2}$, $f''(x) = \dfrac{5e^{-x}\left(e^{-x}-1\right)}{\left(1+e^{-x}\right)^3}$
\end{problem}

\begin{problem}\label{exponentialcurvesketchlast}
$f(x) = e^{-x} - e^{-2x}$, $f'(x) = 2e^{-2x} - e^{-x}$, $f''(x) = e^{-x} - 4 e^{-2x}$
\end{problem}
    
\end{question}




\end{document}
