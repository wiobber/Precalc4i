\documentclass{ximera}

\begin{document}
	\author{Stitz-Zeager}
	\xmtitle{Answers for Applications of Exponential and Logarithmic Functions}{}

\mfpicnumber{1} \opengraphsfile{ExercisesforApplicationsofExponentialandLogarithmicFunctions} % mfpic settings added 


\begin{enumerate}

\item \begin{itemize}  \item $A(t) = 500\left(1 + \frac{0.0075}{12}\right)^{12t}$ 

\item $A(5) \approx \$ 519.10$, $A(10) \approx \$ 538.93$, $A(30) \approx \$ 626.12$, $A(35) \approx \$ 650.03$ 

\item It will take approximately $92$ years for the investment to double.

\item  The average rate of change from the end of the fourth year to the end of the fifth year is approximately $3.88$.  This means that the investment is growing at an average rate of $\$3.88$ per year at this point.  The average rate of change from the end of the thirty-fourth year to the end of the thirty-fifth year is approximately $4.85$.  This means that the investment is growing at an average rate of $\$4.85$ per year at this point. 

\end{itemize}

\item \begin{itemize}  \item $A(t) = 500e^{0.0075t}$ 

\item $A(5) \approx \$ 519.11$, $A(10) \approx \$ 538.94$, $A(30) \approx \$ 626.16$, $A(35) \approx \$ 650.09$ 

\item It will take approximately $92$ years for the investment to double.

\item  The average rate of change from the end of the fourth year to the end of the fifth year is approximately $3.88$.  This means that the investment is growing at an average rate of $\$3.88$ per year at this point.  The average rate of change from the end of the thirty-fourth year to the end of the thirty-fifth year is approximately $4.86$.  This means that the investment is growing at an average rate of $\$4.86$ per year at this point. 

\end{itemize}

\item \begin{itemize}  \item $A(t) = 1000\left(1 + \frac{0.0125}{12}\right)^{12t}$ 

\item $A(5) \approx \$ 1064.46$, $A(10) \approx \$ 1133.07$, $A(30) \approx \$ 1454.71$, $A(35) \approx \$ 1548.48$ 

\item  It will take approximately $55$ years for the investment to double.

\item  The average rate of change from the end of the fourth year to the end of the fifth year is approximately $13.22$.  This means that the investment is growing at an average rate of $\$13.22$ per year at this point.  The average rate of change from the end of the thirty-fourth year to the end of the thirty-fifth year is approximately $19.23$.  This means that the investment is growing at an average rate of $\$19.23$ per year at this point. 

\end{itemize}



\item \begin{itemize}  \item $A(t) = 1000e^{0.0125t}$ 

\item $A(5) \approx \$ 1064.49$, $A(10) \approx \$ 1133.15$, $A(30) \approx \$ 1454.99$, $A(35) \approx \$ 1548.83$ 

\item It will take approximately $55$ years for the investment to double.

\item  The average rate of change from the end of the fourth year to the end of the fifth year is approximately $13.22$.  This means that the investment is growing at an average rate of $\$13.22$ per year at this point.  The average rate of change from the end of the thirty-fourth year to the end of the thirty-fifth year is approximately $19.24$.  This means that the investment is growing at an average rate of $\$19.24$ per year at this point. 

\end{itemize}

\pagebreak

\item \begin{itemize}  \item $A(t) = 5000\left(1 + \frac{0.02125}{12}\right)^{12t}$ 

\item $A(5) \approx \$ 5559.98$, $A(10) \approx \$ 6182.67$, $A(30) \approx \$ 9453.40$, $A(35) \approx \$ 10512.13$ 

\item  It will take approximately $33$ years for the investment to double.

\item  The average rate of change from the end of the fourth year to the end of the fifth year is approximately $116.80$.  This means that the investment is growing at an average rate of $\$116.80$ per year at this point.  The average rate of change from the end of the thirty-fourth year to the end of the thirty-fifth year is approximately $220.83$.  This means that the investment is growing at an average rate of $\$220.83$ per year at this point. 

\end{itemize}

\item \begin{itemize}  \item $A(t) = 5000e^{0.02125t}$ 

\item $A(5) \approx \$ 5560.50$, $A(10) \approx \$ 6183.83$, $A(30) \approx \$ 9458.73$, $A(35) \approx \$ 10519.05$ 

\item  It will take approximately $33$ years for the investment to double.

\item  The average rate of change from the end of the fourth year to the end of the fifth year is approximately $116.91$.  This means that the investment is growing at an average rate of $\$116.91$ per year at this point.  The average rate of change from the end of the thirty-fourth year to the end of the thirty-fifth year is approximately $221.17$.  This means that the investment is growing at an average rate of $\$221.17$ per year at this point. 

\end{itemize}

\setcounter{HW}{\value{enumi}}
\end{enumerate}

\begin{enumerate}
\setcounter{enumi}{\value{HW}}

\addtocounter{enumi}{1}

\item  $P = \frac{2000}{e^{0.0025 \cdot 3}} \approx \$ 1985.06$

\item  $P = \frac{5000}{\left(1 + \frac{0.0225}{12}\right)^{12 \cdot 10}} \approx \$ 3993.42$

\item \begin{enumerate}

\item $A(8) = 2000\left(1 + \frac{0.0025}{12}\right)^{12 \cdot 8} \approx \$2040.40$
\item $t = \dfrac{\ln(2)}{12 \ln\left(1 + \frac{0.0025}{12}\right)} \approx 277.29$ years
\item $P = \dfrac{2000}{\left(1 + \frac{0.0025}{12}\right)^{36}} \approx \$1985.06$

\end{enumerate}

\item \begin{enumerate}

\item $A(8) = 2000\left(1 + \frac{0.0225}{12}\right)^{12 \cdot 8} \approx \$2394.03$
\item $t = \dfrac{\ln(2)}{12 \ln\left(1 + \frac{0.0225}{12}\right)} \approx 30.83$ years
\item $P = \dfrac{2000}{\left(1 + \frac{0.0225}{12}\right)^{36}} \approx \$1869.57$
\item $\left(1 + \frac{0.0225}{12}\right)^{12} \approx 1.0227$ so the APY is 2.27\%

\end{enumerate}

\item  $A(3) = 5000e^{0.299 \cdot 3} \approx \$12,226.18$,  $A(6) = 5000e^{0.299 \cdot 6} \approx \$30,067.29$

\setcounter{HW}{\value{enumi}}
\end{enumerate}


\begin{multicols}{2}
\begin{enumerate}
\setcounter{enumi}{\value{HW}}
\addtocounter{enumi}{1}

\item  \begin{itemize}  \item $k = \frac{\ln(1/2)}{5.27} \approx -0.1315$

\item $A(t) = 50e^{-0.1315t}$

\item  $t = \frac{\ln(0.1)}{-0.1315} \approx 17.51$ years.

\end{itemize}



\item  \begin{itemize}  \item $k = \frac{\ln(1/2)}{14} \approx -0.0495$

\item $A(t) = 2e^{-0.0495t}$

\item  $t = \frac{\ln(0.1)}{-0.0495} \approx 46.52$ days.

\end{itemize}

\setcounter{HW}{\value{enumi}}
\end{enumerate}
\end{multicols}

\begin{multicols}{2}
\begin{enumerate}
\setcounter{enumi}{\value{HW}}


\item  \begin{itemize}  \item $k = \frac{\ln(1/2)}{27.7} \approx -0.0250$

\item $A(t) = 75e^{-0.0250t}$

\item  $t = \frac{\ln(0.1)}{-0.025} \approx 92.10$ days.

\end{itemize}

\item  \begin{itemize}  \item $k = \frac{\ln(1/2)}{432.7} \approx -0.0016$

\item $A(t) = 0.29e^{-0.0016t}$

\item  $t = \frac{\ln(0.1)}{-0.0016} \approx 1439.11$ years.

\end{itemize}


\setcounter{HW}{\value{enumi}}
\end{enumerate}
\end{multicols}

\begin{enumerate}
\setcounter{enumi}{\value{HW}}

\item  \begin{itemize}  \item $k = \frac{\ln(1/2)}{704} \approx -0.0009846$

\item $A(t) = e^{-0.0009846t}$

\item $t = \frac{\ln(0.1)}{-0.0009846} \approx 2338.60$ million years, or $2.339$ billion years.

\end{itemize}


\setcounter{HW}{\value{enumi}}
\end{enumerate}

\begin{multicols}{2}
\begin{enumerate}
\setcounter{enumi}{\value{HW}}


\item  $t = \frac{\ln(0.1)}{k} = -\frac{\ln(10)}{k}$

\item $V(t) = 25e^{\ln\left(\frac{4}{5}\right)t} \approx 25e^{-0.22314355t}$

\setcounter{HW}{\value{enumi}}
\end{enumerate}
\end{multicols}


\begin{enumerate}
\setcounter{enumi}{\value{HW}}


\item \begin{enumerate}  \item  $G(0) = 9743.77$  This means that the GDP of the US in 2000 was $\$9743.77$ billion dollars.

\item  $G(7) = 13963.24$ and $G(10) = 16291.25$, so the model predicted a GDP of $\$ 13,963.24$ billion in 2007 and $\$ 16,291.25$ billion in 2010. 

\end{enumerate}

\item \begin{enumerate} \item $D(0) = 15$, so the tumor was 15 millimeters in diameter when it was first detected.

\item  $t = \frac{\ln(2)}{0.0277} \approx 25$ days.

\end{enumerate}

\setcounter{HW}{\value{enumi}}
\end{enumerate}

\begin{multicols}{2}
\begin{enumerate}
\setcounter{enumi}{\value{HW}}

\item  \begin{enumerate} \item  $k = \frac{\ln(2)}{20} \approx 0.0346$

\item  $N(t) = 1000e^{0.0346 t}$

\item  $t = \frac{\ln(9)}{0.0346} \approx 63$ minutes

\end{enumerate}

\item  \begin{enumerate} \item  $k = \frac{1}{2}\frac{\ln(6)}{2.5} \approx 0.4377$

\item  $N(t) = 2.5e^{0.4377 t}$

\item  $t = \frac{\ln(2)}{0.4377} \approx 1.58$ hours

\end{enumerate}

\setcounter{HW}{\value{enumi}}
\end{enumerate}
\end{multicols}

\begin{enumerate}
\setcounter{enumi}{\value{HW}}


\item  $N_{\text{\tiny $0$}} = 52$,  $k = \frac{1}{3} \ln\left( \frac{118}{52}\right) \approx 0.2731$, $N(t) = 52e^{0.2731t}$.  $N(6) \approx 268$. 

\item  $N_{\text{\tiny $0$}} = 2649$,  $k = \frac{1}{60} \ln\left( \frac{7272}{2649}\right) \approx 0.0168$, $N(t) = 2649e^{0.0168t}$.  $N(150) \approx 32923$, so the population of Painesville in 2010 based on this model would have been 32,923.



\item  \begin{enumerate}  \item  $P(0) = \frac{120}{4.167} \approx 29$.  There are 29 Sasquatch in Bigfoot County in 2010.

\item  $P(3) = \frac{120}{1+3.167e^{-0.05(3)}} \approx 32$ Sasquatch.

\item  $t = 20 \ln(3.167) \approx 23$ years.

\item  We find $\ds{\lim_{t \rightarrow \infty} P(t) =  120}$.  As time goes by, the Sasquatch Population in Bigfoot County will approach 120.  Graphically,  $y = P(x)$ has a horizontal asymptote $y=120$.

\end{enumerate}

\pagebreak

\item 

\begin{enumerate}

\addtocounter{enumii}{1}

\item The average rates of change are listed in order below. They suggest  slope at $(1,5)$ is $2.5$.  

\begin{multicols}{6}

\begin{itemize}

\item $\approx 2.487$

\item $\approx 2.498$

\item $\approx 2.500$

\item $\approx 2.500$

\item $\approx 2.498$

\item$\approx 2.487$

\end{itemize}

\end{multicols}

\end{enumerate}


\item \begin{enumerate}

\item $A(t) = Ne^{-\left(\frac{\ln(2)}{5730}\right)t} \approx Ne^{-0.00012097t}$
\item $A(20000) \approx 0.088978 \cdot N$ so about 8.9\% remains
\item $t \approx \dfrac{\ln(.42)}{-0.00012097} \approx 7171$ years old

\end{enumerate}

\item $A(t) = 2.3e^{-0.0138629t}$

\item  The relative rate of change of $A(t)$ over $\left[t, t+\frac{1}{n} \right]$ is $\frac{r}{n}$ which is the annual percentage rate divided by the number of compoundings per year -- that is,  the percentage growth rate over one compounding.

\addtocounter{enumi}{1}

\item \begin{enumerate}

\item $T(t) = 75 + 105e^{-0.005005t}$

\item The roast would have cooled to $140^{\circ}$F in about 95 minutes.

\end{enumerate}

\item From the graph, it appears that $\ds{\lim_{x \rightarrow 0^{+}} y(x) = \infty}$.  This is due to the presence of the $\ln(x)$ term in the function.  This means that Fritzy will never catch Chewbacca, which makes sense since Chewbacca has a head start and Fritzy only runs as fast as he does.

\begin{center}

\includegraphics[height=1.5in]{./ApplicationsofExponentialandLogarithmicFunctionsGraphics/PURSUIT03.jpg}

\smallskip

$y(x) = \frac{1}{4} x^2-\frac{1}{4} \ln(x)-\frac{1}{4}$

\end{center}

\item The steady state current is 2 amps.

\addtocounter{enumi}{1}

\item  630 feet.

\item The linear regression on the data below is $y = 1.74899x + 0.70739$ with $r^{2} \approx 0.999995$.  

This is an excellent fit.

\scriptsize

\noindent \begin{tabular}{|l|r|r|r|r|r|r|r|r|r|r|} \hline
$x$ & 1 & 2 & 3 & 4 & 5 & 6 & 7 & 8 & 9 & 10 \\ 
\hline 
$\ln(N(x))$ & 2.4849 & 4.1897 & 5.9454 & 7.6967 & 9.4478 & 11.1988 & 12.9497 & 14.7006 & 16.4523 & 18.2025 \\ \hline
\end{tabular}

\normalsize

$N(x) = 2.02869(5.74879)^{x} = 2.02869e^{1.74899x}$ with $r^{2} \approx 0.999995$.  This is also an excellent fit and corresponds to our linearized model because $\ln(2.02869) \approx 0.70739$.

\item  \begin{enumerate}
\addtocounter{enumii}{1}

\item  The linearized model is: $\ln(L(x)) \approx 2.106 \ln(x) - 5.268$ with an $r^2 \approx  0.9914$.

\item  $L(x) = 0.005154 x^{2.106}$.  $L(90) \approx 67.3$ meaning the bottom $90 \%$ of wage earners take home $67.3 \%$ of the total national income.  Said differently, according to this model, the top $10 \%$ of wage earners take home $32.7 \%$ of the total national income. 



\item  We graph our answer to Example \ref{LorenzEx} in Section \ref{PowerFunctions}, $L(x) = 0.00027901x^{2.7738}$, below on the left.  Below on the right is the model we derived in this exercise. 

\begin{center}

\begin{tabular}{cc}

\includegraphics[height=1.5in]{./ApplicationsofExponentialandLogarithmicFunctionsGraphics/OldLorenzModel.jpg}  &

\hspace{1in}

\includegraphics[height=1.5in]{./ApplicationsofExponentialandLogarithmicFunctionsGraphics/NewLorenzModel.jpg}  \\

 $L(x) = 0.00027901x^{2.7738}$

&

\hspace{1in}
$L(x) = 0.005154 x^{2.106}$ \\

\end{tabular}

\end{center}

\end{enumerate}


\item  \begin{enumerate}  \item   We get:  $y = 2895.06 (1.0147)^{x}$.  Graphing this along with our answer from Exercise \ref{PainesvillePopulationTwoPoint} over the interval $[0,60]$ shows that they are pretty close. From this model, $y(150) \approx 25840$ which once again overshoots the actual data value.

\item $P(150) \approx 18717$, so this model predicts 17,914 people in Painesville in 2010, a more conservative number than was recorded in the 2010 census.  We have $\ds{\lim_{t \rightarrow \infty} P(t) =  18691}$,  so the limiting population of Painesville based on this model is 18,691 people.

\enlargethispage{\baselineskip}

\end{enumerate}

\item \begin{enumerate}  \item  $y = \dfrac{242526}{1+874.63e^{-0.07113x}}$, where $x$ is the number of years since 1860.

\item  The plot of the data and the curve is below.

\centerline{\includegraphics[height=2in]{./ApplicationsofExponentialandLogarithmicFunctionsGraphics/LAKECOUNTYLOGISTIC.jpg}} 

\item  $y(140) \approx 232884$, so this model predicts 232,884 people in Lake County in 2010.

\item  We get $\ds{\lim_{x \rightarrow \infty} y = 242526}$, so the limiting population of Lake County based on this model is 242,526 people.

\end{enumerate}

\end{enumerate}


\end{document}
