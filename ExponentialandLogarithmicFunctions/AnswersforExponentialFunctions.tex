\documentclass{ximera}

\begin{document}
	\author{Stitz-Zeager}
	\xmtitle{Answers for Exponential Functions}{}

\mfpicnumber{1} \opengraphsfile{ExercisesforExponentialFunctions} % mfpic settings added 


\label{AnswersforExponentialFunctions}


\subsection{Answers}

\begin{multicols}{2}
\begin{enumerate}


\item  Domain of $g$:  $(-\infty, \infty)$\\
 Range of $g$:  $(-1, \infty)$\\
 Points:  $\left(-1, -\frac{1}{2} \right)$, $(0,0)$, $(1,1)$\\
Asymptote: $y = -1$\\
 
\begin{mfpic}[15]{-4}{4}{-2}{9}
\point[4pt]{(-1,-0.5), (0,0), (1,1)}
\axes
\tlabel[cc](4,-0.5){\scriptsize $x$}
\tlabel[cc](0.5,9){\scriptsize $y$}
\tcaption{$y = g(x) = 2^{x}-1$}
\xmarks{-3,-2,-1,1,2,3}
\ymarks{-1,1,2,3,4,5,6,7,8}
\tlabel[cc](-3,0.5){\tiny $-3 \hspace{7pt}$}
\tlabel[cc](-2,0.5){\tiny $-2 \hspace{7pt}$}
\tlabel[cc](-1,0.5){\tiny $-1 \hspace{7pt}$}
\tlpointsep{4pt}
\axislabels {x}{{\tiny $1$} 1, {\tiny $2$} 2, {\tiny $3$} 3}
\axislabels {y}{{\tiny $1$} 1, {\tiny $2$} 2, {\tiny $3$} 3, {\tiny $4$} 4, {\tiny $5$} 5, {\tiny $6$} 6, {\tiny $7$} 7, {\tiny $8$} 8}
\dashed \polyline{(-4,-1),(4,-1)}
\penwd{1.25pt}
\arrow \reverse \arrow \function{-3.5, 3.1, 0.1}{(2**(x))-1}
\end{mfpic}

\vfill

\columnbreak

\item  Domain of $g$:  $(-\infty, \infty)$ \\
 Range of $g$:  $(0, \infty)$ \\
  Points:  $(0,3)$, $(1,1)$, $\left(2, \frac{1}{3} \right)$\\
 Asymptote: $y = 0$\\
 
\begin{mfpic}[15]{-4}{4}{-1}{10}
\point[4pt]{(0,3), (1,1), (2,0.3333)}
\axes
\tlabel[cc](4,-0.5){\scriptsize $x$}
\tlabel[cc](0.5,10){\scriptsize $y$}
\tcaption{$y = g(x) = \left(\frac{1}{3}\right)^{x-1}$}
\xmarks{-3,-2,-1,1,2,3}
\ymarks{1,2,3,4,5,6,7,8,9}
\tlpointsep{4pt}
\axislabels {x}{{\tiny $-3 \hspace{7pt}$} -3, {\tiny $-2 \hspace{7pt}$} -2, {\tiny $-1 \hspace{7pt}$} -1, {\tiny $1$} 1, {\tiny $2$} 2, {\tiny $3$} 3}
\axislabels {y}{{\tiny $1$} 1, {\tiny $2$} 2, {\tiny $3$} 3, {\tiny $4$} 4, {\tiny $5$} 5, {\tiny $6$} 6, {\tiny $7$} 7, {\tiny $8$} 8, {\tiny $9$} 9}
\penwd{1.25pt}
\arrow \reverse \arrow \function{-1.05, 3.5, 0.1}{3**(1-x)}
\end{mfpic} 

\setcounter{HW}{\value{enumi}}
\end{enumerate}
\end{multicols}

\begin{multicols}{2}
\begin{enumerate}
\setcounter{enumi}{\value{HW}}

\item  Domain of $g$:  $(-\infty, \infty)$\\
 Range of $g$:  $(2, \infty)$\\
  Points:  $\left(1, \frac{7}{3} \right)$, $(0,3)$, $(-1,5)$\\
  Asymptote:  $y = 2$ \\

\begin{mfpic}[15]{-4}{4}{-1}{12}
\point[4pt]{(1,2.3333), (0,3), (-1,5)}
\axes
\tlabel[cc](4,-0.5){\scriptsize $x$}
\tlabel[cc](0.5,12){\scriptsize $y$}
\tcaption{$y = g(x) = 3^{-x}+2$}
\xmarks{-3,-2,-1,1,2,3}
\ymarks{1,2,3,4,5,6,7,8,9,10,11}
\tlpointsep{4pt}
\axislabels {x}{{\tiny $-3 \hspace{7pt}$} -3, {\tiny $-2 \hspace{7pt}$} -2, {\tiny $-1 \hspace{7pt}$} -1, {\tiny $1$} 1, {\tiny $2$} 2, {\tiny $3$} 3}
\axislabels {y}{{\tiny $1$} 1, {\tiny $2$} 2, {\tiny $3$} 3, {\tiny $4$} 4, {\tiny $5$} 5, {\tiny $6$} 6, {\tiny $7$} 7, {\tiny $8$} 8, {\tiny $9$} 9, {\tiny $10$} 10, {\tiny $11$} 11}
\dashed \polyline{(-4,2),(4,2)}
\penwd{1.25pt}
\arrow \reverse \arrow \function{-2.05, 2.5, 0.1}{2+3**(0-x)}
\end{mfpic}

\vfill

\columnbreak

\item  Domain of $g$:  $(-\infty, \infty)$\\
 Range of $g$:  $(-20, \infty)$\\
  Points:  $\left(-1,-19 \right)$, $(1,-10)$, $(3,80)$\\
  Asymptote:  $y = -20$\\
 
\begin{mfpic}[15]{-3}{4}{-2}{9}
\point[4pt]{(-1,-1.9), (1,-1), (3,8)}
\axes
\tlabel[cc](4,-0.5){\scriptsize $x$}
\tlabel[cc](0.5,9){\scriptsize $y$}
\tcaption{$y = g(x) = 10^{\frac{x+1}{2}}-20$}
\xmarks{-3,-2,-1,1,2,3}
\ymarks{-2,-1,1,2,3,4,5,6,7,8}
\tlpointsep{4pt}
\axislabels {x}{{\tiny $-3 \hspace{7pt}$} -3, {\tiny $-2 \hspace{7pt}$} -2,  {\tiny $1$} 1, {\tiny $2$} 2, {\tiny $3$} 3}
\axislabels {y}{{\tiny $-10$} -1,{\tiny $10$} 1, {\tiny $20$} 2, {\tiny $30$} 3, {\tiny $40$} 4, {\tiny $50$} 5, {\tiny $60$} 6, {\tiny $70$} 7, {\tiny $80$} 8}
\dashed \polyline{(-4,-2), (4,-2)}
\penwd{1.25pt}
\arrow \reverse \arrow \function{-3, 3.06, 0.1}{((10**((x+1)/2))-20)/10}
\end{mfpic}

\setcounter{HW}{\value{enumi}}
\end{enumerate}
\end{multicols}

\begin{multicols}{2}
\begin{enumerate}
\setcounter{enumi}{\value{HW}}

\item  Domain of $g$:  $(-\infty, \infty)$\\
  Range of $g$:  $(0, \infty)$ \\
  Points:  $(-10, 200)$, $(0, 100)$, $(10, 50)$\\
  Asymptote: $y = 0$\\
 
\begin{mfpic}[15]{-4}{4}{-1}{9}
\point[4pt]{(-1,2), (0,1), (1, 0.5)}
\axes
\tlabel[cc](4,-0.5){\scriptsize $t$}
\tlabel[cc](0.5,9){\scriptsize $y$}
\tcaption{ $y = g(t) = 100(0.5)^{0.1t}$}
\xmarks{-3,-2,-1,1,2,3}
\ymarks{1,2,3,4,5,6,7,8}
\tlpointsep{4pt}
\axislabels {x}{{\tiny $-30 \hspace{7pt}$} -3, {\tiny $-20 \hspace{7pt}$} -2, {\tiny $-10 \hspace{7pt}$} -1, {\tiny $10$} 1, {\tiny $20$} 2, {\tiny $30$} 3}
\axislabels {y}{{\tiny $100$} 1, {\tiny $200$} 2, {\tiny $300$} 3, {\tiny $400$} 4, {\tiny $500$} 5, {\tiny $600$} 6, {\tiny $700$} 7, {\tiny $800$} 8}
\penwd{1.25pt}
\arrow \reverse \arrow \function{-3, 3, 0.1}{0.5**x}
\end{mfpic}

\vfill

\columnbreak


\item  Domain of $g$:  $(-\infty, \infty)$\\
  Range of $g$:  $(-\infty, 1)$ \\
  Points:  $(1, 0.2)$, $(2,0)$, $(3,-0.25)$\\
  Asymptote: $y = 1$\\
 
\begin{mfpic}[10][8]{-8}{5}{-11}{11}
\point[4pt]{(1,2), (2,0), (3, -2.5)}
\axes
\tlabel[cc](5,-0.5){\scriptsize $t$}
\tlabel[cc](0.5,11){\scriptsize $y$}
\tcaption{ $y = g(t) = 1-(1.25)^{t-2}$}
\xmarks{-8 step 1 until 4}
\ymarks{-10 step 1 until 10}
\tlpointsep{4pt}
\axislabels {x}{{\tiny $-8 \hspace{7pt}$} -8,{\tiny $-6 \hspace{7pt}$} -6,{\tiny $-4 \hspace{7pt}$} -4,{\tiny $-2 \hspace{7pt}$} -2, {\tiny $2$} 2, {\tiny $4$} 4}
\axislabels {y}{ {\tiny $0.2$} 2,  {\tiny $0.4$} 4,  {\tiny $0.6$} 6, {\tiny $0.8$} 8, {\tiny $1$} 10, {\tiny $-0.2$} -2, {\tiny $-0.4$} -4,  {\tiny $-0.6$} -6, {\tiny $-0.8$} -8, {\tiny $-1$} -10}
 
\dashed \polyline{(-8, 10), (5, 10)}
\penwd{1.25pt}
\arrow \reverse \arrow \function{-8, 5, 0.1}{(1- (1.25**(x-2)))*10}
\end{mfpic}

\setcounter{HW}{\value{enumi}}
\end{enumerate}
\end{multicols}



\begin{multicols}{2}
\begin{enumerate}
\setcounter{enumi}{\value{HW}}

\item  Domain of $g$:  $(-\infty, \infty)$\\
 Range of $g$:  $(-\infty, 8)$ \\
  Points:  $\left(1, 8-e^{-1} \right) \approx (1, 7.63)$,  \\
  $(0,7)$, $\left(-1,  8-e \right) \approx (1,5.28) $\\
 Asymptote:  $y = 8$ \\
 
\begin{mfpic}[15]{-4}{4}{-1}{9}
\point[4pt]{(1,7.6321), (0,7), (-1,5.282)}
\axes
\tlabel[cc](4,-0.5){\scriptsize $t$}
\tlabel[cc](0.5,9){\scriptsize $y$}
\tcaption{ $y = g(t) = 8-e^{-t}$}
\xmarks{-3,-2,-1,1,2,3}
\ymarks{1,2,3,4,5,6,7,8}
\tlpointsep{4pt}
\axislabels {x}{{\tiny $-3 \hspace{7pt}$} -3, {\tiny $-2 \hspace{7pt}$} -2, {\tiny $-1 \hspace{7pt}$} -1, {\tiny $1$} 1, {\tiny $2$} 2, {\tiny $3$} 3}
\axislabels {y}{{\tiny $1$} 1, {\tiny $2$} 2, {\tiny $3$} 3, {\tiny $4$} 4, {\tiny $5$} 5, {\tiny $6$} 6, {\tiny $7$} 7, {\tiny $8$} 8}
\dashed \polyline{(-4,8), (4,8)}
\penwd{1.25pt}
\arrow \reverse \arrow \function{-2.25, 2, 0.1}{8-exp(0-x)}
\end{mfpic}

\vfill

\columnbreak


\item  Domain of $g$:  $(-\infty, \infty)$\\
 Range of $g$:  $(0, \infty)$\\
 Points:  $\left(10, 10e^{-1} \right) \approx (10, 3.68)$ \\
 $(0,10)$, $\left(-10, 10e \right) \approx (-10, 27.18)$  \\
 Asymptote:  $y = 0$\\
 
\begin{mfpic}[15]{-4}{4}{-1}{9}
\point[4pt]{(1,0.3679), (0,1), (-1,2.718)}
\axes
\tlabel[cc](4,-0.5){\scriptsize $t$}
\tlabel[cc](0.5,9){\scriptsize $y$}
\tcaption{$y = g(t) = 10e^{-0.1t}$}
\xmarks{-3,-2,-1,1,2,3}
\ymarks{1,2,3,4,5,6,7,8}
\tlpointsep{4pt}
\axislabels {x}{{\tiny $-10 \hspace{7pt}$} -1, {\tiny $10$} 1, {\tiny $20$} 2, {\tiny $30$} 3}
\axislabels {y}{{\tiny $10$} 1, {\tiny $20$} 2, {\tiny $30$} 3, {\tiny $40$} 4, {\tiny $50$} 5, {\tiny $60$} 6, {\tiny $70$} 7, {\tiny $80$} 8}
\penwd{1.25pt}
\arrow \reverse \arrow \function{-2.15, 2, 0.1}{exp(0-x)}
\end{mfpic}

\setcounter{HW}{\value{enumi}}
\end{enumerate}
\end{multicols}

\begin{multicols}{4}
\begin{enumerate}
\setcounter{enumi}{\value{HW}}

\item $F(x) = 2^{x+1}-3$

\item  $F(x) = -2^{-x} + 3$

\item $F(x) = 2^{2x-6}$

\item  $F(x) =3 \cdot 2^{-2x}$

\setcounter{HW}{\value{enumi}}
\end{enumerate}
\end{multicols}

\begin{enumerate}
\setcounter{enumi}{\value{HW}}

\item  Since $2 = 4^{\frac{1}{2}}$, one way to obtain the formulas for $G(x)$ is to use properties of exponents.  For example, $F(x) = 2^{x+1}-3 = \left(4^{\frac{1}{2}}\right)^{x+1} -3 = 4^{\frac{1}{2}(x+1)} - 3 = 4^{\frac{1}{2} x + \frac{1}{2}} - 3$.  In order, the formulas for $G(x)$ are:

\begin{multicols}{4}
\begin{itemize}

\item $G(x) = 4^{\frac{1}{2}x+\frac{1}{2}}-3$

\item  $G(x) = -4^{-\frac{1}{2} x} + 3$

\item $G(x) = 4^{x-3}$

\item  $G(x) =3 \cdot 4^{-x}$

\end{itemize}
\end{multicols}

\setcounter{HW}{\value{enumi}}
\end{enumerate}

\begin{multicols}{2}

\begin{enumerate}
\setcounter{enumi}{\value{HW}}

\addtocounter{enumi}{1}


\item\footnote{Answers for Exercises \ref{decomposebasicexpfirst} - \ref{decomposebasicexplast} vary.  We list one solution for each problem.}$g(x)  = e^{-x}$ and $h(x) = 1$.
\item $g(x) = e^{2x}$ and $h(x) = x$.


\setcounter{HW}{\value{enumi}}
\end{enumerate}

\end{multicols}

\begin{multicols}{2}

\begin{enumerate}
\setcounter{enumi}{\value{HW}}


\item   $g(t) = t^2$ and $h(t) = e^{-t}$.
\item   $f(x) = e^{x} - e^{-x}$ and $g(x) = e^{x}+e^{-x}$.


\setcounter{HW}{\value{enumi}}
\end{enumerate}

\end{multicols}

\begin{multicols}{2}

  \begin{enumerate}
\setcounter{enumi}{\value{HW}}

\item   $f(x) = -x^2$ and $g(x) = e^{x}$.
\item   $f(x) = e^{2x} -1$ and $g(x) = \sqrt{x}$.  

\setcounter{HW}{\value{enumi}}
\end{enumerate}

\end{multicols}

\begin{enumerate}
\setcounter{enumi}{\value{HW}}

\item  \begin{enumerate}

\item   $A(0) = 500$, so the principal is $\$500$.  $A(1) = 525$, so after $1$ year, there is $\$525$ in the savings account.   $A(2) =551.25$, so after $2$ years, there is $\$551.25$ in the savings account. 

\item  The relative rate of change of $A$ over the intervals $[0,1]$ and $[1,2]$ is $0.05$ which means the savings account is growing by $5 \%$ each year for those two years.  Over the interval $[0,2]$, the relative rate of change is $0.1025$ meaning the account has grown by $10.25 \%$ over the course of the first two years.  Note this is greater than the sum of the two rates $5 \% + 5 \% = 10 \%$.  This is due to the `compounding effect' and will be discussed in greater detail in Section \ref{ExpLogApplications}.

\item  The relative rate of change of $A$ over the $[t, t+1]$ is $0.05$. This means over the course of one year, the savings account grows by $5 \%$.

\item Graphing $y= A(t)$ and $y = 1500$, we find they intersect when $t \approx 22.5$ so it takes approximately $22-23$ years for the savings account to grow to $\$1500$ in value.

\end{enumerate}

\item  \begin{enumerate}

\item $\frac{229437-230041}{230041} \approx 0.263 \%$.

\item  Since 2020 is five years after 2015, we expect the population to decrease by  $0.263 \%$ of 229437, or approximately 603 people.  Hence, we approximate the population in 2020 as 228834.

\item  $P(t) = 230041(1-0.00263)^t = 230041(0.99737)^{t}$, $t \geq 0$.

\item  Since 2017 is 7 years after 2010, we set $t = \frac{7}{5} = 1.4$ and find $P(1.4) \approx 229194$.  So the population is approximately 229, 194 in 2017.  

\item  $A(t) = 230041(1 - 0.0005626)^{t} =  230041 (0.999474)^{t}$, $t \geq 0$.  Since 2050 is 40 years after 2010, using the model $P(t)$, we divide $\frac{40}{5} = 8$ and find $P(8) \approx 225,245$. On the other hand, $A(40) \approx 225,250$.  This is more than roundoff error.  There is a compounding effect which makes the functions $A(t)$ and $P(t)$ different. \footnote{See number \ref{preludetocompoundingexercise} above or, for more, see Section \ref{ExpLogApplications}.}

\end{enumerate}

\addtocounter{enumi}{3}

\enlargethispage{0.25in}

\item  \begin{enumerate}
\addtocounter{enumii}{3}

\item \begin{multicols}{3} \begin{enumerate} \item $y = x+1$ \item  $y = e \, x $  \item  $y = e^{-1} \, x + 2e^{-1}$  \end{enumerate}  \end{multicols} \end{enumerate}


\setcounter{HW}{\value{enumi}}
\end{enumerate}





\end{document}
