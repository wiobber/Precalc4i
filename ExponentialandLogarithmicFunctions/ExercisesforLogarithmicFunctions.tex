\documentclass{ximera}

\begin{document}
	\author{Stitz-Zeager}
	\xmtitle{Exercises}
\mfpicnumber{1} \opengraphsfile{ExercisesforLogarithmicFunctions} % mfpic settings added 


\label{ExercisesforLogarithmicFunctions}

\label{IntroExpLogsExercises}

In Exercises \ref{rewritefirstex} - \ref{rewritelastex}, use the property: $b^{a} = c$ if and only if $\log_{b}(c) = a$ from Theorem \ref{logfcnprops} to rewrite the given equation in the other form.  That is, rewrite the exponential equations as logarithmic equations and rewrite the logarithmic equations as exponential equations.

\begin{multicols}{3}
\begin{enumerate}

\item  $2^{3} = 8$ \label{rewritefirstex}

\item  $5^{-3} = \frac{1}{125}$  

\item  $4^{5/2} = 32$  

\setcounter{HW}{\value{enumi}}
\end{enumerate}
\end{multicols}

\begin{multicols}{3}
\begin{enumerate}
\setcounter{enumi}{\value{HW}}

\item  $\left(\frac{1}{3}\right)^{-2} = 9$  

\item  $\left(\frac{4}{25}\right)^{-1/2} = \frac{5}{2}$  

\item  $10^{-3} = 0.001$ 

\setcounter{HW}{\value{enumi}}
\end{enumerate}
\end{multicols}

\begin{multicols}{3}
\begin{enumerate}
\setcounter{enumi}{\value{HW}}

\item  $e^{0}  = 1$  

\item  $\log_{5}(25) = 2$  

\item  $\log_{25} (5) = \frac{1}{2}$  

\setcounter{HW}{\value{enumi}}
\end{enumerate}
\end{multicols}

\begin{multicols}{3}
\begin{enumerate}
\setcounter{enumi}{\value{HW}}

\item  $\log_{3} \left(\frac{1}{81} \right) = -4$  

\item  $\log_{\frac{4}{3}} \left(\frac{3}{4} \right) = -1$  

\item  $\log(100) = 2$  

\setcounter{HW}{\value{enumi}}
\end{enumerate}
\end{multicols}

\begin{multicols}{3}
\begin{enumerate}
\setcounter{enumi}{\value{HW}}

\item  $\log (0.1) = -1$  

\item  $\ln(e) = 1$ 

\item  $\ln\left(\frac{1}{\sqrt{e}}\right) = -\frac{1}{2}$  \label{rewritelastex}

\setcounter{HW}{\value{enumi}}
\end{enumerate}
\end{multicols}

In Exercises \ref{simplifylogfirst} - \ref{simplifyloglast}, evaluate the expression without using a calculator.

\begin{multicols}{3}
\begin{enumerate}
\setcounter{enumi}{\value{HW}}

\item $\log_{3} (27)$  \label{simplifylogfirst}
\item $\log_{6} (216)$
\item $\log_{2} (32)$

\setcounter{HW}{\value{enumi}}
\end{enumerate}
\end{multicols}


\begin{multicols}{3}
\begin{enumerate}
\setcounter{enumi}{\value{HW}}

\item  $\log_{6} \left( \frac{1}{36} \right)$
\item $\log_{8} (4)$
\item $\log_{36} (216)$

\setcounter{HW}{\value{enumi}}
\end{enumerate}
\end{multicols}


\begin{multicols}{3}
\begin{enumerate}
\setcounter{enumi}{\value{HW}}

\item $\log_{\frac{1}{5}} (625)$
\item  $\log_{\frac{1}{6}} (216)$
\item $\log_{36} (36)$ 

\setcounter{HW}{\value{enumi}}
\end{enumerate}
\end{multicols}


\begin{multicols}{3}
\begin{enumerate}
\setcounter{enumi}{\value{HW}}

\item $\log \left(\frac{1}{1000000}\right)$
\item $\log(0.01)$
\item $\ln\left(e^3\right)$

\setcounter{HW}{\value{enumi}}
\end{enumerate}
\end{multicols}


\begin{multicols}{3}
\begin{enumerate}
\setcounter{enumi}{\value{HW}}

\item $\log_{4} (8)$
\item $\log_{6} (1)$
\item $\log_{13} \left(\sqrt{13}\right)$

\setcounter{HW}{\value{enumi}}
\end{enumerate}
\end{multicols}


\begin{multicols}{3}
\begin{enumerate}
\setcounter{enumi}{\value{HW}}

\item $\log_{36} \left(\sqrt[4]{36}\right)$
\item $7^{\log_{7} (3)}$
\item  $36^{\log_{36}(216)}$

\setcounter{HW}{\value{enumi}}
\end{enumerate}
\end{multicols}


\begin{multicols}{3}
\begin{enumerate}
\setcounter{enumi}{\value{HW}}

\item  $\log_{36} \left(36^{216}\right)$
\item $\ln \left(e^{5} \right)$
\item $\log \left(\sqrt[9]{10^{11}}\right)$

\setcounter{HW}{\value{enumi}}
\end{enumerate}
\end{multicols}


\begin{multicols}{3}
\begin{enumerate}
\setcounter{enumi}{\value{HW}}

\item  $\log\left( \sqrt[3]{10^5} \right)$
\item  $\ln \left( \frac{1}{\sqrt{e}}\right)$
\item $\log_{5} \left(3^{\log_{3} (5)}\right)$

\setcounter{HW}{\value{enumi}}
\end{enumerate}
\end{multicols}


\begin{multicols}{3}
\begin{enumerate}
\setcounter{enumi}{\value{HW}}

\item $\log\left(e^{\ln(100)}\right)$ 
\item $\log_{2}\left(3^{-\log_{3}(2)}\right)$
\item $\ln\left(42^{6\log(1)}\right)$ \label{simplifyloglast}

\setcounter{HW}{\value{enumi}}
\end{enumerate}
\end{multicols}

In Exercises \ref{domainlogfirst} - \ref{domainloglast}, find the domain of the function.

\begin{multicols}{2}
\begin{enumerate}
\setcounter{enumi}{\value{HW}}


\item $f(x) = \ln(x^{2} + 1)$ \label{domainlogfirst}
\item $f(x) = \log_{7}(4x + 8)$

\setcounter{HW}{\value{enumi}}
\end{enumerate}
\end{multicols}

\begin{multicols}{2}
\begin{enumerate}
\setcounter{enumi}{\value{HW}}


\item $g(t) = \ln(4t-20)$
\item $g(t) = \log \left(t^2+9t+18\right)$

\setcounter{HW}{\value{enumi}}
\end{enumerate}
\end{multicols}

\begin{multicols}{2}
\begin{enumerate}
\setcounter{enumi}{\value{HW}}

\item $f(x) = \log \left(\dfrac{x + 2}{x^{2} - 1}\right)$
\item $f(x) = \log\left(\dfrac{x^2+9x+18}{4x-20}\right)$

\setcounter{HW}{\value{enumi}}
\end{enumerate}
\end{multicols}

\begin{multicols}{2}
\begin{enumerate}
\setcounter{enumi}{\value{HW}}

\item $g(t) = \ln(7 - t) + \ln(t - 4)$
\item $g(t) = \ln(4t-20) + \ln\left(t^2+9t+18\right)$

\setcounter{HW}{\value{enumi}}
\end{enumerate}
\end{multicols}

\begin{multicols}{2}
\begin{enumerate}
\setcounter{enumi}{\value{HW}}

\item $f(x) = \log\left(x^2+x+1\right)$
\item $f(x) = \sqrt[4]{\log_{4} (x)}$

\setcounter{HW}{\value{enumi}}
\end{enumerate}
\end{multicols}

\begin{multicols}{2}
\begin{enumerate}
\setcounter{enumi}{\value{HW}}

\item $g(t) = \log_{9}(|t + 3| - 4)$
\item $g(t) = \ln(\sqrt{t - 4} - 3)$

\setcounter{HW}{\value{enumi}}
\end{enumerate}
\end{multicols}

\begin{multicols}{2}
\begin{enumerate}
\setcounter{enumi}{\value{HW}}

\item $f(x) = \dfrac{1}{3 - \log_{5} (x)}$
\item $f(x) = \dfrac{\sqrt{-1 - x}}{\log_{\frac{1}{2}} (x)}$

\setcounter{HW}{\value{enumi}}
\end{enumerate}
\end{multicols}

\begin{multicols}{2}
\begin{enumerate}
\setcounter{enumi}{\value{HW}}

\item $f(x) = \ln(-2x^{3} - x^{2} + 13x - 6)$  \label{domainloglast}

\setcounter{HW}{\value{enumi}}
\end{enumerate}
\end{multicols}


In Exercises \ref{graphlogfirst} - \ref{graphloglast}, sketch the graph of $g$ by starting with the graph of $f$ and using transformations.  Track at least three points of your choice and the vertical asymptote through the transformations. State the domain and range of $g$.


\begin{multicols}{2}
\begin{enumerate}
\setcounter{enumi}{\value{HW}}

\item  $f(x) = \log_{2}(x)$, $g(x) = \log_{2}(x+1)$ \label{graphlogfirst}

\item  $f(x) = \log_{\frac{1}{3}}(x)$, $g(x) = \log_{\frac{1}{3}}(x)+1$

\setcounter{HW}{\value{enumi}}
\end{enumerate}
\end{multicols}

\begin{multicols}{2}
\begin{enumerate}
\setcounter{enumi}{\value{HW}}


\item  $f(x) = \log_{3}(x)$, $g(x) = -\log_{3}(x-2)$

\item  $f(x) = \log(x)$, $g(x) = 2\log(x+20) -1$  

\setcounter{HW}{\value{enumi}}
\end{enumerate}
\end{multicols}

\begin{multicols}{2}
\begin{enumerate}
\setcounter{enumi}{\value{HW}}

\item  $f(t) = \log_{0.5}(t)$, $g(t) = 10 \log_{0.5}\left(\frac{t}{100}\right)$

\item  $f(t) = \log_{1.25}(t)$, $g(t) = \log_{1.25}(-t+1) + 2$

\setcounter{HW}{\value{enumi}}
\end{enumerate}
\end{multicols}

\begin{multicols}{2}
\begin{enumerate}
\setcounter{enumi}{\value{HW}}


\item  $f(t) = \ln(t)$, $g(t) = -\ln(8-t)$

\item  $f(t) = \ln(t)$, $g(t) = -10\ln\left(\frac{t}{10}\right)$ \label{graphloglast}

\setcounter{HW}{\value{enumi}}
\end{enumerate}
\end{multicols}

\begin{enumerate}
\setcounter{enumi}{\value{HW}}

\smallskip

\item  Verify that each function in Exercises \ref{graphlogfirst} - \ref{graphloglast} is the inverse of the corresponding function in Exercises \ref{graphexpfirsta} - \ref{graphexplasta} in Section \ref{ExponentialFunctions}.  (Match up \#\ref{graphexpfirsta} and \#\ref{graphlogfirst}, and so on.)

\setcounter{HW}{\value{enumi}}
\end{enumerate}

In Exercises, \ref{logformfirsta} - \ref{logformlasta}, the graph of a logarithmic function is given.  Find a formula for the function in the form $F(x) = a \cdot \log_{2}(bx-h)+k$.

\begin{multicols}{2}
\begin{enumerate}
\setcounter{enumi}{\value{HW}}

\item  \label{logformfirsta}  Points:  $\left( -\frac{5}{2}, -2 \right)$,  $\left(-2, -1 \right)$, $\left(-1,0 \right)$, \\
Asymptote:  $x = -3$. \\

\begin{mfpic}[13]{-4}{6}{-5}{3}
\axes
\tlabel[cc](6,-0.5){\scriptsize $x$}
\tlabel[cc](0.5,3){\scriptsize $y$}
\ymarks{-4, -3,-2,-1,1,2}
\xmarks{-3, -2, -1, 1,2,3,4,5}
\tlpointsep{4pt}
\axislabels {y}{{\scriptsize $-4$} -4,{\scriptsize $-3$} -3, {\scriptsize $-2$} -2, {\scriptsize $-1$} -1, {\scriptsize $1$} 1, {\scriptsize $2$} 2}
\axislabels {x}{{\scriptsize $1$} 1, {\scriptsize $2$} 2, {\scriptsize $3$} 3, {\scriptsize $4$} 4, {\scriptsize $5$} 5, {\scriptsize $-1 \hspace{7pt}$} -1, {\scriptsize $-2 \hspace{7pt}$} -2, {\scriptsize $-3 \hspace{7pt}$} -3}
\dashed \polyline{(-3,-5), (-3,3)}
\penwd{1.25pt}
\arrow \reverse \arrow \parafcn{-4.5, 2.1, 0.1}{( (2**(t+1))-3, t)}
\point[4pt]{(-2.5, -2), (-2,-1),(-1,0)}
\end{mfpic}

\vfill

\columnbreak

\item  Points:  $\left( 1, -1 \right)$, $\left( 2, 0 \right)$, $\left(\frac{5}{2}, 1 \right)$ \\
Asymptote:  $x = 3$. \\

\begin{mfpic}[13]{-6}{4}{-4}{4}
\axes
\tlabel[cc](4,-0.5){\scriptsize $x$}
\tlabel[cc](0.5,4){\scriptsize $y$}
\ymarks{-3,-2,-1,1,2,3}
\xmarks{-5 step 1 until 3}
\tlpointsep{4pt}
\axislabels {y}{{\scriptsize $-3$} -3, {\scriptsize $-2$} -2, {\scriptsize $-1$} -1, {\scriptsize $1$} 1, {\scriptsize $2$} 2, {\scriptsize $3$} 3}
\axislabels {x}{{\scriptsize $1$} 1, {\scriptsize $2$} 2, {\scriptsize $3$} 3, {\scriptsize $-1 \hspace{7pt}$} -1, {\scriptsize $-2 \hspace{7pt}$} -2, {\scriptsize $-3 \hspace{7pt}$} -3, {\scriptsize $-4 \hspace{7pt}$} -4, {\scriptsize $-5 \hspace{7pt}$} -5}
\dashed \polyline{(3,-4), (3,4)}
\penwd{1.25pt}
\arrow \reverse \arrow \parafcn{-3, 3.1, 0.1}{(3-(2**(-t)), t)}
\point[4pt]{(1,-1), (2,0), ( 2.5,1)}
\end{mfpic}


\setcounter{HW}{\value{enumi}}
\end{enumerate}
\end{multicols}

\begin{multicols}{2}
\begin{enumerate}
\setcounter{enumi}{\value{HW}}

\item  Points:  $\left(\frac{1}{2}, \frac{5}{2} \right)$, $\left(1, 3 \right)$, $\left(2, \frac{7}{2} \right)$, \\
Asymptote:  $x = 0$. \\

\begin{mfpic}[13]{-1}{9}{-1}{5}
\axes
\tlabel[cc](9,-0.5){\scriptsize $x$}
\tlabel[cc](0.5,5){\scriptsize $y$}
\ymarks{1 step 1 until 4}
\xmarks{1 step 1 until 8}
\tlpointsep{4pt}
\axislabels {y}{{\scriptsize $1$} 1, {\scriptsize $2$} 2, {\scriptsize $3$} 3, {\scriptsize $4$} 4}
\axislabels {x}{{\scriptsize $1$} 1, {\scriptsize $2$} 2, {\scriptsize $3$} 3, {\scriptsize $4$} 4, {\scriptsize $5$} 5, {\scriptsize $6$} 6, {\scriptsize $7$} 7, {\scriptsize $8$} 8}
\penwd{1.25pt}
\arrow \reverse \arrow \parafcn{1.25, 4.55, 0.1}{(2**(2*t-6), t)}
\point[4pt]{(0.5, 2.5), (1,3), (2,3.5) }
\end{mfpic}

\vfill

\columnbreak

\item  Points:  $\left( 6, -\frac{1}{2} \right)$, $\left(3,0 \right)$, $\left( \frac{3}{2}, \frac{1}{2} \right)$, \\
Asymptote:  $x = 0$.   \\

\begin{mfpic}[13]{-1}{9}{-4}{4}
\axes
\tlabel[cc](9,-0.5){\scriptsize $x$}
\tlabel[cc](0.5,4){\scriptsize $y$}
\ymarks{-3,-2,-1,1,2,3}
\xmarks{1,2,3,4,5,6,7,8}
\tlpointsep{4pt}
\axislabels {y}{{\scriptsize $-3$} -3, {\scriptsize $-2$} -2, {\scriptsize $-1$} -1, {\scriptsize $1$} 1, {\scriptsize $2$} 2, {\scriptsize $3$} 3}
\axislabels {x}{{\scriptsize $1$} 1, {\scriptsize $2$} 2, {\scriptsize $3$} 3,  {\scriptsize $7$} 7, {\scriptsize $8$} 8}
\penwd{1.25pt}
\arrow \reverse \arrow \parafcn{-0.79, 2, 0.1}{(3*(2**(-2*t)), t)}
\point[4pt]{(6, -0.5), (3,0), (1.5, 0.5)}
\end{mfpic}


\label{logformlasta} 

\setcounter{HW}{\value{enumi}}
\end{enumerate}
\end{multicols}


\begin{enumerate}
\setcounter{enumi}{\value{HW}}


\item \label{logformbase4exercise} Find a formula for each graph in Exercises \ref{logformfirsta} - \ref{logformlasta} of the form $G(x) =  a \cdot \log_{4}(bx-h)+k$.

\setcounter{HW}{\value{enumi}}
\end{enumerate}

In Exercises \ref{inverselogexpfirst} - \ref{inverselogexplast},  find the inverse of the function from the `procedural perspective' discussed in Example \ref{proceduralinverse} and graph the function and its inverse on the same set of axes.

\begin{multicols}{2}
\begin{enumerate}
\setcounter{enumi}{\value{HW}}

\item $f(x) = 3^{x + 2} - 4$  \label{inverselogexpfirst} 
\item $f(x) = \log_{4}(x - 1)$

\setcounter{HW}{\value{enumi}}
\end{enumerate}
\end{multicols}

\enlargethispage{.5in}
\vspace{-.2in}

\begin{multicols}{2}
\begin{enumerate}
\setcounter{enumi}{\value{HW}}

\item $g(t)= -2^{-t} + 1$
\item $g(t) = 5\log(t) - 2$ \label{inverselogexplast}

\setcounter{HW}{\value{enumi}}
\end{enumerate}
\end{multicols}

In Exercises \ref{decomposebasiclogfirst} - \ref{decomposebasicloglast}, write the given function as a nontrivial decomposition of functions as directed.

\begin{enumerate}
\setcounter{enumi}{\value{HW}}

\item  For $f(x) = \log_{2}(x+3) + 4$, find functions $g$ and $h$ so that $f=g+h$. \label{decomposebasiclogfirst}
\item  For $f(x) = \log(2x) - e^{-x}$, find functions $g$ and $h$ so that $f=g-h$. 
\item  For $f(t) = 3t \log(t)$, find functions $g$ and $h$ so that $f=gh$.
\item  For $r(x) = \dfrac{\ln(x)}{x}$, find functions $f$ and $g$ so $r = \dfrac{f}{g}$.
\item  For $k(t) = \ln(t^2+1)$, find functions $f$ and $g$  so that $k = g \circ f$.
\item  For $p(z) = (\ln(z))^2$, find functions $f$ and $g$ so $p = g \circ f$. \label{decomposebasicloglast}

\setcounter{HW}{\value{enumi}}
\end{enumerate}

\phantomsection
\label{logarithmicscales}

(Logarithmic Scales) In Exercises \ref{Richterexercise} - \ref{pHexercise}, we introduce three widely used measurement scales which involve common logarithms: the Richter scale, the decibel scale and the pH scale.  The computations involved in all three scales are nearly identical so pay attention to the subtle differences. \index{logarithmic scales}

\begin{enumerate}
\setcounter{enumi}{\value{HW}}

\item \label{Richterexercise} \index{Richter Scale} \index{earthquake ! Richter Scale} Earthquakes are complicated events and it is not our intent to provide a complete discussion of the science involved in them.  Instead, we refer the interested reader to a solid course in Geology\footnote{Rock-solid, perhaps?} or the U.S. Geological Survey's Earthquake Hazards Program found \href{http://earthquake.usgs.gov/}{\underline{here}} and present only a simplified version of the \href{http://en.wikipedia.org/wiki/Richter_scale}{\underline{Richter scale}}.  The Richter scale measures the magnitude of an earthquake by comparing the amplitude of the seismic waves of the given earthquake to those of a ``magnitude 0 event'', which was chosen to be a seismograph reading of $0.001$ millimeters recorded on a seismometer 100 kilometers from the earthquake's epicenter.  Specifically, the magnitude of an earthquake is given by \[M(x) = \log \left(\dfrac{x}{0.001}\right)\] where $x$ is the seismograph reading in millimeters of the earthquake recorded 100 kilometers from the epicenter.  

\begin{enumerate}

\item Show that $M(0.001) = 0$.
\item Compute $M(80,000)$.
\item Show that an earthquake which registered 6.7 on the Richter scale had a seismograph reading ten times larger than one which measured 5.7.
\item Find two news stories about recent earthquakes which give their magnitudes on the Richter scale.  How many times larger was the seismograph reading of the earthquake with larger magnitude?

\end{enumerate}

\item \label{decibelexercise} \index{decibel} \index{sound intensity level ! decibel} While the decibel scale can be used in many disciplines,\footnote{See this  \href{http://en.wikipedia.org/wiki/Decibel}{\underline{webpage}} for more information.} we shall restrict our attention to its use in acoustics, specifically its use in measuring the intensity level of sound. The Sound Intensity Level $L$ (measured in decibels) of a sound intensity $I$ (measured in watts per square meter) is given by \[L(I) = 10\log\left( \dfrac{I}{10^{-12}} \right).\] Like the Richter scale, this scale compares $I$ to baseline: $10^{-12} \frac{W}{m^{2}}$ is the threshold of human hearing. 

\begin{enumerate}

\item Compute $L(10^{-6})$.
\item Damage to your hearing can start with short term exposure to sound levels around 115 decibels.  What intensity $I$ is needed to produce this level? 
\item Compute $L(1)$.  How does this compare with the threshold of pain which is around 140 decibels?

\end{enumerate}

\item \label{pHexercise} \index{pH} \index{acidity of a solution ! pH} \index{alkalinity of a solution ! pH} The pH of a solution is a measure of its acidity or alkalinity.  Specifically, $\mbox{pH} = -\log[\mbox{H}^{+}]$ where $[\mbox{H}^{+}]$ is the hydrogen ion concentration in moles per liter.  A solution with a pH less than 7 is an acid, one with a pH greater than 7 is a base (alkaline) and a pH of 7 is regarded as neutral.

\begin{enumerate}

\item The hydrogen ion concentration of pure water is $[\mbox{H}^{+}] = 10^{-7}$.  Find its pH.
\item Find the pH of a solution with $[\mbox{H}^{+}] = 6.3 \times 10^{-13}$.
\item The pH of gastric acid (the acid in your stomach) is about $0.7$.  What is the corresponding hydrogen ion concentration?

\end{enumerate}

\item Use the definition of logarithm to explain why  $\log_{b} 1 = 0$ and $\log_{b} b = 1$ for every $b > 0, \; b \neq 1$.


\setcounter{HW}{\value{enumi}}
\end{enumerate}


\newpage

\subsection{Answers}

\begin{multicols}{3}
\begin{enumerate}

\item $\log_{2}(8) = 3$

\item  $\log_{5}\left(\frac{1}{125}\right) = -3$

\item  $\log_{4}(32) = \frac{5}{2}$

\setcounter{HW}{\value{enumi}}
\end{enumerate}
\end{multicols}

\begin{multicols}{3}
\begin{enumerate}
\setcounter{enumi}{\value{HW}}


\item  $\log_{\frac{1}{3}}(9) = -2$

\item  $\log_{\frac{4}{25}}\left(\frac{5}{2}\right) = -\frac{1}{2}$

\item  $\log(0.001) = -3$

\setcounter{HW}{\value{enumi}}
\end{enumerate}
\end{multicols}

\begin{multicols}{3}
\begin{enumerate}
\setcounter{enumi}{\value{HW}}


\item  $\ln(1) = 0$

\item  $5^{2} = 25$

\item  $(25)^{\frac{1}{2}} = 5$

\setcounter{HW}{\value{enumi}}
\end{enumerate}
\end{multicols}

\begin{multicols}{3}
\begin{enumerate}
\setcounter{enumi}{\value{HW}}


\item  $3^{-4} = \frac{1}{81}$

\item  $\left(\frac{4}{3} \right)^{-1} = \frac{3}{4}$

\item  $10^{2} = 100$

\setcounter{HW}{\value{enumi}}
\end{enumerate}
\end{multicols}

\begin{multicols}{3}
\begin{enumerate}
\setcounter{enumi}{\value{HW}}


\item  $10^{-1} = 0.1$

\item  $e^{1} = e$

\item  $e^{-\frac{1}{2}} = \frac{1}{\sqrt{e}}$

\setcounter{HW}{\value{enumi}}
\end{enumerate}
\end{multicols}

\begin{multicols}{3}
\begin{enumerate}
\setcounter{enumi}{\value{HW}}

\item $\log_{3} (27) = 3$
\item $\log_{6} (216) = 3$
\item $\log_{2} (32) = 5$

\setcounter{HW}{\value{enumi}}
\end{enumerate}
\end{multicols}

\begin{multicols}{3}
\begin{enumerate}
\setcounter{enumi}{\value{HW}}


\item  $\log_{6} \left( \frac{1}{36} \right) = -2$
\item $\log_{8} (4) = \frac{2}{3}$
\item $\log_{36} (216) = \frac{3}{2}$

\setcounter{HW}{\value{enumi}}
\end{enumerate}
\end{multicols}

\begin{multicols}{3}
\begin{enumerate}
\setcounter{enumi}{\value{HW}}


\item $\log_{\frac{1}{5}} (625) = -4$
\item  $\log_{\frac{1}{6}} (216) = -3$
\item $\log_{36} (36)=1$ 

\setcounter{HW}{\value{enumi}}
\end{enumerate}
\end{multicols}

\begin{multicols}{3}
\begin{enumerate}
\setcounter{enumi}{\value{HW}}


\item $\log \frac{1}{1000000} = -6$
\item $\log(0.01) = -2$
\item $\ln\left(e^3\right) = 3$

\setcounter{HW}{\value{enumi}}
\end{enumerate}
\end{multicols}

\begin{multicols}{3}
\begin{enumerate}
\setcounter{enumi}{\value{HW}}


\item $\log_{4} (8) = \frac{3}{2}$
\item $\log_{6} (1) = 0$
\item $\log_{13} \left(\sqrt{13}\right) = \frac{1}{2}$

\setcounter{HW}{\value{enumi}}
\end{enumerate}
\end{multicols}

\begin{multicols}{3}
\begin{enumerate}
\setcounter{enumi}{\value{HW}}


\item $\log_{36} \left(\sqrt[4]{36}\right) = \frac{1}{4}$
\item $7^{\log_{7} (3)} = 3$
\item  $36^{\log_{36}(216)} = 216$

\setcounter{HW}{\value{enumi}}
\end{enumerate}
\end{multicols}

\begin{multicols}{3}
\begin{enumerate}
\setcounter{enumi}{\value{HW}}


\item  $\log_{36} \left(36^{216}\right) = 216$
\item $\ln(e^{5}) = 5$
\item $\log \left(\sqrt[9]{10^{11}}\right) = \frac{11}{9}$

\setcounter{HW}{\value{enumi}}
\end{enumerate}
\end{multicols}

\begin{multicols}{3}
\begin{enumerate}
\setcounter{enumi}{\value{HW}}


\item  $\log\left( \sqrt[3]{10^5} \right) = \frac{5}{3}$
\item  $\ln \left( \frac{1}{\sqrt{e}}\right) = -\frac{1}{2} $
\item $\log_{5} \left(3^{\log_{3} 5}\right) = 1$

\setcounter{HW}{\value{enumi}}
\end{enumerate}
\end{multicols}

\begin{multicols}{3}
\begin{enumerate}
\setcounter{enumi}{\value{HW}}


\item $\log\left(e^{\ln(100)}\right) = 2$
\item $\log_{2}\left(3^{-\log_{3}(2)}\right) = -1$
\item $\ln\left(42^{6\log(1)}\right) = 0$

\setcounter{HW}{\value{enumi}}
\end{enumerate}
\end{multicols}


\begin{multicols}{3}
\begin{enumerate}
\setcounter{enumi}{\value{HW}}


\item $(-\infty, \infty)$
\item $(-2, \infty)$
\item $(5, \infty)$

\setcounter{HW}{\value{enumi}}
\end{enumerate}
\end{multicols}


\begin{multicols}{3}
\begin{enumerate}
\setcounter{enumi}{\value{HW}}


\item $(-\infty, -6) \cup (-3, \infty)$
\item $(-2, -1) \cup (1, \infty)$
\item $(-6,-3) \cup (5, \infty)$

\setcounter{HW}{\value{enumi}}
\end{enumerate}
\end{multicols}


\begin{multicols}{3}
\begin{enumerate}
\setcounter{enumi}{\value{HW}}


\item $(4, 7)$
\item $(5, \infty)$
\item $(-\infty, \infty)$

\setcounter{HW}{\value{enumi}}
\end{enumerate}
\end{multicols}


\begin{multicols}{3}
\begin{enumerate}
\setcounter{enumi}{\value{HW}}


\item $[1, \infty)$
\item $(-\infty, -7) \cup (1, \infty)$
\item $(13, \infty)$

\setcounter{HW}{\value{enumi}}
\end{enumerate}
\end{multicols}


\begin{multicols}{3}
\begin{enumerate}
\setcounter{enumi}{\value{HW}}


\item $(0, 125) \cup (125, \infty)$
\item No domain
\item $(-\infty, -3) \cup \left(\frac{1}{2}, 2\right)$

\setcounter{HW}{\value{enumi}}
\end{enumerate}
\end{multicols}

\newpage

\begin{multicols}{2}
\begin{enumerate}
\setcounter{enumi}{\value{HW}}


\item  Domain of $g$: $(-1, \infty)$\\
 Range of $g$:  $(-\infty, \infty)$ \\
 Points:  $\left( -\frac{1}{2}, -1 \right)$, $(0,0)$, $(1,1)$\\
 Asymptote: $x = -1$ \\

\begin{mfpic}[15]{-2}{9}{-4}{4}
\point[4pt]{(-0.5, -1), (0,0), (1,1)}
\axes
\tlabel[cc](0.5,4){\scriptsize $y$}
\tlabel[cc](9,-0.5){\scriptsize $x$}
\tlabel[cc](0.6, -3.1){\tiny $-3$}
\tlabel[cc](0.6, -2.1){\tiny $-2$}
\tlabel[cc](0.6, -1.1){\tiny $-1$}
\tcaption{$y = g(x) = \log_{2}(x+1)$}
\ymarks{-3,-2,-1,1,2,3}
\xmarks{-1,1,2,3,4,5,6,7,8}
\tlpointsep{4pt}
\axislabels {y}{{\tiny $1$} 1, {\tiny $2$} 2, {\tiny $3$} 3}
\axislabels {x}{{\tiny $1$} 1, {\tiny $2$} 2, {\tiny $3$} 3, {\tiny $4$} 4, {\tiny $5$} 5, {\tiny $6$} 6, {\tiny $7$} 7, {\tiny $8$} 8}
\dashed \polyline{(-1,-3),(-1,4)}
\penwd{1.25pt}
\arrow \reverse \arrow \parafcn{-3.5, 3.1, 0.1}{((2**(t))-1,t)}
\end{mfpic}

\vfill

\columnbreak

\item  Domain of $g$:  $(0, \infty)$\\
 Range of $g$:  $(-\infty, \infty)$ \\
 Points:   $\left(\frac{1}{3}, 2 \right)$, $(1,1)$, $(3,0)$ \\
 Asymptote:  $x = 0$ \\

\begin{mfpic}[15]{-1}{10}{-4}{4}
\point[4pt]{(3,0), (1,1), (0.3333,2)}
\axes
\tlabel[cc](0.5, 4){\scriptsize $y$}
\tlabel[cc](10,-0.5){\scriptsize $x$}
\tcaption{$y = g(x) = \log_{\frac{1}{3}}(x)+1$}
\ymarks{-3,-2,-1,1,2,3}
\xmarks{1,2,3,4,5,6,7,8,9}
\tlpointsep{4pt}
\axislabels {y}{{\tiny $-3$} -3, {\tiny $-2$} -2, {\tiny $-1$} -1, {\tiny $1$} 1, {\tiny $2$} 2, {\tiny $3$} 3}
\axislabels {x}{{\tiny $1$} 1, {\tiny $2$} 2, {\tiny $3$} 3, {\tiny $4$} 4, {\tiny $5$} 5, {\tiny $6$} 6, {\tiny $7$} 7, {\tiny $8$} 8, {\tiny $9$} 9}
\penwd{1.25pt}
\arrow \reverse \arrow \parafcn{-1.05, 3.5, 0.1}{(3**(1-t),t)}
\end{mfpic} 

\setcounter{HW}{\value{enumi}}
\end{enumerate}
\end{multicols}

\begin{multicols}{2}
\begin{enumerate}
\setcounter{enumi}{\value{HW}}


\item  Domain of $g$: $(2, \infty)$\\
 Range of $g$:  $(-\infty, \infty)$ \\
 Points:  $\left( \frac{7}{3},1 \right)$, $(3,0)$, $(5,-1)$ \\
 Asymptote: $x = 2$ \\
 
\begin{mfpic}[15]{-1}{12}{-4}{4}
\point[4pt]{(2.3333,1), (3,0), (5,-1)}
\axes
\tlabel[cc](0.5,4){\scriptsize $y$}
\tlabel[cc](12,-0.5){\scriptsize $x$}
\tcaption{$y = g(x) = -\log_{3}(x-2)$}
\ymarks{-3,-2,-1,1,2,3}
\xmarks{1,2,3,4,5,6,7,8,9,10,11}
\tlpointsep{4pt}
\axislabels {y}{{\tiny $-3$} -3, {\tiny $-2$} -2, {\tiny $-1$} -1, {\tiny $1$} 1, {\tiny $2$} 2, {\tiny $3$} 3}
\axislabels {x}{{\tiny $1$} 1, {\tiny $2$} 2, {\tiny $3$} 3, {\tiny $4$} 4, {\tiny $5$} 5, {\tiny $6$} 6, {\tiny $7$} 7, {\tiny $8$} 8, {\tiny $9$} 9, {\tiny $10$} 10, {\tiny $11$} 11}
\dashed \polyline{(2,-3),(2,4)}
\penwd{1.25pt}
\arrow \reverse \arrow \parafcn{-2.05, 2.5, 0.1}{(2+3**(0-t),t)}
\end{mfpic} 

\vfill

\columnbreak


\item  Domain of $g$: $(-20, \infty)$\\
 Range of $g$:  $(-\infty, \infty)$\\
 Points:  $(-19, -1)$, $(-10,1)$, $(80,3)$\\
 Asymptote:  $x = -20$ \\

\begin{mfpic}[15]{-2}{11}{-3}{4}
\point[4pt]{(-1.9,-1), (-1,1), (8,3)}
\axes
\tlabel[cc](0.5,4){\scriptsize $y$}
\tlabel[cc](11,-0.5){\scriptsize $x$}
\tcaption{$y = g(x) = 2\log(x+20) -1$}
\ymarks{-3,-2,-1,1,2,3}
\xmarks{-2,-1,1,2,3,4,5,6,7,8,9,10}
\tlpointsep{4pt}
\axislabels {y}{{\tiny $-3$} -3, {\tiny $-2$} -2, {\tiny $-1$} -1, {\tiny $1$} 1, {\tiny $2$} 2, {\tiny $3$} 3}
\axislabels {x}{{\tiny $-10 \hspace{5pt}$} -1,{\tiny $10$} 1, {\tiny $20$} 2, {\tiny $30$} 3, {\tiny $40$} 4, {\tiny $50$} 5, {\tiny $60$} 6, {\tiny $70$} 7, {\tiny $80$} 8, {\tiny $90$} 9, {\tiny $100$} 10}
\dashed \polyline{(-2,-3), (-2,4)}
\penwd{1.25pt}
\arrow \reverse \arrow \parafcn{-3, 3.06, 0.1}{(((10**((t+1)/2))-20)/10,t)}
\end{mfpic}

\setcounter{HW}{\value{enumi}}
\end{enumerate}
\end{multicols}

\newpage

\begin{multicols}{2}
\begin{enumerate}
\setcounter{enumi}{\value{HW}}

\item  Domain of $g$: $(0, \infty)$ \\
  Range of $g$:  $(-\infty, \infty)$ \\
  Points:  $(50,10)$,   $(100,0)$, $(200,-10)$  \\
  Asymptote: $t= 0$\\
 
\begin{mfpic}[15]{-1}{9}{-4}{4}
\point[4pt]{(2,-1), (1,0), (0.5,1)}
\axes
\tlabel[cc](9,-0.5){\scriptsize $t$}
\tlabel[cc](0.5,4){\scriptsize $y$}
\tcaption{ $y = g(t) = 10 \log_{0.5} \left( \frac{t}{100} \right)$}
\ymarks{-3,-2,-1,1,2,3}
\xmarks{1,2,3,4,5,6,7,8}
\tlpointsep{4pt}
\axislabels {y}{{\tiny $-30$} -3, {\tiny $-20$} -2, {\tiny $-10$} -1, {\tiny $10$} 1, {\tiny $20$} 2, {\tiny $30$} 3}
\axislabels {x}{{\tiny $100$} 1, {\tiny $200$} 2, {\tiny $300$} 3, {\tiny $400$} 4, {\tiny $500$} 5, {\tiny $600$} 6, {\tiny $700$} 7, {\tiny $800$} 8}
\penwd{1.25pt}
\arrow \reverse \arrow \parafcn{-3, 3, 0.1}{(0.5**t, t)}
\end{mfpic}

\vfill

\columnbreak


\item  Domain of $g$: $(-\infty, 1)$ \\
  Range of $g$: $(-\infty, \infty)$  \\
  Points:  $(-0.25,3)$,  $(0,2)$, $(0.2,1)$ \\
  Asymptote: $t = 1$\\
 
\begin{mfpic}[8][10]{-11}{11}{-8}{5}
\point[4pt]{(2,1), (0,2), ( -2.5,3)}
\axes
\tlabel[cc](11,-0.5){\scriptsize $t$}
\tlabel[cc](0.5,5){\scriptsize $y$}
\tcaption{ $y = g(t) = \log_{1.25}(-t+1) + 2$}
\ymarks{-8 step 1 until 4}
\xmarks{-10 step 1 until 10}
\tlpointsep{4pt}
\axislabels {y}{{\tiny $-8$} -8,{\tiny $-6$} -6,{\tiny $-4$} -4,{\tiny $-2$} -2, {\tiny $2$} 2, {\tiny $4$} 4}
\axislabels {x}{ {\tiny $0.2$} 2,  {\tiny $0.4$} 4,  {\tiny $0.6$} 6, {\tiny $0.8$} 8, {\tiny $1$} 10, {\tiny $-0.2  \hspace{7pt}$} -2, {\tiny $-0.4  \hspace{7pt}$} -4,  {\tiny $-0.6  \hspace{7pt}$} -6, {\tiny $-0.8  \hspace{7pt}$} -8, {\tiny $-1  \hspace{7pt}$} -10}
 
\dashed \polyline{(10,-8), (10,5)}
\penwd{1.25pt}
\arrow \reverse \arrow \parafcn{-8, 5, 0.1}{((1- (1.25**(t-2)))*10,t)}
\end{mfpic}

\setcounter{HW}{\value{enumi}}
\end{enumerate}
\end{multicols}


\begin{multicols}{2}
\begin{enumerate}
\setcounter{enumi}{\value{HW}}


\item  Domain of $g$:  $(-\infty, 8)$\\
 Range of $g$:$(-\infty, \infty)$\\
 Points:  $(8-e, -1) \approx (5.28, -1)$,\\
 $(7,0)$, $(8 - e^{-1}, 1) \approx (7.63,1)$ \\
 Asymptote:  $t = 8$ \\

\begin{mfpic}[15]{-1}{9}{-4}{4}
\point[4pt]{(7.6321,1), (7,0), (5.282,-1)}
\axes
\tlabel[cc](0.5,4){\scriptsize $y$}
\tlabel[cc](9,-0.5){\scriptsize $t$}
\tcaption{$y = g(t) = -\ln(8-t)$}
\ymarks{-3,-2,-1,1,2,3}
\xmarks{1,2,3,4,5,6,7,8}
\tlpointsep{4pt}
\axislabels {y}{{\tiny $-3$} -3, {\tiny $-2$} -2, {\tiny $-1$} -1, {\tiny $1$} 1, {\tiny $2$} 2, {\tiny $3$} 3}
\axislabels {x}{{\tiny $1$} 1, {\tiny $2$} 2, {\tiny $3$} 3, {\tiny $4$} 4, {\tiny $5$} 5, {\tiny $6$} 6, {\tiny $7$} 7, {\tiny $8$} 8}
\dashed \polyline{(8,-3), (8,4)}
\penwd{1.25pt}
\arrow \reverse \arrow \parafcn{-2.25, 2, 0.1}{(8-exp(0-t),t)}
\end{mfpic} 

\vfill

\columnbreak

\item  Domain of $g$:  $(0, \infty)$\\
 Range of $g$:  $(-\infty, \infty)$\\
 Points:  $(10e^{-1}, 10) \approx (3.68. 10)$ \\
 $(10,0)$, $(10e, -10) \approx (27.18, -10)$ \\
 Asymptote: $t = 0$ \\

\begin{mfpic}[15]{-1}{9}{-4}{4}
\point[4pt]{(0.3679,1), (1,0), (2.718,-1)}
\axes
\tlabel[cc](0.5,4){\scriptsize $y$}
\tlabel[cc](9,-0.5){\scriptsize $t$}
\tcaption{$y = g(t) = -10\ln\left(\frac{t}{10}\right)$}
\ymarks{-3,-2,-1,1,2,3}
\xmarks{1,2,3,4,5,6,7,8}
\tlpointsep{4pt}
\axislabels {y}{{\tiny $-30$} -3,{\tiny $-20$} -2,{\tiny $-10$} -1, {\tiny $10$} 1, {\tiny $20$} 2, {\tiny $30$} 3}
\axislabels {x}{{\tiny $10$} 1, {\tiny $20$} 2, {\tiny $30$} 3, {\tiny $40$} 4, {\tiny $50$} 5, {\tiny $60$} 6, {\tiny $70$} 7, {\tiny $80$} 8}
\penwd{1.25pt}
\arrow \reverse \arrow \parafcn{-2.15, 2, 0.1}{(exp(0-t),t)}
\end{mfpic}

\setcounter{HW}{\value{enumi}}
\end{enumerate}
\end{multicols}

\newpage


\begin{multicols}{2}
\begin{enumerate}
\setcounter{enumi}{\value{HW}}
\addtocounter{enumi}{1}

\item $F(x) = \log_{2}(x+3) - 1$

\item  $F(x) = -\log_{2}(-x+3)$


\setcounter{HW}{\value{enumi}}
\end{enumerate}
\end{multicols}

\begin{multicols}{2}
\begin{enumerate}
\setcounter{enumi}{\value{HW}}


\item $F(x) = \frac{1}{2} \log_{2}(x) + 3$

\item  $F(x) = -\frac{1}{2} \log_{2} \left( \frac{x}{3} \right)$

\setcounter{HW}{\value{enumi}}
\end{enumerate}
\end{multicols}

\begin{enumerate}
\setcounter{enumi}{\value{HW}}

\item  In order, the formulas for $G(x)$ are:

\begin{multicols}{2}
\begin{itemize}

\item $G(x) = 2\log_{4}(x+3) - 1$

\item  $G(x) = -2\log_{4}(-x+3)$

\end{itemize}
\end{multicols}

\begin{multicols}{2}
\begin{itemize}

\item $G(x) =  \log_{4}(x) + 3$

\item  $G(x) = - \log_{4} \left( \frac{x}{3} \right)$

\end{itemize}
\end{multicols}


\setcounter{HW}{\value{enumi}}
\end{enumerate}

\begin{multicols}{2}
\begin{enumerate}
\setcounter{enumi}{\value{HW}}


\item $y = f(x) = 3^{x + 2} - 4$\\
{\boldmath $y = f^{-1}(x) = \log_{3}(x + 4) - 2$}\\

\begin{mfpic}[13.5]{-5}{7}{-5}{7}
\axes
\tlabel[cc](7,-0.5){\scriptsize $x$}
\tlabel[cc](0.5,7){\scriptsize $y$}
\xmarks{-4 step 1 until 6}
\ymarks{-4 step 1 until 6}
\tlpointsep{4pt}
\tiny
\axislabels {x}{{$-4 \hspace{7pt}$} -4, {$-3 \hspace{7pt}$} -3, {$-2 \hspace{7pt}$} -2, {$-1 \hspace{7pt}$} -1, {$1$} 1, {$2$} 2, {$3$} 3, {$4$} 4, {$5$} 5, {$6$} 6}
\axislabels {y}{{$-4$} -4, {$-3$} -3, {$-2$} -2, {$-1$} -1,{$1$} 1, {$2$} 2, {$3$} 3, {$4$} 4, {$5$} 5, {$6$} 6}
\normalsize
\arrow \reverse \arrow \function{-5, 0.15, 0.1}{(3**(x+2))-4}
\dashed \polyline{(-4,-5),(-4,7)}
\dashed \polyline{(-5,-5),(5,5)}
\dashed \polyline{(-5,-4),(7,-4)}
\penwd{1.5pt}
\arrow \reverse \arrow \function{-3.95, 7, 0.1}{(ln(x+4))/(ln(3.0)) - 2}
\end{mfpic}

\vfill

\columnbreak

\item $y = f(x) = \log_{4}(x - 1)$\\
{\boldmath $y = f^{-1}(x) = 4^{x} + 1$}\\

\begin{mfpic}[18]{-3}{7}{-3}{7}
\axes
\tlabel[cc](7,-0.5){\scriptsize $x$}
\tlabel[cc](0.5,7){\scriptsize $y$}
\xmarks{-2 step 1 until 6}
\ymarks{-2 step 1 until 6}
\tlpointsep{4pt}
\tiny
\axislabels {x}{{$-2 \hspace{7pt}$} -2, {$-1 \hspace{7pt}$} -1, {$1$} 1, {$2$} 2, {$3$} 3, {$4$} 4, {$5$} 5, {$6$} 6}
\axislabels {y}{{$-2$} -2, {$-1$} -1,{$1$} 1, {$2$} 2, {$3$} 3, {$4$} 4, {$5$} 5, {$6$} 6}
\normalsize
\arrow \reverse \arrow \function{1.03, 7, 0.1}{(ln(x-1))/(ln(4.0))}
\dashed \polyline{(-3,1),(7,1)}
\dashed \polyline{(1,7),(1,-3)}
\dashed \polyline{(-2,-2),(5,5)}
\penwd{1.5pt}
\arrow \reverse \arrow \function{-3, 1.25, 0.1}{(4**x) + 1}
\end{mfpic}

\setcounter{HW}{\value{enumi}}
\end{enumerate}
\end{multicols}


\begin{multicols}{2}
\begin{enumerate}
\setcounter{enumi}{\value{HW}}

\item $y = g(t) = -2^{-t} + 1$\\
{ \boldmath $y = g^{-1}(t) = -\log_{2}(-t+1)$} \\

\begin{mfpic}[22.5]{-3}{3}{-3}{3}
\axes
\tlabel[cc](3,-0.25){\scriptsize $t$}
\tlabel[cc](0.25,3){\scriptsize $y$}
\xmarks{-2 step 1 until 2}
\ymarks{-2 step 1 until 2}
\tlpointsep{4pt}
\tiny
\axislabels {x}{{$-2 \hspace{7pt}$} -2, {$-1 \hspace{7pt}$} -1, {$1$} 1, {$2$} 2}
\axislabels {y}{{$-2$} -2, {$-1$} -1,{$1$} 1, {$2$} 2}
\normalsize
\arrow \reverse \arrow \function{-2, 3, 0.1}{1-(2**(-x))}
\dashed \polyline{(-3,1),(3,1)}
\dashed \polyline{(1,-3),(1,3)}
\dashed \polyline{(-2.5,-2.5),(2.5,2.5)}
\penwd{1.5pt}
\arrow \reverse \arrow \function{-3, 0.87, 0.1}{-(ln(1-x))/(ln(2.0))}
\end{mfpic}


\vfill

\columnbreak

\item $y = g(t) = 5\log(t) - 2$\\
{\boldmath $y = g^{-1}(t) = 10^{\frac{t + 2}{5}}$}\\

\begin{mfpic}[13.5]{-5}{6}{-5}{6}
\axes
\tlabel[cc](6,-0.5){\scriptsize $t$}
\tlabel[cc](0.5,6){\scriptsize $y$}
\xmarks{-4 step 1 until 5}
\ymarks{-4 step 1 until 5}
\tlpointsep{4pt}
\tiny
\axislabels {x}{{$-4 \hspace{7pt}$} -4, {$-3 \hspace{7pt}$} -3, {$-2 \hspace{7pt}$} -2, {$-1 \hspace{7pt}$} -1, {$1$} 1, {$2$} 2, {$3$} 3, {$4$} 4, {$5$} 5}
\axislabels {y}{{$-4$} -4, {$-3$} -3, {$-2$} -2, {$-1$} -1,{$1$} 1, {$2$} 2, {$3$} 3, {$4$} 4, {$5$} 5}
\normalsize
\arrow \reverse \arrow \function{0.3, 6, 0.1}{((5*ln(x))/ln(10.0)) - 2}
\dashed \polyline{(-4,-4),(4,4)}
\penwd{1.5pt}
\arrow \reverse \arrow \function{-5, 1.8, 0.1}{10**((x+2)/5)}
\end{mfpic}

\setcounter{HW}{\value{enumi}}
\end{enumerate}
\end{multicols}

\begin{enumerate}
\setcounter{enumi}{\value{HW}}

\item  One solution is $g(x) = \log_{2}(x+3)$ and $h(x) = 4$.
\item  One solution is $g(x) = \log(2x)$ and $h(x) = e^{-x}$. 
\item  One solution is $g(t) = 3t$ and $h(t) = \log(t)$.
\item  One solution is $f(x) = \ln(x)$ and $g(x)=x$.
\item  One solution is $f(t) = t^2+1$ and $g(t) = \ln(t)$.
\item  One solution is $f(z) = \ln(z)$ and $g(z) = z^2$. 

\setcounter{HW}{\value{enumi}}
\end{enumerate}


\begin{enumerate}
\setcounter{enumi}{\value{HW}}


\item \begin{enumerate}

\item $M(0.001) = \log \left(\frac{0.001}{0.001} \right) = \log(1) = 0$.
\item $M(80,000) = \log \left(\frac{80,000}{0.001} \right) = \log(80,000,000) \approx 7.9$.

\end{enumerate}

\item \begin{enumerate}

\item $L(10^{-6}) = 60$ decibels.
\item $I = 10^{-.5} \approx 0.316$ watts per square meter.
\item Since $L(1) = 120$ decibels and $L(100) = 140$ decibels, a sound with intensity level 140 decibels has an intensity 100 times greater than a sound with intensity level 120 decibels.

\end{enumerate}

\item \begin{enumerate}

\item The pH of pure water is 7.
\item If $[\mbox{H}^{+}] = 6.3 \times 10^{-13}$ then the solution has a pH of 12.2.
\item $[\mbox{H}^{+}] = 10^{-0.7} \approx .1995$ moles per liter.

\end{enumerate}
\setcounter{HW}{\value{enumi}}
\end{enumerate}



\end{document}
