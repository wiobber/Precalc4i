\documentclass{ximera}

\begin{document}
	\author{Stitz-Zeager}
	\xmtitle{Exercises for The Circular Functions Sineand Cosine}{}

\mfpicnumber{1} \opengraphsfile{ExercisesforTheCircularFunctionsSineandCosine} % mfpic settings added 



In Exercises \ref{valuefirst} - \ref{valuelast}, find the exact value of the cosine and sine of the given angle.

\begin{multicols}{4}

\begin{enumerate}

\item $\theta = 0$ \vphantom{$\dfrac{\pi}{4}$} \label{valuefirst}
\item $\theta = \dfrac{\pi}{4}$
\item $\theta = \dfrac{\pi}{3}$
\item $\theta = \dfrac{\pi}{2}$

\setcounter{HW}{\value{enumi}}

\end{enumerate}

\end{multicols}

\begin{multicols}{4}

\begin{enumerate}

\setcounter{enumi}{\value{HW}}

\item $\theta = \dfrac{2\pi}{3}$
\item $\theta = \dfrac{3\pi}{4}$
\item $\theta = \pi$ \vphantom{$\dfrac{7\pi}{4}$}
\item $\theta = \dfrac{7\pi}{6}$

\setcounter{HW}{\value{enumi}}

\end{enumerate}

\end{multicols}

\begin{multicols}{4}

\begin{enumerate}

\setcounter{enumi}{\value{HW}}

\item $\theta = \dfrac{5\pi}{4}$
\item $\theta = \dfrac{4\pi}{3}$
\item $\theta = \dfrac{3\pi}{2}$
\item $\theta = \dfrac{5\pi}{3}$

\setcounter{HW}{\value{enumi}}

\end{enumerate}

\end{multicols}

\begin{multicols}{4}

\begin{enumerate}

\setcounter{enumi}{\value{HW}}

\item $\theta = \dfrac{7\pi}{4}$
\item $\theta = \dfrac{23\pi}{6}$
\item $\theta = -\dfrac{13\pi}{2}$
\item $\theta = -\dfrac{43\pi}{6}$

\setcounter{HW}{\value{enumi}}

\end{enumerate}

\end{multicols}

\begin{multicols}{4}

\begin{enumerate}

\setcounter{enumi}{\value{HW}}

\item $\theta = -\dfrac{3\pi}{4}$
\item $\theta = -\dfrac{\pi}{6}$ \vphantom{$\dfrac{7\pi}{4}$}
\item $\theta = \dfrac{10\pi}{3}$
\item $\theta = 117\pi$ \vphantom{$\dfrac{7\pi}{4}$} \label{valuelast}

\setcounter{HW}{\value{enumi}}

\end{enumerate}

\end{multicols}

In Exercises \ref{solveforanglefirst} - \ref{solveforanglelast}, find all of the angles which satisfy the given equation.

\begin{multicols}{3}

\begin{enumerate}

\setcounter{enumi}{\value{HW}}

\item $\sin(\theta) = \dfrac{1}{2}$ \vphantom{$\dfrac{2}{2}$} \label{solveforanglefirst}
\item $\cos(\theta) = -\dfrac{\sqrt{3}}{2}$
\item $\sin(\theta) = 0$ \vphantom{$\dfrac{2}{2}$}

\setcounter{HW}{\value{enumi}}

\end{enumerate}

\end{multicols}

\begin{multicols}{3}

\begin{enumerate}

\setcounter{enumi}{\value{HW}}

\item $\cos(\theta) = \dfrac{\sqrt{2}}{2}$
\item $\sin(\theta) = \dfrac{\sqrt{3}}{2}$
\item $\cos(\theta) = -1$ \vphantom{$\dfrac{\sqrt{2}}{2}$}

\setcounter{HW}{\value{enumi}}

\end{enumerate}

\end{multicols}

\begin{multicols}{3}

\begin{enumerate}

\setcounter{enumi}{\value{HW}}

\item  $\sin(\theta) = -1$ \vphantom{$\dfrac{\sqrt{2}}{2}$}
\item  $\cos(\theta) = \dfrac{\sqrt{3}}{2}$
\item  $\cos(\theta) = -1.001$ \vphantom{$\dfrac{\sqrt{2}}{2}$} \label{solveforanglelast}

\setcounter{HW}{\value{enumi}}

\end{enumerate}

\end{multicols}

In Exercises \ref{solvefortfirst} - \ref{solvefortlast}, solve the equation for $t$.  (See the remarks on page \ref{cosinesineequationsrealnumbers}.)

\begin{multicols}{3}

\begin{enumerate}

\setcounter{enumi}{\value{HW}}

\item $\cos(t) = 0$ \vphantom{$\dfrac{\sqrt{2}}{2}$} \label{solvefortfirst}
\item $\sin(t) = -\dfrac{\sqrt{2}}{2}$
\item $\cos(t) = 3$ \vphantom{$\dfrac{\sqrt{2}}{2}$}

\setcounter{HW}{\value{enumi}}

\end{enumerate}

\end{multicols}

\begin{multicols}{3}

\begin{enumerate}

\setcounter{enumi}{\value{HW}}

\item $\sin(t) = -\dfrac{1}{2}$
\item $\cos(t) = \dfrac{1}{2}$
\item $\sin(t) = -2$ \vphantom{$\dfrac{1}{2}$}

\setcounter{HW}{\value{enumi}}

\end{enumerate}

\end{multicols}

\begin{multicols}{3}

\begin{enumerate}

\setcounter{enumi}{\value{HW}}

\item $\cos(t) = 1$ \vphantom{$\dfrac{\sqrt{2}}{2}$}
\item $\sin(t) = 1$ \vphantom{$\dfrac{\sqrt{2}}{2}$}
\item $\cos(t) = -\dfrac{\sqrt{2}}{2}$ \label{solvefortlast}

\setcounter{HW}{\value{enumi}}

\end{enumerate}

\end{multicols}

In Exercises \ref{pointsfirst} - \ref{pointslast}, let $\theta$ be the angle in standard position whose terminal side contains the given point then compute $\cos(\theta)$ and $\sin(\theta)$.

\begin{multicols}{4}

\begin{enumerate}

\setcounter{enumi}{\value{HW}}

\item $P(-7, 24)$ \label{pointsfirst} 
\item $Q(3, 4)$
\item $R(5, -9)$
\item $T(-2, -11)$ \label{pointslast}

\setcounter{HW}{\value{enumi}}

\end{enumerate}

\end{multicols}

\newpage

In Exercises \ref{findthevaluefirst} - \ref{findthevaluelast}, use the results developed throughout the section to find the requested value.

\begin{enumerate}

\setcounter{enumi}{\value{HW}}

\item If $\sin(\theta) = -\dfrac{7}{25}$ with $\theta$ in Quadrant IV, what is $\cos(\theta)$? \label{findthevaluefirst}
\item If $\cos(\theta) = \dfrac{4}{9}$ with $\theta$ in Quadrant I, what is $\sin(\theta)$?
\item If $\sin(\theta) = \dfrac{5}{13}$ with $\theta$ in Quadrant II, what is $\cos(\theta)$?
\item If $\cos(\theta) = -\dfrac{2}{11}$ with $\theta$ in Quadrant III, what is $\sin(\theta)$?
\item If $\sin(\theta) = -\dfrac{2}{3}$ with $\theta$ in Quadrant III, what is $\cos(\theta)$?
\item If $\cos(\theta) = \dfrac{28}{53}$ with $\theta$ in Quadrant IV, what is $\sin(\theta)$?
\item  If $\sin(\theta) = \dfrac{2\sqrt{5}}{5}$ and $\dfrac{\pi}{2} < \theta < \pi$, what is $\cos(\theta)$?
\item  If $\cos(\theta) = \dfrac{\sqrt{10}}{10}$ and $2\pi < \theta < \dfrac{5\pi}{2}$, what is $\sin(\theta)$?
\item  If $\sin(\theta) = -0.42$ and $\pi < \theta < \dfrac{3\pi}{2}$, what is  $\cos(\theta)$?
\item  If $\cos(\theta) = -0.98$ and $\dfrac{\pi}{2} < \theta < \pi$, what is $\sin(\theta)$? \label{findthevaluelast}

\setcounter{HW}{\value{enumi}}

\end{enumerate}

In Exercises \ref{calculatorfirst} - \ref{calculatorlast}, use your calculator to approximate the given value to three decimal places.  Make sure your calculator is in the proper angle measurement mode!

\begin{multicols}{3}

\begin{enumerate}

\setcounter{enumi}{\value{HW}}

\item $\sin(78.95^{\circ})$ \label{calculatorfirst}
\item $\cos(-2.01)$
\item $\sin(392.994)$

\setcounter{HW}{\value{enumi}}

\end{enumerate}

\end{multicols}

\begin{multicols}{3}

\begin{enumerate}

\setcounter{enumi}{\value{HW}}

\item $\cos(207^{\circ})$
\item $\sin\left( \pi^{\circ} \right)$
\item $\cos(e)$ \label{calculatorlast} 

\setcounter{HW}{\value{enumi}}

\end{enumerate}

\end{multicols}

In Exercises \ref{decomposebasicsinecosinefirst} - \ref{decomposebasicsinecosinelast}, write the given function as a nontrivial decomposition of functions as directed.

\begin{enumerate}
\setcounter{enumi}{\value{HW}}

\item  For $f(t) = 3t + \sin(2t)$, find functions $g$ and $h$ so that $f=g+h$. \label{decomposebasicsinecosinefirst}
\item  For $f(\theta) = 3\cos(\theta) - \sin(4\theta)$, find functions $g$ and $h$ so that $f=g-h$. 
\item  For $f(t) = e^{-0.1t} \sin(3t)$, find functions $g$ and $h$ so that $f=gh$.
\item  For $r(t) = \dfrac{\sin(t)}{t}$, find functions $f$ and $g$ so $r = \dfrac{f}{g}$.
\item  For $r(\theta) =\sqrt{3 \cos(\theta)}$, find functions $f$ and $g$ so $r = g \circ f$. \label{decomposebasicsinecosinelast}

\setcounter{HW}{\value{enumi}}
\end{enumerate}

\begin{enumerate}
\setcounter{enumi}{\value{HW}}

\item \label{sinearcexercise}For each function $S(t)$ listed below, compute the average rate of change over the indicated interval.\footnote{See Definition \ref{arc} in Section \ref{AverageRateofChange} for a review of this concept, as needed.}  What trends do you notice? Be sure your calculator is in radian mode!

\[ \begin{array}{|r||c|c|c|}  \hline

 S(t) &  [-0.1, 0.1] & [-0.01, 0.01] &[-0.001, 0.001] \\ \hline
 \sin(t)     &&&  \\  \hline
 \sin(2t)   &&&  \\ \hline
 \sin(3t)   &&&   \\  \hline
\sin(4t)   &&&   \\  \hline

\end{array} \]

\setcounter{HW}{\value{enumi}}
\end{enumerate}

In Exercises \ref{motionfirst} - \ref{motionlast}, find the equations of motion for the given scenario.  Assume that the center of the motion is the origin, the motion is counter-clockwise and that $t = 0$ corresponds to a position along the positive $x$-axis.  (See Equation \ref{equationsforcircularmotion} and Example \ref{EarthRotationEx}.)

\begin{enumerate}

\setcounter{enumi}{\value{HW}}

\item  \label{motionfirst} A point on the edge of the spinning yo-yo in Exercise \ref{spinningyoyo} from Section \ref{RadianMeasure}. 

Recall: The diameter of the yo-yo is 2.25 inches and it spins at 4500 revolutions per minute.

\item  The yo-yo in exercise \ref{yoyotrick} from Section \ref{RadianMeasure}.

Recall: The radius of the circle is 28 inches and it completes one revolution in 3 seconds.

\item A point on the edge of the hard drive in Exercise \ref{harddrive} from Section \ref{RadianMeasure}.

Recall:  The diameter of the hard disk is 2.5 inches and it spins at 7200 revolutions per minute.

\item  \label{motionlast} A passenger on the Big Wheel in Exercise \ref{giantwheelmotion} from Section \ref{RadianMeasure}.

Recall: The diameter is 128 feet and completes 2 revolutions in 2 minutes, 7 seconds.

\setcounter{HW}{\value{enumi}}

\end{enumerate}

\begin{enumerate}

\setcounter{enumi}{\value{HW}}

\item Consider the numbers:  $0$, $1$, $2$, $3$, $4$.  Take the square root of each of these numbers, then divide each by $2$. The resulting numbers should look hauntingly familiar. (See the values in the table on \pageref{CosineSineFacts}.) 


\item  On page \pageref{cosinesineequationsrealnumbers}, we see that the sine and cosine functions of \textit{angles} can be considered functions of \textit{real numbers}.  With help from your classmates, discuss the domains and ranges of  $f(t) = \sin(t)$ and $g(t) = \cos(t)$.  Write your answers using interval notation.


\item  \label{circleofradiusrsinecosinetrans}  Another way to establish  Theorem \ref{cosinesinecircle} is to use transformations. Re-read the discussion following  Theorem \ref{standardcirclealternate} in Chapter \ref{TheConicSections}  and transform  the Unit Circle, $x^2+y^2 = 1$, to $x^2 + y^2 = r^2$ using horizontal and vertical stretches.  Show if the coordinates on the Unit Circle are $(\cos(\theta), \sin(\theta))$, then the corresponding coordinates on $x^2+y^2 = r^2$ are $(r \cos(\theta), r \sin(\theta))$.

\item  In the scenario of Equation \ref{equationsforcircularmotion}, we assumed that at $t=0$, the object was at the point $(r,0)$.  If this is not the case,  we can adjust the equations of motion by introducing a `time delay.'   If $t_{0} > 0$ is the first time the object passes through the point $(r,0)$, show, with the help of your classmates, the equations of motion are $x = r \cos(\omega (t - t_{0}))$ and $y = r \sin(\omega (t-t_{0}))$.


\end{enumerate}

\newpage

\subsection{Answers}

\begin{multicols}{2}

\begin{enumerate}

\item $\cos(0) = 1$, $\; \sin(0) = 0$ \vphantom{$\dfrac{\sqrt{2}}{2}$}

\item $\cos \left(\dfrac{\pi}{4} \right) = \dfrac{\sqrt{2}}{2}$, $\; \sin \left(\dfrac{\pi}{4} \right) = \dfrac{\sqrt{2}}{2}$

\setcounter{HW}{\value{enumi}}

\end{enumerate}

\end{multicols}

\begin{multicols}{2}

\begin{enumerate}

\setcounter{enumi}{\value{HW}}

\item $\cos \left(\dfrac{\pi}{3}\right) = \dfrac{1}{2}$, $\; \sin \left(\dfrac{\pi}{3}\right) = \dfrac{\sqrt{3}}{2}$

\item $\cos \left(\dfrac{\pi}{2}\right) = 0$, $\; \sin \left(\dfrac{\pi}{2}\right) = 1$ \vphantom{$\dfrac{\sqrt{2}}{2}$}

\setcounter{HW}{\value{enumi}}

\end{enumerate}

\end{multicols}

\begin{multicols}{2}

\begin{enumerate}

\setcounter{enumi}{\value{HW}}

\item $\cos\left(\dfrac{2\pi}{3}\right) = -\dfrac{1}{2}$, $\; \sin \left(\dfrac{2\pi}{3}\right) = \dfrac{\sqrt{3}}{2}$

\item $\cos \left(\dfrac{3\pi}{4} \right) = -\dfrac{\sqrt{2}}{2}$, $\; \sin \left(\dfrac{3\pi}{4} \right) = \dfrac{\sqrt{2}}{2}$

\setcounter{HW}{\value{enumi}}

\end{enumerate}

\end{multicols}

\begin{multicols}{2}

\begin{enumerate}

\setcounter{enumi}{\value{HW}}

\item $\cos(\pi) = -1$, $\; \sin(\pi) = 0$ \vphantom{$\dfrac{\sqrt{3}}{2}$}

\item $\cos\left(\dfrac{7\pi}{6}\right) = -\dfrac{\sqrt{3}}{2}$, $\; \sin\left(\dfrac{7\pi}{6}\right) = -\dfrac{1}{2}$

\setcounter{HW}{\value{enumi}}

\end{enumerate}

\end{multicols}

\begin{multicols}{2}

\begin{enumerate}

\setcounter{enumi}{\value{HW}}

\item $\cos \left(\dfrac{5\pi}{4} \right) = -\dfrac{\sqrt{2}}{2}$, $\; \sin \left(\dfrac{5\pi}{4} \right) = -\dfrac{\sqrt{2}}{2}$

\item $\cos\left(\dfrac{4\pi}{3}\right) = -\dfrac{1}{2}$, $\; \sin \left(\dfrac{4\pi}{3}\right) = -\dfrac{\sqrt{3}}{2}$

\setcounter{HW}{\value{enumi}}

\end{enumerate}

\end{multicols}

\begin{multicols}{2}

\begin{enumerate}

\setcounter{enumi}{\value{HW}}

\item $\cos \left(\dfrac{3\pi}{2}\right) = 0$, $\; \sin \left(\dfrac{3\pi}{2}\right) = -1$

\item $\cos\left(\dfrac{5\pi}{3}\right) = \dfrac{1}{2}$, $\; \sin \left(\dfrac{5\pi}{3}\right) = -\dfrac{\sqrt{3}}{2}$

\setcounter{HW}{\value{enumi}}

\end{enumerate}

\end{multicols}

\begin{multicols}{2}

\begin{enumerate}

\setcounter{enumi}{\value{HW}}

\item $\cos \left(\dfrac{7\pi}{4} \right) = \dfrac{\sqrt{2}}{2}$, $\; \sin \left(\dfrac{7\pi}{4} \right) = -\dfrac{\sqrt{2}}{2}$

\item $\cos\left(\dfrac{23\pi}{6}\right) = \dfrac{\sqrt{3}}{2}$, $\; \sin\left(\dfrac{23\pi}{6}\right) = -\dfrac{1}{2}$

\setcounter{HW}{\value{enumi}}

\end{enumerate}

\end{multicols}

\begin{multicols}{2}

\begin{enumerate}

\setcounter{enumi}{\value{HW}}

\item $\cos \left(-\dfrac{13\pi}{2}\right) = 0$, $\; \sin \left(-\dfrac{13\pi}{2}\right) = -1$ \vphantom{$\dfrac{\sqrt{3}}{2}$}

\item $\cos\left(-\dfrac{43\pi}{6}\right) = -\dfrac{\sqrt{3}}{2}$, $\; \sin\left(-\dfrac{43\pi}{6}\right) = \dfrac{1}{2}$

\setcounter{HW}{\value{enumi}}

\end{enumerate}

\end{multicols}

\begin{multicols}{2}

\begin{enumerate}

\setcounter{enumi}{\value{HW}}

\item $\cos \left(-\dfrac{3\pi}{4} \right) = -\dfrac{\sqrt{2}}{2}$, $\; \sin \left(-\dfrac{3\pi}{4} \right) = -\dfrac{\sqrt{2}}{2}$

\item $\cos\left(-\dfrac{\pi}{6}\right) = \dfrac{\sqrt{3}}{2}$, $\; \sin\left(-\dfrac{\pi}{6}\right) = -\dfrac{1}{2}$

\setcounter{HW}{\value{enumi}}

\end{enumerate}

\end{multicols}

\begin{multicols}{2}

\begin{enumerate}

\setcounter{enumi}{\value{HW}}

\item $\cos\left(\dfrac{10\pi}{3}\right) = -\dfrac{1}{2}$, $\; \sin \left(\dfrac{10\pi}{3}\right) = -\dfrac{\sqrt{3}}{2}$

\item $\cos(117\pi) = -1$, $\; \sin(117\pi) = 0$ \vphantom{$\dfrac{\sqrt{3}}{2}$}

\setcounter{HW}{\value{enumi}}

\end{enumerate}

\end{multicols}

\begin{enumerate}

\setcounter{enumi}{\value{HW}}

\item $\sin(\theta) = \dfrac{1}{2}$ when $\theta = \dfrac{\pi}{6} + 2\pi k$ or $\theta = \dfrac{5\pi}{6} + 2\pi k$ for any integer $k$.
\item $\cos(\theta) = -\dfrac{\sqrt{3}}{2}$ when $\theta = \dfrac{5\pi}{6} + 2\pi k$ or $\theta = \dfrac{7\pi}{6} + 2\pi k$ for any integer $k$.
\item $\sin(\theta) = 0$ when $\theta = \pi k$ for any integer $k$.
\item $\cos(\theta) = \dfrac{\sqrt{2}}{2}$ when $\theta = \dfrac{\pi}{4} + 2\pi k$ or $\theta = \dfrac{7\pi}{4} + 2\pi k$ for any integer $k$.
\item $\sin(\theta) = \dfrac{\sqrt{3}}{2}$ when $\theta = \dfrac{\pi}{3} + 2\pi k$ or $\theta = \dfrac{2\pi}{3} + 2\pi k$ for any integer $k$.
\item $\cos(\theta) = -1$ when $\theta = (2k + 1)\pi$ for any integer $k$.
\item  $\sin(\theta) = -1$ when $\theta = \dfrac{3\pi}{2} + 2\pi k$ for any integer $k$.
\item  $\cos(\theta) = \dfrac{\sqrt{3}}{2}$ when $\theta = \dfrac{\pi}{6} + 2\pi k$ or  $\theta = \dfrac{11\pi}{6} + 2\pi k$ for any integer $k$.
%\item  $\sin(\theta) = \dfrac{\sqrt{2}}{2}$ when $\theta = \dfrac{\pi}{4} + 2\pi k$ or  $\theta = \dfrac{3\pi}{4} + 2\pi k$ for any integer $k$.
\item  $\cos(\theta) = -1.001$ never happens

\setcounter{HW}{\value{enumi}}

\end{enumerate}

\begin{enumerate}

\setcounter{enumi}{\value{HW}}

\item $\cos(t) = 0$ when $t = \dfrac{\pi}{2} + \pi k$ for any integer $k$.
\item $\sin(t) = -\dfrac{\sqrt{2}}{2}$ when $t = \dfrac{5\pi}{4} + 2\pi k$ or $t = \dfrac{7\pi}{4} + 2\pi k$ for any integer $k$.
\item $\cos(t) = 3$ never happens.  
\item $\sin(t) = -\dfrac{1}{2}$ when $t = \dfrac{7\pi}{6} + 2\pi k$ or $t = \dfrac{11\pi}{6} + 2\pi k$ for any integer $k$.
\item $\cos(t) = \dfrac{1}{2}$ when $t = \dfrac{\pi}{3} + 2\pi k$ or $t = \dfrac{5\pi}{3} + 2\pi k$ for any integer $k$.
\item $\sin(t) = -2$ never happens
\item $\cos(t) = 1$ when $t = 2\pi k$ for any integer $k$.
\item $\sin(t) = 1$ when $t = \dfrac{\pi}{2} + 2\pi k$ for any integer $k$.
\item $\cos(t) = -\dfrac{\sqrt{2}}{2}$ when $t = \dfrac{3\pi}{4} + 2\pi k$ or $t = \dfrac{5\pi}{4} + 2\pi k$ for any integer $k$.
%\item  $\sin(t) = -\dfrac{\sqrt{3}}{2}$ when $t = \dfrac{4\pi}{3} + 2\pi k$ or  $t = \dfrac{5\pi}{3} + 2\pi k$ for any integer $k$.

\setcounter{HW}{\value{enumi}}

\end{enumerate}

\begin{enumerate}

\setcounter{enumi}{\value{HW}}

\item $\cos(\theta) = -\dfrac{7}{25}, \; \sin(\theta) = \dfrac{24}{25}$

\item $\cos(\theta) = \dfrac{3}{5}, \; \sin(\theta) = \dfrac{4}{5}$

\item $\cos(\theta) = \dfrac{5\sqrt{106}}{106}, \; \sin(\theta) = -\dfrac{9\sqrt{106}}{106}$

\item $\cos(\theta) = -\dfrac{2\sqrt{5}}{25}, \; \sin(\theta) = -\dfrac{11\sqrt{5}}{25}$

\setcounter{HW}{\value{enumi}}

\end{enumerate}


\begin{enumerate}

\setcounter{enumi}{\value{HW}}

\item If $\sin(\theta) = -\dfrac{7}{25}$ with $\theta$ in Quadrant IV, then $\cos(\theta) = \dfrac{24}{25}$.
\item If $\cos(\theta) = \dfrac{4}{9}$ with $\theta$ in Quadrant I, then $\sin(\theta) = \dfrac{\sqrt{65}}{9}$.
\item If $\sin(\theta) = \dfrac{5}{13}$ with $\theta$ in Quadrant II, then $\cos(\theta) = -\dfrac{12}{13}$.
\item If $\cos(\theta) = -\dfrac{2}{11}$ with $\theta$ in Quadrant III, then $\sin(\theta) = -\dfrac{\sqrt{117}}{11}$.
\item If $\sin(\theta) = -\dfrac{2}{3}$ with $\theta$ in Quadrant III, then $\cos(\theta) = -\dfrac{\sqrt{5}}{3}$.
\item If $\cos(\theta) = \dfrac{28}{53}$ with $\theta$ in Quadrant IV, then $\sin(\theta) = -\dfrac{45}{53}$.
\item  If $\sin(\theta) = \dfrac{2\sqrt{5}}{5}$ and $\dfrac{\pi}{2} < \theta < \pi$, then $\cos(\theta) = -\dfrac{\sqrt{5}}{5}$.
\item  If $\cos(\theta) = \dfrac{\sqrt{10}}{10}$ and $2\pi < \theta < \dfrac{5\pi}{2}$, then $\sin(\theta)  = \dfrac{3 \sqrt{10}}{10}$.
\item  If $\sin(\theta) = -0.42$ and $\pi < \theta < \dfrac{3\pi}{2}$, then $\cos(\theta) = -\sqrt{0.8236} \approx -0.9075$.
\item  If $\cos(\theta) = -0.98$ and $\dfrac{\pi}{2} < \theta < \pi$, then $\sin(\theta) = \sqrt{0.0396} \approx 0.1990$.

\setcounter{HW}{\value{enumi}}

\end{enumerate}


\begin{multicols}{3}

\begin{enumerate}

\setcounter{enumi}{\value{HW}}

\item $\sin(78.95^{\circ}) \approx 0.981$
\item $\cos(-2.01) \approx -0.425$
\item $\sin(392.994) \approx -0.291$

\setcounter{HW}{\value{enumi}}

\end{enumerate}

\end{multicols}

\begin{multicols}{3}

\begin{enumerate}

\setcounter{enumi}{\value{HW}}

\item $\cos(207^{\circ}) \approx -0.891$
\item $\sin\left( \pi^{\circ} \right) \approx 0.055$
\item $\cos(e) \approx -0.912$

\setcounter{HW}{\value{enumi}}

\end{enumerate}
\end{multicols}

\begin{enumerate}
\setcounter{enumi}{\value{HW}}

\item One solution is $g(t) = 3t$ and $h(t) = \sin(2t)$.
\item One solution is $g(\theta) = 3 \cos(\theta)$ and $h(\theta) = \sin(4 \theta)$.
\item One solution is $g(t) =  e^{-0.1t}$ and $h(t) = \sin(3t)$. 
\item One solution is $f(t) = \sin(t)$ and $g(t) = t$.
\item One solution is $f(\theta) = 3 \cos(\theta)$ and $g(\theta) = \sqrt{\theta}$.

\setcounter{HW}{\value{enumi}}
\end{enumerate}

\begin{enumerate}
\setcounter{enumi}{\value{HW}}

\item  As we zoom in towards $0$, the average rate of change of $\sin(k t)$ approaches $k$.

\[ \begin{array}{|r||c|c|c|}  \hline

 S(t) &  [-0.1, 0.1] & [-0.01, 0.01] &[-0.001, 0.001] \\ \hline
 \sin(t)     & \approx 0.9983  &  \approx 1  &  \approx 1 \\  \hline
 \sin(2t)   & \approx 1.9867 & \approx  1.9999 & \approx 2 \\ \hline
 \sin(3t)   & \approx 2.9552 & \approx 2.9995 &  \approx 3  \\  \hline
 \sin(4t)  & \approx 3.8942 & \approx 3.9989 &  \approx 4 \\  \hline

\end{array} \]

\setcounter{HW}{\value{enumi}}
\end{enumerate}


\begin{enumerate}

\setcounter{enumi}{\value{HW}}

\item   $r = 1.125$ inches, $\omega = 9000 \pi \, \frac{\text{radians}}{\text{minute}}$,  $x = 1.125 \cos(9000 \pi \, t)$, $y = 1.125 \sin(9000 \pi \, t)$.  Here $x$ and $y$ are measured in inches and $t$ is measured in minutes.

\item   $r = 28$ inches, $\omega = \frac{2\pi}{3} \, \frac{\text{radians}}{\text{second}}$,  $x = 28 \cos\left(\frac{2\pi}{3} \, t \right)$, $y = 28 \sin\left(\frac{2\pi}{3} \, t \right)$.  Here $x$ and $y$ are measured in inches and $t$ is measured in seconds.

\item $r = 1.25$ inches, $\omega = 14400 \pi \, \frac{\text{radians}}{\text{minute}}$,  $x = 1.25 \cos(14400 \pi \, t)$, $y = 1.25 \sin(14400 \pi \, t)$.  Here $x$ and $y$ are measured in inches and $t$ is measured in minutes.

\item  $r = 64$ feet, $\omega = \frac{4\pi}{127} \, \frac{\text{radians}}{\text{second}}$,  $x = 64 \cos\left(\frac{4\pi}{127} \, t \right)$, $y = 64 \sin\left(\frac{4\pi}{127} \, t \right)$.  Here $x$ and $y$ are measured in feet and $t$ is measured in seconds

\end{enumerate}


\end{document}
