\documentclass{ximera}

\begin{document}
	\author{Stitz-Zeager}
	\xmtitle{Exercises for App Right Trig}{}

\mfpicnumber{1} \opengraphsfile{ExercisesforAppRightTrig} % mfpic settings added 


\label{ExercisesforAppRightTrig}
In Exercises \ref{trianglecircfirst} - \ref{trianglecirclast},  find the requested quantities.

\begin{multicols}{2} \raggedcolumns

\begin{enumerate}



\item Find $\theta$, $a$, and $c$.  \label{trianglecircfirst}

 \begin{mfpic}[15]{-5}{5}{-5}{5}

\arrow \reverse \arrow \shiftpath{(-4.330,0)} \parafcn{5, 25, 5}{3*dir(t)}
\arrow \reverse \arrow \shiftpath{(4.330,5)}  \parafcn{215, 265, 5}{1.5*dir(t)}
\tlabel(-1.25, 0.6){$\theta$}
\tlabel(0,-0.75){$9$}
\tlabel(4.75,2.25){$a$}
\tlabel(-0.5,3){$c$}
\tlabel(2.75,2.85){$60^{\circ}$}
\polyline{(3.93, 0), (3.93, 0.4), (4.33, 0.4)}
\penwd{1.25pt}
\polyline{(-4.330,0), (4.330,0), (4.330,5), (-4.330,0)}
\end{mfpic}

\vspace{.5in}
 
\item  Find $\alpha$, $b$, and $c$.

\begin{mfpic}[15]{-1}{5}{-1}{7}
\arrow \reverse \arrow \parafcn{60, 87, 5}{1.75*dir(t)}
\arrow \reverse \arrow \shiftpath{(4.357,6.709)}  \parafcn{185, 232, 5}{1.5*dir(t)}
\tlabel(0.25, 2){$34^{\circ}$}
\tlabel(2.5,3){$c$}
\tlabel(2,7){$b$}
\tlabel(-0.85,4){$12$}
\tlabel(2.25,5.75){$\alpha$}
\polyline{(0,6.304), (0.4, 6.304),  (0.4, 6.704)}
\penwd{1.25pt}
\polyline{(0,0), (0,6.709), (4.357, 6.709), (0,0)}
\end{mfpic}

\setcounter{HW}{\value{enumi}}

\end{enumerate}

\end{multicols}

\enlargethispage{.3in}

\begin{multicols}{2}

\begin{enumerate}

\setcounter{enumi}{\value{HW}}

\item  Find $\theta$, $a$, and $c$.

\begin{mfpic}[18]{-5}{5}{-5}{5}
\arrow \reverse \arrow \shiftpath{(2.5,0)} \parafcn{140, 175, 5}{1.5*dir(t)}
\arrow \reverse \arrow \shiftpath{(-2.5,5)}  \parafcn{275, 310, 5}{1.5*dir(t)}
\tlabel(-2, 2.75){$47^{\circ}$}
\tlabel(-0.5,-0.75){$6$}
\tlabel(-3.25,2.25){$a$}
\tlabel(0,3){$c$}
\tlabel(0.5,0.5){$\theta$}
\polyline{(-2.5, 0.4), (-2.1, 0.4), (-2.1, 0)}
\penwd{1.25pt}
\polyline{(-2.5, 0), (2.5,0), (-2.5,5), (-2.5,0)}
\end{mfpic}

\item Find $\beta$, $b$, and $c$.  \label{trianglecirclast}

\begin{mfpic}[18]{-6}{1}{-1}{9}
\arrow \reverse \arrow \parafcn{95, 127, 5}{1.75*dir(t)}
\arrow \reverse \arrow \shiftpath{(-5.402,6)}  \parafcn{317, 355, 5}{1.5*dir(t)}
\tlabel(-3.75, 5){$\beta$}
\tlabel(0.5,3){$2.5$}
\tlabel(-2.6,6.25){$b$}
\tlabel(-3.25,2.5){$c$}
\tlabel(-1.2,2){$50^{\circ}$}
\polyline{(0,5.6), (-0.4, 5.6),  (-0.4, 6)}
\penwd{1.25pt}
\polyline{(0,0), (0,6), (-5.402, 6), (0,0)}
\end{mfpic} 

\setcounter{HW}{\value{enumi}}

\end{enumerate}

\end{multicols}

In Exercises \ref{moretrianglecircfirst} - \ref{moretrianglecirclast}, answer the following questions assuming  $\theta$ is an angle in a right triangle.

\begin{enumerate}

\setcounter{enumi}{\value{HW}}

\item  If $\theta = 30^{\circ}$ and the side opposite $\theta$ has length $4$, how long is the side adjacent to $\theta$? \label{moretrianglecircfirst}

\item  If $\theta = 15^{\circ}$ and the hypotenuse has length $10$, how long is the side opposite $\theta$?

\item  If $\theta = 87^{\circ}$ and the side adjacent to $\theta$ has length $2$, how long is the side opposite $\theta$?

\item  If $\theta = 38.2^{\circ}$ and the side opposite $\theta$ has lengh $14$, how long is the hypoteneuse?

\item  If $\theta = 2.05^{\circ}$ and the hypotenuse has length $3.98$, how long is the side adjacent to $\theta$?

\item  If $\theta = 42^{\circ}$ and the side adjacent to $\theta$ has length $31$, how long is the side opposite $\theta$? \label{moretrianglecirclast}

\setcounter{HW}{\value{enumi}}

\end{enumerate}

In Exercises \ref{trianglesidesfirst} - \ref{trianglesideslast}, find the two acute angles in the right triangle whose sides have the given lengths.  Express your answers using degree measure rounded to two decimal places.

\begin{multicols}{3}

\begin{enumerate}

\setcounter{enumi}{\value{HW}}

\item 3, 4 and 5 \label{trianglesidesfirst}

\item 5, 12 and 13

\item 336, 527 and 625 \label{trianglesideslast}

\setcounter{HW}{\value{enumi}}

\end{enumerate}

\end{multicols}


\newpage

In Exercises \ref{findothercircfirstapprighttrig} - \ref{findothercirclastapprighttrig}, $\theta$ is an acute angle.  Use the given trigonometric ratio to find the exact values of the remaining trigonometric ratios of $\theta$.  Find a decimal approximation to $\theta$, rounded to two decimal places.

\begin{multicols}{3}

\begin{enumerate}
\setcounter{enumi}{\value{HW}}

\item $\sin(\theta) = \dfrac{3}{5}$  \label{findothercircfirstapprighttrig}
\item $\tan(\theta) = \dfrac{12}{5}$
\item $\csc(\theta) = \dfrac{25}{24}$

\setcounter{HW}{\value{enumi}}

\end{enumerate}

\end{multicols}

\begin{multicols}{3}

\begin{enumerate}

\setcounter{enumi}{\value{HW}}


\item $\sec(\theta) = 7$  \vphantom{$\dfrac{10}{\sqrt{91}}$}
\item $\csc(\theta) = \dfrac{10\sqrt{91}}{91}$ 
\item $\cot(\theta) = 23$ 

\setcounter{HW}{\value{enumi}}

\end{enumerate}

\end{multicols}

\begin{multicols}{3}

\begin{enumerate}

\setcounter{enumi}{\value{HW}}




\item  $\tan(\theta) = 2$  \vphantom{$\sqrt{5}$}
\item  $\sec(\theta) = 4$  \vphantom{$\sqrt{5}$}
\item $\cot(\theta) = \sqrt{5}$ 

\setcounter{HW}{\value{enumi}}

\end{enumerate}

\end{multicols}


\begin{multicols}{3}

\begin{enumerate}

\setcounter{enumi}{\value{HW}}


\item  $\cos(\theta) = \dfrac{1}{3}$ 

\item  $\cot(\theta) = 2$ \vphantom{$ \dfrac{1}{3}$}

\item  $\csc(\theta) = 5$ \vphantom{$ \dfrac{1}{3}$}

\setcounter{HW}{\value{enumi}}

\end{enumerate}

\end{multicols}

\begin{multicols}{3}

\begin{enumerate}

\setcounter{enumi}{\value{HW}}

\item  $\tan(\theta) = \sqrt{10}$ 
\item  $\sec(\theta) = 2\sqrt{5}$ 
\item  $\cos(\theta) = 0.4$  \vphantom{$\sqrt{10}$}  \label{findothercirclastapprighttrig}

\setcounter{HW}{\value{enumi}}

\end{enumerate}

\end{multicols}


\begin{enumerate}

\setcounter{enumi}{\value{HW}}

\item A tree standing vertically on level ground casts a 120 foot long shadow.  The angle of elevation from the end of the shadow to the top of the tree is $21.4^{\circ}$.  Find the height of the tree to the nearest foot.  With the help of your classmates, research the term \emph{umbra versa} and see what it has to do with the shadow in this problem.

\item The broadcast tower for radio station WSAZ (Home of ``Algebra in the Morning with Carl and Jeff'') has two enormous flashing red lights on it: one at the very top and one a few feet below the top.  From a point 5000 feet away from the base of the tower on level ground the angle of elevation to the top light is $7.970^{\circ}$ and to the second light is $7.125^{\circ}$.  Find the distance between the lights to the nearest foot.

\item On page \pageref{angleofelevation} we defined the angle of inclination (also known as the angle of elevation) and in this exercise we introduce a related angle - \index{angle ! of depression} the angle of depression (also known as \index{angle ! of declination} the angle of declination).  The angle of depression of an object refers to the angle whose initial side is a horizontal line above the object and whose terminal side is the line-of-sight to the object below the horizontal.  This is represented schematically below.
\label{angleofdepression}

\begin{center}

\begin{mfpic}[18]{-5}{5}{-5}{5}
\polyline{(-5,5), (4.330,5)}
\point[3pt]{(4.330,5)}
\dashed \polyline{(-4.330,0), (4.330,5)}
\reverse \arrow \shiftpath{(4.330,5)} \parafcn{185, 205, 5}{3*dir(t)}
\tlabel(0.75, 4){$\theta$}
\tlabel[cc](-1,5.5){horizontal}
\tlabel[cc](5.25,4.5){observer}
\plotsymbol[3pt]{Asterisk}{(-4.330,0)}
\tlabel(-5.0,-0.75){object}
\end{mfpic} 

\smallskip

The angle of depression from the horizontal to the object is $\theta$

\end{center}

\begin{enumerate}

\item Show that if the horizontal is above and parallel to level ground then the angle of depression (from observer to object) and the angle of inclination (from object to observer) will be congruent because they are alternate interior angles.

\item \label{sasquatchfire} From a firetower 200 feet above level ground in the Sasquatch National Forest, a ranger spots a fire off in the distance.  The angle of depression to the fire is $2.5^{\circ}$.  How far away from the base of the tower is the fire?

\item  The ranger in part \ref{sasquatchfire} sees a Sasquatch running directly from the fire towards the firetower.  The ranger takes two sightings.  At the first sighting, the angle of depression from the tower to the Sasquatch is $6^{\circ}$.  The second sighting, taken just 10 seconds later, gives the the angle of depression as $6.5^{\circ}$.  How far did the Saquatch travel in those 10 seconds?  Round your answer to the nearest foot.  How fast is it running in miles per hour? Round your answer to the nearest mile per hour.  If the Sasquatch keeps up this pace, how long will it take for the Sasquatch to reach the firetower from his location at the second sighting?  Round your answer to the nearest minute.

\end{enumerate}

\item  When I stand 30 feet away from a tree at home, the angle of elevation to the top of the tree is $50^{\circ}$ and the angle of depression to the base of the tree is $10^{\circ}$.  What is the height of the tree?  Round your answer to the nearest foot.

\item From the observation deck of the lighthouse at Sasquatch Point 50 feet above the surface of Lake Ippizuti, a lifeguard spots a boat out on the lake sailing directly toward the lighthouse.  The first sighting had an angle of depression of $8.2^{\circ}$ and the second sighting had an angle of depression of $25.9^{\circ}$.  How far had the boat traveled between the sightings?

\item A guy wire 1000 feet long is attached to the top of a tower.  When pulled taut it makes a $43^{\circ}$ angle with the ground.  How tall is the tower?  How far away from the base of the tower does the wire hit the ground?

\setcounter{HW}{\value{enumi}}

\end{enumerate}

\begin{enumerate}

\setcounter{enumi}{\value{HW}}

\item A guy wire 1000 feet long is attached to the top of a tower.  When pulled taut it touches level ground 360 feet from the base of the tower.  What angle does the wire make with the ground?  Express your answer using degree measure rounded to one decimal place.

\item At Cliffs of Insanity Point, The Great Sasquatch Canyon is 7117 feet deep.  From that point, a fire is seen at a location known to be 10 miles away from the base of the sheer canyon wall.  What angle of depression is made by the line of sight from the canyon edge to the fire?  Express your answer using degree measure rounded to one decimal place.

\item Shelving is being built at the Utility Muffin Research Library which is to be 14 inches deep.  An 18-inch rod will be attached to the wall and the underside of the shelf at its edge away from the wall, forming a right triangle under the shelf to support it.  What angle, to the nearest degree, will the rod make with the wall?

\item A parasailor is being pulled by a boat on Lake Ippizuti.  The cable is 300 feet long and the parasailor is 100 feet above the surface of the water.  What is the angle of elevation from the boat to the parasailor?  Express your answer using degree measure rounded to one decimal place.

\item  A tag-and-release program to study the Sasquatch population of the eponymous Sasquatch National Park is begun.  From a 200 foot tall tower, a ranger spots a Sasquatch lumbering through the wilderness directly towards the tower.  Let $\theta$ denote the angle of depression from the top of the tower to a point on the ground.  If the range of the rifle with a tranquilizer dart is 300 feet, find the smallest value of $\theta$ for which the corresponding point on the ground is in range of the rifle.  Round your answer to the nearest hundreth of a degree.

\item  The rule of thumb for safe ladder use states that the length of the ladder should be at least four times as long as the distance from the base of the ladder to the wall. Assuming the ladder is resting against a wall which is `plumb' (that is, makes a $90^{\circ}$ angle with the ground), determine the acute angle the ladder makes with the ground, rounded to the nearest tenth of a degree.

\setcounter{HW}{\value{enumi}}

\end{enumerate}

As you may have already noticed in working through the exercises, since the six trigonometric ratios are all defined in terms of the three sides of a right triangle, there are several relationships between them.  In Exercises \ref{rtidentityfirst} - \ref{rtidentitylast}, use the diagram on page \pageref{righttranglediagram} along with Definitions \ref{righttrianglesinecosinetangent} and \ref{righttriangletherest} to show the following relationships hold for all acute angles.\footnote{These are called trigonometric \textit{identities} and will be studied in greater detail in Section \ref{FundamentalTrigonometricIdentities}.}

\begin{multicols}{3}

\begin{enumerate}

\setcounter{enumi}{\value{HW}}

\item  $\tan(\theta) = \dfrac{\sin(\theta)}{\cos(\theta)}$ \label{rtidentityfirst}

\item  $\csc(\theta) = \dfrac{1}{\sin(\theta)}$

\item  $\sec(\theta) = \dfrac{1}{\cos(\theta)}$

\setcounter{HW}{\value{enumi}}

\end{enumerate}

\end{multicols}

For Exercises \ref{rtcofunctionfirst} - \ref{rrcofunctionlast}, it may be helpful to recall that $90^{\circ} - \theta$ is the measure of the `other' acute angle in the right triangle besides $\theta$.


\begin{multicols}{3}

\begin{enumerate}

\setcounter{enumi}{\value{HW}}

\item  $\cos(\theta) = \sin\left( 90^{\circ} - \theta \right) $ \label{rtcofunctionfirst} \label{cofunctionforeshadowing}

\item  $\csc(\theta) = \sec\left( 90^{\circ} - \theta \right) $

\item  $\cot(\theta) = \tan\left( 90^{\circ} - \theta \right) $ \label{rrcofunctionlast}

\setcounter{HW}{\value{enumi}}

\end{enumerate}

\end{multicols}

For Exercises \ref{rtpythexercisefirst} - \ref{rtpythexerciselast}, it may be helpful to remember that $a^2+b^2 = c^2$:

\begin{multicols}{3}

\begin{enumerate}

\setcounter{enumi}{\value{HW}}

\item  $(\cos(\theta))^2 + (\sin(\theta))^2 = 1$ \label{rtpythexercisefirst}

\item  $1 + (\tan(\theta))^2 = (\sec(\theta))^2$

\item  $ 1 + (\cot(\theta))^2 = (\csc(\theta))^2$ \label{rtpythexerciselast} \label{rtidentitylast}

\setcounter{HW}{\value{enumi}}

\end{enumerate}

\end{multicols}

\newpage

\subsection{Answers}

\begin{enumerate}



\item  $\theta = 30^{\circ}$, $a = 3\sqrt{3}$, $c = \sqrt{108} = 6\sqrt{3}$

\item  $\alpha = 56^{\circ}$, $b = 12 \tan(34^{\circ}) =  8.094$, $c = 12\sec(34^{\circ}) = \dfrac{12}{\cos(34^{\circ})} \approx 14.475$

\item  $\theta = 43^{\circ}$, $a = 6\cot(47^{\circ}) = \dfrac{6}{\tan(47^{\circ})} \approx 5.595$, $c = 6\csc(47^{\circ}) = \dfrac{6}{\sin(47^{\circ})} \approx 8.204$

\item  $\beta = 40^{\circ}$, $b = 2.5 \tan(50^{\circ}) \approx 2.979$, $c = 2.5\sec(50^{\circ}) = \dfrac{2.5}{\cos(50^{\circ})} \approx 3.889$

\setcounter{HW}{\value{enumi}}

\end{enumerate}

\begin{enumerate}

\setcounter{enumi}{\value{HW}}

\item  The side adjacent to $\theta$ has length $4\sqrt{3} \approx 6.928$

\item  The side opposite $\theta$ has length $10 \sin(15^{\circ}) \approx 2.588$

\item  The side opposite $\theta$ is $2\tan(87^{\circ}) \approx 38.162$

\item  The hypoteneuse has length $14 \csc(38.2^{\circ}) = \dfrac{14}{\sin(38.2^{\circ})} \approx 22.639$

\item  The side adjacent to $\theta$ has length $3.98 \cos(2.05^{\circ}) \approx 3.977$

\item  The side opposite $\theta$ has length $31\tan(42^{\circ}) \approx 27.912$

\setcounter{HW}{\value{enumi}}

\end{enumerate}

\begin{multicols}{3}

\begin{enumerate}

\setcounter{enumi}{\value{HW}}

\item $36.87^{\circ}$ and $53.13^{\circ}$
\item $22.62^{\circ}$ and $67.38^{\circ}$
\item $32.52^{\circ}$ and $57.48^{\circ}$

\setcounter{HW}{\value{enumi}}

\end{enumerate}

\end{multicols}


\begin{enumerate}

\setcounter{enumi}{\value{HW}}

\item $\sin(\theta) = \frac{3}{5}, \cos(\theta) = \frac{4}{5}, \tan(\theta) = \frac{3}{4}, \csc(\theta) = \frac{5}{3}, \sec(\theta) = \frac{5}{4}, \cot(\theta) = \frac{4}{3}$, $\theta \approx 36.87^{\circ}$

\item $\sin(\theta) = \frac{12}{13}, \cos(\theta) = \frac{5}{13}, \tan(\theta) = \frac{12}{5}, \csc(\theta) = \frac{13}{12}, \sec(\theta) = \frac{13}{5}, \cot(\theta) = \frac{5}{12}$, $\theta \approx 67.38^{\circ}$

\item $\sin(\theta) = \frac{24}{25}, \cos(\theta) = \frac{7}{25}, \tan(\theta) = \frac{24}{7}, \csc(\theta) = \frac{25}{24}, \sec(\theta) = \frac{25}{7}, \cot(\theta) = \frac{7}{24}$, $\theta \approx 73.74^{\circ}$

\item $\sin(\theta) = \frac{4\sqrt{3}}{7}, \cos(\theta) = \frac{1}{7}, \tan(\theta) = 4\sqrt{3}, \csc(\theta) = \frac{7\sqrt{3}}{12}, \sec(\theta) = 7, \cot(\theta) = \frac{\sqrt{3}}{12}$,  $\theta \approx 81.79^{\circ}$

\item $\sin(\theta) = \frac{\sqrt{91}}{10}, \cos(\theta) = \frac{3}{10}, \tan(\theta) = \frac{\sqrt{91}}{3}, \csc(\theta) = \frac{10\sqrt{91}}{91}, \sec(\theta) = \frac{10}{3}, \cot(\theta) = \frac{3\sqrt{91}}{91}$, $\theta \approx 72.54^{\circ}$

\item $\sin(\theta) = \frac{\sqrt{530}}{530}, \cos(\theta) = \frac{23\sqrt{530}}{530}, \tan(\theta) = \frac{1}{23}, \csc(\theta) = \sqrt{530}, \sec(\theta) = \frac{\sqrt{530}}{23}, \cot(\theta) = 23$, $\theta \approx 2.49^{\circ}$

\item $\sin(\theta) = \frac{2\sqrt{5}}{5}, \cos(\theta) = \frac{\sqrt{5}}{5}, \tan(\theta) = 2, \csc(\theta) = \frac{\sqrt{5}}{2}, \sec(\theta) = \sqrt{5}, \cot(\theta) = \frac{1}{2}$, $\theta \approx 63.43^{\circ}$

\item  $\sin(\theta) = \frac{\sqrt{15}}{4}, \cos(\theta) = \frac{1}{4}, \tan(\theta) = \sqrt{15}, \csc(\theta) = \frac{4\sqrt{15}}{15}, \sec(\theta) = 4, \cot(\theta) = \frac{\sqrt{15}}{15}$, $\theta \approx 75.52^{\circ}$

\item $\sin(\theta) = \frac{\sqrt{6}}{6}, \cos(\theta) = \frac{\sqrt{30}}{6}, \tan(\theta) = \frac{\sqrt{5}}{5}, \csc(\theta) = \sqrt{6}, \sec(\theta) = \frac{\sqrt{30}}{5}, \cot(\theta) = \sqrt{5}$, $\theta \approx 24.09^{\circ}$

\item $\sin(\theta) = \frac{2\sqrt{2}}{3}, \cos(\theta) = \frac{1}{3}, \tan(\theta) = 2\sqrt{2}, \csc(\theta) = \frac{3\sqrt{2}}{4}, \sec(\theta) = 3, \cot(\theta) = \frac{\sqrt{2}}{4}$, $\theta \approx 70.53^{\circ}$

\item $\sin(\theta) = \frac{\sqrt{5}}{5}, \cos(\theta) = \frac{2\sqrt{5}}{5}, \tan(\theta) = \frac{1}{2}, \csc(\theta) = \sqrt{5}, \sec(\theta) = \frac{\sqrt{5}}{2}, \cot(\theta) = 2$, $\theta \approx 26.57^{\circ}$

\item $\sin(\theta) = \frac{1}{5}, \cos(\theta) = \frac{2\sqrt{6}}{5}, \tan(\theta) = \frac{\sqrt{6}}{12}, \csc(\theta) = 5, \sec(\theta) = \frac{5\sqrt{6}}{12}, \cot(\theta) = 2\sqrt{6}$, $\theta \approx 11.54^{\circ}$

\item $\sin(\theta) = \frac{\sqrt{110}}{11}, \cos(\theta) = \frac{\sqrt{11}}{11}, \tan(\theta) = \sqrt{10}, \csc(\theta) = \frac{\sqrt{110}}{10}, \sec(\theta) = \sqrt{11}, \cot(\theta) = \frac{\sqrt{10}}{10}$, $\theta \approx 72.45^{\circ}$

\item $\sin(\theta) = \frac{\sqrt{95}}{10}, \cos(\theta) = \frac{\sqrt{5}}{10}, \tan(\theta) = \sqrt{19}, \csc(\theta) = \frac{2\sqrt{95}}{19}, \sec(\theta) = 2\sqrt{5}, \cot(\theta) = \frac{\sqrt{19}}{19}$, $\theta \approx 77.08^{\circ}$

\item $\sin(\theta) = \frac{\sqrt{21}}{5}, \cos(\theta) = \frac{2}{5}, \tan(\theta) = \frac{\sqrt{21}}{2}, \csc(\theta) = \frac{5\sqrt{21}}{21}, \sec(\theta) = \frac{5}{2}, \cot(\theta) = \frac{2\sqrt{21}}{21}$, $\theta \approx  66.42^{\circ}$

\setcounter{HW}{\value{enumi}}

\end{enumerate}

\begin{enumerate}

\setcounter{enumi}{\value{HW}}

\item The tree is about 47 feet tall.

\item The lights are about 75 feet apart.

\setcounter{HW}{\value{enumi}}

\end{enumerate}

\begin{enumerate}

\setcounter{enumi}{\value{HW}}

\item \begin{enumerate}

\addtocounter{enumii}{1}

\item The fire is about 4581 feet from the base of the tower.

\item  The Sasquatch ran $200\cot(6^{\circ}) - 200\cot(6.5^{\circ}) \approx 147$ feet in those 10 seconds. This translates to $\approx 10$ miles per hour.  At the scene of the second sighting, the Sasquatch was $\approx 1755$ feet from the tower, which means, if it keeps up this pace, it will reach the tower in about $2$ minutes.

\end{enumerate}

\setcounter{HW}{\value{enumi}}

\end{enumerate}

\begin{enumerate}

\setcounter{enumi}{\value{HW}}

\item  The tree is about 41 feet tall.

\item The boat has traveled about 244 feet.

\item  The tower is about 682 feet tall. The guy wire hits the ground about  731 feet away from the base of the tower.

\setcounter{HW}{\value{enumi}}

\end{enumerate}

\begin{multicols}{3}

\begin{enumerate}

\setcounter{enumi}{\value{HW}}

\item $68.9^{\circ}$

\item $7.7^{\circ}$

\item $51^{\circ}$

\setcounter{HW}{\value{enumi}}

\end{enumerate}

\end{multicols}

\begin{multicols}{3}

\begin{enumerate}

\setcounter{enumi}{\value{HW}}

\item $19.5^{\circ}$

\item  $41.81^{\circ}$

\item  $75.5^{\circ}$.

\setcounter{HW}{\value{enumi}}

\end{enumerate}

\end{multicols}

\end{document}
