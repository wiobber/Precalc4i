\documentclass{ximera}

\begin{document}
	\author{Stitz-Zeager}
	\xmtitle{Exercises for Function Arithmetic}{}

\mfpicnumber{1} \opengraphsfile{ExercisesforFunctionArithmetic} % mfpic settings added 


\label{ExercisesforFunctionArithmetic}

In Exercises \ref{basicarithonefirst} - \ref{basicarithonelast}, use the pair of functions $f$ and $g$ to find the following values if they exist.


\begin{itemize}

\item  $(f+g)(2)$ 
\item  $(f-g)(-1)$
\item  $(g-f)(1)$

\end{itemize}



\begin{itemize}

\item  $(fg)\left(\frac{1}{2}\right)$
\item  $\left(\frac{f}{g}\right)(0)$
\item  $\left(\frac{g}{f}\right)\left(-2\right)$

\end{itemize}





% Transformed Exercises with Solutions

\begin{question}
$f(x) = 3x+1$ and  $g(t) = 4-t$
\begin{solution}
For  $f(x) = 3x+1$ and $g(x) = 4-x$


\end{solution}

\end{question}

\begin{question}
$f(x) = x^2$ and $g(t) = -2t+1$

\begin{solution}
$(f+g)(2) = 9$
\end{solution}

\end{question}

\begin{question}
$f(x) = x^2 - x$ and  $g(t) = 12-t^2$
\begin{solution}
$(f-g)(-1) = -7$
\end{solution}

\end{question}

\begin{question}
$f(x) = 2x^3$ and $g(t) = -t^2-2t-3$

\begin{solution}
$(g-f)(1) = -1$

\end{solution}

\end{question}

\begin{question}
$f(x) = \sqrt{x+3}$ and  $g(t) = 2t-1$
\begin{solution}
$(fg)\left(\frac{1}{2}\right) = \frac{35}{4}$
\end{solution}

\end{question}

\begin{question}
$f(x) = \sqrt{4-x}$ and $g(t) = \sqrt{t+2}$

\begin{solution}
$\left(\frac{f}{g}\right)(0) = \frac{1}{4}$
\end{solution}

\end{question}

\begin{question}
$f(x) = 2x$ and  $g(t) = \dfrac{1}{2t+1}$
\begin{solution}
$\left(\frac{g}{f}\right)\left(-2\right) = -\frac{6}{5}$

\end{solution}

\end{question}

\begin{question}
$f(x) = x^2$ and $g(t) = \dfrac{3}{2t-3}$

\begin{solution}
For  $f(x) = x^2$ and $g(x) = -2x+1$
\end{solution}

\end{question}

\begin{question}
$f(x) = x^2$ and  $g(t) = \dfrac{1}{t^2}$
\begin{solution}
$(f+g)(2) = 1$
\end{solution}

\end{question}

\begin{question}
$f(x) = x^2+1$ and $g(t) = \dfrac{1}{t^2+1}$ 

\begin{solution}
$(f-g)(-1) = -2$
\end{solution}

\end{question}

\begin{question}
$(f + g)(-4)$
\begin{solution}
$(g-f)(1) = -2$

\end{solution}

\end{question}

\begin{question}
$(f + g)(0)$
\begin{solution}
$(fg)\left(\frac{1}{2}\right) = 0$
\end{solution}

\end{question}

\begin{question}
$(f- g)(4)$

\begin{solution}
$\left(\frac{f}{g}\right)(0) = 0$
\end{solution}

\end{question}

\begin{question}
$(fg)(-4)$
\begin{solution}
$\left(\frac{g}{f}\right)\left(-2\right) = \frac{5}{4}$

\end{solution}

\end{question}

\begin{question}
$(fg)(-2)$
\begin{solution}
For  $f(x) = x^2 - x$ and  $g(x) = 12-x^2$


\end{solution}

\end{question}

\begin{question}
$(fg)(4)$

\begin{solution}
$(f+g)(2) = 10$
\end{solution}

\end{question}

\begin{question}
$\left(\dfrac{f}{g}\right)(0)$
\begin{solution}
$(f-g)(-1) = -9$
\end{solution}

\end{question}

\begin{question}
$\left(\dfrac{f}{g}\right)(2)$
\begin{solution}
$(g-f)(1) = 11$

\end{solution}

\end{question}

\begin{question}
$\left(\dfrac{g}{f}\right)(-1)$ 

\begin{solution}
$(fg)\left(\frac{1}{2}\right) = -\frac{47}{16}$
\end{solution}

\end{question}

\begin{question}
Find the domains of $f+g$, $f-g$,  $fg$, $\dfrac{f}{g}$ and $\dfrac{g}{f}$.  

\begin{solution}
$\left(\frac{f}{g}\right)(0) = 0$
\end{solution}

\end{question}

\begin{question}
$(f + g)(-3)$
\begin{solution}
$\left(\frac{g}{f}\right)\left(-2\right) = \frac{4}{3}$

\end{solution}

\end{question}

\begin{question}
$(f - g)(2)$
\begin{solution}
For $f(x) = 2x^3$ and  $g(x) = -x^2-2x-3$


\end{solution}

\end{question}

\begin{question}
$(fg)(-1)$

\begin{solution}
$(f+g)(2) = 5$
\end{solution}

\end{question}

\begin{question}
$(g + f)(1)$
\begin{solution}
$(f-g)(-1) = 0$
\end{solution}

\end{question}

\begin{question}
$(g - f)(3)$
\begin{solution}
$(g-f)(1) = -8$

\end{solution}

\end{question}

\begin{question}
$(gf)(-3)$

\begin{solution}
$(fg)\left(\frac{1}{2}\right) = -\frac{17}{16}$
\end{solution}

\end{question}

\begin{question}
$\left(\frac{f}{g}\right)(-2)$
\begin{solution}
$\left(\frac{f}{g}\right)(0) = 0$
\end{solution}

\end{question}

\begin{question}
$\left(\frac{f}{g}\right)(-1)$
\begin{solution}
$\left(\frac{g}{f}\right)\left(-2\right) = \frac{3}{16}$

\end{solution}

\end{question}

\begin{question}
$\left(\frac{f}{g}\right)(2)$

\begin{solution}
For $f(x) = \sqrt{x+3}$ and  $g(x) = 2x-1$


\end{solution}

\end{question}

\begin{question}
$\left(\frac{g}{f}\right)(-1)$
\begin{solution}
$(f+g)(2) = 3+\sqrt{5}$
\end{solution}

\end{question}

\begin{question}
$\left(\frac{g}{f}\right)(3)$
\begin{solution}
$(f-g)(-1) = 3+\sqrt{2}$
\end{solution}

\end{question}

\begin{question}
$\left(\frac{g}{f}\right)(-3)$ 

\begin{solution}
$(g-f)(1) = -1$

\end{solution}

\end{question}

\begin{question}
$f(x) = 2x+1$ and $g(x) = x-2$
\begin{solution}
$(fg)\left(\frac{1}{2}\right) = 0$
\end{solution}

\end{question}

\begin{question}
$f(x) = 1-4x$ and $g(x) = 2x-1$

\begin{solution}
$\left(\frac{f}{g}\right)(0) = -\sqrt{3}$
\end{solution}

\end{question}

\begin{question}
$f(x) = x^2$ and $g(x) = 3x-1$
\begin{solution}
$\left(\frac{g}{f}\right)\left(-2\right) = -5$

\end{solution}

\end{question}

\begin{question}
$f(x) = x^2-x$ and $g(x) = 7x$

\begin{solution}
For $f(x) = \sqrt{4-x}$ and $g(x) = \sqrt{x+2}$


\end{solution}

\end{question}

\begin{question}
$f(x) = x^2-4$ and $g(x) = 3x+6$
\begin{solution}
$(f+g)(2) = 2+\sqrt{2}$
\end{solution}

\end{question}

\begin{question}
$f(x) = -x^2+x+6$ and $g(x) = x^2-9$

\begin{solution}
$(f-g)(-1) = -1+\sqrt{5}$
\end{solution}

\end{question}

\begin{question}
$f(x) = \dfrac{x}{2}$ and $g(x) = \dfrac{2}{x}$
\begin{solution}
$(g-f)(1) = 0$

\end{solution}

\end{question}

\begin{question}
$f(x) =x-1$ and $g(x) = \dfrac{1}{x-1}$

\begin{solution}
$(fg)\left(\frac{1}{2}\right) = \frac{\sqrt{35}}{2}$
\end{solution}

\end{question}

\begin{question}
$f(x) = x$ and $g(x) = \sqrt{x+1}$
\begin{solution}
$\left(\frac{f}{g}\right)(0) = \sqrt{2}$
\end{solution}

\end{question}

\begin{question}
$f(x) =\sqrt{x-5}$ and $g(x) = f(x) = \sqrt{x-5}$ 

\begin{solution}
$\left(\frac{g}{f}\right)\left(-2\right) = 0$

\end{solution}

\end{question}

\begin{question}
For $p(z) = 4z-z^3$, find functions $f$ and $g$ so that $p=f-g$.
\begin{solution}
For  $f(x) = 2x$ and  $g(x) = \frac{1}{2x+1}$


\end{solution}

\end{question}

\begin{question}
For $p(z) = 4z-z^3$, find functions $f$ and $g$ so that $p=f+g$.
\begin{solution}
$(f+g)(2) = \frac{21}{5}$
\end{solution}

\end{question}

\begin{question}
For $g(t) = 3t|2t-1|$, find functions $f$ and $h$  so that $g = fh$.
\begin{solution}
$(f-g)(-1) = -1$
\end{solution}

\end{question}

\begin{question}
For $r(x) = \dfrac{3-x}{x+1}$, find functions $f$ and $g$ so $r = \dfrac{f}{g}$.
\begin{solution}
$(g-f)(1) = -\frac{5}{3}$

\end{solution}

\end{question}

\begin{question}
For $r(x) = \dfrac{3-x}{x+1}$, find functions $f$ and $g$ so $r = fg$. 

\begin{solution}
$(fg)\left(\frac{1}{2}\right) = \frac{1}{2}$
\end{solution}

\end{question}

\begin{question}
Can $f(x) = x$ be decomposed as $f = g-h$ where $g(x) = x+\dfrac{1}{x}$ and $h(x) = \dfrac{1}{x}$?
\begin{solution}
$\left(\frac{f}{g}\right)(0) = 0$
\end{solution}

\end{question}

\begin{question}
Discuss with your classmates how to phrase the quantities revenue and profit in Definition \ref{revenueprofitdefns} terms of function arithmetic as defined in Definition \ref{functionarithmeticdefn}.
 
\begin{solution}
$\left(\frac{g}{f}\right)\left(-2\right) = \frac{1}{12}$

\end{solution}

\end{question}

\begin{question}
In this exercise, we explore decomposing a function into its positive and negative parts.  Given a function $f$, we define the \index{positive part of a function}\textbf{positive part} of $f$, denoted $f_{+}$ and \index{negative part of a function}\textbf{negative part} of $f$, denoted $f_{-}$ by:

\[ f_{+}(x) = \dfrac{f(x) + |f(x)|}{2}, \qquad \text{and} \qquad f_{-}(x) = \dfrac{f(x) - |f(x)|}{2}. \]

\begin{solution}
For  $f(x) = x^2$ and $g(x) = \frac{3}{2x-3}$


\end{solution}

\end{question}

\begin{question}
Using a graphing utility, graph each of the functions $f$ below along with $f_{+}$ and $f_{-}$.



\begin{solution}
$(f+g)(2) = 7$
\end{solution}

\end{question}

\begin{question}
$f(x) = x-3$
\begin{solution}
$(f-g)(-1) = \frac{8}{5}$
\end{solution}

\end{question}

\begin{question}
$f(x) = x^2-x-6$
\begin{solution}
$(g-f)(1) = -4$

\end{solution}

\end{question}

\begin{question}
$f(x) = 4x-x^3$

Why is $f_{+}$ called the `positive part' of $f$ and $f_{-}$ called the `negative part' of $f$?
\begin{solution}
$(fg)\left(\frac{1}{2}\right) = -\frac{3}{8}$
\end{solution}

\end{question}

\begin{question}
Show that $f = f_{+} + f_{-}$.
\begin{solution}
$\left(\frac{f}{g}\right)(0) = 0$
\end{solution}

\end{question}

\begin{question}
Use Definition \ref{absolutevaluepiecewise} to rewrite the expressions for $f_{+}(x)$ and $f_{-}(x)$ as piecewise defined functions.
\begin{solution}
$\left(\frac{g}{f}\right)\left(-2\right) = -\frac{3}{28}$

\end{solution}

\end{question}

\begin{question}
Let $U$ be the unit step function defined in Exercise \ref{unitstepexercise} in Section \ref{ConstantandLinearFunctions}.  For each function $f(t)$ below:

\begin{solution}
For  $f(x) = x^2$ and $g(x) = \frac{1}{x^2}$


\end{solution}

\end{question}

\begin{question}
Write $(Uf)(t)$ as a piecewise-defined function.
\begin{solution}
$(f+g)(2) =\frac{17}{4}$
\end{solution}

\end{question}

\begin{question}
Graph $y = f(t)$ and $y = (Uf)(t)$.

\begin{solution}
$(f-g)(-1) = 0$
\end{solution}

\end{question}

\begin{question}
$f(t) = t-3$
\begin{solution}
$(g-f)(1) = 0$

\end{solution}

\end{question}

\begin{question}
$f(t) = |t+2|$
\begin{solution}
$(fg)\left(\frac{1}{2}\right) =1$
\end{solution}

\end{question}

\begin{question}
$f(t) =(t-1)^2$



\begin{solution}
$\left(\frac{f}{g}\right)(0)$ is undefined.
\end{solution}

\end{question}

\begin{question}
$f(t) =(t+1)^{-1}$
\begin{solution}
$\left(\frac{g}{f}\right)\left(-2\right) = \frac{1}{16}$

\end{solution}

\end{question}

\begin{question}
$f(t) = \sqrt[3]{t-1}$ \vphantom{ $f(t) =(t+1)^{-1}$}
\begin{solution}
For  $f(x) = x^2+1$ and $g(x) = \frac{1}{x^2+1}$


\end{solution}

\end{question}

\begin{question}
$f(t) = (t-2)^{\frac{2}{3}}$ \vphantom{ $f(t) =(t+1)^{-1}$}


\begin{solution}
$(f+g)(2) =\frac{26}{5}$
\end{solution}

\end{question}

\begin{question}
Write a general formula for $(Uf)(t)$ for a function $f$.  (Assume the domain of $f$ is $(-\infty, \infty)$.)
\begin{solution}
$(f-g)(-1) = \frac{3}{2}$
\end{solution}

\end{question}

\begin{question}
Explain how to obtain the graph of $y=(Uf)(t)$ from $y=f(t)$.
\begin{solution}
$(g-f)(1) = -\frac{3}{2}$

\end{solution}

\end{question}

\begin{question}
The function $U(t)$ is used to model a change in state from `off' to `on' (like flipping a light switch.)  How does this relate to your observations?
\begin{solution}
$(fg)\left(\frac{1}{2}\right) =1$
\end{solution}

\end{question}

\begin{question}
Use the graph of $y=f(t)$ below to graph $y=(Uf)(t)$.

\begin{center}

% \input{ExercisesforFunctionArithmetic_pic3.tex}
\begin{tikzpicture}
\begin{axis}[fplot, xmin=-5, xmax=5, ymin=-4, ymax=4, xlabel={$t$}]
  \addplot[fgraph, domain=-4:2] {3*sin(deg(pi*x/4))};
  \addplot[only marks, mark=*] coordinates {(-2,-3) (2,3) (-4,0) (0,0)};
  \node[flabel, label=below left :{$(-2,-3)$}] at (axis cs:-2,-3) {};
  \node[flabel, label=above right:{$(2,3)$}]   at (axis cs: 2, 3) {};
  \node[flabel, label=above left :{$(-4,0)$}]  at (axis cs:-4, 0) {};
  \node[flabel, label=below right:{$(0,0)$}]   at (axis cs: 0, 0) {};
  \node at (rel axis cs:0.5,0) [below] {$y = f(t)$};
\end{axis}
\end{tikzpicture}


\end{center}
\begin{solution}
$\left(\frac{f}{g}\right)(0) = 1$
\end{solution}

\end{question}

\begin{question}
A chemist combines the solutions from two graduated cylinders into a beaker.  The volume of the first solution, $A$, an acid,  is read as $A_{1} = 101 \pm 0.5$ milliliters (mL). The volume of the second solution,  a base, $B$,  is measured to be $B_{1} = 16 \pm 0.5$ mL.    Estimate the percent propagated error in calculating the volume of the combined solution as $V = A_{1} + B_{1} = 101 + 16 = 117$ mL.
\begin{solution}
$\left(\frac{g}{f}\right)\left(-2\right) = \frac{1}{25}$

\end{solution}

\end{question}

\begin{question}
A student measures the length, $\ell$, and width, $w$,  of a piece of paper.  They find  $\ell_{1} = 280 \pm 0.5$ millimeters (mm) $w_{1} = 216 \pm 0.5$ mm.    Estimate the percent propagated error in calculating the area of the piece of paper as $A = \ell_{1} \, w_{1} = 280 \times 216 = 60480 \, \text{mm}^2$.
\begin{solution}
$(f + g)(-4) = -5$
\end{solution}

\end{question}

\begin{question}
An airplane passenger  observers a car travel a distance $d_{1} = 1320 \pm 2$ feet (ft) in time $t_{1}  = 15 \pm 0.5$ seconds (s).  Estimate the percent propagated error in calculating the speed of the car as $v = \frac{d_{1}}{t_{1}} = \frac{1320}{15} = 88 \, \frac{\text{ft}}{\text{s}}$.
\begin{solution}
$(f + g)(0) = 5$
\end{solution}

\end{question}

\begin{question}
Find and interpret $\overline{C}(75)$.
\begin{solution}
$(f-g)(4) = -5$

\end{solution}

\end{question}

\begin{question}
Define  the \index{marginal cost}\index{cost ! marginal}\textbf{marginal cost} $MC(x) = C(x+1) - C(x)$.   Find and interpret $MC(75)$.
\begin{solution}
$(fg)(-4) = 6$
\end{solution}

\end{question}

\begin{question}
How do your answers to parts \ref{ACexercise1} and \ref{MCexercise1} compare?
\begin{solution}
$(fg)(-2) = 0$
\end{solution}

\end{question}

\begin{question}
Graph $y = \overline{C}(x)$ with help from a graphing utility.  What is happening graphically near $x = 75$?
\begin{solution}
$(fg)(4) = -6$

\end{solution}

\end{question}

\begin{question}
Use Theorem \ref{functionarithmeticaroc} to show that, in general, $\text{ARoC}[ \overline{C}(x)] = 0$ when $MC(x) = \overline{C}(x)$.

\smallskip

\textbf{HINT:}  Note that, by definition, $MC(x) = C(x+1) - C(x) = \Delta[C(x)]$ when $\Delta x = 1$.  

\smallskip

Hence,  $\text{ARoC}[C(x)] = \frac{\Delta[C(x)]}{\Delta x} = \frac{\Delta[C(x)]}{1} = \Delta[C(x)] = MC(x)$ in this case \ldots
\begin{solution}
$\left(\dfrac{f}{g}\right)(0) = \dfrac{3}{2}$
\end{solution}

\end{question}

\end{document}