\documentclass{ximera}

\begin{document}
	\author{Stitz-Zeager}
	\xmtitle{Exercises for Function Arithmetic}{}

\mfpicnumber{1} \opengraphsfile{ExercisesforFunctionArithmetic} % mfpic settings added 


\label{ExercisesforFunctionArithmetic}

\begin{question}
    
In Exercises \ref{basicarithonefirst} - \ref{basicarithonelast}, use the pair of functions $f$ and $g$ to find each of the following  if they exist.


\begin{itemize}
\item  $(f+g)(2)$ 
\item  $(f-g)(-1)$
\item  $(g-f)(1)$
\item  $(fg)\left(\frac{1}{2}\right)$
\item  $\left(\frac{f}{g}\right)(0)$
\item  $\left(\frac{g}{f}\right)\left(-2\right)$
\end{itemize}

\begin{problem}\label{basicarithonefirst}
$f(x) = 3x+1$ and  $g(t) = 4-t$ 


\begin{itemize}
\item  $(f+g)(2)=\answer{9}$ 
\item  $(f-g)(-1)=\answer{-7}$
\item  $(g-f)(1)=\answer{-1}$
\item  $(fg)\left(\frac{1}{2}\right)=\answer{\frac{35}{4}}$
\item  $\left(\frac{f}{g}\right)(0)=\answer{\frac{1}{4}}$
\item  $\left(\frac{g}{f}\right)\left(-2\right)=\answer{-\frac{6}{5}}$
\end{itemize}

\end{problem}

\begin{problem}
$f(x) = x^2$ and $g(t) = -2t+1$

\begin{itemize}
\item  $(f+g)(2)=\answer{1}$ 
\item  $(f-g)(-1)=\answer{-2}$
\item  $(g-f)(1)=\answer{-2}$
\item  $(fg)\left(\frac{1}{2}\right)=\answer{0}$
\item  $\left(\frac{f}{g}\right)(0)=\answer{0}$
\item  $\left(\frac{g}{f}\right)\left(-2\right)=\answer{\frac{5}{4}}$
\end{itemize}

\end{problem}

\begin{problem}
$f(x) = x^2 - x$ and  $g(t) = 12-t^2$

\begin{itemize}
\item  $(f+g)(2)=\answer{10}$ 
\item  $(f-g)(-1)=\answer{-9}$
\item  $(g-f)(1)=\answer{11}$
\item  $(fg)\left(\frac{1}{2}\right)=\answer{-\frac{47}{16}}$
\item  $\left(\frac{f}{g}\right)(0)=\answer{0}$
\item  $\left(\frac{g}{f}\right)\left(-2\right)=\answer{\frac{4}{3}}$
\end{itemize}

\end{problem}

\begin{problem}
$f(x) = 2x^3$ and $g(t) = -t^2-2t-3$

\begin{itemize}
\item  $(f+g)(2)=\answer{5}$ 
\item  $(f-g)(-1)=\answer{0}$
\item  $(g-f)(1)=\answer{-8}$
\item  $(fg)\left(\frac{1}{2}\right)=\answer{-\frac{17}{16}}$
\item  $\left(\frac{f}{g}\right)(0)=\answer{0}$
\item  $\left(\frac{g}{f}\right)\left(-2\right)=\answer{\frac{3}{16}}$
\end{itemize}

\end{problem}

\begin{problem}
$f(x) = \sqrt{x+3}$ and  $g(t) = 2t-1$

\begin{itemize}
\item  $(f+g)(2)=\answer{3+\sqrt{5}}$ 
\item  $(f-g)(-1)=\answer{3+\sqrt{2}}$
\item  $(g-f)(1)=\answer{-1}$
\item  $(fg)\left(\frac{1}{2}\right)=\answer{0}$
\item  $\left(\frac{f}{g}\right)(0)=\answer{-\sqrt{3}}$
\item  $\left(\frac{g}{f}\right)\left(-2\right)=\answer{-5}$
\end{itemize}

\end{problem}

\begin{problem}
$f(x) = \sqrt{4-x}$ and $g(t) = \sqrt{t+2}$

\begin{itemize}
\item  $(f+g)(2)=\answer{2+\sqrt{2}}$ 
\item  $(f-g)(-1)=\answer{-1+\sqrt{5}}$
\item  $(g-f)(1)=\answer{0}$
\item  $(fg)\left(\frac{1}{2}\right)=\answer{\frac{\sqrt{35}}{2}}$
\item  $\left(\frac{f}{g}\right)(0)=\answer{\sqrt{2}}$
\item  $\left(\frac{g}{f}\right)\left(-2\right)=\answer{0}$
\end{itemize}

\end{problem}

\begin{problem}
$f(x) = 2x$ and  $g(t) = \dfrac{1}{2t+1}$

\begin{itemize}
\item  $(f+g)(2)=\answer{\frac{21}{5}}$ 
\item  $(f-g)(-1)=\answer{-1}$
\item  $(g-f)(1)=\answer{-\frac{5}{3}}$
\item  $(fg)\left(\frac{1}{2}\right)=\answer{\frac{1}{2}}$
\item  $\left(\frac{f}{g}\right)(0)=\answer{0}$
\item  $\left(\frac{g}{f}\right)\left(-2\right)=\answer{\frac{1}{12}}$
\end{itemize}

\end{problem} 

\begin{problem}
$f(x) = x^2$ and $g(t) = \dfrac{3}{2t-3}$

\begin{itemize}
\item  $(f+g)(2)=\answer{7}$ 
\item  $(f-g)(-1)=\answer{\frac{8}{5}}$
\item  $(g-f)(1)=\answer{-4}$
\item  $(fg)\left(\frac{1}{2}\right)=\answer{-\frac{3}{8}}$
\item  $\left(\frac{f}{g}\right)(0)=\answer{0}$
\item  $\left(\frac{g}{f}\right)\left(-2\right)=\answer{-\frac{3}{28}}$
\end{itemize}

\end{problem}   

\begin{problem}
$f(x) = x^2$ and  $g(t) = \dfrac{1}{t^2}$

\begin{itemize}
\item  $(f+g)(2)=\answer{\frac{17}{4}}$ 
\item  $(f-g)(-1)=\answer{0}$
\item  $(g-f)(1)=\answer{0}$
\item  $(fg)\left(\frac{1}{2}\right)=\answer{1}$
\item  $\left(\frac{f}{g}\right)(0)$
\begin{solution}
is undefined
\end{solution}
\item  $\left(\frac{g}{f}\right)\left(-2\right)=\answer{\frac{1}{16}}$
\end{itemize}

\end{problem} 

\begin{problem}\label{basicarithonelast}
$f(x) = x^2+1$ and $g(t) = \dfrac{1}{t^2+1}$

\begin{itemize}
\item  $(f+g)(2)=\answer{\frac{26}{5}}$ 
\item  $(f-g)(-1)=\answer{\frac{3}{2}}$
\item  $(g-f)(1)=\answer{-\frac{3}{2}}$
\item  $(fg)\left(\frac{1}{2}\right)=\answer{1}$
\item  $\left(\frac{f}{g}\right)(0)=\answer{1}$
\item  $\left(\frac{g}{f}\right)\left(-2\right)=\answer{\frac{1}{25}}$
\end{itemize}

\end{problem}   

\end{question}


\begin{question}
Exercises \ref{arithfromgraphfirst} - \ref{arithfromgraphlast} refer to the functions $f$ and $g$ whose graphs are below. 


\begin{tikzpicture}[baseline]
  % === Left Graph: y = f(x) ===
  \begin{axis}[
      name=leftplot,
      axis lines=middle,
      xmin=-5, xmax=5,
      ymin=-5, ymax=5,
      xtick={-4,-3,-2,-1,1,2,3,4},
      ytick={-4,-3,-2,-1,1,2,4},
      xlabel={$x$},
      ylabel={$y$},
      width=7.5cm,
      height=7.5cm,
      grid=none,
      ticklabel style={font=\scriptsize},
      label style={font=\scriptsize},
    ]

    % Function curve
    \addplot[thick, domain=-4:4, samples=200] {3*cos(deg(pi*x/4))};

    % Points
    \addplot[only marks, mark=*] coordinates {
      (-2,0) (2,0) (4,-3) (-4,-3) (0,3)
    };

    % Labels — all slightly nudged inward
    \node[font=\small, anchor=west] at (axis cs:-3.6,-2.6) {$( -4, -3 )$};
    \node[font=\small, anchor=east] at (axis cs:-2.2,0.4) {$( -2, 0 )$};
    \node[font=\small, anchor=west] at (axis cs:2.3,0.4) {$( 2, 0 )$};
    \node[font=\small, anchor=east] at (axis cs:3.5,-2.5) {$( 4, -3 )$}; % nudged inward
    \node[font=\small, anchor=south east] at (axis cs:-0.3,3) {$( 0, 3 )$};

    % Function name on plot
    \node[font=\scriptsize, anchor=west, fill=white, inner sep=1pt]
      at (axis cs:1.2,2.5) {$y = f(x)$};

  \end{axis}

  % === Right Graph: y = g(t) ===
  \begin{axis}[
      at={(leftplot.east)},
      anchor=west,
      xshift=2.5cm,
      axis lines=middle,
      xmin=-5, xmax=5,
      ymin=-5, ymax=5,
      xtick={-4,-3,-2,-1,1,2,3,4},
      ytick={-4,-3,-2,-1,1,3,4},
      xlabel={$t$},
      ylabel={$y$},
      width=7.5cm,
      height=7.5cm,
      grid=none,
      ticklabel style={font=\scriptsize},
      label style={font=\scriptsize},
    ]

    % Line segments and arrow
    \addplot[thick] coordinates {(-4,-2) (-1,0) (0,2)};
    \addplot[thick, -{Stealth[length=2mm]}] coordinates {(0,2) (5,2)};

    % Points
    \addplot[only marks, mark=*] coordinates {
      (-4,-2) (-1,0) (0,2)
    };

    % Labels (nudged inward)
    \node[font=\small, anchor=west] at (axis cs:-3.8,-2.4) {$( -4, -2 )$};
    \node[font=\small, anchor=west] at (axis cs:-0.9,0.5) {$( -1, 0 )$};
    \node[font=\small, anchor=west] at (axis cs:-0.3,2.3) {$( 0, 2 )$};

    % Function name on plot
    \node[font=\scriptsize, anchor=west, fill=white, inner sep=1pt]
      at (axis cs:1.3,2.1) {$y = g(t)$};

  \end{axis}
\end{tikzpicture}

\begin{problem}\label{arithfromgraphfirst}
$(f + g)(-4)=\answer{-5}$
\end{problem}

\begin{problem}
$(f + g)(0)=\answer{5}$
\end{problem}
 
\begin{problem}
$(f- g)(4)=\answer{-5}$
\end{problem} 

\begin{problem}
$(fg)(-4)=\answer{6}$
\end{problem}  

\begin{problem}
$(fg)(-2)=\answer{0}$
\end{problem} 

\begin{problem}
$(fg)(4)=\answer{-6}$
\end{problem} 

\begin{problem}
$\left(\dfrac{f}{g}\right)(0)=\answer{\dfrac{3}{2}}$
\end{problem} 

\begin{problem}
$\left(\dfrac{f}{g}\right)(2)=\answer{0}$
\end{problem}

\begin{problem}
$\left(\dfrac{g}{f}\right)(-1)=\answer{0}$ 
\end{problem}

\begin{problem}\label{arithfromgraphlast}
Find the domains of $f+g$, $f-g$,  $fg$, $\dfrac{f}{g}$ and $\dfrac{g}{f}$.

\begin{solution}
 The domains of $f+g$, $f-g$ and $fg$ are all $[-4,4]$.  The domain of $\frac{f}{g}$ is $[-4, -1) \cup (-1,4]$ and the domain of $\frac{g}{f}$ is $[-4, -2) \cup (-2,2) \cup (2, 4]$.
\end{solution}

\end{problem}

\end{question}

\begin{question}
In Exercises \ref{reformarithfirst} - \ref{reformarithlast}, let $f$ be the function defined by \[f = \{(-3, 4), (-2, 2), (-1, 0), (0, 1), (1, 3), (2, 4), (3, -1)\}\] and let $g$ be the function defined by \[g = \{(-3, -2), (-2, 0), (-1, -4), (0, 0), (1, -3), (2, 1), (3, 2)\}\] Compute the indicated value if it exists.

\begin{problem}\label{reformarithfirst}
$(f + g)(-3)=\answer{2}$
\end{problem}

\begin{problem}
$(f - g)(2)=\answer{3}$
\end{problem}

\begin{problem}
$(fg)(-1)=\answer{0}$
\end{problem} 

\begin{problem}
$(g + f)(1)=\answer{0}$
\end{problem}

\begin{problem}
$(g - f)(3)=\answer{3}$
\end{problem}

\begin{problem}
$(gf)(-3)=\answer{-8}$
\end{problem} 

\begin{problem}
$\left(\frac{f}{g}\right)(-2)$

\begin{solution}
Does not exist.
\end{solution}

\end{problem} 

\begin{problem}
$\left(\frac{f}{g}\right)(-1)=\answer{0}$
\end{problem}  

\begin{problem}
$\left(\frac{f}{g}\right)(2)=\answer{4}$
\end{problem}  

\begin{problem}
$\left(\frac{g}{f}\right)(-1)$

\begin{solution}
Does not exist.
\end{solution}

\end{problem}  

\begin{problem}
$\left(\frac{g}{f}\right)(3)=\answer{-2}$
\end{problem}


\begin{problem}\label{reformarithlast}
$\left(\frac{g}{f}\right)(-3)=\answer{-\frac{1}{2}}$ 
\end{problem}

\end{question}

\begin{question}
In Exercises \ref{basicarithtwofirst} - \ref{basicarithtwolast}, use the pair of functions $f$ and $g$ to find the domain of the indicated function then find and simplify an expression for it.
\begin{itemize}
\item  $(f+g)(x)$
\item  $(f-g)(x)$
\item  $(fg)(x)$
\item  $\left(\frac{f}{g}\right)(x)$
\end{itemize}

\begin{problem}\label{basicarithtwofirst}
$f(x) = 2x+1$ and $g(x) = x-2$ 

\begin{itemize}
\item  $(f+g)(x)=\answer{3x-1}$
\begin{solution}
Domain: $(-\infty, \infty)$
\end{solution}
\item  $(f-g)(x)=\answer{x+3}$
\begin{solution}
Domain:  $(-\infty, \infty)$
\end{solution}
\item  $(fg)(x)=\answer{2x^2-3x-2}$
\begin{solution}
Domain: $(-\infty, \infty)$
\end{solution}
\item  $\left(\frac{f}{g}\right)(x)=\answer{\frac{2x+1}{x-2}}$
\begin{solution}
Domain:  $(-\infty, 2) \cup (2, \infty)$
\end{solution}
\end{itemize}

\end{problem}

\begin{problem}
$f(x) = 1-4x$ and $g(x) = 2x-1$

\begin{itemize}
\item  $(f+g)(x)=\answer{-2x}$
\begin{solution}
 Domain: $(-\infty, \infty)$
\end{solution}
\item  $(f-g)(x)=\answer{2-6x}$
\begin{solution}
Domain:  $(-\infty, \infty)$
\end{solution}
\item  $(fg)(x)=\answer{-8x^2+6x-1}$
\begin{solution}
Domain: $(-\infty, \infty)$
\end{solution}
\item  $\left(\frac{f}{g}\right)(x)=\answer{ \frac{1-4x}{2x-1}}$
\begin{solution}
 Domain:  $\left(-\infty, \frac{1}{2} \right) \cup \left(\frac{1}{2}, \infty \right)$
\end{solution}
\end{itemize}

\end{problem}

\begin{problem}
$f(x) = x^2$ and $g(x) = 3x-1$

\begin{itemize}
\item  $(f+g)(x)=\answer{x^2+3x-1}$
\begin{solution}
Domain: $(-\infty, \infty)$
\end{solution}
\item  $(f-g)(x)=\answer{x^2-3x+1}$
\begin{solution}
Domain:  $(-\infty, \infty)$
\end{solution}
\item  $(fg)(x)=\answer{3x^3-x^2}$
\begin{solution}
Domain: $(-\infty, \infty)$
\end{solution}
\item  $\left(\frac{f}{g}\right)(x)=\answer{\frac{x^2}{3x-1}}$
\begin{solution}
Domain:  $\left(-\infty, \frac{1}{3} \right) \cup \left(\frac{1}{3}, \infty \right)$
\end{solution}
\end{itemize}

\end{problem}

\begin{problem}
$f(x) = x^2-x$ and $g(x) = 7x$

\begin{itemize}
\item  $(f+g)(x)=\answer{x^2+6x}$
\begin{solution}
Domain: $(-\infty, \infty)$
\end{solution}
\item  $(f-g)(x)=\answer{x^2-8x}$
\begin{solution}
Domain:  $(-\infty, \infty)$
\end{solution}
\item  $(fg)(x)=\answer{7x^3-7x^2}$
\begin{solution}
Domain: $(-\infty, \infty)$
\end{solution}
\item  $\left(\frac{f}{g}\right)(x)=\answer{\frac{x-1}{7}}$
\begin{solution}
Domain:  $\left(-\infty, 0 \right) \cup \left(0, \infty \right)$
\end{solution}
\end{itemize}

\end{problem}

\begin{problem}
$f(x) = x^2-4$ and $g(x) = 3x+6$

\begin{itemize}
\item  $(f+g)(x)=\answer{x^2+3x+}$
Domain: $(-\infty, \infty)$
\item  $(f-g)(x)=\answer{x^2-3x-10}$
\begin{solution}
Domain:  $(-\infty, \infty)$
\end{solution}
\item  $(fg)(x)=\answer{3x^3+6x^2-12x-24}$
\begin{solution}
Domain: $(-\infty, \infty)$
\end{solution}
\item  $\left(\frac{f}{g}\right)(x)=\answer{\frac{x-2}{3}}$
\begin{solution}
Domain:  $\left(-\infty, -2 \right) \cup \left(-2, \infty \right)$
\end{solution}
\end{itemize}

\end{problem} 

\begin{problem}
$f(x) = -x^2+x+6$ and $g(x) = x^2-9$

\begin{itemize}
\item  $(f+g)(x)=\answer{x-3}$
\begin{solution}
Domain: $(-\infty, \infty)$
\end{solution}
\item  $(f-g)(x)=\answer{-2x^2+x+15}$
\begin{solution}
Domain:  $(-\infty, \infty)$
\end{solution}
\item  $(fg)(x)=\answer{-x^4+x^3+15x^2-9x-54}$
\begin{solution}
Domain: $(-\infty, \infty)$
\end{solution}
\item  $\left(\frac{f}{g}\right)(x)=\answer{-\frac{x+2}{x+3}}$
\begin{solution}
Domain:  $\left(-\infty, -3 \right) \cup \left(-3, 3 \right) \cup (3, \infty)$
\end{solution}
\end{itemize}

\end{problem} 

\begin{problem}
$f(x) = \dfrac{x}{2}$ and $g(x) = \dfrac{2}{x}$

\begin{itemize}
\item  $(f+g)(x)=\answer{\frac{x^2+4}{2x}}$
\begin{solution}
Domain: $(-\infty, 0) \cup (0, \infty)$
\end{solution}
\item  $(f-g)(x)=\answer{\frac{x^2-4}{2x}}$
\begin{solution}
Domain:  $(-\infty,0) \cup (0, \infty)$
\end{solution}
\item  $(fg)(x)=\answer{1}$
\begin{solution}
Domain: $(-\infty,0) \cup (0, \infty)$
\end{solution}
\item  $\left(\frac{f}{g}\right)(x)=\answer{\frac{x^2}{4}}$
\begin{solution}
Domain: $(-\infty,0) \cup (0, \infty)$
\end{solution}
\end{itemize}

\end{problem} 

\begin{problem}
$f(x) =x-1$ and $g(x) = \dfrac{1}{x-1}$

\begin{itemize}
\item  $(f+g)(x)=\answer{\frac{x^2-2x+2}{x-1}}$
\begin{solution}
Domain: $(-\infty, 1) \cup (1, \infty)$
\end{solution}
\item  $(f-g)(x)=\answer{\frac{x^2-2x}{x-1}}$
\begin{solution}
Domain:  $(-\infty,1) \cup (1, \infty)$
\end{solution}
\item  $(fg)(x)=\answer{1}$
\begin{solution}
Domain: $(-\infty,1) \cup (1, \infty)$
\end{solution}
\item  $\left(\frac{f}{g}\right)(x)=\answer{x^2-2x+1}$
\begin{solution}
Domain: $(-\infty,1) \cup (1, \infty)$
\end{solution}
\end{itemize}

\end{problem}

\begin{problem}
$f(x) = x$ and $g(x) = \sqrt{x+1}$

\begin{itemize}
\item  $(f+g)(x)=\answer{x+\sqrt{x+1}}$
\begin{solution}
Domain: $[-1,\infty)$
\end{solution}
\item  $(f-g)(x)=\answer{x-\sqrt{x+1}}$
\begin{solution}
Domain: $[-1,\infty)$
\end{solution}
\item  $(fg)(x)=\answer{x\sqrt{x+1}}$
\begin{solution}
Domain: $[-1,\infty)$
\end{solution}
\item  $\left(\frac{f}{g}\right)(x)=\answer{\frac{x}{\sqrt{x+1}}}$
\begin{solution}
Domain: $(-1,\infty)$
\end{solution}
\end{itemize}

\end{problem}

\begin{problem}\label{basicarithtwolast}
$f(x) =\sqrt{x-5}$ and $g(x) = f(x) = \sqrt{x-5}$

\begin{itemize}
\item  $(f+g)(x)=\answer{2\sqrt{x-5}}$
Domain: $[5,\infty)$
\item  $(f-g)(x)=\answer{0}$
Domain: $[5,\infty)$
\item  $(fg)(x)=\answer{x-5}$
\begin{solution}
Domain: $[5,\infty)$
\end{solution}
\item  $\left(\frac{f}{g}\right)(x)=\answer{1}$
\begin{solution}
    Domain: $(5,\infty)$
\end{solution}
\end{itemize}

\end{problem}

\end{question}

\begin{question}
In Exercises \ref{decomposebasicfirst} - \ref{decomposebasiclast}, write the given function as a nontrivial decomposition of functions as directed.

\begin{problem}\label{decomposebasicfirst}
For $p(z) = 4z-z^3$, find functions $f$ and $g$ so that $p=f-g$.
\begin{solution}
One solution is $f(z) = 4z$ and $g(z) = z^3$.
\end{solution}
\end{problem}
  
\begin{problem}
For $p(z) = 4z-z^3$, find functions $f$ and $g$ so that $p=f+g$.
\begin{solution}
    One solution is $f(z) = 4z$ and $g(z) = - z^3$.
\end{solution}
\end{problem} 

\begin{problem}
For $g(t) = 3t|2t-1|$, find functions $f$ and $h$  so that $g = fh$.
\begin{solution}
    One solution is  $f(t) = 3t$ and $h(t) = |2t-1|$ 
\end{solution}
\end{problem}

\begin{problem}
For $r(x) = \dfrac{3-x}{x+1}$, find functions $f$ and $g$ so $r = \dfrac{f}{g}$.
\begin{solution}
    One solution is $f(x) = 3-x$ and $g(x) = x+1$. 
\end{solution}
\end{problem} 

\begin{problem}\label{decomposebasiclast}
For $r(x) = \dfrac{3-x}{x+1}$, find functions $f$ and $g$ so $r = fg$.
\begin{solution}
    One solution is $f(x) = 3-x$ and $g(x) = (x+1)^{-1}$. 
\end{solution}
\end{problem} 



\end{question}

\begin{problem}
Can $f(x) = x$ be decomposed as $f = g-h$ where $g(x) = x+\dfrac{1}{x}$ and $h(x) = \dfrac{1}{x}$? 
\begin{solution}
    No.  The equivalence does not hold when $x = 0$.
\end{solution}
\end{problem}

\begin{problem}
Discuss with your classmates how to phrase the quantities revenue and profit in Definition \ref{revenueprofitdefns} terms of function arithmetic as defined in Definition \ref{functionarithmeticdefn}.
\end{problem}


\begin{problem}\label{posnegdecompexercise}  
In this exercise, we explore decomposing a function into its positive and negative parts.  Given a function $f$, we define the \index{positive part of a function}\textbf{positive part} of $f$, denoted $f_{+}$ and \index{negative part of a function}\textbf{negative part} of $f$, denoted $f_{-}$ by:

\[ f_{+}(x) = \dfrac{f(x) + |f(x)|}{2}, \qquad \text{and} \qquad f_{-}(x) = \dfrac{f(x) - |f(x)|}{2}. \]

\begin{enumerate}

\item Using a graphing utility, graph each of the functions $f$ below along with $f_{+}$ and $f_{-}$.

%\begin{multicols}{3}

\begin{itemize}

\item  $f(x) = x-3$

\item  $f(x) = x^2-x-6$

\item  $f(x) = 4x-x^3$

\end{itemize}

%\end{multicols}

Why is $f_{+}$ called the `positive part' of $f$ and $f_{-}$ called the `negative part' of $f$?

\item Show that $f = f_{+} + f_{-}$.

\item Use Definition \ref{absolutevaluepiecewise} to rewrite the expressions for $f_{+}(x)$ and $f_{-}(x)$ as piecewise defined functions.

\end{enumerate}  
\end{problem} 


\end{document}
