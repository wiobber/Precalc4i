
\documentclass{ximera}

\begin{document}
	\author{Stitz-Zeager}
	\xmtitle{Exercises for The Algebra of Difference Quotients}{}

\mfpicnumber{1} \opengraphsfile{ExercisesforFunctionArithmetic} % mfpic settings added 


\label{ExercisesforFunctionArithmetic}

\begin{question}
    
In Exercises \ref{diffquotexerfirst} - \ref{diffquotexerlast}, find and simplify the difference quotient $\dfrac{f(x+h) - f(x)}{h}$ for the given function.

\begin{problem}\label{diffquotexerfirst}
$f(x) = 2x - 5$

\begin{solution}
    $2$
\end{solution}
\end{problem}
 
\begin{problem}
$f(x) = -3x + 5$   

\begin{solution}
    $-3$
\end{solution}
\end{problem} 

\begin{problem}
$f(x) = 6$   

\begin{solution}
    $0$
\end{solution}
\end{problem}  

\begin{problem}
$f(x) = 3x^2 - x$   

\begin{solution}
    $6x+3h-1$
\end{solution}
\end{problem} 

\begin{problem}
$f(x) = -x^2 + 2x - 1$   

\begin{solution}
    $-2x-h+2$
\end{solution}
\end{problem} 

\begin{problem}
$f(x) = 4x^2$   

\begin{solution}
    $8x+4h$
\end{solution}
\end{problem}

\begin{problem}
$f(x) = x-x^2$

\begin{solution}
$-2x-h+1$
\end{solution}
\end{problem} 

\begin{problem}
$f(x) = x^{3} + 1$

\begin{solution}
    $3x^{2} + 3xh + h^{2}$
\end{solution}
\end{problem}  

\begin{problem}
$f(x) = mx + b\;$ where $m \neq 0$

\begin{solution}
    $m$
\end{solution}
\end{problem} 

\begin{problem}
$f(x) = ax^{2} + bx + c\;$ where $a \neq 0$

\begin{solution}
    $2ax + ah + b$
\end{solution}
\end{problem}  

\begin{problem}
$f(x) = \dfrac{2}{x}$

\begin{solution}
    $\dfrac{-2}{x(x+h)}$
\end{solution}
\end{problem} 

\begin{problem}
$f(x) = \dfrac{3}{1-x}$

\begin{solution}
    $\dfrac{3}{(1-x-h)(1-x)}$
\end{solution}
\end{problem}

\begin{problem}
$f(x) = \dfrac{1}{x^2}$

\begin{solution}
$\dfrac{-(2x+h)}{x^2(x+h)^2}$
\end{solution}
\end{problem}  

\begin{problem}
 $f(x) = \dfrac{2}{x+5}$

 \begin{solution}
     $\dfrac{-2}{(x+5)(x+h+5)}$
 \end{solution}
\end{problem}

\begin{problem}
$f(x) = \dfrac{1}{4x-3}$ 

\begin{solution}
    $\dfrac{-4}{(4x-3)(4x+4h-3)}$
\end{solution}
\end{problem} 

\begin{problem}
$f(x) = \dfrac{3x}{x+1}$  

\begin{solution}
    $\dfrac{3}{(x+1)(x+h+1)}$
\end{solution}
\end{problem}  

\begin{problem}
$f(x) = \dfrac{x}{x - 9}$  

\begin{solution}
    $\dfrac{-9}{(x - 9)(x + h - 9)}$
\end{solution}
\end{problem}

\begin{problem}
$f(x) = \dfrac{x^2}{2x+1}$  

\begin{solution}
    $\dfrac{2x^2+2xh+2x+h}{(2x+1)(2x+2h+1)}$
\end{solution}
\end{problem} 

\begin{problem}
$f(x) = \sqrt{x-9}$

\begin{solution}
$\dfrac{1}{\sqrt{x+h-9} + \sqrt{x-9}}$
\end{solution}
\end{problem}   

\begin{problem}
$f(x) = \sqrt{2x+1}$ 

\begin{solution}
    $\dfrac{2}{\sqrt{2x+2h+1} + \sqrt{2x+1}}$
\end{solution}
\end{problem}   

\begin{problem}
$f(x) = \sqrt{-4x+5}$ 

\begin{solution}
    $\dfrac{-4}{\sqrt{-4x-4h+5} + \sqrt{-4x+5}}$
\end{solution}
\end{problem}

\begin{problem}
$f(x) = \sqrt{4-x}$ 

\begin{solution}
    $\dfrac{-1}{\sqrt{4-x-h} + \sqrt{4-x}}$
\end{solution}
\end{problem}

\begin{problem}
$f(x) = \sqrt{ax+b}$, where $a \neq 0$.

\begin{solution}
    $\dfrac{a}{\sqrt{ax+ah+b} + \sqrt{ax+b}}$
\end{solution}
\end{problem}

\begin{problem}
$f(x) = x \sqrt{x}$

\begin{solution}
    $\dfrac{3x^2+3xh+h^2}{(x+h)^{3/2} + x^{3/2}}$
\end{solution}
\end{problem}

\begin{problem}\label{diffquotexerlast}
$f(x) = \sqrt[3]{x}$

\begin{hint}
$(a-b)\left(a^2+ab+b^2\right) = a^3 - b^3$
\end{hint}

\begin{solution}
    $\dfrac{1}{(x+h)^{2/3} + (x+h)^{1/3} x^{1/3} + x^{2/3}}$
\end{solution}
\end{problem}
  

\end{question}

\begin{question}
    

In Exercises \ref{econexerfirst} - \ref{econexerlast}, $C(x)$ denotes the cost to produce $x$ items and $p(x)$ denotes the price-demand function in the given economic scenario.  In each Exercise, do the following:

%\begin{multicols}{2}
\begin{itemize}

\item  Find and interpret $C(0)$.
\item  Find and interpret $\overline{C}(10)$.
\item  Find and interpret $p(5)$
\item  Find and simplify $R(x)$.
\item  Find and simplify $P(x)$.
\item  Solve $P(x) = 0$ and interpret.
\end{itemize}
%\end{multicols}

\begin{problem}\label{econexerfirst}
The cost, in dollars, to produce $x$ ``I'd rather be a Sasquatch'' T-Shirts is $C(x) = 2x+26$, $x \geq 0$ and the price-demand function, in dollars per shirt,  is $p(x) = 30 - 2x$, $0 \leq x \leq 15$. 

\begin{itemize}

\item  Find and interpret $C(0)$.

\begin{solution}
    $C(0) = 26$, so the fixed costs are $\$26$.
\end{solution}
\item  Find and interpret $\overline{C}(10)$.

\begin{solution}
    $\overline{C}(10) = 4.6$, so when 10 shirts are produced, the cost per shirt is $\$4.60$.
\end{solution}
\item  Find and interpret $p(5)$

\begin{solution}
    $p(5) = 20$, so to sell $5$ shirts, set the price at $\$20$ per shirt.
\end{solution}
\item  Find and simplify $R(x)$.

\begin{solution}
    $R(x) = -2x^2+30x$, $0 \leq x \leq 15$
\end{solution}
\item  Find and simplify $P(x)$.

\begin{solution}
    $P(x) = -2x^2+28x-26$, $0 \leq x \leq 15$
\end{solution}
\item  Solve $P(x) = 0$ and interpret.

\begin{solution}
    $P(x) = 0$ when $x = 1$ and $x=13$.  These are the `break even' points, so selling $1$ shirt or $13$ shirts will guarantee the revenue earned exactly recoups the cost of production.
\end{solution}
\end{itemize}

\end{problem}

\begin{problem}
The cost, in dollars, to produce $x$ bottles of $100 \%$ All-Natural Certified Free-Trade Organic Sasquatch Tonic is $C(x) = 10x+100$, $x \geq 0$ and the price-demand function, in dollars per bottle,  is $p(x) = 35 - x$, $0 \leq x \leq 35$.

\begin{itemize}

\item  Find and interpret $C(0)$.

\begin{solution}
    $C(0) = 100$, so the fixed costs are $\$100$.
\end{solution}
\item  Find and interpret $\overline{C}(10)$.

\begin{solution}
    $\overline{C}(10) = 20$, so when 10 bottles of tonic are produced, the cost per bottle is $\$20$.
\end{solution}
\item  Find and interpret $p(5)$

\begin{solution}
    $p(5) = 30$, so to sell $5$ bottles of tonic, set the price at $\$30$ per bottle.
\end{solution}
\item  Find and simplify $R(x)$.

\begin{solution}
    $R(x) = -x^2+35x$, $0 \leq x \leq 35$
\end{solution}
\item  Find and simplify $P(x)$.

\begin{solution}
    $P(x) = -x^2+25x-100$, $0 \leq x \leq 35$
\end{solution}
\item  Solve $P(x) = 0$ and interpret.

\begin{solution}
    $P(x) = 0$ when $x = 5$ and $x=20$.  These are the `break even' points, so selling $5$ bottles of tonic or $20$ bottles of tonic will guarantee the revenue earned exactly recoups the cost of production.
\end{solution}
\end{itemize}
\end{problem}  

\begin{problem}
The cost, in cents, to produce $x$ cups of Mountain Thunder Lemonade at Junior's Lemonade Stand  is $C(x) = 18x + 240$, $x \geq 0$ and the price-demand function, in cents per cup,  is $p(x) = 90-3x$, $0 \leq x \leq 30$.

\begin{itemize}

\item  Find and interpret $C(0)$.

\begin{solution}
    $C(0) = 240$, so the fixed costs are $240$\textcent \,  or $\$2.40$.
\end{solution}
\item  Find and interpret $\overline{C}(10)$.

\begin{solution}
    $\overline{C}(10) = 42$, so when 10 cups of lemonade are made, the cost per cup is $42$\textcent.
\end{solution}
\item  Find and interpret $p(5)$

\begin{solution}
    $p(5) = 75$, so to sell $5$ cups of lemonade, set the price at $75$\textcent \,  per cup.
\end{solution}
\item  Find and simplify $R(x)$.

\begin{solution}
    $R(x) = -3x^2+90x$, $0 \leq x \leq 30$
\end{solution}
\item  Find and simplify $P(x)$.

\begin{solution}
    $P(x) = -3x^2+72x-240$, $0 \leq x \leq 30$
\end{solution}
\item  Solve $P(x) = 0$ and interpret.

\begin{solution}
    $P(x) = 0$ when $x = 4$ and $x=20$.  These are the `break even' points, so selling $4$ cups of lemonade or $20$ cups of lemonade will guarantee the revenue earned exactly recoups the cost of production.
\end{solution}
\end{itemize}
\end{problem} 

\begin{problem}
The daily cost, in dollars, to produce $x$ Sasquatch Berry Pies $C(x) = 3x + 36$, $x \geq 0$ and the price-demand function, in  dollars per pie,  is $p(x) = 12-0.5x$, $0 \leq x \leq 24$.

\begin{itemize}

\item  Find and interpret $C(0)$.

\begin{solution}
    $C(0) = 36$, so the daily fixed costs are $\$36$.
\end{solution}
\item  Find and interpret $\overline{C}(10)$.

\begin{solution}
    $\overline{C}(10) = 6.6$, so when 10 pies are made, the cost per pie is $\$6.60$.
\end{solution}
\item  Find and interpret $p(5)$

\begin{solution}
    $p(5) = 9.5$, so to sell $5$ pies a day, set the price at $\$9.50$  per pie.
\end{solution}
\item  Find and simplify $R(x)$.
\begin{solution}
    $R(x) = -0.5 x^2 + 12x$, $0 \leq x \leq 24$
\end{solution}
\item  Find and simplify $P(x)$.

\begin{solution}
    $P(x) = -0.5 x^2+9x-36$, $0 \leq x \leq 24$
\end{solution}
\item  Solve $P(x) = 0$ and interpret.

\begin{solution}
    $P(x) = 0$ when $x = 6$ and $x=12$.  These are the `break even' points, so selling $6$ pies or $12$ pies a day will guarantee the revenue earned exactly recoups the cost of production.
\end{solution}
\end{itemize}
\end{problem} 

\begin{problem}\label{econexerlast}
The monthly cost, in hundreds of dollars, to produce $x$ custom built electric scooters is $C(x) = 20x + 1000$, $x \geq 0$ and the price-demand function, in  hundreds of dollars per scooter,  is $p(x) = 140-2x$, $0 \leq x \leq 70$.

\begin{itemize}

\item  Find and interpret $C(0)$.

\begin{solution}
    $C(0) = 1000$, so the monthly fixed costs are $1000$ \textit{hundred} dollars, or $\$100,\!000$.
\end{solution}
\item  Find and interpret $\overline{C}(10)$.

\begin{solution}
    $\overline{C}(10) = 120$, so when 10 scooters are made, the cost per scooter is $120$ hundred dollars, or $\$12,\!000$.
\end{solution}
\item  Find and interpret $p(5)$

\begin{solution}
    $p(5) = 130$, so to sell $5$ scooters a month, set the price at $130$ hundred dollars, or $\$13,\!000$ per scooter.
\end{solution}
\item  Find and simplify $R(x)$.

\begin{solution}
    $R(x) = -2x^2+140x$, $0 \leq x \leq 70$
\end{solution}
\item  Find and simplify $P(x)$.

\begin{solution}
    $P(x) = -2x^2+120x-1000$, $0 \leq x \leq 70$
\end{solution}
\item  Solve $P(x) = 0$ and interpret.

\begin{solution}
    $P(x) = 0$ when $x = 10$ and $x=50$.  These are the `break even' points, so selling $10$ scooters or $50$ scooters a month will guarantee the revenue earned exactly recoups the cost of production.
\end{solution}
\end{itemize}
\end{problem} 


\end{question}

\end{document}