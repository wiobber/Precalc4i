\documentclass{ximera}

\begin{document}
	\author{Stitz-Zeager}
	\xmtitle{Exercises for Graphs of Functions}{}

\mfpicnumber{1} \opengraphsfile{ExercisesforGraphsofFunctions} % mfpic settings added 


In Exercises \ref{usefuncgraphfirst} - \ref{usefuncgraphlast}, use the graph of $y = f(x)$ given below to answer the question.


\begin{center}

\begin{tikzpicture}
\begin{axis}[fplot, xmin=-6, xmax=6, ymin=-6, ymax=6]
  \addplot[fgraph] coordinates {(-5,-5) (-4,0) (-3,4) (-2,2) (-1,0)};
  \addplot[fgraph, domain=-1:1] {x^2 - 1};
  \addplot[fgraph] coordinates {(1,0) (2,3) (3,1)};
  \addplot[only marks, mark=*] coordinates {(-5,-5) (-4,0) (-3,4) (-2,2) (-1,0) (0,-1) (1,0) (2,3) (3,1)};
  \node[flabel, label=below left :{$(-5,-5)$}] at (axis cs:-5,-5) {};
  \node[flabel, label=above left :{$(-4,0)$}]  at (axis cs:-4, 0) {};
  \node[flabel, label=above left :{$(-3,4)$}]  at (axis cs:-3, 4) {};
  \node[flabel, label=above left :{$(-2,2)$}]  at (axis cs:-2, 2) {};
  \node[flabel, label=below left :{$(-1,0)$}]  at (axis cs:-1, 0) {};
  \node[flabel, label=below right:{$(0,-1)$}]  at (axis cs: 0,-1) {};
  \node[flabel, label=below right:{$(1,0)$}]   at (axis cs: 1, 0) {};
  \node[flabel, label=above right:{$(2,3)$}]   at (axis cs: 2, 3) {};
  \node[flabel, label=above right:{$(3,1)$}]   at (axis cs: 3, 1) {};
  \node at (rel axis cs:0.5,0) [below] {$y=f(x)$};
\end{axis}
\end{tikzpicture}



\end{center}




% Transformed Exercises with Solutions

\begin{problem}
Find the domain of $f$.
$\answer{[-5,3]}$


\end{problem}

\begin{problem}
Find the range of $f$.
$\answer{[-5,4]}$


\end{problem}

\begin{problem}
Find the maximum, if it exists.
\begin{solution}
$f(-3) = 4$
\end{solution}

\end{problem}

\begin{problem}
Find the minimum, if it exists. 
\begin{solution}
$f(-5) = -5$
\end{solution}

\end{problem}

\begin{problem}
List the local maximums, if any exist.
\begin{solution}
$(-3,4)$,  $(2,3)$
\end{solution}

\end{problem}

\begin{problem}
List the local minimums, if any exist.
\begin{solution}
$(0,-1)$
\end{solution}

\end{problem}

\begin{problem}
List the intervals where $f$ is increasing.
\begin{solution}
$[-5,-3]$, $[0,2]$
\end{solution}

\end{problem}

\begin{problem}
List the intervals where $f$ is decreasing.
\begin{solution}
$[-3,0]$, $[2,3]$
\end{solution}

\end{problem}

\begin{problem}
Determine $f(-2)$.  $f(-2) = \answer{2}$

\end{problem}

\begin{problem}
Solve $f(x) = 4$.
$x=\answer{-3}$

\end{problem}

\begin{problem}
List the $x$-intercepts, if any exist.
\begin{solution}
$(-4,0)$, $(-1,0)$, $(1,0)$
\end{solution}

\end{problem}

\begin{problem}
List the $y$-intercepts, if any exist.
$\answer{(0,-1)}$

\end{problem}

\begin{problem}
Find the zeros of $f$.
\begin{solution}
$-4$, $-1$, $1$
\end{solution}

\end{problem}

\begin{problem}
Solve $f(x) \geq 0$.
\begin{solution}
$[-4,-1]$, $[1,3]$
\end{solution}

\end{problem}

\begin{problem}
Find the number of solutions to $f(x) = 1$.
$\answer{4}$
\end{problem}

\begin{problem}
Find the number of solutions to $|f(x)| = 1$.
$\answer{6}$

\end{problem}

\begin{problem}
Solve $(x^2-x-2)f(x) = 0$
\begin{solution}
$x=-4, -1,1,2$
\end{solution}

\end{problem}

\begin{problem}
Solve  $(x^2-x-2)f(x) > 0$

\begin{solution}
$(-4,-1) \cup (-1,1) \cup (2,3)$ 
\end{solution}

\end{problem}

\begin{problem}
Find the domain of $R(x) = \dfrac{1}{f(x)}$
\begin{solution}
To find the domain of $R(x) = \frac{1}{f(x)}$, we start with the domain of $f$ and exclude values where $f(x) = 0$.  Hence, the domain of $R$ is $[-5,-4) \cup (-4,-1) \cup (-1,1) \cup (1,3]$.
\end{solution}

\end{problem}

\begin{problem}
Find the range of $R(x) = \dfrac{1}{f(x)}$

\begin{solution}
To find the range of $R(x) = \frac{1}{f(x)}$, we start with the range of $f$ (excluding $0$)  and take reciprocals.  If $-5 \leq y < 0$, then $\frac{1}{y} \leq -\frac{1}{5}$.  If $0 < y \leq 4$, then $\frac{1}{y} \geq \frac{1}{4}$. Hence the range of $R$ is $\left(-\infty, -\frac{1}{5} \right] \cup \left[ \frac{1}{4}, \infty \right)$. 

\end{solution}

\end{problem}

In Exercises \ref{usesecondfuncgraphfirst} - \ref{usesecondfuncgraphlast}, use the graph of $y =g(t)$ given below to answer the  question.
\begin{center}
    
\begin{tikzpicture}
  \begin{axis}[
    axis lines=middle,
    xmin=-5, xmax=5,
    ymin=-6, ymax=6,
    xtick={-3,-2,-1,1,2,3,4},
    xticklabels={$-3$,$-2$,$-1$,$1$,$2$,$3$,$4$},
    ytick={-5,-4,-3,-2,-1,1,2,3,4,5},
    yticklabels={$-5$,$-4$,$-3$,$-2$,$-1$,$1$,$2$,$3$,$4$,$5$},
    axis line style={->},
    width=11cm, height=11cm,
    clip=false
  ]
    % Axis labels
    \node at (axis cs:5,-0.5) {\scriptsize $t$};
    \node at (axis cs:0.5,6) {\scriptsize $y$};

    % Labels near points
    \node at (axis cs:-4,0.5)  {\scriptsize $(-4,0)$};
    \node at (axis cs:-2,-5.5) {\scriptsize $(-2,-5)$};
    \node at (axis cs:1,-0.5)  {\scriptsize $(0,0)$};
    \node at (axis cs:2,2.5)   {\scriptsize $(2,3)$};
    \node at (axis cs:2,5.5)   {\scriptsize $(2,5)$};
    \node at (axis cs:4,-0.5)  {\scriptsize $(4,0)$};

    % Function y = 5 sin(pi/4 * x)
    \addplot[thick, domain=-4:4, samples=200] {5*sin(deg(x*pi/4))};

    % Solid points
    \addplot[only marks, mark=*] coordinates {
      (-2,-5) (0,0) (4,0) (2,3) (-4,0)
    };

    % Open circle at (2,5)
    \addplot[only marks, mark=o, mark size=2pt, thick] coordinates {(2,5)};

    % Caption
    \node[below] at (rel axis cs:0.5,0) {$y=g(t)$};
  \end{axis}
\end{tikzpicture}
\end{center}

\begin{problem}\label{usesecondfuncgraphfirst}
Find the domain of $g$.
\begin{solution}
$[-4,4]$
\end{solution}

\end{problem}

\begin{problem}
Find the range of $g$.

\begin{solution}
$[-5,5)$
\end{solution}

\end{problem}

\begin{problem}
Find the maximum, if it exists.
\begin{solution}
none

\end{solution}

\end{problem}

\begin{problem}
Find the minimum, if it exists. 

\begin{solution}
$g(-2) = -5$
\end{solution}

\end{problem}

\begin{problem}
List the local maximums, if any exist.
\begin{solution}
none
\end{solution}

\end{problem}

\begin{problem}
List the local minimums, if any exist.

\begin{solution}
$(-2,-5)$, $(2,3)$

\end{solution}

\end{problem}

\begin{problem}
List the intervals where $g$ is increasing.
\begin{solution}
$[-2,2)$
\end{solution}

\end{problem}

\begin{problem}
List the intervals where $g$ is decreasing.

\begin{solution}
$[-4, -2]$, $(2,4]$
\end{solution}

\end{problem}

\begin{problem}
Determine $g(2)$.
\begin{solution}
$g(2) = 3$

\end{solution}

\end{problem}

\begin{problem}
Solve $g(t) = -5$.

\begin{solution}
$t=-2$
\end{solution}

\end{problem}

\begin{problem}
List the $t$-intercepts, if any exist.
\begin{solution}
$(-4,0)$, $(0,0)$, $(4,0)$
\end{solution}

\end{problem}

\begin{problem}
List the $y$-intercepts, if any exist.

\begin{solution}
$(0,0)$

\end{solution}

\end{problem}

\begin{problem}
Find the zeros of $g$.
\begin{solution}
$-4$, $0$, $4$
\end{solution}

\end{problem}

\begin{problem}
Solve $g(t) \leq 0$.

\begin{solution}
$[-4,0] \cup \{4\}$
\end{solution}

\end{problem}

\begin{problem}
Find the domain of $G(t) = \dfrac{g(t)}{t+2}$.
\begin{solution}
$[-4,-2) \cup (-2.4]$

\end{solution}

\end{problem}

\begin{problem}
Solve $\dfrac{g(t)}{t+2} \leq 0$.

\begin{solution}
$\{-4\} \cup (-2,0] \cup \{4\}$
\end{solution}

\end{problem}

\begin{problem}
How many solutions are there to $[g(t)]^2 = 9$?
\begin{solution}
$5$
\end{solution}

\end{problem}

\begin{problem}\label{usesecondfuncgraphlast}
Does $g$ appear to be even, odd, or neither?

\begin{solution}
Neither.

\end{solution}

\end{problem}

\begin{problem}
Prove that if $f$ is an odd function and $0$ is in the domain of $f$, then $f(0) = 0$.
\begin{solution}
\end{solution}

\end{problem}

\begin{problem}
Let $R(x)$ be the function defined as:  $R(x) = 1$ if $x$ is a rational number, $R(x) = 0$ if $x$ is an irrational number. With help from your classmates, try to graph $R$.  What difficulties do you encounter?

NOTE:  Between every pair of real numbers, there is both a rational and an irrational number \ldots


\begin{solution}
Local maximum: $(0,1)$, no local minimum.  Increasing: $(0,2]$, decreasing: $[-2,0)$.
\end{solution}

\end{problem}

\begin{problem}
Consider the graph of the function $f$ given below.  

\begin{center}

% \input{ExercisesforGraphsofFunctions_pic3.tex}
\begin{tikzpicture}
\begin{axis}[fplot, xmin=-3, xmax=3, ymin=-3.5, ymax=4]
  \addplot[fgraph, domain=-3:-1] {2*x + 3};
  \addplot[fgraph] coordinates {(-1,1) (1,1)};
  \addplot[fgraph, domain=1:3] {x};
  \addplot[only marks, mark=*] coordinates {(-1,1) (0,1) (1,1)};
  \node[flabel, label=above left :{$(-1,1)$}] at (axis cs:-1,1) {};
  \node[flabel, label=above right:{$(1,1)$}]  at (axis cs: 1,1) {};
\end{axis}
\end{tikzpicture}


\end{center}

\begin{solution}
No local maximum,  local minimum: $(0,1)$.  Increasing: $[-2,0)$, decreasing: $(0,2]$.
\end{solution}

\end{problem}

\begin{problem}
Explain why $f$ has a local maximum but not a local minimum at the point $(-1, 1)$.
\begin{solution}
No local maximum,  local minimum: $(0,-1)$.  Increasing: $[0,2]$, decreasing: $[-2,0]$.
\end{solution}

\end{problem}

\begin{problem}
Explain why  $f$ has a local minimum but not a local maximum at the point $(1, 1)$.
\begin{solution}
Local maximum: $(0,5)$, no local minimum.  Increasing: $[-2,0]$, decreasing: $[0,2]$.
\end{solution}

\end{problem}

\begin{problem}
Explain why $f$ has a local maximum AND a local minimum at the point $(0, 1)$.
\end{problem}

\begin{problem}
Explain why $f$ is constant on the interval $[-1, 1]$ and thus has both a local maximum AND a local minimum at every point $(x, f(x))$ where $-1 < x < 1$.
\end{problem}

\begin{problem}
For each function below, find the local maximum or local minimum and list the interval over which the function is increasing and the interval over which the function is decreasing. 


\end{problem}

\begin{problem}
% \input{ExercisesforGraphsofFunctions_pic5.tex}
\begin{tikzpicture}
\begin{axis}[fplot, xmin=-3, xmax=3, ymin=-2, ymax=5]
  \addplot[fgraph, domain=-2:2] {x^2};
  \addplot[only marks, mark=*] coordinates {(-2,4) (0,1) (2,4)};
  \addplot[only marks, mark=o] coordinates {(0,0)};
  \node at (rel axis cs:0.5,0) [below] {Function I};
\end{axis}
\end{tikzpicture}
\end{problem}

\begin{problem}
% \input{ExercisesforGraphsofFunctions_pic6.tex}
\begin{tikzpicture}
\begin{axis}[fplot, xmin=-3, xmax=3, ymin=-2, ymax=5]
  \addplot[fgraph, domain=-2:2] {4 - x^2};
  \addplot[only marks, mark=*] coordinates {(-2,0) (0,1) (2,0)};
  \addplot[only marks, mark=o] coordinates {(0,4)};
  \node at (rel axis cs:0.5,0) [below] {Function II};
\end{axis}
\end{tikzpicture}


\end{problem}

\begin{problem}
% \input{ExercisesforGraphsofFunctions_pic7.tex}
\begin{tikzpicture}
\begin{axis}[fplot, xmin=-3, xmax=3, ymin=-2, ymax=5]
  \addplot[fgraph, domain=-2:2] {x^2};
  \addplot[only marks, mark=*] coordinates {(-2,4) (0,-1) (2,4)};
  \addplot[only marks, mark=o] coordinates {(0,0)};
  \node at (rel axis cs:0.5,0) [below] {Function III};
\end{axis}
\end{tikzpicture}
\end{problem}

\begin{problem}
% \input{ExercisesforGraphsofFunctions_pic8.tex}
\begin{tikzpicture}
\begin{axis}[fplot, xmin=-3, xmax=3, ymin=-1, ymax=6]
  \addplot[fgraph, domain=-2:2] {4 - x^2};
  \addplot[only marks, mark=*] coordinates {(-2,0) (0,5) (2,0)};
  \addplot[only marks, mark=o] coordinates {(0,4)};
  \node at (rel axis cs:0.5,0) [below] {Function IV};
\end{axis}
\end{tikzpicture}
\end{problem}

\end{document}