\documentclass{ximera}

\begin{document}
	\author{Stitz-Zeager}
	\xmtitle{The Algebra of Difference Quotients}
    
\section{The Algebra of Difference Quotients}


\begin{definition}\label{differencequotient}

Given a function $f$, the \index{function ! difference quotient}\index{difference quotient}\textbf{difference quotient} of $f$ is the expression \[ \dfrac{f(x+h) - f(x)}{h} \]

\end{definition}


We will revisit this concept in Chapter \ref{IntroductiontoDerivatives}, but for now, we use it as a way to practice function notation and function arithmetic.  For reasons which will become clear in Calculus,  `simplifying' a difference quotient means rewriting it in a form where the `$h$' in the definition of the difference quotient cancels from the denominator. Once that happens, we consider our work to be done.

\begin{example} \label{differencequotientex} Find and simplify the difference quotients for the following functions

\begin{multicols}{3}

\begin{enumerate}

\item  $f(x) = x^2-x-2$ \vphantom{$\dfrac{3}{2x+1}$}

\item  $g(x) = \dfrac{3}{2x+1}$

\item  $r(x) = \sqrt{x}$ \vphantom{$\dfrac{3}{2x+1}$}

\end{enumerate}

\end{multicols}

{\bf Solution.}
 
\begin{enumerate}

\item To find $f(x+h)$, we replace every occurrence of $x$ in the formula $f(x) = x^2-x-2$ with the quantity $(x+h)$ to get \[ \begin{array}{rclr}  
 
 f(x+h) & = & (x+h)^2 - (x+h) -2 & \\ [8pt]
 & = & x^2 + 2xh + h^2 - x - h - 2.
 \end{array} \]

So the difference quotient is

\setlength{\extrarowheight}{12pt}

\begin{longtable}{rclr}  

$\dfrac{f(x+h)-f(x)}{h}$ & = & $\dfrac{\left(x^2+2xh+h^2-x-h-2 \right)-\left(x^{2}-x-2 \right)}{h}$ & \\[8pt] 
& = & $\dfrac{x^2+2xh+h^2-x-h-2-x^2+x+2}{h}$ & \\ [8pt]
& = & $\dfrac{2xh+h^2-h}{h}$ & \\ [8pt]
& = & $\dfrac{h \left(2x+h-1\right)}{h}$ & factor \\ [8pt]
& = & $\dfrac{\cancel{h} \left(2x+h-1\right)}{\cancel{h}}$ & cancel \\ [8pt]
& = & $2x+h-1$. \\

\end{longtable} 

\pagebreak

\item To find $g(x+h)$, we replace every occurrence of $x$ in the formula $g(x) = \frac{3}{2x+1}$ with the quantity $(x+h)$ to get 

 \[ \begin{array}{rclr}  
 g(x+h) & = & \dfrac{3}{2(x+h)+1} & \\
 & = & \dfrac{3}{2x+2h+1}, 
 \end{array} \]

which yields 

\begin{longtable}{rclr}  

$\dfrac{g(x+h)-g(x)}{h}$ & = & $\dfrac{\dfrac{3}{2x+2h+1}-\dfrac{3}{2x+1}}{h}$ & \\ [3pt]
& = &  $\dfrac{\dfrac{3}{2x+2h+1}-\dfrac{3}{2x+1}}{h} \cdot \dfrac{(2x+2h+1)(2x+1)}{(2x+2h+1)(2x+1)}$ & \\ [3pt]
& = &  $\dfrac{3(2x+1)-3(2x+2h+1)}{h(2x+2h+1)(2x+1)}$  & \\ [3pt]
& = &  $\dfrac{6x+3-6x-6h-3}{h(2x+2h+1)(2x+1)}$  & \\ [3pt]
& = &  $\dfrac{-6h}{h(2x+2h+1)(2x+1)}$  & \\ [3pt]
& = &  $\dfrac{-6\cancel{h}}{\cancel{h}(2x+2h+1)(2x+1)}$  & \\ [3pt]
& = &  $\dfrac{-6}{(2x+2h+1)(2x+1)}$.  & \\ 

\end{longtable}

Since we have managed to cancel the original `$h$' from the denominator, we are done.

\item  For $r(x) = \sqrt{x}$, we get $r(x+h) = \sqrt{x+h}$ so the difference quotient is 

\[ \dfrac{r(x+h) - r(x)}{h} = \dfrac{\sqrt{x+h} - \sqrt{x}}{h} \]

In order to cancel the `$h$' from the denominator, we rationalize the \textit{numerator} by multiplying by its conjugate.\footnote{Rationalizing the \textit{numerator}!?  How's that for a twist!}

\[ \begin{array}{rcll}  

\dfrac{r(x+h) - r(x)}{h} & = & \dfrac{\sqrt{x+h} - \sqrt{x}}{h} & \\ [15pt]
												
												 & = & \dfrac{\left(\sqrt{x+h} - \sqrt{x}\right)}{h} \cdot \dfrac{\left(\sqrt{x+h} + \sqrt{x}\right)}{\left(\sqrt{x+h} + \sqrt{x}\right)} & \text{Multiply by the conjugate.} \\ [15pt]
										
												 & = &  \dfrac{\left(\sqrt{x+h}\right)^2 - \left(\sqrt{x}\right)^2}{h\left(\sqrt{x+h} + \sqrt{x}\right)} & \text{Difference of Squares.}\\ [15pt]
												
												& = &  \dfrac{(x+h) - x}{h\left(\sqrt{x+h} + \sqrt{x}\right)} & \\ [15pt]
												
												 & = &  \dfrac{h}{h\left(\sqrt{x+h} + \sqrt{x}\right)} & \\ [15pt]
												 
												 												 & = &  \dfrac{\cancelto{1}{h}}{\cancel{h}\left(\sqrt{x+h} + \sqrt{x}\right)} & \\ [15pt]
										
										
												 & = &  \dfrac{1}{\sqrt{x+h} + \sqrt{x}} & \\ 
												\end{array}\]	


Since we have removed the original `$h$' from the denominator, we are done.  \qed

\end{enumerate}

\end{example}

\phantomsection
\label{diffquotgeompromise}

As mentioned before, we will revisit difference quotients in Section \ref{LinearFunctions} where we will explain them geometrically.  For now, we want to move on to some classic applications of function arithmetic from Economics and for that, we need to think like an entrepreneur.\footnote{Not really, but ``entrepreneur'' is the buzzword of the day and we're trying to be trendy.}

\medskip

Suppose you are a manufacturer making a certain product.\footnote{Poorly designed resin Sasquatch statues, for example.  Feel free to choose your own entrepreneurial fantasy.}  Let $x$ be the \textbf{production level}, that is, the number of items produced in a given time period. It is customary to let $C(x)$ denote the function which calculates the total \textbf{cost} of producing the $x$ items.  The quantity $C(0)$, which represents the cost of producing no items, is called the \textbf{fixed} cost, and represents the amount of money required to begin production. Associated with the total cost $C(x)$ is cost per item, or \textbf{average cost}, denoted $\overline{C}(x)$ and read `$C$-bar' of $x$.  To compute $\overline{C}(x)$,  we take the total cost $C(x)$ and divide by the number of items produced $x$ to get \[ \overline{C}(x) = \dfrac{C(x)}{x}\] On the retail end, we have the \textbf{price} $p$ charged per item.  To simplify the dialog and computations in this text, we assume that \textit{the number of items sold equals the number of items produced}. From a retail perspective, it seems natural to think of the number of items sold, $x$, as a function of the price charged, $p$.  After all, the retailer can easily adjust the price to sell more product.  In the language of functions,  $x$ would be the \textit{dependent} variable and $p$ would be the \textit{independent} variable or, using function notation, we have a function $x(p)$.  While we will adopt this convention later in the text,\footnote{See Example \ref{demandfunctionofprice} in Section \ref{InverseFunctions}.} we will hold with tradition at this point and consider the price $p$ as a function of the number of items sold, $x$.  That is, we regard $x$ as the independent variable and $p$ as the dependent variable and speak of the \textbf{price-demand} function, $p(x)$.  Hence, $p(x)$ returns the price charged per item when $x$ items are produced and sold.   Our next function to consider is the \textbf{revenue} function, $R(x)$.  The function $R(x)$ computes the amount of money collected as a result of selling $x$ items.  Since $p(x)$ is the price charged per item, we have $R(x)= x p(x)$.  Finally, the \textbf{profit} function, $P(x)$ calculates how much money is earned after the costs are paid.  That is, $P(x) = (R-C)(x) = R(x) - C(x)$.  We summarize all of these functions below.


\label{pricerevenuecostprofit}

\centerline{\textbf{Summary of Common Economic Functions}} 

 Suppose $x$ represents the quantity of items produced and sold.

\begin{itemize}

\item  The price-demand function $p(x)$ calculates the price per item.\index{price-demand function}\index{function ! price-demand}

\item  The revenue function $R(x)$ calculates the total money collected by selling $x$ items at a price $p(x)$,  $R(x) = x \, p(x)$.\index{revenue function}\index{function ! revenue}

\item  The cost function $C(x)$  calculates the cost to produce $x$ items.  The value $C(0)$ is called the fixed cost or start-up cost.\index{cost function}\index{function ! cost}\index{cost ! fixed, start-up}\index{fixed cost}\index{start-up cost}

\item  The average cost function $\overline{C}(x) = \frac{C(x)}{x}$ calculates the cost per item when making $x$ items.  Here, we necessarily assume $x > 0$.\index{average cost function}\index{cost ! average}\index{function ! average cost}

\item  The profit function $P(x)$ calculates the money earned after costs are paid when $x$ items are produced and sold, $P(x) = (R-C)(x) = R(x) - C(x)$.\index{profit function}\index{function ! profit}

\end{itemize}


It is high time for an example.

%\vspace{-.1in}

\begin{example} \label{costrevenueprofitex1}  Let $x$ represent the number of dOpi media players (`dOpis'\footnote{Pronounced `dopeys' \ldots}) produced and sold in a typical week.  Suppose the cost, in dollars, to produce $x$ dOpis  is given by   $C(x) = 100x + 2000$, for $x \geq 0$, and the price, in dollars per dOpi, is given by $p(x) = 450-15x$ for $0 \leq x \leq 30$.

\begin{multicols}{2}
\begin{enumerate}

\item  Find and interpret $C(0)$.
\item  Find and interpret $\overline{C}(10)$.

\setcounter{HW}{\value{enumi}}
\end{enumerate}
\end{multicols}
\vspace{-.2in}
\begin{multicols}{2}
\begin{enumerate}
\setcounter{enumi}{\value{HW}}

\item  Find and interpret $p(0)$ and $p(20)$.
\item  Solve $p(x) = 0$ and interpret the result.

\setcounter{HW}{\value{enumi}}
\end{enumerate}
\end{multicols}
\vspace{-.2in}
\begin{enumerate}
\setcounter{enumi}{\value{HW}}

\item  Find and simplify expressions for the revenue function $R(x)$ and the profit function $P(x)$.

\setcounter{HW}{\value{enumi}}
\end{enumerate}
\enlargethispage{.3in} \vspace{-.2in}
\begin{multicols}{2}
\begin{enumerate}
\setcounter{enumi}{\value{HW}}

\item  Find and interpret $R(0)$ and $P(0)$.
\item  Solve $P(x) = 0$ and interpret the result.

\end{enumerate}
\end{multicols}

\vspace{-.2in}

\pagebreak

{\bf Solution.}

\begin{enumerate}

\item \label{fixedcostex} We substitute $x=0$ into the formula for $C(x)$ and get $C(0) = 100(0) + 2000 = 2000$.  This means to produce $0$ dOpis, it costs $\$2000$.  In other words, the fixed (or start-up) costs are $\$2000$. The reader is encouraged to contemplate what sorts of expenses these might be.

\item  Since $\overline{C}(x) = \frac{C(x)}{x}$, $\overline{C}(10) = \frac{C(10)}{10} = \frac{3000}{10} = 300$.  This means when $10$ dOpis are produced, the cost to manufacture them amounts to $\$ 300$ per dOpi.

\item Plugging $x=0$ into the expression for $p(x)$ gives $p(0) = 450 - 15(0) = 450$.  This means no dOpis are sold if the price is  $\$450$ per dOpi.  On the other hand, $p(20) = 450-15(20) = 150$ which means to sell $20$ dOpis in a typical week, the price should be set at $\$150$ per dOpi.

\item  Setting $p(x) = 0$ gives $450-15x = 0$.  Solving gives $x = 30$.  This means in order to sell $30$ dOpis in a typical week, the price needs to be set to $\$ 0$. What's more, this means that even if dOpis were given away for free, the retailer would only be able to move $30$ of them.\footnote{Imagine that!  Giving something away for free and hardly anyone taking advantage of it \ldots}

\item  To find the revenue, we compute $R(x) = x p(x) = x (450 - 15x) = 450x - 15x^2$.  Since the formula for $p(x)$ is valid only for $0 \leq x \leq 30$, our formula $R(x)$ is also restricted to $0 \leq x \leq 30$.  For the profit, $P(x) = (R-C)(x) = R(x) - C(x)$.  Using the given formula for $C(x)$ and the derived formula for $R(x)$, we get $P(x) = \left(450x - 15x^2\right) -(100x+2000) = -15x^2+350x-2000$. As before, the validity of this formula is for $0 \leq x \leq 30$ only.

\item  We find $R(0) = 0$ which means if no dOpis are sold, we have no revenue, which makes sense.  Turning to profit, $P(0) = -2000$ since $P(x) = R(x) - C(x)$ and $P(0) = R(0) - C(0) = -2000$.  This means that if no dOpis are sold, more money ($\$2000$ to be exact!) was put into producing the dOpis than was recouped in sales. In number \ref{fixedcostex}, we found the fixed costs to be $\$2000$, so it makes sense that if we sell no dOpis, we are out those start-up costs. 

\item Setting $P(x) = 0$ gives $-15x^2+350x-2000 = 0$.  Factoring gives $-5(x-10)(3x-40) = 0$ so $x = 10$ or $x = \frac{40}{3}$.  What do these values mean in the context of the problem?  Since $P(x) = R(x) - C(x)$, solving $P(x) = 0$ is the same as solving $R(x) = C(x)$. This means that the solutions to $P(x) = 0$ are the production (and sales) figures for which the sales revenue exactly balances the total production costs.  These are the so-called `\textbf{break even}' points.  The solution $x=10$ means $10$ dOpis should be produced (and sold) during the week to recoup the cost of production. For $x = \frac{40}{3} = 13.\overline{3}$, things are a bit more complicated.  Even though $x = 13.\overline{3}$ satisfies $0 \leq x \leq 30$, and hence is in the domain of $P$, it doesn't make sense in the context of this problem to produce a fractional part of a dOpi.\footnote{We've seen this sort of thing before in Section \ref{modeling}.}  Evaluating $P(13) = 15$ and $P(14) = -40$, we see that producing and selling $13$ dOpis per week makes a (slight) profit, whereas producing just one more puts us back into the red. While breaking even is nice, we ultimately would like to find what production level (and price) will result in the largest profit, and we'll do just that \ldots in Section \ref{QuadraticFunctions}. \qed


\end{enumerate}

\end{example}

\end{document}
