\documentclass{ximera}

\begin{document}
	\author{Stitz-Zeager}
	\xmtitle{Exercises for Transformations}{}

\mfpicnumber{1} \opengraphsfile{ExercisesforTransformations} % mfpic settings added 


\label{ExercisesforTransformations}

Suppose $(2,-3)$ is on the graph of $y = f(x)$.  In Exercises \ref{transformpointfirst} - \ref{transformpointlast}, use Theorem \ref{transformationsthm} to find a point on the graph of the given transformed function.  



% Transformed Exercises with Solutions

\begin{question}
$y = f(x)+3$
\begin{solution}
$(2,0)$
\end{solution}

\end{question}

\begin{question}
$y = f(x+3)$
\begin{solution}
$(-1,-3)$
\end{solution}

\end{question}

\begin{question}
$y = f(x)-1$

\begin{solution}
$(2,-4)$

\end{solution}

\end{question}

\begin{question}
$y = f(x-1)$
\begin{solution}
$(3,-3)$
\end{solution}

\end{question}

\begin{question}
$y = 3f(x)$
\begin{solution}
$(2,-9)$
\end{solution}

\end{question}

\begin{question}
$y = f(3x)$

\begin{solution}
$\left(\frac{2}{3}, -3\right)$

\end{solution}

\end{question}

\begin{question}
$y = -f(x)$
\begin{solution}
$(2,3)$
\end{solution}

\end{question}

\begin{question}
$y = f(-x)$
\begin{solution}
$(-2,-3)$
\end{solution}

\end{question}

\begin{question}
$y = f(x-3)+1$

\begin{solution}
$(5,-2)$

\end{solution}

\end{question}

\begin{question}
$y = 2f(x+1)$
\begin{solution}
$(1,-6)$
\end{solution}

\end{question}

\begin{question}
$y = 10 - f(x)$
\begin{solution}
$(2,13)$
\end{solution}

\end{question}

\begin{question}
$y = 3f(2x) - 1$

\begin{solution}
$y = (1,-10)$

\end{solution}

\end{question}

\begin{question}
$y = \frac{1}{2} f(4-x)$
\begin{solution}
$\left(2, -\frac{3}{2}\right)$
\end{solution}

\end{question}

\begin{question}
$y = 5f(2x+1) + 3$
\begin{solution}
$\left(\frac{1}{2}, -12 \right)$
\end{solution}

\end{question}

\begin{question}
$y = 2f(1-x) -1$

\begin{solution}
$(-1,-7)$

\end{solution}

\end{question}

\begin{question}
$y =f\left(\dfrac{7-2x}{4}\right)$
\begin{solution}
$\left(-\frac{1}{2}, -3\right)$
\end{solution}

\end{question}

\begin{question}
$y = \dfrac{f(3x) - 1}{2}$
\begin{solution}
$\left(\frac{2}{3}, -2 \right)$
\end{solution}

\end{question}

\begin{question}
$y = \dfrac{4-f(3x-1)}{7}$ 

\begin{solution}
$(1,1)$

\end{solution}

\end{question}

\begin{question}
$y = f(x) + 1$
\begin{solution}
$y = f(x) + 1$

% \input{ExercisesforTransformations_pic15.tex}
\begin{mfpic}[15]{-5}{5}{-1}{5}
\axes
\tlabel[cc](5,-0.25){\scriptsize $x$}
\tlabel[cc](0.25,5){\scriptsize $y$}
\tlabel[cc](-2.5,2.25){\scriptsize $(-2,3)$}
\tlabel[cc](0.75,0.5){\scriptsize $(0,1)$}
\tlabel[cc](2.25,2.25){\scriptsize $(2,3)$}
\xmarks{-4,-3,-2,-1,1,2,3,4}
\ymarks{1,2,3,4}
\tlpointsep{5pt}
\scriptsize
\axislabels {x}{{$-4 \hspace{7pt}$} -4,{$-3 \hspace{7pt}$} -3, {$-2 \hspace{7pt}$} -2, {$-1 \hspace{7pt}$} -1, {$1$} 1, {$2$} 2,{$3$} 3,{$4$} 4}
\axislabels {y}{{$1$} 1, {$2$} 2, {$3$} 3, {$4$} 4}
\normalsize
\penwd{1.25pt}
\arrow \reverse \arrow \polyline{(-4,5), (0,1), (4,5)}
\point[4pt]{(-2,3), (0,1), (2,3)}
\end{mfpic}
 

\vfill
\end{solution}

\end{question}

\begin{question}
$y = f(x) - 2$
\begin{solution}
$y = f(x) - 2$

% \input{ExercisesforTransformations_pic16.tex}
\begin{mfpic}[15]{-5}{5}{-3}{3}
\axes
\tlabel[cc](5,-0.25){\scriptsize $x$}
\tlabel[cc](0.25,3){\scriptsize $y$}
\tlabel[cc](-2.5,-0.75){\scriptsize $(-2,2)$}
\tlabel[cc](0.75,-2.5){\scriptsize $(0,-2)$}
\tlabel[cc](2.25,-0.75){\scriptsize $(2,2)$}
\xmarks{-4,-3,-2,-1,1,2,3,4}
\ymarks{-2,-1,1,2}
\tlpointsep{5pt}
\scriptsize
\axislabels {x}{{$-4 \hspace{7pt}$} -4, {$-1 \hspace{7pt}$} -1, {$1$} 1,{$4$} 4}
\axislabels {y}{{$-2$} -2, {$-1$} -1, {$1$} 1, {$2$} 2}
\normalsize
\penwd{1.25pt}
\arrow \reverse \arrow \polyline{(-4,2), (0,-2), (4,2)}
\point[4pt]{(-2,0), (0,-2), (2,0)}
\end{mfpic}
 

\end{solution}

\end{question}

\begin{question}
$y = f(x+1)$

\begin{solution}
$y = f(x+1)$

% \input{ExercisesforTransformations_pic17.tex}
\begin{mfpic}[15]{-6}{4}{-1}{5}
\axes
\tlabel[cc](4,-0.25){\scriptsize $x$}
\tlabel[cc](0.25,5){\scriptsize $y$}
\tlabel[cc](-3.5,1.25){\scriptsize $(-3,2)$}
\tlabel[cc](-1,-0.5){\scriptsize $(-1,0)$}
\tlabel[cc](1.25,1.25){\scriptsize $(1,2)$}
\xmarks{-5,-4,-3,-2,-1,1,2,3}
\ymarks{1,2,3,4}
\tlpointsep{5pt}
\scriptsize
\axislabels {x}{{$-5 \hspace{7pt}$} -5,{$-4 \hspace{7pt}$} -4,{$-3 \hspace{7pt}$} -3, {$1$} 1, {$2$} 2,{$3$} 3}
\axislabels {y}{{$1$} 1, {$2$} 2, {$3$} 3, {$4$} 4}
\normalsize
\penwd{1.25pt}
\arrow \reverse \arrow \polyline{(-5,4), (-1,0), (3,4)}
\point[4pt]{(-3,2), (-1,0), (1,2)}
\end{mfpic}
 

\vfill
\end{solution}

\end{question}

\begin{question}
$y = f(x - 2)$
\begin{solution}
$y = f(x - 2)$

% \input{ExercisesforTransformations_pic18.tex}
\begin{mfpic}[15]{-3}{7}{-1}{5}
\axes
\tlabel[cc](7,-0.25){\scriptsize $x$}
\tlabel[cc](0.25,5){\scriptsize $y$}
\tlabel[cc](0.75,2){\scriptsize $(0,2)$}
\tlabel[cc](2,-0.5){\scriptsize $(2,0)$}
\tlabel[cc](3,2){\scriptsize $(4,2)$}
\xmarks{-2,-1,1,2,3,4,5,6}
\ymarks{1,2,3,4}
\tlpointsep{5pt}
\scriptsize
\axislabels {x}{{$-2 \hspace{7pt}$} -2, {$-1 \hspace{7pt}$} -1, {$3$} 3,{$4$} 4,{$5$} 5,{$6$} 6}
\axislabels {y}{{$1$} 1, {$2$} 2,  {$4$} 4}
\normalsize
\penwd{1.25pt}
\arrow \reverse \arrow \polyline{(-2,4), (2,0), (6,4)}
\point[4pt]{(0,2), (2,0), (4,2)}
\end{mfpic}
 

\end{solution}

\end{question}

\begin{question}
$y = 2f(x)$
\begin{solution}
$y = 2f(x)$

% \input{ExercisesforTransformations_pic19.tex}
\begin{mfpic}[15]{-5}{5}{-1}{5}
\axes
\tlabel[cc](5,-0.25){\scriptsize $x$}
\tlabel[cc](0.25,5){\scriptsize $y$}
\tlabel[cc](-2.75,3.5){\scriptsize $(-2,4)$}
\tlabel[cc](0.75,-0.5){\scriptsize $(0,0)$}
\tlabel[cc](2.5,3.5){\scriptsize $(2,4)$}
\xmarks{-4,-3,-2,-1,2,3,4}
\ymarks{1,2,3,4}
\tlpointsep{5pt}
\scriptsize
\axislabels {x}{{$-4 \hspace{7pt}$} -4,{$-3 \hspace{7pt}$} -3, {$-2 \hspace{7pt}$} -2, {$-1 \hspace{7pt}$} -1, {$2$} 2,{$3$} 3,{$4$} 4}
\axislabels {y}{{$1$} 1, {$2$} 2, {$3$} 3, {$4$} 4}
\normalsize
\penwd{1.25pt}
\arrow \reverse \arrow \polyline{(-2.5,5), (0,0), (2.5,5)}
\point[4pt]{(-2,4), (0,0), (2,4)}
\end{mfpic}
 


\vfill
\end{solution}

\end{question}

\begin{question}
$y = f(2x)$

\begin{solution}
$y = f(2x)$

% \input{ExercisesforTransformations_pic20.tex}
\begin{mfpic}[15]{-5}{5}{-1}{5}
\axes
\penwd{1.25pt}
\arrow \reverse \arrow \polyline{(-2.5,5), (0,0), (2.5,5)}
\point[4pt]{(-1,2), (0,0), (1,2)}
\tlabel[cc](5,-0.25){\scriptsize $x$}
\tlabel[cc](0.25,5){\scriptsize $y$}
\tlabel[cc](-1.75,1.5){\scriptsize $(-1,2)$}
\tlabel[cc](0.75,-0.5){\scriptsize $(0,0)$}
\tlabel[cc](1.5,1.5){\scriptsize $(1,2)$}
\xmarks{-4,-3,-2,-1,2,3,4}
\ymarks{1,2,3,4}
\tlpointsep{5pt}
\scriptsize
\axislabels {x}{{$-4 \hspace{7pt}$} -4,{$-3 \hspace{7pt}$} -3, {$-2 \hspace{7pt}$} -2, {$-1 \hspace{7pt}$} -1, {$2$} 2,{$3$} 3,{$4$} 4}
\axislabels {y}{{$1$} 1, {$2$} 2, {$3$} 3, {$4$} 4}
\normalsize
\end{mfpic}
 


\end{solution}

\end{question}

\begin{question}
$y = 2 - f(x)$
\begin{solution}
$y = 2 - f(x)$

% \input{ExercisesforTransformations_pic21.tex}
\begin{mfpic}[15]{-5}{5}{-2}{3}
\axes
\tlabel[cc](5,-0.25){\scriptsize $x$}
\tlabel[cc](0.25,3){\scriptsize $y$}
\tlabel[cc](-1.25,-0.5){\scriptsize $(-2,0)$}
\tlabel[cc](0.75,2){\scriptsize $(0,2)$}
\tlabel[cc](1.5,-0.5){\scriptsize $(2,0)$}
\xmarks{-4,-3,-2,-1,2,3,4}
\ymarks{-1,1,2}
\tlpointsep{5pt}
\scriptsize
\axislabels {x}{{$-4 \hspace{7pt}$} -4,{$-3 \hspace{7pt}$} -3,  {$3$} 3,{$4$} 4}
\axislabels {y}{{$1$} 1, {$2$} 2}
\normalsize
\penwd{1.25pt}
\arrow \reverse \arrow \polyline{(-4,-2), (0,2), (4,-2)}
\point[4pt]{(-2,0), (0,2), (2,0)}
\end{mfpic}
 

\vfill
\end{solution}

\end{question}

\begin{question}
$y = f(2-x)$
\begin{solution}
$y = f(2-x)$

% \input{ExercisesforTransformations_pic22.tex}
\begin{mfpic}[15]{-3}{7}{-1}{5}
\axes
\tlabel[cc](7,-0.25){\scriptsize $x$}
\tlabel[cc](0.25,5){\scriptsize $y$}
\tlabel[cc](0.75,2){\scriptsize $(0,2)$}
\tlabel[cc](2,-0.5){\scriptsize $(2,0)$}
\tlabel[cc](3,2){\scriptsize $(4,2)$}
\xmarks{-2,-1,1,2,3,4,5,6}
\ymarks{1,2,3,4}
\tlpointsep{5pt}
\scriptsize
\axislabels {x}{{$-2 \hspace{7pt}$} -2, {$-1 \hspace{7pt}$} -1, {$3$} 3,{$4$} 4,{$5$} 5,{$6$} 6}
\axislabels {y}{{$1$} 1, {$2$} 2, {$4$} 4}
\normalsize
\penwd{1.25pt}
\arrow \reverse \arrow \polyline{(-2,4), (2,0), (6,4)}
\point[4pt]{(0,2), (2,0), (4,2)}
\end{mfpic}
 

\end{solution}

\end{question}

\begin{question}
$y = 2-f(2-x)$ 

\begin{solution}
$y = 2-f(2-x)$

% \input{ExercisesforTransformations_pic23.tex}
\begin{mfpic}[15]{-3}{7}{-1}{5}
\axes
\tlabel[cc](7,-0.25){\scriptsize $x$}
\tlabel[cc](0.25,5){\scriptsize $y$}
\tlabel[cc](0.75,-0.5){\scriptsize $(0,0)$}
\tlabel[cc](2,2.5){\scriptsize $(2,2)$}
\tlabel[cc](3.5,-0.5){\scriptsize $(4,0)$}
\xmarks{-2,-1,1,2,3,4,5,6}
\ymarks{1,2,3,4}
\tlpointsep{5pt}
\scriptsize
\axislabels {x}{{$-2 \hspace{7pt}$} -2, {$-1 \hspace{7pt}$} -1, {$2$} 2,{$5$} 5, {$6$} 6}
\axislabels {y}{{$1$} 1, {$2$} 2, {$3$} 3, {$4$} 4}
\normalsize
\penwd{1.25pt}
\arrow \reverse \arrow \polyline{(-2,-2), (2,2), (6,-2)}
\point[4pt]{(0,0), (2,2), (4,0)}
\end{mfpic}
 

\vfill



\addtocounter{enumi}{2}
\end{solution}

\end{question}

\begin{question}
Some of the answers to Exercises \ref{transformgraphfirst} - \ref{transformgraphlast} above should be the same.  Which ones match up?  What properties of the graph of $y=f(x)$ contribute to the duplication?
\begin{solution}
$y = g(t) - 1$

% \input{ExercisesforTransformations_pic24.tex}
\begin{mfpic}[15]{-5}{5}{-5}{5}
\axes
\tlabel[cc](5,-0.25){\scriptsize $t$}
\tlabel[cc](0.25,5){\scriptsize $y$}
\tlabel[cc](-2.25,-1.5){\scriptsize $(-2,-1)$}
\tlabel[cc](1,3){\scriptsize $(0,3)$}
\tlabel[cc](1,-1.5){\scriptsize $(2,-1)$}
\tlabel[cc](4,-3.5){\scriptsize $(4,-3)$}
\xmarks{-4,-3,-2,-1,1,2,3,4}
\ymarks{-4,-3,-2,-1,1,2,3,4}
\tlpointsep{5pt}
\scriptsize
\axislabels {x}{{$-4 \hspace{7pt}$} -4,{$-3 \hspace{7pt}$} -3, {$-1 \hspace{7pt}$} -1,{$-2 \hspace{7pt}$} -2,{$1$} 1,{$2$} 2,{$3$} 3,{$4$} 4}
\axislabels {y}{{$-4$} -4,{$-3$} -3,{$-2$} -2, {$-1$} -1, {$1$} 1, {$2$} 2, {$3$} 3, {$4$} 4}
\normalsize
\penwd{1.25pt}
\polyline{(-2,-1), (0,3), (2,-1), (4,-3)}
\point[4pt]{(-2,-1), (0,3), (2,-1), (4,-3)}
\end{mfpic}
 


\end{solution}

\end{question}

\begin{question}
The function $f$ used in  Exercises \ref{transformgraphfirst} - \ref{transformgraphlast} should look familiar.  What is $f(x)$?  How does this this explain some of the duplication in the answers to Exercises \ref{transformgraphfirst} - \ref{transformgraphlast} mentioned in Exercise \ref{somegraphsthesame}?

\begin{solution}
$y = g(t + 1)$

% \input{ExercisesforTransformations_pic25.tex}
\begin{mfpic}[15]{-5}{5}{-5}{5}
\axes
\tlabel[cc](5,-0.25){\scriptsize $t$}
\tlabel[cc](0.25,5){\scriptsize $y$}
\tlabel[cc](-3.25,-1.25){\scriptsize $(-3,0)$}
\tlabel[cc](-3,4){\scriptsize $(-1,4)$}
\tlabel[cc](1,-1.25){\scriptsize $(1,0)$}
\tlabel[cc](3,-2.5){\scriptsize $(3,-2)$}
\xmarks{-4,-3,-2,-1,1,2,3,4}
\ymarks{-4,-3,-2,-1,1,2,3,4}
\tlpointsep{5pt}
\scriptsize
\axislabels {x}{{$-4 \hspace{7pt}$} -4,{$-3 \hspace{7pt}$} -3, {$-1 \hspace{7pt}$} -1,{$-2 \hspace{7pt}$} -2,{$1$} 1,{$2$} 2,{$3$} 3,{$4$} 4}
\axislabels {y}{{$-4$} -4,{$-3$} -3,{$-2$} -2, {$-1$} -1, {$1$} 1, {$2$} 2, {$3$} 3, {$4$} 4}
\normalsize
\penwd{1.25pt}
\polyline{(-3,0), (-1,4), (1,0), (3,-2)}
\point[4pt]{(-3,0), (-1,4), (1,0), (3,-2)}
\end{mfpic}
 

\vfill
\end{solution}

\end{question}

\begin{question}
$y = g(t) - 1$
\begin{solution}
$y = \frac{1}{2} g(t)$

% \input{ExercisesforTransformations_pic26.tex}
\begin{mfpic}[15]{-5}{5}{-5}{5}
\axes
\tlabel[cc](5,-0.25){\scriptsize $t$}
\tlabel[cc](0.25,5){\scriptsize $y$}
\tlabel[cc](-2.25,-1.25){\scriptsize $(-2,0)$}
\tlabel[cc](1,2){\scriptsize $(0,2)$}
\tlabel[cc](2,-1.25){\scriptsize $(2,0)$}
\tlabel[cc](4,-1.5){\scriptsize $(4,-1)$}
\xmarks{-4,-3,-2,-1,1,2,3,4}
\ymarks{-4,-3,-2,-1,1,2,3,4}
\tlpointsep{5pt}
\scriptsize
\axislabels {x}{{$-4 \hspace{7pt}$} -4,{$-3 \hspace{7pt}$} -3, {$-1 \hspace{7pt}$} -1,{$1$} 1,{$3$} 3,{$4$} 4}
\axislabels {y}{{$-4$} -4,{$-3$} -3,{$-2$} -2, {$-1$} -1, {$1$} 1, {$2$} 2, {$3$} 3, {$4$} 4}
\normalsize
\penwd{1.25pt}
\polyline{(-2,0), (0,2), (2,0), (4,-1)}
\point[4pt]{(-2,0), (0,2), (2,0), (4,-1)}
\end{mfpic}
 

\end{solution}

\end{question}

\begin{question}
$y = g(t + 1)$
\begin{solution}
$y =g(2t)$

% \input{ExercisesforTransformations_pic27.tex}
\begin{mfpic}[15]{-5}{5}{-5}{5}
\axes
\tlabel[cc](5,-0.25){\scriptsize $t$}
\tlabel[cc](0.25,5){\scriptsize $y$}
\tlabel[cc](-1,-0.75){\scriptsize $(-1,0)$}
\tlabel[cc](1,4){\scriptsize $(0,4)$}
\tlabel[cc](1.75,0.5){\scriptsize $(1,0)$}
\tlabel[cc](2,-2.5){\scriptsize $(2,-2)$}
\xmarks{-4,-3,-2,-1,1,2,3,4}
\ymarks{-4,-3,-2,-1,1,2,3,4}
\tlpointsep{5pt}
\scriptsize
\axislabels {x}{{$-4 \hspace{7pt}$} -4,{$-3 \hspace{7pt}$} -3, {$-2 \hspace{7pt}$} -2,{$2$} 2,{$3$} 3,{$4$} 4}
\axislabels {y}{{$-4$} -4,{$-3$} -3,{$-2$} -2, {$1$} 1, {$2$} 2, {$3$} 3, {$4$} 4}
\normalsize
\penwd{1.25pt}
\polyline{(-1,0), (0,4), (1,0), (2,-2)}
\point[4pt]{(-1,0), (0,4), (1,0), (2,-2)}
\end{mfpic}
 


\vfill
\end{solution}

\end{question}

\begin{question}
$y = \frac{1}{2} g(t)$

\begin{solution}
$y = - g(t)$

% \input{ExercisesforTransformations_pic28.tex}
\begin{mfpic}[15]{-5}{5}{-5}{5}
\axes
\tlabel[cc](5,-0.25){\scriptsize $t$}
\tlabel[cc](0.25,5){\scriptsize $y$}
\tlabel[cc](-2.25,.75){\scriptsize $(-2,0)$}
\tlabel[cc](1.25,-4){\scriptsize $(0,-4)$}
\tlabel[cc](1.75,.75){\scriptsize $(2,0)$}
\tlabel[cc](4,2.5){\scriptsize $(4,2)$}
\xmarks{-4,-3,-2,-1,1,2,3,4}
\ymarks{-4,-3,-2,-1,1,2,3,4}
\tlpointsep{5pt}
\scriptsize
\axislabels {x}{{$-4 \hspace{7pt}$} -4,{$-3 \hspace{7pt}$} -3, {$-1 \hspace{7pt}$} -1,{$-2 \hspace{7pt}$} -2,{$1$} 1,{$2$} 2,{$3$} 3,{$4$} 4}
\axislabels {y}{{$-4$} -4,{$-3$} -3,{$-2$} -2, {$-1$} -1, {$1$} 1, {$2$} 2, {$3$} 3, {$4$} 4}
\normalsize
\penwd{1.25pt}
\polyline{(-2,0), (0,-4), (2,0), (4,2)}
\point[4pt]{(-2,0), (0,-4), (2,0), (4,2)}
\end{mfpic}
 

\end{solution}

\end{question}

\begin{question}
$y = g(2t)$
\begin{solution}
$y = g(-t)$

% \input{ExercisesforTransformations_pic29.tex}
\begin{mfpic}[15]{-5}{5}{-5}{5}
\axes
\tlabel[cc](5,-0.25){\scriptsize $t$}
\tlabel[cc](0.25,5){\scriptsize $y$}
\tlabel[cc](2.25,-1.25){\scriptsize $(2,0)$}
\tlabel[cc](1,4){\scriptsize $(0,4)$}
\tlabel[cc](-2,-1.25){\scriptsize $(-2,0)$}
\tlabel[cc](-4,-2.5){\scriptsize $(-4,-2)$}
\xmarks{-4,-3,-2,-1,1,2,3,4}
\ymarks{-4,-3,-2,-1,1,2,3,4}
\tlpointsep{5pt}
\scriptsize
\axislabels {x}{{$-4 \hspace{7pt}$} -4,{$-3 \hspace{7pt}$} -3, {$-1 \hspace{7pt}$} -1,{$1$} 1,{$3$} 3,{$4$} 4}
\axislabels {y}{{$-4$} -4,{$-3$} -3,{$-2$} -2, {$-1$} -1, {$1$} 1, {$2$} 2, {$3$} 3, {$4$} 4}
\normalsize
\penwd{1.25pt}
\polyline{(2,0), (0,4), (-2,0), (-4,-2)}
\point[4pt]{(2,0), (0,4), (-2,0), (-4,-2)}
\end{mfpic}
 


\vfill
\end{solution}

\end{question}

\begin{question}
$y = - g(t)$
\begin{solution}
$y = g(t+1) - 1$

% \input{ExercisesforTransformations_pic30.tex}
\begin{mfpic}[15]{-5}{5}{-5}{5}
\axes
\tlabel[cc](5,-0.25){\scriptsize $t$}
\tlabel[cc](0.25,5){\scriptsize $y$}
\tlabel[cc](-3.25,-2.25){\scriptsize $(-3,-1)$}
\tlabel[cc](-3,3){\scriptsize $(-1,3)$}
\tlabel[cc](2.5,-1){\scriptsize $(1,-1)$}
\tlabel[cc](3,-3.5){\scriptsize $(3,-3)$}
\xmarks{-4,-3,-2,-1,1,2,3,4}
\ymarks{-4,-3,-2,-1,1,2,3,4}
\tlpointsep{5pt}
\scriptsize
\axislabels {x}{{$-4 \hspace{7pt}$} -4,{$-3 \hspace{7pt}$} -3, {$-1 \hspace{7pt}$} -1,{$-2 \hspace{7pt}$} -2,{$1$} 1,{$2$} 2,{$3$} 3,{$4$} 4}
\axislabels {y}{{$-4$} -4,{$-3$} -3,{$-2$} -2, {$-1$} -1, {$1$} 1, {$2$} 2, {$3$} 3, {$4$} 4}
\normalsize
\penwd{1.25pt}
\polyline{(-3,-1), (-1,3), (1,-1), (3,-3)}
\point[4pt]{(-3,-1), (-1,3), (1,-1), (3,-3)}
\end{mfpic}
 

\end{solution}

\end{question}

\begin{question}
$y = g(-t)$

\begin{solution}
$y = 1 -g(t)$

% \input{ExercisesforTransformations_pic31.tex}
\begin{mfpic}[15]{-5}{5}{-5}{5}
\axes
\tlabel[cc](5,-0.25){\scriptsize $t$}
\tlabel[cc](0.25,5){\scriptsize $y$}
\tlabel[cc](-2.25,1.75){\scriptsize $(-2,1)$}
\tlabel[cc](1.25,-3){\scriptsize $(0,-3)$}
\tlabel[cc](1.75,1.75){\scriptsize $(2,1)$}
\tlabel[cc](4,3.5){\scriptsize $(4,3)$}
\xmarks{-4,-3,-2,-1,1,2,3,4}
\ymarks{-4,-3,-2,-1,1,2,3,4}
\tlpointsep{5pt}
\scriptsize
\axislabels {x}{{$-4 \hspace{7pt}$} -4,{$-3 \hspace{7pt}$} -3, {$-1 \hspace{7pt}$} -1,{$-2 \hspace{7pt}$} -2,{$1$} 1,{$2$} 2,{$3$} 3,{$4$} 4}
\axislabels {y}{{$-4$} -4,{$-3$} -3,{$-2$} -2, {$-1$} -1, {$1$} 1, {$2$} 2, {$3$} 3, {$4$} 4}
\normalsize
\penwd{1.25pt}
\polyline{(-2,1), (0,-3), (2,1), (4,3)}
\point[4pt]{(-2,1), (0,-3), (2,1), (4,3)}
\end{mfpic}



\vfill
\end{solution}

\end{question}

\begin{question}
$y = g(t+1) - 1$
\begin{solution}
$y = \frac{1}{2}g(t+1)-1$


% \input{ExercisesforTransformations_pic32.tex}
\begin{mfpic}[15]{-5}{5}{-5}{5}
\axes
\tlabel[cc](5,-0.25){\scriptsize $t$}
\tlabel[cc](0.25,5){\scriptsize $y$}
\tlabel[cc](-3.25,-1.5){\scriptsize $(-3,-1)$}
\tlabel[cc](-2.25,1){\scriptsize $(-1,1)$}
\tlabel[cc](2.5,-1){\scriptsize $(1,-1)$}
\tlabel[cc](3,-2.5){\scriptsize $(3,-2)$}
\xmarks{-4,-3,-2,-1,1,2,3,4}
\ymarks{-4,-3,-2,-1,1,2,3,4}
\tlpointsep{5pt}
\scriptsize
\axislabels {x}{{$-4 \hspace{7pt}$} -4,{$-3 \hspace{7pt}$} -3, {$-1 \hspace{7pt}$} -1,{$-2 \hspace{7pt}$} -2,{$1$} 1,{$2$} 2,{$3$} 3,{$4$} 4}
\axislabels {y}{{$-4$} -4,{$-3$} -3,{$-2$} -2, {$-1$} -1, {$1$} 1, {$2$} 2, {$3$} 3, {$4$} 4}
\normalsize
\penwd{1.25pt}
\polyline{(-3,-1), (-1,1), (1,-1), (3,-2)}
\point[4pt]{(-3,-1), (-1,1), (1,-1), (3,-2)}
\end{mfpic}
 


\end{solution}

\end{question}

\begin{question}
$y = 1 - g(t)$
\begin{solution}
$g(x) = f(x) + 3$\\
% \input{ExercisesforTransformations_pic33.tex}
\begin{mfpic}[15]{-4}{4}{-1.5}{7}
\tlabel[cc](-3,2){\tiny $\left(-3, 3 \right)$}
\tlabel[cc](0.8,6.3){\tiny $\left(0, 6 \right)$}
\tlabel[cc](3,2){\tiny $\left(3, 3 \right)$}
\axes
\tlabel[cc](4,-0.5){\scriptsize $x$}
\tlabel[cc](0.5,7){\scriptsize $y$}
\xmarks{-3,-2,-1,1,2,3}
\ymarks{-1,1,2,3,4,5,6}
\tlpointsep{4pt}
\tiny
\axislabels {x}{{$-3 \hspace{7pt}$} -3, {$-2 \hspace{7pt}$} -2, {$-1 \hspace{7pt}$} -1, {$1$} 1, {$2$} 2, {$3$} 3}
\axislabels {y}{{$-1$} -1, {$1$} 1, {$2$} 2, {$3$} 3, {$4$} 4, {$5$} 5, {$6$} 6}
\normalsize
\point[4pt]{(-3,3),(0,6)}
\penwd{1.25pt}
\parafcn{0,3.14159,0.1}{(3*cos(t), (3*sin(t)) + 3)}
%\function{-3,3,.1}{3 + sqrt(9 - (x**2))}
\pointfillfalse
\point[4pt]{(3,3)}
\end{mfpic}


\vfill
\end{solution}

\end{question}

\begin{question}
$y = \frac{1}{2}g(t+1)-1$ 

\begin{solution}
$h(x) = f(x) - \frac{1}{2}$\\
% \input{ExercisesforTransformations_pic34.tex}
\begin{mfpic}[15]{-4}{4}{-1.5}{4}
\tlabel[cc](-3,-1){\tiny $\left(-3, -\frac{1}{2} \right)$}
\tlabel[cc](0.8,3){\tiny $\left(0, \frac{5}{2} \right)$}
\tlabel[cc](3,-1){\tiny $\left(3, -\frac{1}{2} \right)$}
\axes
\tlabel[cc](4,-0.5){\scriptsize $x$}
\tlabel[cc](0.5,4){\scriptsize $y$}
\xmarks{-3,-2,-1,1,2,3}
\ymarks{-1,1,2,3}
\tlpointsep{4pt}
\tiny
\axislabels {x}{{$-3 \hspace{7pt}$} -3, {$-2 \hspace{7pt}$} -2, {$-1 \hspace{7pt}$} -1, {$1$} 1, {$2$} 2}
\axislabels {y}{{$-1$} -1, {$1$} 1, {$2$} 2, {$3$} 3}
\normalsize
\point[4pt]{(-3,-0.5),(0,2.5)}
%\function{-3,3,.1}{sqrt(9 - (x**2)) - 0.5}
\penwd{1.25pt}
\parafcn{0,3.14159,0.1}{(3*cos(t), (3*sin(t)) - 0.5)}
\pointfillfalse
\point[4pt]{(3,-0.5)}
\end{mfpic}




\end{solution}

\end{question}

\begin{question}
$g(x) = f(x) + 3$
\begin{solution}
$j(x) = f\left(x - \frac{2}{3}\right)$\\
% \input{ExercisesforTransformations_pic35.tex}
\begin{mfpic}[15]{-4}{4.5}{-1.5}{4}
\tlabel[cc](-2.333,-1){\tiny $\left(-\frac{7}{3}, 0 \right)$}
\tlabel[cc](1.5,3.5){\tiny $\left(\frac{2}{3}, 3 \right)$}
\tlabel[cc](3.6667,-1){\tiny $\left(\frac{11}{3}, 0 \right)$}
\axes
\tlabel[cc](4.5,-0.5){\scriptsize $x$}
\tlabel[cc](0.5,4){\scriptsize $y$}
\xmarks{-3,-2,-1,1,2,3}
\ymarks{-1,1,2,3}
\tlpointsep{4pt}
\tiny
\axislabels {x}{{$-3 \hspace{7pt}$} -3, {$-2 \hspace{7pt}$} -2, {$-1 \hspace{7pt}$} -1, {$1$} 1, {$2$} 2, {$3$} 3}
\axislabels {y}{{$-1$} -1, {$1$} 1, {$2$} 2, {$3$} 3}
\normalsize
\point[4pt]{(-2.333,0), (.6667, 3)}
%\function{-2.333,3.6667,.1}{sqrt(9 - ((x - 0.6667)**2))}
\penwd{1.25pt}
\parafcn{0,3.14159,0.1}{((3*cos(t)) + 0.6667, 3*sin(t))}
\pointfillfalse
\point[4pt]{(3.6667,0)}
\end{mfpic}


\vfill
\end{solution}

\end{question}

\begin{question}
$h(x) = f(x) - \frac{1}{2}$
\begin{solution}
$a(x) = f(x + 4)$\\
% \input{ExercisesforTransformations_pic36.tex}
\begin{mfpic}[15]{-8}{1}{-1.5}{4}
\tlabel[cc](-7,-1){\tiny $\left(-7, 0 \right)$}
\tlabel[cc](-3,3.5){\tiny $\left(-4, 3 \right)$}
\tlabel[cc](-1,-1){\tiny $\left(-1, 0 \right)$}
\axes
\tlabel[cc](1,-0.5){\scriptsize $x$}
\tlabel[cc](0.5,4){\scriptsize $y$}
\xmarks{-7,-6,-5,-4,-3,-2,-1}
\ymarks{-1,1,2,3}
\tlpointsep{4pt}
\tiny
\axislabels {x}{{$-7 \hspace{7pt}$} -7, {$-6 \hspace{7pt}$} -6, {$-5 \hspace{7pt}$} -5, {$-4 \hspace{7pt}$} -4, {$-3 \hspace{7pt}$} -3, {$-2 \hspace{7pt}$} -2, {$-1 \hspace{7pt}$} -1}
\axislabels {y}{{$1$} 1, {$2$} 2, {$3$} 3}
\normalsize
\point[4pt]{(-7,0),(-4, 3)}
%\function{-7,-1,.1}{sqrt(9 - ((x + 4)**2))}
\penwd{1.25pt}
\parafcn{0,3.14159,0.1}{((3*cos(t)) - 4, 3*sin(t))}
\pointfillfalse
\point[4pt]{(-1,0)}
\end{mfpic}



\end{solution}

\end{question}

\begin{question}
$j(x) = f\left(x - \frac{2}{3}\right)$

\begin{solution}
$b(x) = f(x + 1) - 1$\\
% \input{ExercisesforTransformations_pic37.tex}
\begin{mfpic}[15]{-5}{3}{-2}{3}
\tlabel[cc](-4,-1.5){\tiny $\left(-4, -1 \right)$}
\tlabel[cc](-1.5,2.5){\tiny $\left(-1,2 \right)$}
\tlabel[cc](2,-1.5){\tiny $\left(2, -1 \right)$}
\axes
\tlabel[cc](3,-0.5){\scriptsize $x$}
\tlabel[cc](0.5,3){\scriptsize $y$}
\xmarks{-4,-3,-2,-1,1,2}
\ymarks{-1,1,2}
\tlpointsep{4pt}
\tiny
\axislabels {x}{{$-4 \hspace{7pt}$} -4, {$-3 \hspace{7pt}$} -3, {$-2 \hspace{7pt}$} -2, {$-1 \hspace{7pt}$} -1, {$1$} 1, {$2$} 2}
\axislabels {y}{{$-1$} -1, {$1$} 1, {$2$} 2}
\normalsize
\point[4pt]{(-4,-1),(-1, 2)}
%\function{-4,2,.1}{sqrt(9 - ((x + 1)**2)) - 1}
\penwd{1.25pt}
\parafcn{0,3.14159,0.1}{((3*cos(t)) - 1, (3*sin(t)) - 1)}
\pointfillfalse
\point[4pt]{(2,-1)}
\end{mfpic}


\vfill
\end{solution}

\end{question}

\begin{question}
$a(x) = f(x + 4)$
\begin{solution}
$c(x) = \frac{3}{5}f(x)$\\
% \input{ExercisesforTransformations_pic38.tex}
\begin{mfpic}[15]{-4}{4}{-1.5}{3}
\tlabel[cc](-3,-1){\tiny $\left(-3, 0 \right)$}
\tlabel[cc](0.8,2.3){\tiny $\left(0, \frac{9}{5} \right)$}
\tlabel[cc](3,-1){\tiny $\left(3, 0 \right)$}
\axes
\tlabel[cc](4,-0.5){\scriptsize $x$}
\tlabel[cc](0.5,3){\scriptsize $y$}
\xmarks{-3,-2,-1,1,2,3}
\ymarks{-1,1,2}
\tlpointsep{4pt}
\tiny
\axislabels {x}{{$-3 \hspace{7pt}$} -3, {$-2 \hspace{7pt}$} -2, {$-1 \hspace{7pt}$} -1, {$1$} 1, {$2$} 2, {$3$} 3}
\axislabels {y}{{$-1$} -1, {$1$} 1, {$2$} 2}
\normalsize
\point[4pt]{(-3,0),(0, 1.8)}
%\function{-3,3,.1}{0.6*sqrt(9 - (x**2))}
\penwd{1.25pt}
\parafcn{0,3.14159,0.1}{(3*cos(t), 1.8*sin(t))}
\pointfillfalse
\point[4pt]{(3,0)}
\end{mfpic}



\end{solution}

\end{question}

\begin{question}
$b(x) = f(x + 1) - 1$
\begin{solution}
$d(x) = -2f(x)$\\
% \input{ExercisesforTransformations_pic39.tex}
\begin{mfpic}[15]{-4}{4}{-7}{1}
\tlabel[cc](-3,0.5){\tiny $\left(-3, 0 \right)$}
\tlabel[cc](0.8,-6.5){\tiny $\left(0, -6 \right)$}
\tlabel[cc](3,0.5){\tiny $\left(3, 0 \right)$}
\axes
\tlabel[cc](4,-0.5){\scriptsize $x$}
\tlabel[cc](0.5,1){\scriptsize $y$}
\xmarks{-3,-2,-1,1,2,3}
\ymarks{-6,-5,-4,-3,-2,-1}
\tlpointsep{4pt}
\tiny
\axislabels {x}{{$-3 \hspace{7pt}$} -3, {$-2 \hspace{7pt}$} -2, {$-1 \hspace{7pt}$} -1, {$1$} 1, {$2$} 2, {$3$} 3}
\axislabels {y}{{$-6$} -6, {$-5$} -5, {$-4$} -4, {$-3$} -3, {$-2$} -2, {$-1$} -1}
\normalsize
\point[4pt]{(-3,0),(0, -6)}
%\function{-3,3,.1}{-2*sqrt(9 - (x**2))}
\penwd{1.25pt}
\parafcn{0,3.14159,0.1}{(3*cos(t), -6*sin(t))}
\pointfillfalse
\point[4pt]{(3,0)}
\end{mfpic}


\vfill
\end{solution}

\end{question}

\begin{question}
$c(x) = \frac{3}{5}f(x)$


\begin{solution}
$k(x) = f\left(\frac{2}{3}x\right)$\\
% \input{ExercisesforTransformations_pic40.tex}
\begin{mfpic}[15]{-5}{5}{-1.5}{4}
\tlabel[cc](-4.5,-1){\tiny $\left(-\frac{9}{2}, 0 \right)$}
\tlabel[cc](0.8,3.5){\tiny $\left(0, 3 \right)$}
\tlabel[cc](4.5,-1){\tiny $\left(\frac{9}{2}, 0 \right)$}
\axes
\tlabel[cc](5,-0.5){\scriptsize $x$}
\tlabel[cc](0.5,4){\scriptsize $y$}
\xmarks{-4,-3,-2,-1,1,2,3,4}
\ymarks{-1,1,2,3}
\tlpointsep{4pt}
\tiny
\axislabels {x}{{$-4 \hspace{7pt}$} -4, {$-3 \hspace{7pt}$} -3, {$-2 \hspace{7pt}$} -2, {$-1 \hspace{7pt}$} -1, {$1$} 1, {$2$} 2, {$3$} 3, {$4$} 4}
\axislabels {y}{{$-1$} -1, {$1$} 1, {$2$} 2, {$3$} 3}
\normalsize
\point[4pt]{(-4.5,0),(0, 3)}
%\function{-4.5,4.5,.1}{sqrt(9 - ((0.66666*x)**2))}
\penwd{1.25pt}
\parafcn{0,3.14159,0.1}{(4.5*cos(t), 3*sin(t))}
\pointfillfalse
\point[4pt]{(4.5,0)}
\end{mfpic}



\end{solution}

\end{question}

\begin{question}
$d(x) = -2f(x)$
\begin{solution}
$m(x) = -\frac{1}{4}f(3x)$\\
% \input{ExercisesforTransformations_pic41.tex}
\begin{mfpic}[30]{-2}{2}{-1.5}{1}
\tlabel[cc](-1,0.25){\scriptsize $\left( -1, 0 \right)$}
\tlabel[cc](0.8,-1){\scriptsize  $\left(0, -\frac{3}{4} \right)$}
\tlabel[cc](1,0.25){\scriptsize  $\left( 1, 0 \right)$}
\axes
\tlabel[cc](2,-0.25){\scriptsize $x$}
\tlabel[cc](0.25,1){\scriptsize $y$}
\xmarks{-1,1}
\ymarks{-1}
\tlpointsep{4pt}
\tiny
\axislabels {x}{{$-1 \hspace{7pt}$} -1, {$1$} 1}
\axislabels {y}{{$-1$} -1}
\normalsize
\point[4pt]{(-1,0),(0, -0.75)}
%\function{-1,1,.1}{-0.25*sqrt(9 - ((3*x)**2))}
\penwd{1.25pt}
\parafcn{0,3.14159,0.1}{(cos(t), -0.75*sin(t))}
\pointfillfalse
\point[4pt]{(1,0)}
\end{mfpic}


\vfill
\end{solution}

\end{question}

\begin{question}
$k(x) = f\left(\frac{2}{3}x\right)$
\begin{solution}
$n(x) = 4f(x - 3) - 6$\\
% \input{ExercisesforTransformations_pic42.tex}
\begin{mfpic}[15]{-1}{7}{-7}{7}
\tlabel[cc](1,-6.5){\tiny $\left(0, -6 \right)$}
\tlabel[cc](3,6.5){\tiny $\left(3, 6 \right)$}
\tlabel[cc](5.5,-6.5){\tiny $\left(6, -6 \right)$}
\axes
\tlabel[cc](7,-0.5){\scriptsize $x$}
\tlabel[cc](0.5,7){\scriptsize $y$}
\xmarks{1,2,3,4,5,6}
\ymarks{-6 step 1 until 6}
\tlpointsep{4pt}
\tiny
\axislabels {x}{{$1$} 1, {$2$} 2, {$3$} 3, {$4$} 4, {$5$} 5, {$6$} 6}
\axislabels {y}{{$-6$} -6, {$-5$} -5, {$-4$} -4, {$-3$} -3, {$-2$} -2, {$-1$} -1, {$1$} 1, {$2$} 2, {$3$} 3, {$4$} 4, {$5$} 5, {$6$} 6}
\normalsize
\point[4pt]{(0,-6),(3,6)}
%\function{0,6,.1}{4*sqrt(9 - ((x - 3)**2)) - 6}
\penwd{1.25pt}
\parafcn{0,3.14159,0.1}{(3*cos(t) + 3, (12*sin(t)) - 6)}
\pointfillfalse
\point[4pt]{(6,-6)}
\end{mfpic}



\end{solution}

\end{question}

\begin{question}
$m(x) = -\frac{1}{4}f(3x)$

\begin{solution}
$p(x) = 4 + f(1 - 2x) = f(-2x + 1) + 4$\\
% \input{ExercisesforTransformations_pic43.tex}
\begin{mfpic}[15]{-2}{3}{-1.5}{8}
\tlabel[cc](-1,3.5){\tiny $\left(-1, 4 \right)$}
\tlabel[cc](1.5,7.2){\tiny $\left(\frac{1}{2}, 7 \right)$}
\tlabel[cc](2,3.5){\tiny $\left(2, 4 \right)$}
\axes
\tlabel[cc](3,-0.5){\scriptsize $x$}
\tlabel[cc](0.5,8){\scriptsize $y$}
\xmarks{-1,1,2}
\ymarks{-1,1,2,3,4,5,6,7}
\tlpointsep{4pt}
\tiny
\axislabels {x}{{$-1 \hspace{7pt}$} -1, {$1$} 1, {$2$} 2}
\axislabels {y}{{$-1$} -1, {$1$} 1, {$2$} 2, {$3$} 3, {$4$} 4, {$5$} 5, {$6$} 6, {$7$} 7}
\normalsize
\point[4pt]{(0.5, 7),(2,4)}
%\function{-1,2,.1}{4 + sqrt(9 - (((-2*x) + 1)**2))}
\penwd{1.25pt}
\parafcn{0,3.14159,0.1}{((1 - 3*cos(t))/2, 3*sin(t) + 4)}
\pointfillfalse
\point[4pt]{(-1,4)}
\end{mfpic}


\vfill
\end{solution}

\end{question}

\begin{question}
$n(x) = 4f(x - 3) - 6$
\begin{solution}
\small $q(x) = -\frac{1}{2}f\left(\frac{x + 4}{2}\right) - 3 = -\frac{1}{2}f\left( \frac{1}{2}x + 2 \right) - 3 $\\ \normalsize
% \input{ExercisesforTransformations_pic44.tex}
\begin{mfpic}[10]{-11}{3}{-5.5}{1}
\tlabel[cc](-10,-2.5){\tiny $\left(-10, -3 \right)$}
\tlabel[cc](-4,-5.25){\tiny $\left(-4, -\frac{9}{2} \right)$}
\tlabel[cc](2,-2.5){\tiny $\left(2, -3 \right)$}
\axes
\tlabel[cc](3,-0.5){\scriptsize $x$}
\tlabel[cc](0.5,1){\scriptsize $y$}
\xmarks{-10,-9,-8,-7,-6,-5,-4,-3,-2,-1,1,2}
\ymarks{-4,-3,-2,-1}
\tlpointsep{4pt}
\tiny
\axislabels {x}{{$-10 \hspace{7pt}$} -10, {$-9 \hspace{7pt}$} -9, {$-8 \hspace{7pt}$} -8, {$-7 \hspace{7pt}$} -7, {$-6 \hspace{7pt}$} -6, {$-5 \hspace{7pt}$} -5, {$-4 \hspace{7pt}$} -4, {$-3 \hspace{7pt}$} -3, {$-2 \hspace{7pt}$} -2, {$-1 \hspace{7pt}$} -1, {$1$} 1, {$2$} 2}
\axislabels {y}{{$-4$} -4, {$-3$} -3, {$-2$} -2, {$-1$} -1}
\normalsize
\point[4pt]{(-10,-3),(-4, -4.5)}
%\function{-10,2,.1}{-0.5*sqrt(9 - (((0.5*x) + 2)**2)) - 3}
\penwd{1.25pt}
\parafcn{0,3.14159,0.1}{(6*cos(t) - 4, -1.5*sin(t) - 3)}
\pointfillfalse
\point[4pt]{(2, -3)}
\end{mfpic}



\end{solution}

\end{question}

\begin{question}
$p(x) = 4 + f(1 - 2x)$
\begin{solution}
$y = S_{\text{\tiny $1$}}(t) = S(t + 1)$

% \input{ExercisesforTransformations_pic45.tex}
\begin{mfpic}[20]{-4}{2}{-4}{4}
\axes
\tlabel[cc](2,-0.25){\scriptsize $t$}
\tlabel[cc](0.25,4){\scriptsize $y$}
\tlabel[cc](-3.5,0.5){\scriptsize $(-3,0)$}
\tlabel[cc](-2,-3.5){\scriptsize $(-2,-3)$}
\tlabel[cc](-1.75,0.5){\scriptsize $(-1,0)$}
\tlabel[cc](0.75,3){\scriptsize $(0,3)$}
\tlabel[cc](1,-0.5){\scriptsize $(1,0)$}
\xmarks{-3,-2,-1,1}
\ymarks{-3,-2,-1,1,2,3}
\tlpointsep{5pt}
\scriptsize
\axislabels {x}{{$-3 \hspace{7pt}$} -3,{$-2 \hspace{7pt}$} -2,{$-1 \hspace{7pt}$} -1}
\axislabels {y}{{$-3$} -3,{$-2$} -2, {$-1$} -1, {$1$} 1, {$2$} 2, {$3$} 3}
\normalsize
\penwd{1.25pt}
\function{-3,1,0.1}{3*sin(1.570796327*(x+1))}
\point[4pt]{(-3,0), (-2,-3), (-1,0), (0,3), (1,0)}
\end{mfpic}
 


\vfill
\end{solution}

\end{question}

\begin{question}
$q(x) = -\frac{1}{2}f\left(\frac{x + 4}{2}\right) - 3$ 

\begin{solution}
$y = S_{\text{\tiny $2$}}(t) =  S_{\text{\tiny $1$}}(-t) = S(-t + 1)$

% \input{ExercisesforTransformations_pic46.tex}
\begin{mfpic}[20]{-2}{4}{-4}{4}
\axes
\tlabel[cc](4,-0.25){\scriptsize $t$}
\tlabel[cc](0.25,4){\scriptsize $y$}
\tlabel[cc](3.5,0.5){\scriptsize $(3,0)$}
\tlabel[cc](2,-3.5){\scriptsize $(2,-3)$}
\tlabel[cc](1.75,0.5){\scriptsize $(1,0)$}
\tlabel[cc](0.75,3){\scriptsize $(0,3)$}
\tlabel[cc](-1,-0.5){\scriptsize $(-1,0)$}
\xmarks{3,2,1,-1}
\ymarks{-3,-2,-1,1,2,3}
\tlpointsep{5pt}
\scriptsize
\axislabels {x}{ {$1$} 1, {$2$} 2, {$3$} 3}
\axislabels {y}{{$-3$} -3,{$-2$} -2, {$-1$} -1, {$1$} 1, {$2$} 2, {$3$} 3}
\normalsize
\penwd{1.25pt}
\function{-1,3,0.1}{3*sin(1.570796327*(1-x))}
\point[4pt]{(3,0), (2,-3), (1,0), (0,3), (-1,0)}
\end{mfpic}



\end{solution}

\end{question}

\begin{question}
$y = S_{\text{\tiny $1$}}(t) = S(t + 1)$
\begin{solution}
$y = S_{\text{\tiny $3$}}(t) = \frac{1}{2}  S_{\text{\tiny $2$}}(t) =  \frac{1}{2}S(-t+1)$

% \input{ExercisesforTransformations_pic47.tex}
\begin{mfpic}[20]{-2}{4}{-3}{3}
\axes
\tlabel[cc](4,-0.25){\scriptsize $t$}
\tlabel[cc](0.25,3){\scriptsize $y$}
\tlabel[cc](3,0.5){\scriptsize $(3,0)$}
\tlabel[cc](2,-2){\scriptsize $\left(2,-\frac{3}{2} \right)$}
\tlabel[cc](1.5,0.5){\scriptsize $(1,0)$}
\tlabel[cc](0.75,1.5){\scriptsize $\left(0,\frac{3}{2} \right)$}
\tlabel[cc](-1,-0.5){\scriptsize $(-1,0)$}
\xmarks{3,2,1,-1}
\ymarks{-2,-1,1,2}
\tlpointsep{5pt}
\scriptsize
\axislabels {x}{ {$1$} 1, {$2$} 2, {$3$} 3}
\axislabels {y}{{$-2$} -2,{$-1$} -1, {$1$} 1, {$2$} 2}
\normalsize
\penwd{1.25pt}
\function{-1,3,0.1}{1.5*sin(1.570796327*(1-x))}
\point[4pt]{(3,0), (2,-1.5), (1,0), (0,1.5), (-1,0)}
\end{mfpic}
 

\vfill
\end{solution}

\end{question}

\begin{question}
$y = S_{\text{\tiny $2$}}(t) =  S_{\text{\tiny $1$}}(-t) = S(-t + 1)$

\begin{solution}
$y = S_{\text{\tiny $4$}}(t) = S_{\text{\tiny $3$}}(t) + 1 = \frac{1}{2}S(-t+1) + 1$ 

% \input{ExercisesforTransformations_pic48.tex}
\begin{mfpic}[20]{-2}{4}{-2}{4}
\axes
\tlabel[cc](4,-0.25){\scriptsize $t$}
\tlabel[cc](0.25,4){\scriptsize $y$}
\tlabel[cc](3,1.5){\scriptsize $(3,1)$}
\tlabel[cc](2,-1){\scriptsize $\left(2,-\frac{1}{2} \right)$}
\tlabel[cc](1.5,1.5){\scriptsize $(1,1)$}
\tlabel[cc](0.75,2.5){\scriptsize $\left(0,\frac{5}{2} \right)$}
\tlabel[cc](-1,.5){\scriptsize $(-1,1)$}
\xmarks{3,2,1,-1}
\ymarks{-1,1,2,3}
\tlpointsep{5pt}
\scriptsize
\axislabels {x}{ {$-1 \hspace{7pt}$} -1,{$1$} 1, {$3$} 3}
\axislabels {y}{{$-1$} -1, {$1$} 1, {$2$} 2, {$3$} 3}
\normalsize
\penwd{1.25pt}
\function{-1,3,0.1}{1.5*sin(1.570796327*(1-x))+1}
\point[4pt]{(3,1), (2,-0.5), (1,1), (0,2.5), (-1,1)}
\end{mfpic}
 


\end{solution}

\end{question}

\begin{question}
$y = S_{\text{\tiny $3$}}(t) = \frac{1}{2}  S_{\text{\tiny $2$}}(t) =  \frac{1}{2}S(-t+1)$
\begin{solution}
$g(x) = \sqrt{x-2} - 3$
\end{solution}

\end{question}

\begin{question}
$y = S_{\text{\tiny $4$}}(t) = S_{\text{\tiny $3$}}(t) + 1 = \frac{1}{2}S(-t+1) + 1$ 

\begin{solution}
$g(x) = \sqrt{x-2} - 3$

\end{solution}

\end{question}

\begin{question}
(1) shift right 2 units; (2) shift down 3 units
\begin{solution}
$g(x) = -\sqrt{x} + 1$
\end{solution}

\end{question}

\begin{question}
(1) shift down 3 units; (2) shift right 2 units
\begin{solution}
$g(x) = -(\sqrt{x} + 1) = -\sqrt{x} - 1$

\end{solution}

\end{question}

\begin{question}
(1) reflect across the $x$-axis; (2) shift up 1 unit
\begin{solution}
$g(x) = \sqrt{-x+1} + 2$
\end{solution}

\end{question}

\begin{question}
(1) shift up 1 unit; (2) reflect across the $x$-axis
\begin{solution}
$g(x) = \sqrt{-(x+1)} + 2 = \sqrt{-x-1} + 2$

\end{solution}

\end{question}

\begin{question}
(1) shift left 1 unit; (2) reflect across the $y$-axis; (3) shift up 2 units
\begin{solution}
$g(x) = 2\sqrt{x+3} - 4$
\end{solution}

\end{question}

\begin{question}
(1) reflect across the $y$-axis;  (2) shift left 1 unit;  (3) shift up 2 units
\begin{solution}
$g(x) = 2\left(\sqrt{x+3} - 4\right) = 2\sqrt{x+3} - 8$

\end{solution}

\end{question}

\begin{question}
(1) shift left 3 units; (2) vertical stretch by a factor of 2; (3) shift down 4 units
\begin{solution}
$g(x) = \sqrt{2x-3} + 1$
\end{solution}

\end{question}

\begin{question}
(1) shift left 3 units; (2) shift down 4 units; (3) vertical stretch by a factor of 2
\begin{solution}
$g(x) = \sqrt{2(x-3)} + 1 = \sqrt{2x-6}+1$

\end{solution}

\end{question}

\begin{question}
(1) shift right 3 units; (2) horizontal shrink by a factor of 2; (3) shift up 1 unit
\begin{solution}
$g(x)=f(x)+1$
\end{solution}

\end{question}

\begin{question}
(1) horizontal shrink by a factor of 2; (2) shift right 3 units; (3) shift up 1 unit


\begin{solution}
$h(x) = f(x-2)$ 


\end{solution}

\end{question}

\begin{question}
$y = g(x)$ %$g(x)=f(x)+1$

% \input{ExercisesforTransformations_pic6.tex}
\begin{mfpic}[15]{-5}{5}{-5}{5}
\axes
\tlabel[cc](5,-0.25){\scriptsize $x$}
\tlabel[cc](0.25,5){\scriptsize $y$}
\tlabel[cc](-2,-0.5){\scriptsize $\left(-\frac{1}{2},-\frac{1}{2} \right)$}
\tlabel[cc](-0.75,1.5){\scriptsize $(0,1)$}
\tlabel[cc](1.75,2.5){\scriptsize $\left(\frac{1}{2},\frac{5}{2} \right)$}
\tlabel[cc](3, 4.5){\scriptsize asymptote $y=4$}
\tlabel[cc](-2.75,-2.5){\scriptsize asymptote $y=-2$}
\xmarks{-4,-3,-2,-1,1,2,3,4}
\ymarks{-4,-3,-2,-1,1,2,3,4}
\tlpointsep{5pt}
\scriptsize
%\axislabels {x}{{$-4 \hspace{7pt}$} -4,{$-3 \hspace{7pt}$} -3, {$-1 \hspace{7pt}$} -1,{$1$} 1,{$3$} 3,{$4$} 4}
%\axislabels {y}{{$-4$} -4,{$-3$} -3,{$-2$} -2, {$-1$} -1, {$1$} 1, {$2$} 2, {$3$} 3, {$4$} 4}
\normalsize
\dashed \polyline {(-5,4), (5,4)}
\dashed \polyline {(-5,-2), (5,-2)}
\penwd{1.25pt}
\arrow \reverse \arrow \parafcn{-2.8,2.8,0.1}{( 0.5*(tan( 0.5236*t))   ,   t +1  )}
\point[4pt]{(0,1), (0.5,2.5), (-0.5,-0.5)}
\end{mfpic}
\begin{solution}
$p(x) = f\left( \frac{x}{2} \right) -1$
\end{solution}

\end{question}

\begin{question}
$y = h(x)$ %$h(x) = f(x-2)$

% \input{ExercisesforTransformations_pic7.tex}
\begin{mfpic}[15]{-3}{7}{-5}{5}
\axes
\tlabel[cc](7,-0.25){\scriptsize $x$}
\tlabel[cc](0.25,5){\scriptsize $y$}
\gclear \tlabelrect(0,-1.5){\scriptsize $\left(\frac{3}{2},-\frac{3}{2} \right)$}
\tlabel[cc](1.25,0.5){\scriptsize $(2,0)$}
\tlabel[cc](3.75,1.5){\scriptsize $\left(\frac{5}{2},\frac{3}{2} \right)$}
\tlabel[cc](3, 3.5){\scriptsize asymptote $y=3$}
\gclear \tlabelrect(-0.75,-3.5){\scriptsize asymptote $y=-3$}
\xmarks{-2,-1,1,2,3,4,5,6}
\ymarks{-3, 2,3,4}
\tlpointsep{5pt}
\scriptsize
%\axislabels {x}{{$-4 \hspace{7pt}$} -4,{$-3 \hspace{7pt}$} -3, {$-1 \hspace{7pt}$} -1,{$1$} 1,{$3$} 3,{$4$} 4}
%\axislabels {y}{{$-4$} -4,{$-3$} -3,{$-2$} -2, {$-1$} -1, {$1$} 1, {$2$} 2, {$3$} 3, {$4$} 4}
\normalsize
\dashed \polyline {(-3,3), (7,3)}
\dashed \polyline {(-3,-3), (7,-3)}
\penwd{1.25pt}
\arrow \reverse \arrow \parafcn{-2.8,2.8,0.1}{( 2+(0.5*(tan( 0.5236*t)))    ,   t   )}
\point[4pt]{(2,0), (2.5,1.5), (1.5,-1.5)}
\end{mfpic}
 


\begin{solution}
$q(x) = -2f(x) = 2f(-x)$ \vphantom{$p(x) = f\left( \frac{x}{2} \right) -1$}

\end{solution}

\end{question}

\begin{question}
$y = p(x)$  % $p(x) = f\left( \frac{x}{2} \right) -1$

% \input{ExercisesforTransformations_pic8.tex}
\begin{mfpic}[15]{-5}{5}{-5}{5}
\axes
\tlabel[cc](5,-0.25){\scriptsize $x$}
\tlabel[cc](0.25,5){\scriptsize $y$}
\tlabel[cc](-2,-2.5){\scriptsize $\left(-1,-\frac{5}{2} \right)$}
\tlabel[cc](-1,-1){\scriptsize $(0,-1)$}
\tlabel[cc](1.75,0.5){\scriptsize $\left(1,\frac{1}{2} \right)$}
\tlabel[cc](3, 2.5){\scriptsize asymptote $y=2$}
\tlabel[cc](-2.75,-4.5){\scriptsize asymptote $y=-4$}
\xmarks{-4 step 0.5 until 4}
\ymarks{-4,-3,-2,-1,1,2,3,4}
\tlpointsep{5pt}
\scriptsize
%\axislabels {x}{{$-4 \hspace{7pt}$} -4,{$-3 \hspace{7pt}$} -3, {$-1 \hspace{7pt}$} -1,{$1$} 1,{$3$} 3,{$4$} 4}
%\axislabels {y}{{$-4$} -4,{$-3$} -3,{$-2$} -2, {$-1$} -1, {$1$} 1, {$2$} 2, {$3$} 3, {$4$} 4}
\normalsize
\dashed \polyline {(-5,2), (5,2)}
\dashed \polyline {(-5,-4), (5,-4)}
\penwd{1.25pt}
\arrow \reverse \arrow \parafcn{-2.8,2.8,0.1}{( 0.5*(tan( 0.5236*t))   ,   t -1  )}
\point[4pt]{(0,-1), (0.5,0.5), (-0.5,-2.5)}
\end{mfpic}
\begin{solution}
$r(x) = 2f(x+1)-3$
\end{solution}

\end{question}

\begin{question}
$y = q(x)$  % $q(x) = -2f(x) = 2f(-x)$


% \input{ExercisesforTransformations_pic9.tex}
\begin{mfpic}[15]{-5}{5}{-5}{5}
\axes
\tlabel[cc](5,-0.25){\scriptsize $x$}
\tlabel[cc](0.25,5){\scriptsize $y$}
\tlabel[cc](-2,1.5){\scriptsize $\left(-\frac{1}{2},3 \right)$}
\tlabel[cc](0.75,0.5){\scriptsize $(0,0)$}
\tlabel[cc](1.75,-1.5){\scriptsize $\left(\frac{1}{2} -3 \right)$}
\tlabel[cc](3, 3.5){\scriptsize asymptote $y=6$}
\tlabel[cc](-2.75,-3.5){\scriptsize asymptote $y=-6$}
\xmarks{-4,-3,-2,-1,1,2,3,4}
\ymarks{-4 step 0.5 until 4}
\tlpointsep{5pt}
\scriptsize
%\axislabels {x}{{$-4 \hspace{7pt}$} -4,{$-3 \hspace{7pt}$} -3, {$-1 \hspace{7pt}$} -1,{$1$} 1,{$3$} 3,{$4$} 4}
%\axislabels {y}{{$-4$} -4,{$-3$} -3,{$-2$} -2, {$-1$} -1, {$1$} 1, {$2$} 2, {$3$} 3, {$4$} 4}
\normalsize
\dashed \polyline {(-5,3), (5,3)}
\dashed \polyline {(-5,-3), (5,-3)}
\penwd{1.25pt}
\arrow \reverse \arrow \parafcn{-2.8,2.8,0.1}{( 0.5*(tan( 0.5236*t))   ,  -1*t   )}
\point[4pt]{(0,0), (0.5,-1.5), (-0.5,1.5)}
\end{mfpic}
 



\begin{solution}
$s(x) = 2f(-x+1)-3 = -2f(x-1)+3$  

\end{solution}

\end{question}

\begin{question}
$y = r(x)$ % $r(x) = 2f(x+1)-3$

% \input{ExercisesforTransformations_pic10.tex}
\begin{mfpic}[15]{-6}{4}{-5}{5}
\axes
\tlabel[cc](4,-0.25){\scriptsize $x$}
\tlabel[cc](0.25,5){\scriptsize $y$}
\tlabel[cc](-3,-3){\scriptsize $\left(-\frac{3}{2},-6 \right)$}
\tlabel[cc](-2.5,-1.5){\scriptsize $(-1,-3)$}
\tlabel[cc](-1.5,0.5){\scriptsize $\left(-\frac{1}{2},0\right)$}
\tlabel[cc](3, 2){\scriptsize asymptote $y=3$}
\tlabel[cc](-2.75,-5){\scriptsize asymptote $y=-9$}
\xmarks{-5,-3,-2,-1,1,2,3}
\ymarks{-4 step 0.5 until 4}
\tlpointsep{5pt}
\scriptsize
%\axislabels {x}{{$-4 \hspace{7pt}$} -4,{$-3 \hspace{7pt}$} -3, {$-1 \hspace{7pt}$} -1,{$1$} 1,{$3$} 3,{$4$} 4}
%\axislabels {y}{{$-4$} -4,{$-3$} -3,{$-2$} -2, {$-1$} -1, {$1$} 1, {$2$} 2, {$3$} 3, {$4$} 4}
\normalsize
\dashed \polyline {(-6,1.5), (4,1.5)}
\dashed \polyline {(-6,-4.5), (4,-4.5)}
\penwd{1.25pt}
\arrow \reverse \arrow \parafcn{-2.8,2.8,0.1}{( (0.5*(tan( 0.5236*t)))-1   ,   t -1.5 )}
\point[4pt]{(-1,-1.5), (-0.5,0), (-1.5,-3)}
\end{mfpic}
\begin{solution}
$g(x) = -2\sqrt[3]{x + 3} - 1$ or $g(x) = 2\sqrt[3]{-x - 3} - 1$

\addtocounter{enumi}{4}
\end{solution}

\end{question}

\begin{question}
$y = s(x)$ % $s(x) = 2f(-x+1)-3 = -2f(x-1)+3$  


% \input{ExercisesforTransformations_pic11.tex}
\begin{mfpic}[15]{-4}{6}{-5}{5}
\axes
\tlabel[cc](6,-0.25){\scriptsize $x$}
\tlabel[cc](0.25,5){\scriptsize $y$}
\tlabel[cc](3,-3){\scriptsize $\left(\frac{3}{2},-6 \right)$}
\tlabel[cc](2.5,-1.5){\scriptsize $(1,-3)$}
\tlabel[cc](1.5,0.5){\scriptsize $\left(\frac{1}{2},0\right)$}
\tlabel[cc](3, 2){\scriptsize asymptote $y=3$}
\tlabel[cc](-2.75,-5){\scriptsize asymptote $y=-9$}
\xmarks{-4,-3,-2,-1,1,2,3,4,5}
\ymarks{-4 step 0.5 until 4}
\tlpointsep{5pt}
\scriptsize
%\axislabels {x}{{$-4 \hspace{7pt}$} -4,{$-3 \hspace{7pt}$} -3, {$-1 \hspace{7pt}$} -1,{$1$} 1,{$3$} 3,{$4$} 4}
%\axislabels {y}{{$-4$} -4,{$-3$} -3,{$-2$} -2, {$-1$} -1, {$1$} 1, {$2$} 2, {$3$} 3, {$4$} 4}
\normalsize
\dashed \polyline {(6,1.5), (-4,1.5)}
\dashed \polyline {(6,-4.5), (-4,-4.5)}
\penwd{1.25pt}
\arrow \reverse \arrow \parafcn{-2.8,2.8,0.1}{( (-0.5*(tan( 0.5236*t)))+1   ,   t -1.5 )}
\point[4pt]{(1,-1.5), (0.5,0), (1.5,-3)}
\end{mfpic}
 


\begin{solution}
\end{solution}

\end{question}

\begin{question}
The graph of $y = f(x) = \sqrt[3]{x}$ is given below on the left and the graph of $y = g(x)$ is given on the right. Find a formula for $g$ based on transformations of the graph of $f$.  Check your answer by confirming that the points shown on the graph of $g$ satisfy the equation $y = g(x)$.

\[ \begin{array}{cc}

% \input{ExercisesforTransformations_pic12.tex}
\begin{mfpic}[10]{-12}{9}{-6}{6}
\axes
\tlabel[cc](9,-0.5){\scriptsize $x$}
\tlabel[cc](0.5,6){\scriptsize $y$}
\tlabel[cc](-8,-3){\scriptsize $(-8,-2)$}
\gclear \tlabelrect(-1,-2){\scriptsize $(-1,-1)$ \vphantom{$\dfrac{3}{2}$}}
\tlabel[cc](-1.5,0.5){\scriptsize $(0,0)$}
\tlabel[cc](1,2){\scriptsize $(1,1)$}
\tlabel[cc](8, 3){\scriptsize $(8,2)$}
%\xmarks{-11 step 1 until 8}
%\ymarks{-5 step 1 until 5}
\tlpointsep{4pt}
%\axislabels {x}{{\tiny $-11 \hspace{6pt}$} -11, {\tiny $-10 \hspace{6pt}$} -10, {\tiny $-9 \hspace{6pt}$} -9, {\tiny $-8 \hspace{6pt}$} -8, {\tiny $-7 \hspace{6pt}$} -7, {\tiny $-6 \hspace{6pt}$} -6, {\tiny $-5 \hspace{6pt}$} -5, {\tiny $-4 \hspace{6pt}$} -4, {\tiny $-3 \hspace{6pt}$} -3, {\tiny $-2 \hspace{6pt}$} -2, {\tiny $-1 \hspace{6pt}$} -1, {\tiny $1$} 1, {\tiny $2$} 2, {\tiny $3$} 3, {\tiny $4$} 4, {\tiny $5$} 5, {\tiny $6$} 6, {\tiny $7$} 7, {\tiny $8$} 8}
%\axislabels {y}{{\tiny $-5$} -5, {\tiny $-4$} -4, {\tiny $-3$} -3, {\tiny $-2$} -2, {\tiny $-1$} -1, {\tiny $1$} 1, {\tiny $2$} 2, {\tiny $3$} 3, {\tiny $4$} 4, {\tiny $5$} 5}
\penwd{1.25pt}
\arrow \reverse \arrow \parafcn{-2.1,2.1,0.1}{(t**3,t)}
\point[4pt]{(0,0), (-1, -1), (1, 1), (-8, -2), (8, 2)}
\tcaption{\scriptsize $y = \sqrt[3]{x}$}
\end{mfpic}


&

% \input{ExercisesforTransformations_pic13.tex}
\begin{mfpic}[10]{-12}{9}{-6}{6}
\axes
\tlabel[cc](9,-0.5){\scriptsize $x$}
\tlabel[cc](0.5,6){\scriptsize $y$}
\tlabel[cc](-11,4){\scriptsize $(-11,3)$}
\tlabel[cc](-4,2){\scriptsize $(-4,1)$}
\tlabel[cc](-5,-1){\scriptsize $(-3,-1)$}
\tlabel[cc](-2.5,-4){\scriptsize $(-2,-3)$}
\tlabel[cc](5,-4){\scriptsize $(5,-5)$}
%\xmarks{-11 step 1 until 8}
%\ymarks{-5 step 1 until 5}
\tlpointsep{4pt}
%\axislabels {x}{{\tiny $-11 \hspace{6pt}$} -11, {\tiny $-10 \hspace{6pt}$} -10, {\tiny $-9 \hspace{6pt}$} -9, {\tiny $-8 \hspace{6pt}$} -8, {\tiny $-7 \hspace{6pt}$} -7, {\tiny $-6 \hspace{6pt}$} -6, {\tiny $-5 \hspace{6pt}$} -5, {\tiny $-4 \hspace{6pt}$} -4, {\tiny $-3 \hspace{6pt}$} -3, {\tiny $-2 \hspace{6pt}$} -2, {\tiny $-1 \hspace{6pt}$} -1, {\tiny $1$} 1, {\tiny $2$} 2, {\tiny $3$} 3, {\tiny $4$} 4, {\tiny $5$} 5, {\tiny $6$} 6, {\tiny $7$} 7, {\tiny $8$} 8}
%\axislabels {y}{{\tiny $-5$} -5, {\tiny $-4$} -4, {\tiny $-3$} -3, {\tiny $-2$} -2, {\tiny $-1$} -1, {\tiny $1$} 1, {\tiny $2$} 2, {\tiny $3$} 3, {\tiny $4$} 4, {\tiny $5$} 5}
\penwd{1.25pt}
\arrow \reverse \arrow \parafcn{-2.1,2.1,0.1}{((t**3 - 3),((-2*t) - 1))}
\point[4pt]{(-11,3), (-4,1), (-3,-1), (-2,-3), (5,-5)}
\tcaption{\scriptsize $y = g(x)$}
\end{mfpic}


\end{array} \]
\begin{solution}
$~$


 
\[ \begin{array}{|c||c|c|}

\hline

f(x) & |f(x)| & f\left(|x| \right)  \\ \hline

x+2 &     |x+2|   &       |x|+2      \\ \hline

x^2-4x &    |x^2-4x|      &     |x|^2-3|x| = x^2-4|x|      \\  \hline

x^3-3x^2 &    |x^3-3x^2|     &  |x|^3-3|x|^2 = |x|^3-3x^2      \\  \hline  

(x+1)^{-1}  &     |(x+1)^{-1}|      &    (|x|+1)^{-1}       \\  \hline   

\sqrt{x+2}-3&    | \sqrt{x+2}-3 |       &    \sqrt{|x|+2}-3    \\  \hline   

 \end{array} \]
\end{solution}

\end{question}

\begin{question}
Show that the composition of two linear functions is  a linear function. Hence any (finite) sequence of transformations discussed in this section can be combined into the form given in Theorem \ref{transformationsthm}.

(HINT:  Let $f(x) = ax +b$ and $g(x) = cx + d$.  Find $(f \circ g)(x)$.)
\begin{solution}
\end{solution}

\end{question}

\begin{question}
For many common functions, the properties of Algebra make a horizontal scaling the same as a vertical scaling by (possibly) a different factor.  For example,  $\sqrt{9x} = 3\sqrt{x}$, so a horizontal compression of $y = \sqrt{x}$ by a factor of $9$ results in the same graph as a vertical stretch of $y = \sqrt{x}$ by a factor of $3$.  

With the help of your classmates, find the equivalent vertical scaling produced by the horizontal scalings $y = (2x)^{3}, \, y = |5x|, \, y = \sqrt[3]{27x} \, $ and $\, y = \left(\frac{1}{2} x\right)^{2}$.  

What about $y = (-2x)^{3}, \, y = |-5x|, \, y = \sqrt[3]{-27x}\, $ and $\, y = \left(-\frac{1}{2} x\right)^{2}$?

\newpage
\begin{solution}
To graph $y=|f(x)|$ from the graph of  $y=f(x)$, reflect about the $x$-axis any portion of the graph of $y=f(x)$ which is below the $x$-axis.
\end{solution}

\end{question}

\begin{question}
Discuss the following questions with your classmates.

\begin{solution}
If the graph is below the $x$-axis, then $f(x) < 0$.  Since $|f(x)| = -f(x)$ if $f(x) < 0$, we are graphing $y=-f(x)$ for these values of $x$ which is a reflection across the $x$-axis.
\end{solution}

\end{question}

\begin{question}
If $f$ is even, what happens when you reflect the graph of $y = f(x)$ across the $y$-axis?
\begin{solution}
To  graph  $y=f(|x|)$ from the graph of $y=f(x)$, replace the graph of $y=f(x)$ for $x \leq 0$ with the reflection about the $y$-axis of the graph of $y=f(x)$ for $x \geq 0$.
\end{solution}

\end{question}

\begin{question}
If $f$ is odd, what happens when you reflect the graph of $y = f(x)$ across the $y$-axis?
\begin{solution}
If $x < 0$, then $|x| = -x$, so $f(|x|) = f(-x)$.  Since if $x < 0$, $-x > 0$,  this  means we reflect the graph of $y=f(x)$ about the $y$-axis for $x>0$ only.
\end{solution}

\end{question}

\begin{question}
If $f$ is even, what happens when you reflect the graph of $y = f(x)$ across the $x$-axis?
\end{question}

\begin{question}
If $f$ is odd, what happens when you reflect the graph of $y = f(x)$ across the $x$-axis?
\end{question}

\begin{question}
How would you describe symmetry about the origin in terms of reflections?
\end{question}

\begin{question}
We mentioned earlier in the section that, in general, the order in which transformations are applied matters, yet in our first example with two transformations the order did not matter. (You could perform the shift to the left followed by the shift down or you could shift down and then left to achieve the same result.)  With the help of your classmates, determine the situations in which order does matter and those in which it does not.
\end{question}

\begin{question}
This Exercise is a follow-up to Exercise \ref{makeaveewithabsval} in Section \ref{AbsoluteValueFunctions}.

\end{question}

\begin{question}
Fill in the table below.

\[ \begin{array}{|c||c|c|}

\hline

f(x) & |f(x)| & f\left(|x| \right)  \\ \hline

x+2 &      \hphantom{\sqrt{x+3}-2}   &         \hphantom{\sqrt{x+3}-2}      \\ \hline

x^2-4x &      \hphantom{\sqrt{x+3}-2}      &      \hphantom{\sqrt{x+3}-2}       \\  \hline

x^3-3x^2 &     \hphantom{\sqrt{x+3}-2}       &    \hphantom{\sqrt{x+3}-2}       \\  \hline  

(x+1)^{-1}  &      \hphantom{\sqrt{x+3}-2}      &    \hphantom{\sqrt{x+3}-2}       \\  \hline   

\sqrt{x+2}-3&    \hphantom{\sqrt{x+3}-2}        &     \hphantom{\sqrt{x+3}-2}      \\  \hline   

 \end{array} \]
 
% \[ \begin{array}{|c||c|c|}

%\hline

%f(x) & |f(x)| & f\left(|x| \right)  \\ \hline

%x+2 &     |x+2|   &       |x|+2      \\ \hline

%x^2-4x &    |x^2-4x|      &     |x|^2-3|x| = x^2-4|x|      \\  \hline

%x^3-3x^2 &    |x^3-3x^2|     &  |x|^3-3|x|^2 = |x|^3-3x^2      \\  \hline  

%(x+1)^{-1}  &     |(x+1)^{-1}|      &    (|x|+1)^{-1}       \\  \hline   

%\sqrt{x+2}-3&    | \sqrt{x+2}-3 |       &    \sqrt{|x|+2}-3    \\  \hline   

% \end{array} \]
\end{question}

\begin{question}
For each function $f$ above, graph $y = f(x)$ and $y=|f(x)|$ using a graphing utility.

\end{question}

\begin{question}
Write a sentence (or two!) explaining how to obtain the graph of $y=|f(x)|$ from $y = f(x)$.  

%To obtain the graph of $y=|f(x)|$ from $y=f(x)$, reflect about the $x$-axis any portion of the graph of $y=f(x)$ which is below the $x$-axis.
\end{question}

\begin{question}
How does your explanation relate to Definition \ref{absolutevaluepiecewise}?

%If the graph is below the $x$-axis, then $f(x) < 0$.  Since $|f(x)| = -f(x)$ if $f(x) < 0$, we are graphing $y=-f(x)$ for these values of $x$ which is a reflection across the $x$-axis.
\end{question}

\begin{question}
Write a sentence (or two!) explaining how to obtain the graph of $y=f(|x|)$ from $y = f(x)$.  

%To obtain the graph of $y=f(|x|)$ from $y=f(x)$, replace the portion of the graph of $y=f(x)$ for $x \leq 0$ with the reflection about the $y$-axis of the portion of the graph of $y=f(x)$ for $x \geq 0$.
\end{question}

\begin{question}
How does your explanation relate to Definition \ref{absolutevaluepiecewise}?

%If $x < 0$, then $|x| = -x$, so $f(|x|) = f(-x)$.  Since if $x < 0$, $-x > 0$,  this  means we reflect the graph of $y=f(x)$ about the $y$-axis for $x>0$ only.
\end{question}

\end{document}