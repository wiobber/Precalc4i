\documentclass{ximera}

\begin{document}
	\author{Stitz-Zeager}
	\xmtitle{Exercises for Inverse Functions}{}

\mfpicnumber{1} \opengraphsfile{ExercisesforInverseFunctions} % mfpic settings added 


\label{ExercisesforInverseFunctions}

In Exercises \ref{verifyinversehwfirst} - \ref{verifyinversehwlast}, verify the given pairs of functions are inverses algebraically and graphically.  



% Transformed Exercises with Solutions

\begin{question}
$f(x) = 2x+7$ and $g(x) = \dfrac{x-7}{2}$
\begin{solution}
$f^{-1}(x) = \dfrac{x + 2}{6}$
\end{solution}

\end{question}

\begin{question}
$f(x) = \dfrac{5-3x}{4}$ and $g(x) = -\dfrac{4}{3} x + \dfrac{5}{3}$.


\begin{solution}
$f^{-1}(x) = 42-x$

\end{solution}

\end{question}

\begin{question}
$f(t) = \dfrac{5}{t-1}$ and $g(t) = \dfrac{t+5}{t}$
\begin{solution}
$g^{-1}(t) = 3t-10$
\end{solution}

\end{question}

\begin{question}
$f(t)  = \dfrac{t}{t-1}$ and $g(t) = f(t) =  \dfrac{t}{t-1}$


\begin{solution}
$g^{-1}(t)  = -\frac{5}{3} t + \frac{1}{3}$


\end{solution}

\end{question}

\begin{question}
$f(x) = \sqrt{4-x}$ and $g(x) = -x^2+4$, $x \geq 0$
\begin{solution}
$f^{-1}(x) = \frac{1}{3}(x-5)^2+\frac{1}{3}$, $x \geq 5$
\end{solution}

\end{question}

\begin{question}
$f(x) = 1-\sqrt{x+1}$ and $g(x) = x^2-2x$, $x \leq 1$.

\begin{solution}
$f^{-1}(x) = (x - 2)^{2} + 5, \; x \leq 2$

\end{solution}

\end{question}

\begin{question}
$f(t) = (t-1)^3+5$ and $g(t) = \sqrt[3]{t-5}+1$
\begin{solution}
$g^{-1}(t) = \frac{1}{9}(t+4)^2+1$, $t \geq -4$
\end{solution}

\end{question}

\begin{question}
$f(t) = -\sqrt[4]{t-2}$ and $g(t) = t^4+2$, $t \leq 0$.  


\begin{solution}
$g^{-1}(t) = \frac{1}{8}(t-1)^2-\frac{5}{2}$, $t \leq 1$

\end{solution}

\end{question}

\begin{question}
$f(x) = 6x - 2$
\begin{solution}
$f^{-1}(x) = \frac{1}{3} x^{5} + \frac{1}{3}$
\end{solution}

\end{question}

\begin{question}
$f(x) = 42-x$


\begin{solution}
$f^{-1}(x) = -(x-3)^3+2$

\end{solution}

\end{question}

\begin{question}
$g(t) = \dfrac{t-2}{3} + 4$
\begin{solution}
$g^{-1}(t) = 5 + \sqrt{t+25}$
\end{solution}

\end{question}

\begin{question}
$g(t)  = 1 - \dfrac{4+3t}{5}$


\begin{solution}
$g^{-1}(t) = -\sqrt{\frac{t + 5}{3}} - 4$

\end{solution}

\end{question}

\begin{question}
$f(x) = \sqrt{3x-1}+5$
\begin{solution}
$f^{-1}(x) = 3 - \sqrt{x+4}$
\end{solution}

\end{question}

\begin{question}
$f(x) = 2-\sqrt{x - 5}$

\begin{solution}
$f^{-1}(x) =-\frac{\sqrt{x}+1}{2}$, $x > 1$

\end{solution}

\end{question}

\begin{question}
$g(t) = 3\sqrt{t-1}-4$
\begin{solution}
$g^{-1}(t) = \dfrac{4t-3}{t}$
\end{solution}

\end{question}

\begin{question}
$g(t) = 1 - 2\sqrt{2t+5}$


\begin{solution}
$g^{-1}(t) = \dfrac{t}{3t+1}$

\end{solution}

\end{question}

\begin{question}
$f(x) = \sqrt[5]{3x-1}$
\begin{solution}
$f^{-1}(x) = \dfrac{4x+1}{2-3x}$
\end{solution}

\end{question}

\begin{question}
$f(x) = 3-\sqrt[3]{x-2}$

\begin{solution}
$f^{-1}(x) = \dfrac{6x + 2}{3x - 4}$

\end{solution}

\end{question}

\begin{question}
$g(t) = t^2 - 10t$, $t \geq 5$
\begin{solution}
$g^{-1}(t) = \dfrac{-3t - 2}{t + 3}$
\end{solution}

\end{question}

\begin{question}
$g(t) = 3(t + 4)^{2} - 5, \; t \leq -4$

\begin{solution}
$g^{-1}(t) = \dfrac{t-2}{2t-1}$ 

\end{solution}

\end{question}

\begin{question}
$f(x) = x^2-6x+5, \; x \leq 3$
\begin{solution}
\end{solution}

\end{question}

\begin{question}
$f(x) = 4x^2 + 4x + 1$, $x < -1$

\begin{solution}
None of the first coordinates of the ordered pairs in $F$ are repeated, so $F$ is a function and none of the second coordinates of the ordered pairs of $F$ are repeated, so $F$ is one-to-one.   $F^{-1} = \{ (0,0), (1,1), (-1,2), (2,3), (-2,4), (3,5), (-3,6)  \}$
\end{solution}

\end{question}

\begin{question}
$g(t) = \dfrac{3}{4-t}$
\begin{solution}
Because of the `$\ldots$' it is helpful to determine a formula for the matching. For the even numbers $n$, $n = 0, 2, 4, \ldots$, the ordered pair $\left(n, -\frac{n}{2} \right)$ is in $G$.  For the odd numbers  $n = 1, 3, 5, \ldots$, the ordered pair $\left(n, \frac{n+1}{2} \right)$ is in $G$.  Hence, given any input to $G$, $n$, whether it be even or odd, there is only one output from $G$, either $-\frac{n}{2}$ or $\frac{n+1}{2}$, both of which are functions of $n$. To show $G$ is one to one, we note that if the output from $G$ is $0$ or less, then it must be of the form $-\frac{n}{2}$ for an even number $n$.  Moreover, if $-\frac{n}{2} = -\frac{m}{2}$, then $n = m$. In the case we are looking at outputs from $G$ which are greater than $0$, then it must be of the form $\frac{n+1}{2}$ for an odd number $n$.  In this, too, if  $\frac{n+1}{2} = \frac{m+1}{2}$, then $n = m$.  Hence, in any case, if the outputs from $G$ are the same, then the inputs to $G$ had to be the same so  $G$ is one-to-one and $G^{-1} = \{ (0,0), (1,1), (-1,2), (2,3), (-2,4), (3,5), (-3,6), \ldots \}$
\end{solution}

\end{question}

\begin{question}
$g(t) = \dfrac{t}{1-3t}$

\begin{solution}
To show $P$ is a function we note that if we have the same inputs to $P$, say $2t^{5} = 2u^{5}$, then $t = u$.  Hence the corresponding outputs, $2t-1$ and $3u-1$, are equal, too. To show $P$ is one-to-one, we note that if we have the same outputs from $P$, $3t-1 = 3u-1$, then $t = u$.  Hence, the corresponding  inputs $2t^5$  and $2u^5$ are equal, too. Hence $P$ is one-to-one and $P^{-1} = \{ (3t-1, 2t^5) \, | \, \text{$t$ is a real number.} \}$
\end{solution}

\end{question}

\begin{question}
$f(x) = \dfrac{2x-1}{3x+4}$
\begin{solution}
To show $Q$ is a function, we note that if we have the same inputs to $Q$, say $n = m$, then the outputs from $Q$, namely $n^2$ and $m^2$ are equal. To show $Q$ is one-to-one, we note that if we get the same output from $Q$, namely $n^2 = m^2$, then $n = \pm m$.  However since $n$ and $m$ are \textit{natural} numbers, both $n$ and $m$ are positive so $n = m$. Hence $Q$ is one-to-one and $Q^{-1} = \{ (n^2, n) \, | \, \text{$n$ is a \textit{natural} number.} \}$.
\end{solution}

\end{question}

\begin{question}
$f(x) = \dfrac{4x + 2}{3x - 6}$

\begin{solution}
$y = f^{-1}(x)$. Asymptote: $x = 0$.

% \input{ExercisesforInverseFunctions_pic5.tex}
\begin{tikzpicture}
\begin{axis}[fplot, xmin=-0.5, xmax=5.5, ymin=-3.5, ymax=3.5]
  \addplot[fpplot, domain=-2.5:2.5] ({2^t},{t});
  \node[flabel, label=below right:{(1,0)}] at (axis cs:1,0) {};
  \node[flabel, label=above left:{(2,1)}]  at (axis cs:2,1) {};
  \node[flabel, label=above left:{(4,2)}]  at (axis cs:4,2) {};
\end{axis}
\end{tikzpicture}
\end{solution}

\end{question}

\begin{question}
$g(t) = \dfrac{-3t - 2}{t + 3}$
\begin{solution}
$y = g^{-1}(t)$. Asymptote: $y=2$.

% \input{ExercisesforInverseFunctions_pic6.tex}
\begin{tikzpicture}
\begin{axis}[fplot, xmin=-6, xmax=6, ymin=-4, ymax=3]
  \addplot[fpplot, domain=-2.5:2.5] ({2*t},{2-2^t});
  \addplot[dashed, domain=-6:6] {2};
  \node[flabel, label=above left:{(0,1)}]    at (axis cs:0,1) {};
  \node[flabel, label=below right:{(2,0)}]   at (axis cs:2,0) {};
  \node[flabel, label=below right:{(4,-2)}]  at (axis cs:4,-2) {};
\end{axis}
\end{tikzpicture}


\end{solution}

\end{question}

\begin{question}
$g(t) = \dfrac{t-2}{2t-1}$  

\begin{solution}
$y = S^{-1}(t)$. Domain $[-3,3]$.

% \input{ExercisesforInverseFunctions_pic7.tex}
\begin{tikzpicture}
\begin{axis}[fplot, xmin=-4, xmax=4, ymin=-5, ymax=5]
  \addplot[fpplot, domain=-4:4] ({3*sin(deg(pi*t/8))},{t});
  \node[flabel, label=below left:{(-3,-4)}] at (axis cs:-3,-4) {};
  \node[flabel, label=above right:{(0,0)}]   at (axis cs:0,0) {};
  \node[flabel, label=above right:{(3,4)}]   at (axis cs:3,4) {};
\end{axis}
\end{tikzpicture}
\end{solution}

\end{question}

\begin{question}
Explain why each set of ordered pairs  below represents a  one-to-one function and find the inverse.


\begin{solution}
$y = R^{-1}(s)$.  Asymptotes: $s = \pm 3$.

% \input{ExercisesforInverseFunctions_pic8.tex}
\begin{tikzpicture}
\begin{axis}[fplot, xmin=-4, xmax=4, ymin=-5, ymax=5]
  \addplot[fpplot, domain=-2.8:2.8] ({t},{0.5*tan(0.5236*t)});
  \addplot[dashed] coordinates {(-3,-5) (-3,5)};
  \addplot[dashed] coordinates {( 3,-5) ( 3,5)};
  \node[flabel, label=above left:{(0,0)}]        at (axis cs:0,0) {};
  \node[flabel, label=above left:{(3/2,1/2)}]    at (axis cs:1.5,0.5) {};
  \node[flabel, label=below right:{(-3/2,-1/2)}] at (axis cs:-1.5,-0.5) {};
\end{axis}
\end{tikzpicture}
 

\end{solution}

\end{question}

\begin{question}
$F = \{ (0,0), (1,1), (2,-1), (3,2), (4,-2), (5,3), (6,-3)  \}$
\begin{solution}
\end{solution}

\end{question}

\begin{question}
$G = \{ (0,0), (1,1), (2,-1), (3,2), (4,-2), (5,3), (6,-3), \ldots \}$  

NOTE:  The difference between $F$ and $G$ is the  `$\ldots$.'
\begin{solution}
$p^{-1}(x) = \frac{450-x}{15}$.  The domain of $p^{-1}$ is the range of $p$ which is $[0,450]$
\end{solution}

\end{question}

\begin{question}
$P = \{ (2t^5, 3t-1) \, | \, \text{$t$ is a real number.} \}$
\begin{solution}
$p^{-1}(105) = 23$. This means that if the price is set to $\$105$ then $23$ dOpis will be sold.
\end{solution}

\end{question}

\begin{question}
$Q = \{ (n, n^2) \, | \, \text{$n$ is a \textit{natural} number.} \}$\footnote{Recall this means $n = 0, 1, 2, \ldots$.}
\begin{solution}
$\left(P\circ p^{-1}\right)(x) = -\frac{1}{15} x^2 + \frac{110}{3} x - 5000$, $0 \leq x \leq 450$.  

\smallskip

The graph of $y = \left(P\circ p^{-1}\right)(x)$ is a parabola opening downwards with vertex $\left(275, \frac{125}{3}\right) \approx (275, 41.67)$.  This means that the maximum profit is a whopping $\$41.67$ when the price per dOpi is set to $\$275$.   At this price, we can produce and sell $p^{-1}(275) = 11.\overline{6}$ dOpis.  Since we cannot sell part of a system, we need to adjust the price to sell either $11$ dOpis or $12$ dOpis. We find $p(11) = 285$ and $p(12) = 270$, which means we set the price per dOpi at either $\$285$ or $\$270$, respectively.  The profits at these prices are $\left(P\circ p^{-1}\right)(285) = 35$ and  $\left(P\circ p^{-1}\right)(270) = 40$, so it looks as if the maximum profit is $\$40$ and it is made by producing and selling $12$ dOpis a week at a price of $\$270$ per dOpi.
\end{solution}

\end{question}

\begin{question}
$y = f(x)$  

% \input{ExercisesforInverseFunctions_pic1.tex}
\begin{tikzpicture}
\begin{axis}[fplot, xmin=-3, xmax=3, ymin=-0.5, ymax=5.5]
  \addplot[fgraph, domain=-2.5:2.5] {2^x};
  \addplot[dashed, domain=-3:3] {0};
  \node[flabel, label=above left:{(0,1)}]  at (axis cs:0,1) {};
  \node[flabel, label=above left:{(1,2)}]  at (axis cs:1,2) {};
  \node[flabel, label=above left:{(2,4)}]  at (axis cs:2,4) {};
  \node at (axis description cs:0.5,0.02) {Asymptote: $y = 0$.};
\end{axis}
\end{tikzpicture}
\begin{solution}
If $b =0$, then $m = \pm 1$.  If $b \neq 0$, then $m = -1$ and $b$ can be any real number.
\end{solution}

\end{question}

\begin{question}
$y = g(t)$ 

% \input{ExercisesforInverseFunctions_pic2.tex}
\begin{tikzpicture}
\begin{axis}[fplot, xmin=-4, xmax=3, ymin=-6, ymax=6]
  \addplot[fpplot, domain=-2.5:2.5] ({2-2^t},{2*t});
  \addplot[dashed, domain=-6:6] coordinates {(2,-6) (2,6)};
  \node[flabel, label=below right:{(1,0)}]  at (axis cs:1,0) {};
  \node[flabel, label=above left:{(0,2)}]   at (axis cs:0,2) {};
  \node[flabel, label=above left:{(-2,4)}]  at (axis cs:-2,4) {};
  \node at (axis description cs:0.5,0.02) {Asymptote: $t=2$.};
\end{axis}
\end{tikzpicture}


\end{question}

\begin{question}
$y = S(t) $

% \input{ExercisesforInverseFunctions_pic3.tex}
\begin{tikzpicture}
\begin{axis}[fplot, xmin=-5, xmax=5, ymin=-4, ymax=4]
  \addplot[fgraph, domain=-4:4] {3*sin(deg(pi*x/8))};
  \node[flabel, label=below left:{(-4,-3)}] at (axis cs:-4,-3) {};
  \node[flabel, label=above right:{(0,0)}]  at (axis cs:0,0) {};
  \node[flabel, label=above right:{(4,3)}]  at (axis cs:4,3) {};
  \node at (axis description cs:0.5,0.02) {Domain: $[-4,4]$.};
\end{axis}
\end{tikzpicture}
\end{question}

\begin{question}
$y = R(s)$ 

% \input{ExercisesforInverseFunctions_pic4.tex}
\begin{tikzpicture}
\begin{axis}[fplot, xmin=-5, xmax=5, ymin=-4, ymax=4]
  \addplot[fpplot, domain=-2.8:2.8] ({0.5*tan(0.5236*t)},{t});
  \addplot[dashed, domain=-5:5] {3};
  \addplot[dashed, domain=-5:5] {-3};
  \node[flabel, label=above left:{(0,0)}]        at (axis cs:0,0) {};
  \node[flabel, label=above left:{(1/2,3/2)}]    at (axis cs:0.5,1.5) {};
  \node[flabel, label=below right:{(-1/2,-3/2)}] at (axis cs:-0.5,-1.5) {};
  \node at (axis description cs:0.5,0.02) {Asymptotes: $y=\pm 3$.};
\end{axis}
\end{tikzpicture}
 

\end{question}

\begin{question}
The price of a dOpi media player, in dollars per dOpi, is given as a function of the weekly sales $x$ according to the formula $p(x) = 450-15x$ for $0 \leq x \leq 30$.

\end{question}

\begin{question}
Find $p^{-1}(x)$ and state its domain.
\end{question}

\begin{question}
Find and interpret $p^{-1}(105)$.
\end{question}

\begin{question}
The profit (in dollars) made from producing and selling $x$ dOpis per week is given by the formula $P(x)= -15x^2+350x-2000$, for $0 \leq x \leq 30$.  Find $\left(P \circ p^{-1}\right)(x)$ and determine what price per dOpi would yield the maximum profit.  What is the maximum profit?  How many dOpis need to be produced and sold to achieve the maximum profit?
\end{question}

\begin{question}
Graph $y = f(x)$ using the techniques  in Section \ref{RationalGraphs}.  Check your answer using a graphing utility.
\end{question}

\begin{question}
Verify that $f$ is one-to-one on the interval $(-1,1)$.
\end{question}

\begin{question}
Use the procedure outlined on Page \pageref{inverseprocedure} to find the formula for $f^{-1}(x)$ for $-1 < x < 1$.
\end{question}

\begin{question}
Since $f(0) = 0$, it should be the case that $f^{-1}(0) = 0$.  What goes wrong when you attempt to substitute $x=0$ into $f^{-1}(x)$?  Discuss with your classmates how this problem arose and possible remedies.
\end{question}

\begin{question}
$f(x) = ax + b, \; a \neq 0$
\end{question}

\begin{question}
$f(x) = a\sqrt{x - h} + k, \; a \neq 0, x \geq h$


\end{question}

\begin{question}
$f(x) = ax^{2} + bx + c$ where $a \neq 0, \, x \geq -\dfrac{b}{2a}$.
\end{question}

\begin{question}
$f(x) = \dfrac{ax + b}{cx + d},\;$ (See Exercise \ref{whatconditions} below.) 

\end{question}

\begin{question}
What conditions must you place on the values of $a, b, c$ and $d$ in Exercise \ref{genericinverselast} in order to guarantee that the function is invertible?
\end{question}

\begin{question}
The function given in number \ref{owninverseexample} is an example of a function which is its own inverse.  

\end{question}

\begin{question}
Algebraically verify every function of the form: $f(x) = \dfrac{ax + b}{cx - a}$ is its own inverse.  

What assumptions do you need to make about the values of  $a$, $b$, and $c$?
\end{question}

\begin{question}
Under what conditions is $f(x) = mx + b$, $m \neq 0$ its own inverse?  Prove your answer.
\end{question}

\end{document}