\documentclass{ximera}

\begin{document}
	\author{Stitz-Zeager}
	\xmtitle{Exercises for Function Composition}{}

\mfpicnumber{1} \opengraphsfile{ExercisesforFunctionComposition} % mfpic settings added 


\label{ExercisesforFunctionComposition}


In Exercises \ref{funccompeval1first} - \ref{funccompeval1last}, use the given pair of functions to find the following values if they exist.

\begin{multicols}{3}

\begin{itemize}

\item  $(g\circ f)(0)$

\item  $(f\circ g)(-1)$

\item  $(f \circ f)(2)$

\end{itemize}

\end{multicols}

\begin{multicols}{3}

\begin{itemize}

\item  $(g\circ f)(-3)$

\item  $(f\circ g)\left(\frac{1}{2}\right)$

\item  $(f \circ f)(-2)$

\end{itemize}

\end{multicols}

\begin{multicols}{2}
\begin{enumerate}

\item  $f(x) = x^2$, $g(t) = 2t+1$ \label{funccompeval1first}
\item  $f(x) = 4-x$, $g(t) = 1-t^2$

\setcounter{HW}{\value{enumi}}
\end{enumerate}
\end{multicols}

\begin{multicols}{2}
\begin{enumerate}
\setcounter{enumi}{\value{HW}}

\item  $f(x) = 4-3x$, $g(t) = |t|$
\item  $f(x) = |x-1|$, $g(t) = t^2-5$

\setcounter{HW}{\value{enumi}}
\end{enumerate}
\end{multicols}

\begin{multicols}{2}
\begin{enumerate}
\setcounter{enumi}{\value{HW}}

\item  $f(x) = 4x+5$, $g(t) = \sqrt{t}$
\item  $f(x) = \sqrt{3-x}$, $g(t) = t^2+1$

\setcounter{HW}{\value{enumi}}
\end{enumerate}
\end{multicols}

\begin{multicols}{2}
\begin{enumerate}
\setcounter{enumi}{\value{HW}}

\item  $f(x) = 6-x-x^2$, $g(t) = t\sqrt{t+10}$
\item  $f(x) = \sqrt[3]{x+1}$, $g(t) = 4t^2-t$

\setcounter{HW}{\value{enumi}}
\end{enumerate}
\end{multicols}

\begin{multicols}{2}
\begin{enumerate}
\setcounter{enumi}{\value{HW}}

\item  $f(x) = \dfrac{3}{1-x}$, $g(t) = \dfrac{4t}{t^2+1}$
\item  $f(x) = \dfrac{x}{x+5}$, $g(t) = \dfrac{2}{7-t^2}$


\setcounter{HW}{\value{enumi}}
\end{enumerate}
\end{multicols}

\begin{multicols}{2}
\begin{enumerate}
\setcounter{enumi}{\value{HW}}

\item  $f(x) = \dfrac{2x}{5-x^2}$, $g(t) = \sqrt{4t+1}$
\item  $f(x) =\sqrt{2x+5}$, $g(t) = \dfrac{10t}{t^2+1}$ \label{funccompeval1last}

\setcounter{HW}{\value{enumi}}
\end{enumerate}
\end{multicols}


In Exercises \ref{funccompexp1first} - \ref{funccompexp1last}, use the given pair of functions to find and simplify expressions for the following functions and state the domain of each using interval notation.

\begin{multicols}{3}

\begin{itemize}

\item  $(g \circ f)(x)$

\item  $(f \circ g)(t)$

\item  $(f \circ f)(x)$


\end{itemize}

\end{multicols}


\begin{multicols}{2}
\begin{enumerate}
\setcounter{enumi}{\value{HW}}

\item  $f(x) = 2x+3$, $g(t) = t^2-9$ \label{funccompexp1first}
\item  $f(x) = x^2 -x+1$, $g(t) = 3t-5$ 

\setcounter{HW}{\value{enumi}}
\end{enumerate}
\end{multicols}

\begin{multicols}{2}
\begin{enumerate}
\setcounter{enumi}{\value{HW}}

\item  $f(x) = x^2-4$, $g(t) = |t|$
\item  $f(x) = 3x-5$, $g(t) = \sqrt{t}$ 

\setcounter{HW}{\value{enumi}}
\end{enumerate}
\end{multicols}

\begin{multicols}{2}
\begin{enumerate}
\setcounter{enumi}{\value{HW}}

\item  $f(x) = |x+1|$, $g(t) = \sqrt{t}$
\item  $f(x) = 3-x^2$, $g(t) = \sqrt{t+1}$ 

\setcounter{HW}{\value{enumi}}
\end{enumerate}
\end{multicols}

\begin{multicols}{2}
\begin{enumerate}
\setcounter{enumi}{\value{HW}}

\item  $f(x) = |x|$, $g(t) = \sqrt{4-t}$
\item  $f(x) = x^2-x-1$, $g(t) = \sqrt{t-5}$ 

\setcounter{HW}{\value{enumi}}
\end{enumerate}
\end{multicols}

\begin{multicols}{2}
\begin{enumerate}
\setcounter{enumi}{\value{HW}}

\item  $f(x) = 3x-1$, $g(t) = \dfrac{1}{t+3}$
\item  $f(x) = \dfrac{3x}{x-1}$, $g(t) =\dfrac{t}{t-3}$

\setcounter{HW}{\value{enumi}}
\end{enumerate}
\end{multicols}

\begin{multicols}{2}
\begin{enumerate}
\setcounter{enumi}{\value{HW}}

\item  $f(x) = \dfrac{x}{2x+1}$, $g(t) = \dfrac{2t+1}{t}$
\item  $f(x) =  \dfrac{2x}{x^2-4}$, $g(t) =\sqrt{1-t}$ 
 \label{funccompexp1last}

\setcounter{HW}{\value{enumi}}
\end{enumerate}
\end{multicols}

\enlargethispage{0.5in}

In Exercises \ref{threefunccompfirst} - \ref{threefunccomplast}, use $f(x) = -2x$, $g(t) = \sqrt{t}$ and $h(s) = |s|$ to find and simplify expressions for the following functions and state the domain of each using interval notation.

\begin{multicols}{3}

\begin{enumerate}
\setcounter{enumi}{\value{HW}}

\item $(h\circ g \circ f)(x)$ \label{threefunccompfirst}

\item $(h\circ f \circ g)(t)$

\item $(g\circ f \circ h)(s)$

\setcounter{HW}{\value{enumi}}
\end{enumerate}
\end{multicols}

\begin{multicols}{3}
\begin{enumerate}
\setcounter{enumi}{\value{HW}}

\item $(g\circ h \circ f)(x)$ 

\item $(f\circ h \circ g)(t)$

\item $(f\circ g \circ h)(s)$ \label{threefunccomplast}

\setcounter{HW}{\value{enumi}}
\end{enumerate}
\end{multicols}

\newpage

In Exercises \ref{pointcompexfirst} - \ref{pointcompexlast}, let $f$ be the function defined by \[f = \{(-3, 4), (-2, 2), (-1, 0), (0, 1), (1, 3), (2, 4), (3, -1)\}\] and let $g$ be the function defined by \[g = \{(-3, -2), (-2, 0), (-1, -4), (0, 0), (1, -3), (2, 1), (3, 2)\}.\]  Find the following, if it exists.

\begin{multicols}{3}
\begin{enumerate}
\setcounter{enumi}{\value{HW}}

\item $(f \circ g)(3)$ \label{pointcompexfirst}
\item $f(g(-1))$
\item $(f \circ f)(0)$

\setcounter{HW}{\value{enumi}}
\end{enumerate}
\end{multicols}

\begin{multicols}{3}
\begin{enumerate}
\setcounter{enumi}{\value{HW}}


\item $(f \circ g)(-3)$
\item $(g \circ f)(3)$
\item $g(f(-3))$


\setcounter{HW}{\value{enumi}}
\end{enumerate}
\end{multicols}

\begin{multicols}{3}
\begin{enumerate}
\setcounter{enumi}{\value{HW}}

\item $(g \circ g)(-2)$
\item $(g \circ f)(-2)$
\item $g(f(g(0)))$


\setcounter{HW}{\value{enumi}}
\end{enumerate}
\end{multicols}

\begin{multicols}{3}
\begin{enumerate}
\setcounter{enumi}{\value{HW}}

\item $f(f(f(-1)))$
\item $f(f(f(f(f(1)))))$
\item $\underbrace{(g \circ g \circ \cdots \circ g)}_{\mbox{$n$ times}}(0)$ 

\setcounter{HW}{\value{enumi}}
\end{enumerate}
\end{multicols}


\begin{enumerate}
\setcounter{enumi}{\value{HW}}

\item  Find the domain and range of $f \circ g$ and $g \circ f$. \label{pointcompexlast}


\setcounter{HW}{\value{enumi}}
\end{enumerate}


In Exercises \ref{twofuncgraphcompfirst} - \ref{twofuncgraphcomplast}, use the graphs of $y=f(x)$ and $y=g(x)$ below to find the following if it exists.

\begin{center}

\begin{tabular}{cc}

\begin{mfpic}[20]{-1}{5}{-1}{5}
\axes
\tlabel[cc](5,-0.5){\scriptsize $x$}
\tlabel[cc](0.5,5){\scriptsize $y$}
\tlabel[cc](-0.75,1){\scriptsize $(0,1)$}
\tlabel[cc](1,0.5){\scriptsize $(1,1)$}
\tlabel[cc](1.5,3.5){\scriptsize $(2,3)$}
\tlabel[cc](2.5,4.5){\scriptsize $(2.5,4.5)$}
\tlabel[cc](3.5,3.5){\scriptsize $(3,3)$}
\tlabel[cc](4,-0.5){\scriptsize $(4,0)$}
\xmarks{1,2,3,4}
\ymarks{1,2,3,4}
\tlpointsep{5pt}
\scriptsize
\axislabels {x}{{$1$} 1, {$2$} 2, {$3$} 3}
\axislabels {y} {{$2$} 2, {$3$} 3, {$4$} 4}
\penwd{1.25pt}
\polyline{(0,1), (1,1), (2,3), (2.5, 4),  (3,3), (4,0)}
\point[4pt]{(0,1), (1,1), (2,3), (2.5, 4), (3,3), (4,0)}
\normalsize 
\tcaption{$y = f(x)$}
\end{mfpic}

&

\hspace{1in}

\begin{mfpic}[20]{-1}{5}{-1}{5}
\axes
\tlabel[cc](5,-0.5){\scriptsize $x$}
\tlabel[cc](0.5,5){\scriptsize $y$}
\tlabel[cc](-0.5,-0.5){\scriptsize $(0,0)$}
\tlabel[cc](.5,3.5){\scriptsize $(1,3)$}
\tlabel[cc](2.5,3.5){\scriptsize $(2,3)$}
\tlabel[cc](3,-0.5){\scriptsize $(3,0)$}
\xmarks{1,2,3,4}
\ymarks{1,2,3,4}
\tlpointsep{5pt}
\scriptsize
\axislabels {x}{{$1$} 1, {$2$} 2,  {$4$} 4}
\axislabels {y}{{$1$} 1, {$2$} 2, {$3$} 3, {$4$} 4}
\penwd{1.25pt}
\polyline{(0,0), (1,3), (2,3), (3,0)}
\point[4pt]{(0,0), (1,3), (2,3), (3,0)}
\normalsize 
\tcaption{$y = g(x)$}
\end{mfpic}

\end{tabular}

\end{center}

\smallskip

\begin{multicols}{3}
\begin{enumerate}
\setcounter{enumi}{\value{HW}}

\item  $(g\circ f)(1)$ \label{twofuncgraphcompfirst}
\item  $(f \circ g)(3)$
\item  $(g\circ f)(2)$
\setcounter{HW}{\value{enumi}}
\end{enumerate}
\end{multicols}

\begin{multicols}{3}
\begin{enumerate}
\setcounter{enumi}{\value{HW}}
\item  $(f\circ g)(0)$  
\item  $(f\circ f)(4)$
\item  $(g \circ g)(1)$ 

\setcounter{HW}{\value{enumi}}
\end{enumerate}
\end{multicols}

\begin{enumerate}
\setcounter{enumi}{\value{HW}}

\item \label{twofuncgraphcomplast}  Find the domain and range of $f \circ g$ and $g \circ f$.

\setcounter{HW}{\value{enumi}}
\end{enumerate}

\newpage

In Exercises \ref{breakdowncompexfirst} - \ref{breakdownxomexlast},  write the given function as a composition of two or more non-identity functions.  (There are several correct answers, so check your answer using function composition.)

\begin{multicols}{2}
\begin{enumerate}
\setcounter{enumi}{\value{HW}}

\item  $p(x) = (2x+3)^3$ \label{breakdowncompexfirst}
\item  $P(x) = \left(x^2-x+1\right)^5$

\setcounter{HW}{\value{enumi}}
\end{enumerate}
\end{multicols}

\begin{multicols}{2}
\begin{enumerate}
\setcounter{enumi}{\value{HW}}

\item  $h(t) = \sqrt{2t-1}$
\item  $H(t) = |7-3t|$

\setcounter{HW}{\value{enumi}}
\end{enumerate}
\end{multicols}

\begin{multicols}{2}
\begin{enumerate}
\setcounter{enumi}{\value{HW}}

\item  $r(s) = \dfrac{2}{5s+1}$
\item  $R(s) = \dfrac{7}{s^2-1}$

\setcounter{HW}{\value{enumi}}
\end{enumerate}
\end{multicols}

\begin{multicols}{2}
\begin{enumerate}
\setcounter{enumi}{\value{HW}}

\item  $q(z) = \dfrac{|z|+1}{|z|-1}$
\item  $Q(z) = \dfrac{2z^3+1}{z^3-1}$

\setcounter{HW}{\value{enumi}}
\end{enumerate}
\end{multicols}

\begin{multicols}{2}
\begin{enumerate}
\setcounter{enumi}{\value{HW}}

\item  $v(x) = \dfrac{2x+1}{3-4x}$
\item  $w(x) = \dfrac{x^2}{x^4+1}$ \label{breakdownxomexlast}

\setcounter{HW}{\value{enumi}}
\end{enumerate}
\end{multicols}

\begin{enumerate}
\setcounter{enumi}{\value{HW}}

\item Write the function $F(x) = \sqrt{\dfrac{x^{3} + 6}{x^{3} - 9}}$ as a composition of three or more non-identity functions.

\item Let $g(x) = -x, \, h(x) = x + 2, \, j(x) = 3x$ and $k(x) = x - 4$.  In what order must these functions be composed with $f(x) = \sqrt{x}$ to create $F(x) = 3\sqrt{-x + 2} - 4$?

\item What linear functions could be used to transform $f(x) = x^{3}$ into $F(x) = -\frac{1}{2}(2x - 7)^{3} + 1$?  What is the proper order of composition?

\item Let $f(x) = 3x+1$ and let $g(x) =    \begin{mycases}  2x-1 &  \text{if $x \leq 3$} \\   4-x & \text{if $x > 3$} \\  \end{mycases}$.  Find expressions for $(f \circ g)(x)$ and $(g \circ f)(x)$.

\setcounter{HW}{\value{enumi}}
\end{enumerate}

\begin{enumerate}
\setcounter{enumi}{\value{HW}}


\item The volume $V$ of a cube is a function of its side length $x$.  Let's assume that $x = t + 1$ is also a function of time $t$, where $x$ is measured in inches and $t$ is measured in minutes.  Find a formula for $V$ as a function of $t$.

\item  Suppose a local vendor charges $\$2$ per hot dog and that the number of hot dogs sold per hour $x$ is given by $x(t) = -4t^2+20t+92$, where $t$ is the number of hours since $10$ AM, $0 \leq t \leq 4$.

\begin{enumerate}

\item  Find an expression for the revenue per hour $R$ as a function of $x$.
\item  Find and simplify $\left(R \circ x\right)(t)$.  What does this represent?
\item  What is the revenue per hour at noon?
\item Using Example \ref{surfaceareaex2chainrule} as a guide, verify $\frac{\Delta[R(x)]}{\Delta x} \cdot \frac{\Delta[x(t)]}{\Delta t} = \frac{\Delta[R(t)]}{\Delta t}$.

\end{enumerate}

\item  The book in Example \ref{dragforceex} plunges into a lake and generates a circular wave pattern.  If the waves are tracked as traveling at a constant $0.5$ meters per second $\left( \frac{\text{m}}{\text{s}}\right)$, use Theorem \ref{relatedratesaroc} to find the rate at which the area of the disturbance is changing with respect to time as the radius changes from $r = 1$ to $r = 1.1$ meters (m).  Be sure to include units on your answer.

\smallskip

\textbf{HINT:}  Recall the area, $A$, enclosed by a circle of radius $r$ is given by $A = \pi \, r^2$.  Here, $\frac{\Delta r}{\Delta t} = 0.5 \, \frac{\text{m}}{\text{s}}$.  

\item   Perfectly fine precalculus textbooks which have no Calculus content are being fed into a shredder at a rate of 2 books per minute in order to make room for precalculus textbooks with Calculus content.  The shredder creates a pile of debris which is in the shape of a right circular cone whose height is twice its width. 

\begin{enumerate}

\item  Assume the volume of the conical pile, $V$,  is given by $V = \frac{1}{3} \, \pi r^2 h$ where $r$ is the radius of the base of the pile and $h$ is the height of the pile.  Given the pile is twice as tall as it is wide, show we can write $V = \frac{4}{3} \, \pi r^3$.

\item  Assuming a typical precalculus textbook is $0.10$ cubic feet $\left( \text{ft}^3 \right)$, use Theorem \ref{relatedratesaroc} to find the rate of change of the radius of the pile with respect to time as the radius changes from $2$ to $2.1$ feet. Be sure to include units on your answer.

\end{enumerate}

\item Discuss with your classmates how `real-world' processes such as filling out federal income tax forms or computing your final course grade could be viewed as a use of function composition.  Find a process for which composition with itself (iteration) makes sense.

\end{enumerate}

\newpage

\subsection{Answers}


\begin{enumerate}

\item  For  $f(x) = x^2$ and $g(t) = 2t+1$,
\begin{multicols}{3}

\begin{itemize}

\item  $(g\circ f)(0) = 1$

\item  $(f\circ g)(-1) = 1$

\item  $(f \circ f)(2) = 16$

\end{itemize}

\end{multicols}

\begin{multicols}{3}

\begin{itemize}

\item  $(g\circ f)(-3) = 19$

\item  $(f\circ g)\left(\frac{1}{2}\right) = 4$

\item  $(f \circ f)(-2) = 16$

\end{itemize}

\end{multicols}

\item  For   $f(x) = 4-x$ and $g(t) = 1-t^2$,
\begin{multicols}{3}

\begin{itemize}

\item  $(g\circ f)(0) = -15$

\item  $(f\circ g)(-1) = 4$

\item  $(f \circ f)(2) = 2$

\end{itemize}

\end{multicols}

\begin{multicols}{3}

\begin{itemize}

\item  $(g\circ f)(-3) = -48$

\item  $(f\circ g)\left(\frac{1}{2}\right) = \frac{13}{4}$

\item  $(f \circ f)(-2) = -2$

\end{itemize}

\end{multicols}

\item  For   $f(x) = 4-3x$ and  $g(t) = |t|$,
\begin{multicols}{3}

\begin{itemize}

\item  $(g\circ f)(0) = 4$

\item  $(f\circ g)(-1) = 1$

\item  $(f \circ f)(2) = 10$

\end{itemize}

\end{multicols}

\begin{multicols}{3}

\begin{itemize}

\item  $(g\circ f)(-3) = 13$

\item  $(f\circ g)\left(\frac{1}{2}\right) = \frac{5}{2}$

\item  $(f \circ f)(-2) = -26$

\end{itemize}

\end{multicols}

\item  For   $f(x) = |x-1|$ and $g(t) = t^2-5$,
\begin{multicols}{3}

\begin{itemize}

\item  $(g\circ f)(0) = -4$

\item  $(f\circ g)(-1) = 5$

\item  $(f \circ f)(2) = 0$

\end{itemize}

\end{multicols}

\begin{multicols}{3}

\begin{itemize}

\item  $(g\circ f)(-3) = 11$

\item  $(f\circ g)\left(\frac{1}{2}\right) = \frac{23}{4}$

\item  $(f \circ f)(-2) = 2$

\end{itemize}

\end{multicols}

\item  For $f(x) = 4x+5$ and $g(t) = \sqrt{t}$,
\begin{multicols}{3}

\begin{itemize}

\item  $(g\circ f)(0) = \sqrt{5}$

\item  $(f\circ g)(-1)$ is not real

\item  $(f \circ f)(2) = 57$

\end{itemize}

\end{multicols}

\begin{multicols}{3}

\begin{itemize}

\item  $(g\circ f)(-3)$ is not real

\item  $(f\circ g)\left(\frac{1}{2}\right) = 5+2\sqrt{2}$

\item  $(f \circ f)(-2) = -7$

\end{itemize}

\end{multicols}

\item  For $f(x) = \sqrt{3-x}$ and $g(t) = t^2+1$,
\begin{multicols}{3}

\begin{itemize}

\item  $(g\circ f)(0) = 4$

\item  $(f\circ g)(-1) = 1$

\item  $(f \circ f)(2) = \sqrt{2}$

\end{itemize}

\end{multicols}

\begin{multicols}{3}

\begin{itemize}

\item  $(g\circ f)(-3) = 7$

\item  $(f\circ g)\left(\frac{1}{2}\right) = \frac{\sqrt{7}}{2}$

\item  $(f \circ f)(-2) = \sqrt{3 - \sqrt{5}}$

\end{itemize}

\end{multicols}

\enlargethispage{0.5in}

\item  For  $f(x) = 6-x-x^2$ and $g(t) = t\sqrt{t+10}$,
\begin{multicols}{3}

\begin{itemize}

\item  $(g\circ f)(0) = 24$

\item  $(f\circ g)(-1) = 0$

\item  $(f \circ f)(2) = 6$

\end{itemize}

\end{multicols}

\begin{multicols}{3}

\begin{itemize}

\item  $(g\circ f)(-3) = 0$

\item  $(f\circ g)\left(\frac{1}{2}\right) = \frac{27-2\sqrt{42}}{8}$

\item  $(f \circ f)(-2) = -14$

\end{itemize}

\end{multicols}

\newpage

\item  For  $f(x) = \sqrt[3]{x+1}$ and $g(t) = 4t^2-t$,
\begin{multicols}{3}

\begin{itemize}

\item  $(g\circ f)(0) = 3$

\item  $(f\circ g)(-1) = \sqrt[3]{6}$

\item  $(f \circ f)(2) = \sqrt[3]{\sqrt[3]{3}+1}$

\end{itemize}

\end{multicols}

\begin{multicols}{3}

\begin{itemize}

\item  $(g\circ f)(-3) = 4\sqrt[3]{4}+\sqrt[3]{2}$

\item  $(f\circ g)\left(\frac{1}{2}\right) = \frac{\sqrt[3]{12}}{2}$

\item  $(f \circ f)(-2) = 0$

\end{itemize}

\end{multicols}

\item  For  $f(x) = \frac{3}{1-x}$ and $g(t) = \frac{4t}{t^2+1}$,
\begin{multicols}{3}

\begin{itemize}

\item  $(g\circ f)(0) = \frac{6}{5}$

\item  $(f\circ g)(-1) = 1$

\item  $(f \circ f)(2) = \frac{3}{4}$

\end{itemize}

\end{multicols}

\begin{multicols}{3}

\begin{itemize}

\item  $(g\circ f)(-3) = \frac{48}{25}$

\item  $(f\circ g)\left(\frac{1}{2}\right) = -5$

\item  $(f \circ f)(-2)$ is undefined

\end{itemize}

\end{multicols}

\item  For  $f(x) = \frac{x}{x+5}$ and $g(t) = \frac{2}{7-t^2}$,
\begin{multicols}{3}

\begin{itemize}

\item  $(g\circ f)(0) = \frac{2}{7}$

\item  $(f\circ g)(-1) = \frac{1}{16}$

\item  $(f \circ f)(2) = \frac{2}{37}$

\end{itemize}

\end{multicols}

\begin{multicols}{3}

\begin{itemize}

\item  $(g\circ f)(-3) = \frac{8}{19}$

\item  $(f\circ g)\left(\frac{1}{2}\right) = \frac{8}{143}$

\item  $(f \circ f)(-2) = -\frac{2}{13}$

\end{itemize}

\end{multicols}

\item  For  $f(x) = \frac{2x}{5-x^2}$ and $g(t) = \sqrt{4t+1}$,
\begin{multicols}{3}

\begin{itemize}

\item  $(g\circ f)(0) = 1$

\item  $(f\circ g)(-1)$ is not real

\item  $(f \circ f)(2) = -\frac{8}{11}$

\end{itemize}

\end{multicols}

\begin{multicols}{3}

\begin{itemize}

\item  $(g\circ f)(-3) = \sqrt{7}$

\item  $(f\circ g)\left(\frac{1}{2}\right) = \sqrt{3}$

\item  $(f \circ f)(-2) = \frac{8}{11}$

\end{itemize}

\end{multicols}

\item  For  $f(x) =\sqrt{2x+5}$ and $g(t) = \frac{10t}{t^2+1}$ ,
\begin{multicols}{3}

\begin{itemize}

\item  $(g\circ f)(0) = \frac{5\sqrt{5}}{3}$

\item  $(f\circ g)(-1)$ is not real

\item  $(f \circ f)(2) = \sqrt{11}$

\end{itemize}

\end{multicols}

\begin{multicols}{3}

\begin{itemize}

\item  $(g\circ f)(-3)$ is not real

\item  $(f\circ g)\left(\frac{1}{2}\right) = \sqrt{13}$

\item  $(f \circ f)(-2) = \sqrt{7}$

\end{itemize}

\end{multicols}
\setcounter{HW}{\value{enumi}}
\end{enumerate}

\begin{enumerate}
\setcounter{enumi}{\value{HW}}

\item For $f(x) = 2x+3$ and $g(t) = t^2-9$

\begin{itemize}

\item  $(g \circ f)(x) = 4x^2+12x$, domain: $(-\infty, \infty)$

\item  $(f \circ g)(t) = 2t^2-15$, domain: $(-\infty, \infty)$

\item  $(f \circ f)(x) = 4x+9$, domain: $(-\infty, \infty)$

\end{itemize}


\item For  $f(x) = x^2 -x+1$ and $g(t) = 3t-5$ 

\begin{itemize}

\item  $(g \circ f)(x) = 3x^2-3x-2$, domain: $(-\infty, \infty)$

\item  $(f \circ g)(t) =9t^2-33t+31$, domain: $(-\infty, \infty)$

\item  $(f \circ f)(x) = x^4-2x^3+2x^2-x+1$, domain: $(-\infty, \infty)$

\end{itemize}

\item For  $f(x) = x^2-4$ and $g(t) = |t|$ 

\begin{itemize}

\item  $(g \circ f)(x) = |x^2-4|$, domain: $(-\infty, \infty)$

\item  $(f \circ g)(t) =|t|^2-4 = t^2-4$, domain: $(-\infty, \infty)$

\item  $(f \circ f)(x) =x^4-8x^2+12$, domain: $(-\infty, \infty)$

\end{itemize}

\item For   $f(x) = 3x-5$ and $g(t) = \sqrt{t}$ 

\begin{itemize}

\item  $(g \circ f)(x) = \sqrt{3x-5}$, domain: $\left[ \frac{5}{3}, \infty \right)$

\item  $(f \circ g)(t) = 3\sqrt{t}-5$, domain: $[0,\infty)$

\item  $(f \circ f)(x) = 9x-20$, domain: $(-\infty, \infty)$

\end{itemize}


\item For   $f(x) = |x+1|$ and $g(t) = \sqrt{t}$

\begin{itemize}

\item  $(g \circ f)(x) = \sqrt{|x+1|}$, domain: $(-\infty, \infty)$

\item  $(f \circ g)(t) = |\sqrt{t}+1| = \sqrt{t}+1$, domain: $[0,\infty)$

\item  $(f \circ f)(x) = ||x+1|+1| = |x+1|+1$, domain: $(-\infty, \infty)$

\end{itemize}


\item For   $f(x) = 3-x^2$ and $g(t) = \sqrt{t+1}$ 

\begin{itemize}

\item  $(g \circ f)(x) = \sqrt{4-x^2}$, domain: $[-2,2]$

\item  $(f \circ g)(t) =2-t$, domain: $[-1, \infty)$

\item  $(f \circ f)(x) = -x^4+6x^2-6$, domain: $(-\infty, \infty)$

\end{itemize}

\item For   $f(x) = |x|$ and $g(t) = \sqrt{4-t}$

\begin{itemize}

\item  $(g \circ f)(x) = \sqrt{4-|x|}$, domain: $[-4,4]$

\item  $(f \circ g)(t) =|\sqrt{4-t}| = \sqrt{4-t}$, domain: $(-\infty, 4]$

\item  $(f \circ f)(x) = | |x| | = |x|$, domain: $(-\infty, \infty)$

\end{itemize}


\enlargethispage{0.25in}

\item For  $f(x) = x^2-x-1$ and $g(t) = \sqrt{t-5}$ 

\begin{itemize}

\item  $(g \circ f)(x) = \sqrt{x^2-x-6}$, domain: $(-\infty, -2] \cup [3,\infty)$

\item  $(f \circ g)(t) =t-6-\sqrt{t-5}$, domain: $[5,\infty)$

\item  $(f \circ f)(x) =x^4-2x^3-2x^2+3x+1$, domain: $(-\infty, \infty)$

\end{itemize}


\item For   $f(x) = 3x-1$ and $g(t) = \frac{1}{t+3}$

\begin{itemize}

\item  $(g \circ f)(x) = \frac{1}{3x+2}$, domain: $\left(-\infty, -\frac{2}{3}\right) \cup \left(-\frac{2}{3}, \infty\right)$

\item  $(f \circ g)(t) = -\frac{t}{t+3}$, domain: $\left(-\infty, -3\right) \cup \left(-3, \infty\right)$

\item  $(f \circ f)(x) = 9x-4$, domain: $(-\infty, \infty)$

\end{itemize}


\item For   $f(x) = \frac{3x}{x-1}$ and $g(t) =\frac{t}{t-3}$

\begin{itemize}

\item  $(g \circ f)(x) =x$, domain: $\left(-\infty, 1\right) \cup (1, \infty)$

\item  $(f \circ g)(t) =t$, domain:  $\left(-\infty, 3\right) \cup (3,\infty)$

\item  $(f \circ f)(x) = \frac{9x}{2x+1}$, domain: $\left(-\infty, -\frac{1}{2}\right) \cup \left(-\frac{1}{2}, 1 \right) \cup \left(1,\infty \right)$

\end{itemize}


\item For    $f(x) = \frac{x}{2x+1}$ and $g(t) = \frac{2t+1}{t}$

\begin{itemize}

\item  $(g \circ f)(x) = \frac{4x+1}{x}$, domain: $\left(-\infty, -\frac{1}{2}\right) \cup \left(-\frac{1}{2}, 0), \cup (0, \infty\right)$

\item  $(f \circ g)(t) = \frac{2t+1}{5t+2}$, domain:  $\left(-\infty, -\frac{2}{5}\right) \cup \left(-\frac{2}{5}, 0\right) \cup (0,\infty)$

\item  $(f \circ f)(x) = \frac{x}{4x+1}$, domain: $\left(-\infty, -\frac{1}{2}\right) \cup \left(-\frac{1}{2}, -\frac{1}{4} \right) \cup \left(-\frac{1}{4},\infty\right)$

\end{itemize}


\item For  $f(x) = \frac{2x}{x^2-4}$ and $g(t) =\sqrt{1-t}$ 

\begin{itemize}

\item  $(g \circ f)(x) =\sqrt{\frac{x^2-2x-4}{x^2-4}}$, domain: $\left(-\infty, -2\right) \cup \left[1-\sqrt{5}, 2\right) \cup \left[1+\sqrt{5}, \infty\right)$

\item  $(f \circ g)(t) = -\frac{2\sqrt{1-t}}{t+3}$, domain: $\left(-\infty, -3\right) \cup \left(-3, 1\right]$

\item  $(f \circ f)(x) = \frac{4x-x^3}{x^4-9x^2+16}$, domain: $\left(-\infty, -\frac{1+\sqrt{17}}{2}\right) \cup \left(-\frac{1+\sqrt{17}}{2}, -2\right) \cup \left(-2, \frac{1-\sqrt{17}}{2}\right) \cup \left(\frac{1-\sqrt{17}}{2}, \frac{-1+\sqrt{17}}{2}\right) \cup \left(\frac{-1+\sqrt{17}}{2}, 2\right) \cup \left(2, \frac{1+\sqrt{17}}{2} \right) \cup \left(\frac{1+\sqrt{17}}{2}, \infty\right)$

\end{itemize}
\setcounter{HW}{\value{enumi}}
\end{enumerate}


\begin{enumerate}
\setcounter{enumi}{\value{HW}}

\item $(h\circ g \circ f)(x)= |\sqrt{-2x}|= \sqrt{-2x}$, domain: $(-\infty, 0]$ 

\item $(h\circ f \circ g)(t) = |-2\sqrt{t}|= 2\sqrt{t}$, domain: $[0,\infty)$

\item $(g\circ f \circ h)(s) = \sqrt{-2|s|}$, domain:  $\{0\}$

\item $(g\circ h \circ f)(x) = \sqrt{|-2x|} = \sqrt{2|x|}$, domain: $(-\infty, \infty)$ 

\item $(f\circ h \circ g)(t) = -2|\sqrt{t}| = -2\sqrt{t}$, domain: $[0,\infty)$

\item $(f\circ g \circ h)(s) = -2\sqrt{|s|}$, , domain: $(-\infty,\infty)$

\setcounter{HW}{\value{enumi}}
\end{enumerate}

\begin{multicols}{2}
\begin{enumerate}
\setcounter{enumi}{\value{HW}}

\item $(f \circ g)(3)= f(g(3)) = f(2) = 4$
\item $f(g(-1)) = f(-4)$ which is undefined

\setcounter{HW}{\value{enumi}}
\end{enumerate}
\end{multicols}

\begin{multicols}{2}
\begin{enumerate}
\setcounter{enumi}{\value{HW}}

\item $(f \circ f)(0) = f(f(0)) = f(1) = 3$
\item $(f \circ g)(-3) = f(g(-3)) = f(-2) = 2$

\setcounter{HW}{\value{enumi}}
\end{enumerate}
\end{multicols}

\begin{multicols}{2}
\begin{enumerate}
\setcounter{enumi}{\value{HW}}

\item $(g \circ f)(3) = g(f(3)) = g(-1) = -4$
\item $g(f(-3)) = g(4)$ which is undefined

\setcounter{HW}{\value{enumi}}
\end{enumerate}
\end{multicols}

\begin{multicols}{2}
\begin{enumerate}
\setcounter{enumi}{\value{HW}}

\item $(g \circ g)(-2) = g(g(-2)) = g(0) = 0$
\item $(g \circ f)(-2) = g(f(-2)) = g(2) = 1$

\setcounter{HW}{\value{enumi}}
\end{enumerate}
\end{multicols}

\begin{multicols}{2}
\begin{enumerate}
\setcounter{enumi}{\value{HW}}

\item $g(f(g(0))) = g(f(0)) = g(1) = -3$
\item $f(f(f(-1))) = f(f(0)) = f(1) = 3$

\setcounter{HW}{\value{enumi}}
\end{enumerate}
\end{multicols}

\begin{multicols}{2}
\begin{enumerate}
\setcounter{enumi}{\value{HW}}

\item $f(f(f(f(f(1))))) = f(f(f(f(3)))) =\\ f(f(f(-1))) = f(f(0))  = f(1) = 3$
\item $\underbrace{(g \circ g \circ \cdots \circ g)}_{\mbox{$n$ times}}(0) = 0$

\setcounter{HW}{\value{enumi}}
\end{enumerate}
\end{multicols}

\begin{enumerate}
\setcounter{enumi}{\value{HW}}

\item  \begin{itemize}  \item  The domain of $f \circ g$ is $\{ -3, -2, 0, 1, 2, 3\}$ and the range of $f \circ g$ is $\{1, 2, 3, 4\}$.
\item The domain of $g \circ f$ is $\{ -2, -1, 0, 1, 3 \}$ and the range of $g \circ f$ is $\{ -4, -3, 0, 1, 2 \}$.

\end{itemize}

\setcounter{HW}{\value{enumi}}
\end{enumerate}


\begin{multicols}{3}
\begin{enumerate}
\setcounter{enumi}{\value{HW}}

\item  $(g\circ f)(1) = 3$ 
\item  $(f \circ g)(3) = 1$
\item  $(g\circ f)(2) = 0$
\setcounter{HW}{\value{enumi}}
\end{enumerate}
\end{multicols}

\begin{multicols}{3}
\begin{enumerate}
\setcounter{enumi}{\value{HW}}
\item  $(f\circ g)(0) = 1$  
\item  $(f\circ f)(4) = 1$
\item  $(g \circ g)(1) = 0$

\setcounter{HW}{\value{enumi}}
\end{enumerate}
\end{multicols}

\begin{enumerate}
\setcounter{enumi}{\value{HW}}

\item  \begin{itemize} \item The domain of $f \circ g$ is $[0,3]$ and the range of $f \circ g$ is $[1, 4.5]$.
\item The domain of $g \circ f$ is $[0,2] \cup [3,4]$ and the range is $[0,3]$.

\end{itemize}

\setcounter{HW}{\value{enumi}}
\end{enumerate}

\begin{enumerate}
\setcounter{enumi}{\value{HW}}

\item  Let $f(x) = 2x+3$ and $g(x) = x^3$, then $p(x) = (g\circ f)(x)$.
\item Let $f(x) = x^2-x+1$ and $g(x) = x^5$,  $P(x) =(g\circ f)(x)$.
\item  Let $f(t) = 2t-1$ and $g(t) = \sqrt{t}$, then $h(t) = (g\circ f)(t)$.
\item Let $f(t) = 7-3t$ and $g(t) = |t|$, then  $H(t) =  (g\circ f)(t)$.
\item  Let $f(s) = 5s+1$ and $g(s) = \frac{2}{s}$, then $r(s) =(g\circ f)(s)$.
\item  Let $f(s) = s^2-1$ and $g(s) = \frac{7}{s}$, then $R(s) =(g\circ f)(s)$.
\item  Let $f(z) = |z|$ and $g(z) = \frac{z+1}{z-1}$, then  $q(z) =(g\circ f)(z)$.

\item Let $f(z) = z^3$ and $g(z)= \frac{2z+1}{z-1}$, then  $Q(z) =(g\circ f)(z)$.

\item Let $f(x) =2x$ and $g(x) = \frac{x+1}{3-2x}$, then  $v(x) =(g\circ f)(x)$.

\item  Let $f(x) = x^2$ and $g(x) = \frac{x}{x^2+1}$, then  $w(x) =(g\circ f)(x)$.

\setcounter{HW}{\value{enumi}}
\end{enumerate}

\begin{enumerate}
\setcounter{enumi}{\value{HW}}

\item $F(x) = \sqrt{\frac{x^{3} + 6}{x^{3} - 9}} = (h(g(f(x)))$ where $f(x) = x^{3}, \, g(x) = \frac{x + 6}{x - 9}$ and $h(x) = \sqrt{x}$.

\item $F(x) = 3\sqrt{-x + 2} - 4 = k(j(f(h(g(x)))))$

\item One solution is $F(x) = -\frac{1}{2}(2x - 7)^{3} + 1 = k(j(f(h(g(x)))))$ where $g(x) = 2x, \, h(x) = x - 7, \, j(x) = -\frac{1}{2}x$ and $k(x) = x + 1$.  You could also have $F(x) = H(f(G(x)))$ where $G(x) = 2x - 7$ and $H(x) = -\frac{1}{2}x + 1$.


\item $(f \circ g)(x) =    \begin{mycases} 6x-2 &  \text{if $x \leq 3$} \\   13-3x  & \text{if $x > 3$} \\  \end{mycases}$ and $(g \circ f)(x) =    \begin{mycases} 6x+1 &  \text{if $x \leq \frac{2}{3}$} \\   3-3x  & \text{if $x > \frac{2}{3}$} \\  \end{mycases}$


\setcounter{HW}{\value{enumi}}
\end{enumerate}



\begin{enumerate}
\setcounter{enumi}{\value{HW}}

\item $V(x) = x^{3}$ so $V(x(t)) = (t + 1)^{3}$

\item  \begin{enumerate}

\item  $R(x) = 2x$

\item  $\left(R \circ x \right)(t) =  -8t^2+40t+184$, $0 \leq t \leq 4$.  This gives the revenue per hour as a function of time.

\item  Noon corresponds to $t=2$, so $\left(R \circ x \right)(2) = 232$.  The hourly revenue at noon is $\$232$ per hour. 

\item $\frac{\Delta[R(x)]}{\Delta x} = 2$,  $\frac{\Delta[x(t)]}{\Delta t} = -8t + 4 \Delta t + 20$.   

 $\frac{\Delta[R(x)]}{\Delta x} \cdot \frac{\Delta[x(t)]}{\Delta t} = (2)(-8t + 4 \Delta t + 20) =   -16t + 8 \Delta t + 40 = \frac{\Delta[R(t)]}{\Delta t} \, \checkmark$
\end{enumerate}

\item  $\frac{\Delta A}{\Delta t} = \frac{\Delta A}{\Delta r} \cdot \frac{\Delta r}{\Delta t}$.  $\frac{\Delta A}{\Delta r} = \frac{A(1.1) - A(1)}{1.1-1} = \frac{\pi (1.1)^2 - \pi (1)^2}{0.1} = 2.1 \, \frac{\text{m}^2}{\text{m}}$,  $\frac{\Delta r}{\Delta t} = 0.5 \, \frac{\text{m}}{\text{s}}$.

Hence,  $\frac{\Delta A}{\Delta t} = \left( 2.1 \, \frac{\text{m}^2}{\text{m}} \right) \left(0.5 \, \frac{\text{m}}{\text{s}} \right) = 1.05 \,  \frac{\text{m}^2}{\text{s}}$

\item  \begin{enumerate}  \item  The `width' of the pile is the diameter of the circular base of the pile.  Since the diameter of a circle is twice the radius,  $h = 2 (2r) = 4r$.  Hence, $V = \frac{1}{3} \, \pi r^2 h = \frac{1}{3} \, \pi r^2 (4r) = \frac{4}{3} \, \pi r^3$.

\item  We have $\frac{\Delta V}{\Delta t} = \frac{\Delta V}{ \Delta r} \cdot \frac{\Delta r}{\Delta t}$.    Two textbooks per minute into the shredder amounts to the volume of the cone increasing at a rate of $2(0.1) = 0.2 \, \frac{\text{ft}^3}{\text{min}}$.  $\frac{\Delta V}{\Delta r}= \frac{V(2.1) - V(2)}{2.1 - 2} = 16.81\overline{3}  \, \pi \, \frac{\text{ft}^3}{\text{ft}}$.

Hence, $0.2 \, \frac{\text{ft}^3}{\text{min}} = 16.81\overline{3} \, \pi \, \frac{\text{ft}^3}{\text{ft}} \, \frac{\Delta r}{\Delta t}$ so $\frac{\Delta r}{\Delta t} = \frac{0.2}{ 16.81\overline{3} \, \pi }  \approx 0.0038 \, \frac{\text{ft}}{\text{min}}$.



\end{enumerate}

\end{enumerate}




\end{document}
