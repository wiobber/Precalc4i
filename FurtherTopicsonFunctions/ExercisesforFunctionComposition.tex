\documentclass{ximera}

\begin{document}
	\author{Stitz-Zeager}
	\xmtitle{Exercises for Function Composition}{}

\mfpicnumber{1} \opengraphsfile{ExercisesforFunctionComposition} % mfpic settings added 


\label{ExercisesforFunctionComposition}


\begin{question}
    
In Exercises \ref{funccompeval1first} - \ref{funccompeval1last}, use the given pair of functions to find the following values if they exist.




\begin{itemize}

\item  $(g\circ f)(0)$

\item  $(f\circ g)(-1)$

\item  $(f \circ f)(2)$

\end{itemize}



\begin{itemize}

\item  $(g\circ f)(-3)$

\item  $(f\circ g)\left(\frac{1}{2}\right)$

\item  $(f \circ f)(-2)$

\end{itemize}



\begin{problem}\label{funccompeval1first}
    $f(x) = x^2$, $g(t) = 2t+1$ 
\end{problem}

\begin{problem}

$f(x) = 4-x$, $g(t) = 1-t^2$
    
\end{problem}

\begin{problem}
    $f(x) = 4-3x$, $g(t) = |t|$
\end{problem}

\begin{problem}
    $f(x) = |x-1|$, $g(t) = t^2-5$
\end{problem}


\begin{multicols}{2}
\begin{enumerate}
\setcounter{enumi}{\value{HW}}

\item  $f(x) = 4x+5$, $g(t) = \sqrt{t}$
\item  $f(x) = \sqrt{3-x}$, $g(t) = t^2+1$

\setcounter{HW}{\value{enumi}}
\end{enumerate}
\end{multicols}

\begin{multicols}{2}
\begin{enumerate}
\setcounter{enumi}{\value{HW}}

\item  $f(x) = 6-x-x^2$, $g(t) = t\sqrt{t+10}$
\item  $f(x) = \sqrt[3]{x+1}$, $g(t) = 4t^2-t$

\setcounter{HW}{\value{enumi}}
\end{enumerate}
\end{multicols}

\begin{multicols}{2}
\begin{enumerate}
\setcounter{enumi}{\value{HW}}

\item  $f(x) = \dfrac{3}{1-x}$, $g(t) = \dfrac{4t}{t^2+1}$
\item  $f(x) = \dfrac{x}{x+5}$, $g(t) = \dfrac{2}{7-t^2}$


\setcounter{HW}{\value{enumi}}
\end{enumerate}
\end{multicols}

\begin{multicols}{2}
\begin{enumerate}
\setcounter{enumi}{\value{HW}}

\item  $f(x) = \dfrac{2x}{5-x^2}$, $g(t) = \sqrt{4t+1}$
\item  $f(x) =\sqrt{2x+5}$, $g(t) = \dfrac{10t}{t^2+1}$ \label{funccompeval1last}

\setcounter{HW}{\value{enumi}}
\end{enumerate}
\end{multicols}

\end{question}

In Exercises \ref{funccompexp1first} - \ref{funccompexp1last}, use the given pair of functions to find and simplify expressions for the following functions and state the domain of each using interval notation.

\begin{multicols}{3}

\begin{itemize}

\item  $(g \circ f)(x)$

\item  $(f \circ g)(t)$

\item  $(f \circ f)(x)$


\end{itemize}

\end{multicols}


\begin{multicols}{2}
\begin{enumerate}
\setcounter{enumi}{\value{HW}}

\item  $f(x) = 2x+3$, $g(t) = t^2-9$ \label{funccompexp1first}
\item  $f(x) = x^2 -x+1$, $g(t) = 3t-5$ 

\setcounter{HW}{\value{enumi}}
\end{enumerate}
\end{multicols}

\begin{multicols}{2}
\begin{enumerate}
\setcounter{enumi}{\value{HW}}

\item  $f(x) = x^2-4$, $g(t) = |t|$
\item  $f(x) = 3x-5$, $g(t) = \sqrt{t}$ 

\setcounter{HW}{\value{enumi}}
\end{enumerate}
\end{multicols}

\begin{multicols}{2}
\begin{enumerate}
\setcounter{enumi}{\value{HW}}

\item  $f(x) = |x+1|$, $g(t) = \sqrt{t}$
\item  $f(x) = 3-x^2$, $g(t) = \sqrt{t+1}$ 

\setcounter{HW}{\value{enumi}}
\end{enumerate}
\end{multicols}

\begin{multicols}{2}
\begin{enumerate}
\setcounter{enumi}{\value{HW}}

\item  $f(x) = |x|$, $g(t) = \sqrt{4-t}$
\item  $f(x) = x^2-x-1$, $g(t) = \sqrt{t-5}$ 

\setcounter{HW}{\value{enumi}}
\end{enumerate}
\end{multicols}

\begin{multicols}{2}
\begin{enumerate}
\setcounter{enumi}{\value{HW}}

\item  $f(x) = 3x-1$, $g(t) = \dfrac{1}{t+3}$
\item  $f(x) = \dfrac{3x}{x-1}$, $g(t) =\dfrac{t}{t-3}$

\setcounter{HW}{\value{enumi}}
\end{enumerate}
\end{multicols}

\begin{multicols}{2}
\begin{enumerate}
\setcounter{enumi}{\value{HW}}

\item  $f(x) = \dfrac{x}{2x+1}$, $g(t) = \dfrac{2t+1}{t}$
\item  $f(x) =  \dfrac{2x}{x^2-4}$, $g(t) =\sqrt{1-t}$ 
 \label{funccompexp1last}

\setcounter{HW}{\value{enumi}}
\end{enumerate}
\end{multicols}

\enlargethispage{0.5in}

In Exercises \ref{threefunccompfirst} - \ref{threefunccomplast}, use $f(x) = -2x$, $g(t) = \sqrt{t}$ and $h(s) = |s|$ to find and simplify expressions for the following functions and state the domain of each using interval notation.

\begin{multicols}{3}

\begin{enumerate}
\setcounter{enumi}{\value{HW}}

\item $(h\circ g \circ f)(x)$ \label{threefunccompfirst}

\item $(h\circ f \circ g)(t)$

\item $(g\circ f \circ h)(s)$

\setcounter{HW}{\value{enumi}}
\end{enumerate}
\end{multicols}

\begin{multicols}{3}
\begin{enumerate}
\setcounter{enumi}{\value{HW}}

\item $(g\circ h \circ f)(x)$ 

\item $(f\circ h \circ g)(t)$

\item $(f\circ g \circ h)(s)$ \label{threefunccomplast}

\setcounter{HW}{\value{enumi}}
\end{enumerate}
\end{multicols}

\newpage

In Exercises \ref{pointcompexfirst} - \ref{pointcompexlast}, let $f$ be the function defined by \[f = \{(-3, 4), (-2, 2), (-1, 0), (0, 1), (1, 3), (2, 4), (3, -1)\}\] and let $g$ be the function defined by \[g = \{(-3, -2), (-2, 0), (-1, -4), (0, 0), (1, -3), (2, 1), (3, 2)\}.\]  Find the following, if it exists.

\begin{multicols}{3}
\begin{enumerate}
\setcounter{enumi}{\value{HW}}

\item $(f \circ g)(3)$ \label{pointcompexfirst}
\item $f(g(-1))$
\item $(f \circ f)(0)$

\setcounter{HW}{\value{enumi}}
\end{enumerate}
\end{multicols}

\begin{multicols}{3}
\begin{enumerate}
\setcounter{enumi}{\value{HW}}


\item $(f \circ g)(-3)$
\item $(g \circ f)(3)$
\item $g(f(-3))$


\setcounter{HW}{\value{enumi}}
\end{enumerate}
\end{multicols}

\begin{multicols}{3}
\begin{enumerate}
\setcounter{enumi}{\value{HW}}

\item $(g \circ g)(-2)$
\item $(g \circ f)(-2)$
\item $g(f(g(0)))$


\setcounter{HW}{\value{enumi}}
\end{enumerate}
\end{multicols}

\begin{multicols}{3}
\begin{enumerate}
\setcounter{enumi}{\value{HW}}

\item $f(f(f(-1)))$
\item $f(f(f(f(f(1)))))$
\item $\underbrace{(g \circ g \circ \cdots \circ g)}_{\mbox{$n$ times}}(0)$ 

\setcounter{HW}{\value{enumi}}
\end{enumerate}
\end{multicols}


\begin{enumerate}
\setcounter{enumi}{\value{HW}}

\item  Find the domain and range of $f \circ g$ and $g \circ f$. \label{pointcompexlast}


\setcounter{HW}{\value{enumi}}
\end{enumerate}


In Exercises \ref{twofuncgraphcompfirst} - \ref{twofuncgraphcomplast}, use the graphs of $y=f(x)$ and $y=g(x)$ below to find the following if it exists.

\begin{center}

\begin{tabular}{cc}

\begin{mfpic}[20]{-1}{5}{-1}{5}
\axes
\tlabel[cc](5,-0.5){\scriptsize $x$}
\tlabel[cc](0.5,5){\scriptsize $y$}
\tlabel[cc](-0.75,1){\scriptsize $(0,1)$}
\tlabel[cc](1,0.5){\scriptsize $(1,1)$}
\tlabel[cc](1.5,3.5){\scriptsize $(2,3)$}
\tlabel[cc](2.5,4.5){\scriptsize $(2.5,4.5)$}
\tlabel[cc](3.5,3.5){\scriptsize $(3,3)$}
\tlabel[cc](4,-0.5){\scriptsize $(4,0)$}
\xmarks{1,2,3,4}
\ymarks{1,2,3,4}
\tlpointsep{5pt}
\scriptsize
\axislabels {x}{{$1$} 1, {$2$} 2, {$3$} 3}
\axislabels {y} {{$2$} 2, {$3$} 3, {$4$} 4}
\penwd{1.25pt}
\polyline{(0,1), (1,1), (2,3), (2.5, 4),  (3,3), (4,0)}
\point[4pt]{(0,1), (1,1), (2,3), (2.5, 4), (3,3), (4,0)}
\normalsize 
\tcaption{$y = f(x)$}
\end{mfpic}

&

\hspace{1in}

\begin{mfpic}[20]{-1}{5}{-1}{5}
\axes
\tlabel[cc](5,-0.5){\scriptsize $x$}
\tlabel[cc](0.5,5){\scriptsize $y$}
\tlabel[cc](-0.5,-0.5){\scriptsize $(0,0)$}
\tlabel[cc](.5,3.5){\scriptsize $(1,3)$}
\tlabel[cc](2.5,3.5){\scriptsize $(2,3)$}
\tlabel[cc](3,-0.5){\scriptsize $(3,0)$}
\xmarks{1,2,3,4}
\ymarks{1,2,3,4}
\tlpointsep{5pt}
\scriptsize
\axislabels {x}{{$1$} 1, {$2$} 2,  {$4$} 4}
\axislabels {y}{{$1$} 1, {$2$} 2, {$3$} 3, {$4$} 4}
\penwd{1.25pt}
\polyline{(0,0), (1,3), (2,3), (3,0)}
\point[4pt]{(0,0), (1,3), (2,3), (3,0)}
\normalsize 
\tcaption{$y = g(x)$}
\end{mfpic}

\end{tabular}

\end{center}

\smallskip

\begin{multicols}{3}
\begin{enumerate}
\setcounter{enumi}{\value{HW}}

\item  $(g\circ f)(1)$ \label{twofuncgraphcompfirst}
\item  $(f \circ g)(3)$
\item  $(g\circ f)(2)$
\setcounter{HW}{\value{enumi}}
\end{enumerate}
\end{multicols}

\begin{multicols}{3}
\begin{enumerate}
\setcounter{enumi}{\value{HW}}
\item  $(f\circ g)(0)$  
\item  $(f\circ f)(4)$
\item  $(g \circ g)(1)$ 

\setcounter{HW}{\value{enumi}}
\end{enumerate}
\end{multicols}

\begin{enumerate}
\setcounter{enumi}{\value{HW}}

\item \label{twofuncgraphcomplast}  Find the domain and range of $f \circ g$ and $g \circ f$.

\setcounter{HW}{\value{enumi}}
\end{enumerate}

\newpage

In Exercises \ref{breakdowncompexfirst} - \ref{breakdownxomexlast},  write the given function as a composition of two or more non-identity functions.  (There are several correct answers, so check your answer using function composition.)

\begin{multicols}{2}
\begin{enumerate}
\setcounter{enumi}{\value{HW}}

\item  $p(x) = (2x+3)^3$ \label{breakdowncompexfirst}
\item  $P(x) = \left(x^2-x+1\right)^5$

\setcounter{HW}{\value{enumi}}
\end{enumerate}
\end{multicols}

\begin{multicols}{2}
\begin{enumerate}
\setcounter{enumi}{\value{HW}}

\item  $h(t) = \sqrt{2t-1}$
\item  $H(t) = |7-3t|$

\setcounter{HW}{\value{enumi}}
\end{enumerate}
\end{multicols}

\begin{multicols}{2}
\begin{enumerate}
\setcounter{enumi}{\value{HW}}

\item  $r(s) = \dfrac{2}{5s+1}$
\item  $R(s) = \dfrac{7}{s^2-1}$

\setcounter{HW}{\value{enumi}}
\end{enumerate}
\end{multicols}

\begin{multicols}{2}
\begin{enumerate}
\setcounter{enumi}{\value{HW}}

\item  $q(z) = \dfrac{|z|+1}{|z|-1}$
\item  $Q(z) = \dfrac{2z^3+1}{z^3-1}$

\setcounter{HW}{\value{enumi}}
\end{enumerate}
\end{multicols}

\begin{multicols}{2}
\begin{enumerate}
\setcounter{enumi}{\value{HW}}

\item  $v(x) = \dfrac{2x+1}{3-4x}$
\item  $w(x) = \dfrac{x^2}{x^4+1}$ \label{breakdownxomexlast}

\setcounter{HW}{\value{enumi}}
\end{enumerate}
\end{multicols}

\begin{enumerate}
\setcounter{enumi}{\value{HW}}

\item Write the function $F(x) = \sqrt{\dfrac{x^{3} + 6}{x^{3} - 9}}$ as a composition of three or more non-identity functions.

\item Let $g(x) = -x, \, h(x) = x + 2, \, j(x) = 3x$ and $k(x) = x - 4$.  In what order must these functions be composed with $f(x) = \sqrt{x}$ to create $F(x) = 3\sqrt{-x + 2} - 4$?

\item What linear functions could be used to transform $f(x) = x^{3}$ into $F(x) = -\frac{1}{2}(2x - 7)^{3} + 1$?  What is the proper order of composition?

\item Let $f(x) = 3x+1$ and let $g(x) =    \begin{mycases}  2x-1 &  \text{if $x \leq 3$} \\   4-x & \text{if $x > 3$} \\  \end{mycases}$.  Find expressions for $(f \circ g)(x)$ and $(g \circ f)(x)$.

\setcounter{HW}{\value{enumi}}
\end{enumerate}

\begin{enumerate}
\setcounter{enumi}{\value{HW}}


\item The volume $V$ of a cube is a function of its side length $x$.  Let's assume that $x = t + 1$ is also a function of time $t$, where $x$ is measured in inches and $t$ is measured in minutes.  Find a formula for $V$ as a function of $t$.

\item  Suppose a local vendor charges $\$2$ per hot dog and that the number of hot dogs sold per hour $x$ is given by $x(t) = -4t^2+20t+92$, where $t$ is the number of hours since $10$ AM, $0 \leq t \leq 4$.

\begin{enumerate}

\item  Find an expression for the revenue per hour $R$ as a function of $x$.
\item  Find and simplify $\left(R \circ x\right)(t)$.  What does this represent?
\item  What is the revenue per hour at noon?
\item Using Example \ref{surfaceareaex2chainrule} as a guide, verify $\frac{\Delta[R(x)]}{\Delta x} \cdot \frac{\Delta[x(t)]}{\Delta t} = \frac{\Delta[R(t)]}{\Delta t}$.

\end{enumerate}

\item  The book in Example \ref{dragforceex} plunges into a lake and generates a circular wave pattern.  If the waves are tracked as traveling at a constant $0.5$ meters per second $\left( \frac{\text{m}}{\text{s}}\right)$, use Theorem \ref{relatedratesaroc} to find the rate at which the area of the disturbance is changing with respect to time as the radius changes from $r = 1$ to $r = 1.1$ meters (m).  Be sure to include units on your answer.

\smallskip

\textbf{HINT:}  Recall the area, $A$, enclosed by a circle of radius $r$ is given by $A = \pi \, r^2$.  Here, $\frac{\Delta r}{\Delta t} = 0.5 \, \frac{\text{m}}{\text{s}}$.  

\item   Perfectly fine precalculus textbooks which have no Calculus content are being fed into a shredder at a rate of 2 books per minute in order to make room for precalculus textbooks with Calculus content.  The shredder creates a pile of debris which is in the shape of a right circular cone whose height is twice its width. 

\begin{enumerate}

\item  Assume the volume of the conical pile, $V$,  is given by $V = \frac{1}{3} \, \pi r^2 h$ where $r$ is the radius of the base of the pile and $h$ is the height of the pile.  Given the pile is twice as tall as it is wide, show we can write $V = \frac{4}{3} \, \pi r^3$.

\item  Assuming a typical precalculus textbook is $0.10$ cubic feet $\left( \text{ft}^3 \right)$, use Theorem \ref{relatedratesaroc} to find the rate of change of the radius of the pile with respect to time as the radius changes from $2$ to $2.1$ feet. Be sure to include units on your answer.

\end{enumerate}

\item Discuss with your classmates how `real-world' processes such as filling out federal income tax forms or computing your final course grade could be viewed as a use of function composition.  Find a process for which composition with itself (iteration) makes sense.

\end{enumerate}



\end{document}
