\documentclass{ximera}

\begin{document}
	\author{Stitz-Zeager}
	\xmtitle{Exercises for Function Composition}{}

\mfpicnumber{1} \opengraphsfile{ExercisesforFunctionComposition} % mfpic settings added 


\label{ExercisesforFunctionComposition}


In Exercises \ref{funccompeval1first} - \ref{funccompeval1last}, use the given pair of functions to find the following values if they exist.



\begin{itemize}

\item  $(g\circ f)(0)$

\item  $(f\circ g)(-1)$

\item  $(f \circ f)(2)$

\item  $(g\circ f)(-3)$

\item  $(f\circ g)\left(\frac{1}{2}\right)$

\item  $(f \circ f)(-2)$

\end{itemize}






% Transformed Exercises with Solutions

\begin{question}
$f(x) = x^2$, $g(t) = 2t+1$
\begin{solution}
For  $f(x) = x^2$ and $g(t) = 2t+1$,



\end{solution}

\end{question}

\begin{question}
$f(x) = 4-x$, $g(t) = 1-t^2$

\begin{solution}
$(g\circ f)(0) = 1$
\end{solution}

\end{question}

\begin{question}
$f(x) = 4-3x$, $g(t) = |t|$
\begin{solution}
$(f\circ g)(-1) = 1$
\end{solution}

\end{question}

\begin{question}
$f(x) = |x-1|$, $g(t) = t^2-5$

\begin{solution}
$(f \circ f)(2) = 16$








\end{solution}

\end{question}

\begin{question}
$f(x) = 4x+5$, $g(t) = \sqrt{t}$
\begin{solution}
$(g\circ f)(-3) = 19$
\end{solution}

\end{question}

\begin{question}
$f(x) = \sqrt{3-x}$, $g(t) = t^2+1$

\begin{solution}
$(f\circ g)\left(\frac{1}{2}\right) = 4$
\end{solution}

\end{question}

\begin{question}
$f(x) = 6-x-x^2$, $g(t) = t\sqrt{t+10}$
\begin{solution}
$(f \circ f)(-2) = 16$


\end{solution}

\end{question}

\begin{question}
$f(x) = \sqrt[3]{x+1}$, $g(t) = 4t^2-t$

\begin{solution}
For   $f(x) = 4-x$ and $g(t) = 1-t^2$,



\end{solution}

\end{question}

\begin{question}
$f(x) = \dfrac{3}{1-x}$, $g(t) = \dfrac{4t}{t^2+1}$
\begin{solution}
$(g\circ f)(0) = -15$
\end{solution}

\end{question}

\begin{question}
$f(x) = \dfrac{x}{x+5}$, $g(t) = \dfrac{2}{7-t^2}$


\begin{solution}
$(f\circ g)(-1) = 4$
\end{solution}

\end{question}

\begin{question}
$f(x) = \dfrac{2x}{5-x^2}$, $g(t) = \sqrt{4t+1}$
\begin{solution}
$(f \circ f)(2) = 2$








\end{solution}

\end{question}

\begin{question}
$f(x) =\sqrt{2x+5}$, $g(t) = \dfrac{10t}{t^2+1}$ 

\begin{solution}
$(g\circ f)(-3) = -48$
\end{solution}

\end{question}

\begin{question}
$f(x) = 2x+3$, $g(t) = t^2-9$
\begin{solution}
$(f\circ g)\left(\frac{1}{2}\right) = \frac{13}{4}$
\end{solution}

\end{question}

\begin{question}
$f(x) = x^2 -x+1$, $g(t) = 3t-5$ 

\begin{solution}
$(f \circ f)(-2) = -2$


\end{solution}

\end{question}

\begin{question}
$f(x) = x^2-4$, $g(t) = |t|$
\begin{solution}
For   $f(x) = 4-3x$ and  $g(t) = |t|$,



\end{solution}

\end{question}

\begin{question}
$f(x) = 3x-5$, $g(t) = \sqrt{t}$ 

\begin{solution}
$(g\circ f)(0) = 4$
\end{solution}

\end{question}

\begin{question}
$f(x) = |x+1|$, $g(t) = \sqrt{t}$
\begin{solution}
$(f\circ g)(-1) = 1$
\end{solution}

\end{question}

\begin{question}
$f(x) = 3-x^2$, $g(t) = \sqrt{t+1}$ 

\begin{solution}
$(f \circ f)(2) = 10$








\end{solution}

\end{question}

\begin{question}
$f(x) = |x|$, $g(t) = \sqrt{4-t}$
\begin{solution}
$(g\circ f)(-3) = 13$
\end{solution}

\end{question}

\begin{question}
$f(x) = x^2-x-1$, $g(t) = \sqrt{t-5}$ 

\begin{solution}
$(f\circ g)\left(\frac{1}{2}\right) = \frac{5}{2}$
\end{solution}

\end{question}

\begin{question}
$f(x) = 3x-1$, $g(t) = \dfrac{1}{t+3}$
\begin{solution}
$(f \circ f)(-2) = -26$


\end{solution}

\end{question}

\begin{question}
$f(x) = \dfrac{3x}{x-1}$, $g(t) =\dfrac{t}{t-3}$

\begin{solution}
For   $f(x) = |x-1|$ and $g(t) = t^2-5$,



\end{solution}

\end{question}

\begin{question}
$f(x) = \dfrac{x}{2x+1}$, $g(t) = \dfrac{2t+1}{t}$
\begin{solution}
$(g\circ f)(0) = -4$
\end{solution}

\end{question}

\begin{question}
$f(x) =  \dfrac{2x}{x^2-4}$, $g(t) =\sqrt{1-t}$ 
 

\begin{solution}
$(f\circ g)(-1) = 5$
\end{solution}

\end{question}

\begin{question}
$(h\circ g \circ f)(x)$
\begin{solution}
$(f \circ f)(2) = 0$








\end{solution}

\end{question}

\begin{question}
$(h\circ f \circ g)(t)$
\begin{solution}
$(g\circ f)(-3) = 11$
\end{solution}

\end{question}

\begin{question}
$(g\circ f \circ h)(s)$

\begin{solution}
$(f\circ g)\left(\frac{1}{2}\right) = \frac{23}{4}$
\end{solution}

\end{question}

\begin{question}
$(g\circ h \circ f)(x)$
\begin{solution}
$(f \circ f)(-2) = 2$


\end{solution}

\end{question}

\begin{question}
$(f\circ h \circ g)(t)$
\begin{solution}
For $f(x) = 4x+5$ and $g(t) = \sqrt{t}$,



\end{solution}

\end{question}

\begin{question}
$(f\circ g \circ h)(s)$ 

\begin{solution}
$(g\circ f)(0) = \sqrt{5}$
\end{solution}

\end{question}

\begin{question}
$(f \circ g)(3)$
\begin{solution}
$(f\circ g)(-1)$ is not real
\end{solution}

\end{question}

\begin{question}
$f(g(-1))$
\begin{solution}
$(f \circ f)(2) = 57$








\end{solution}

\end{question}

\begin{question}
$(f \circ f)(0)$

\begin{solution}
$(g\circ f)(-3)$ is not real
\end{solution}

\end{question}

\begin{question}
$(f \circ g)(-3)$
\begin{solution}
$(f\circ g)\left(\frac{1}{2}\right) = 5+2\sqrt{2}$
\end{solution}

\end{question}

\begin{question}
$(g \circ f)(3)$
\begin{solution}
$(f \circ f)(-2) = -7$


\end{solution}

\end{question}

\begin{question}
$g(f(-3))$


\begin{solution}
For $f(x) = \sqrt{3-x}$ and $g(t) = t^2+1$,



\end{solution}

\end{question}

\begin{question}
$(g \circ g)(-2)$
\begin{solution}
$(g\circ f)(0) = 4$
\end{solution}

\end{question}

\begin{question}
$(g \circ f)(-2)$
\begin{solution}
$(f\circ g)(-1) = 1$
\end{solution}

\end{question}

\begin{question}
$g(f(g(0)))$


\begin{solution}
$(f \circ f)(2) = \sqrt{2}$








\end{solution}

\end{question}

\begin{question}
$f(f(f(-1)))$
\begin{solution}
$(g\circ f)(-3) = 7$
\end{solution}

\end{question}

\begin{question}
$f(f(f(f(f(1)))))$
\begin{solution}
$(f\circ g)\left(\frac{1}{2}\right) = \frac{\sqrt{7}}{2}$
\end{solution}

\end{question}

\begin{question}
$\underbrace{(g \circ g \circ \cdots \circ g)}_{\mbox{$n$ times}}(0)$ 

\begin{solution}
$(f \circ f)(-2) = \sqrt{3 - \sqrt{5}}$





\enlargethispage{0.5in}
\end{solution}

\end{question}

\begin{question}
Find the domain and range of $f \circ g$ and $g \circ f$. 


\begin{solution}
For  $f(x) = 6-x-x^2$ and $g(t) = t\sqrt{t+10}$,



\end{solution}

\end{question}

\begin{question}
$(g\circ f)(1)$
\begin{solution}
$(g\circ f)(0) = 24$
\end{solution}

\end{question}

\begin{question}
$(f \circ g)(3)$
\begin{solution}
$(f\circ g)(-1) = 0$
\end{solution}

\end{question}

\begin{question}
$(g\circ f)(2)$
\begin{solution}
$(f \circ f)(2) = 6$








\end{solution}

\end{question}

\begin{question}
$(f\circ g)(0)$
\begin{solution}
$(g\circ f)(-3) = 0$
\end{solution}

\end{question}

\begin{question}
$(f\circ f)(4)$
\begin{solution}
$(f\circ g)\left(\frac{1}{2}\right) = \frac{27-2\sqrt{42}}{8}$
\end{solution}

\end{question}

\begin{question}
$(g \circ g)(1)$ 

\begin{solution}
$(f \circ f)(-2) = -14$





\newpage
\end{solution}

\end{question}

\begin{question}
Find the domain and range of $f \circ g$ and $g \circ f$.

\begin{solution}
For  $f(x) = \sqrt[3]{x+1}$ and $g(t) = 4t^2-t$,



\end{solution}

\end{question}

\begin{question}
$p(x) = (2x+3)^3$
\begin{solution}
$(g\circ f)(0) = 3$
\end{solution}

\end{question}

\begin{question}
$P(x) = \left(x^2-x+1\right)^5$

\begin{solution}
$(f\circ g)(-1) = \sqrt[3]{6}$
\end{solution}

\end{question}

\begin{question}
$h(t) = \sqrt{2t-1}$
\begin{solution}
$(f \circ f)(2) = \sqrt[3]{\sqrt[3]{3}+1}$








\end{solution}

\end{question}

\begin{question}
$H(t) = |7-3t|$

\begin{solution}
$(g\circ f)(-3) = 4\sqrt[3]{4}+\sqrt[3]{2}$
\end{solution}

\end{question}

\begin{question}
$r(s) = \dfrac{2}{5s+1}$
\begin{solution}
$(f\circ g)\left(\frac{1}{2}\right) = \frac{\sqrt[3]{12}}{2}$
\end{solution}

\end{question}

\begin{question}
$R(s) = \dfrac{7}{s^2-1}$

\begin{solution}
$(f \circ f)(-2) = 0$


\end{solution}

\end{question}

\begin{question}
$q(z) = \dfrac{|z|+1}{|z|-1}$
\begin{solution}
For  $f(x) = \frac{3}{1-x}$ and $g(t) = \frac{4t}{t^2+1}$,



\end{solution}

\end{question}

\begin{question}
$Q(z) = \dfrac{2z^3+1}{z^3-1}$

\begin{solution}
$(g\circ f)(0) = \frac{6}{5}$
\end{solution}

\end{question}

\begin{question}
$v(x) = \dfrac{2x+1}{3-4x}$
\begin{solution}
$(f\circ g)(-1) = 1$
\end{solution}

\end{question}

\begin{question}
$w(x) = \dfrac{x^2}{x^4+1}$ 

\begin{solution}
$(f \circ f)(2) = \frac{3}{4}$








\end{solution}

\end{question}

\begin{question}
Write the function $F(x) = \sqrt{\dfrac{x^{3} + 6}{x^{3} - 9}}$ as a composition of three or more non-identity functions.
\begin{solution}
$(g\circ f)(-3) = \frac{48}{25}$
\end{solution}

\end{question}

\begin{question}
Let $g(x) = -x, \, h(x) = x + 2, \, j(x) = 3x$ and $k(x) = x - 4$.  In what order must these functions be composed with $f(x) = \sqrt{x}$ to create $F(x) = 3\sqrt{-x + 2} - 4$?
\begin{solution}
$(f\circ g)\left(\frac{1}{2}\right) = -5$
\end{solution}

\end{question}

\begin{question}
What linear functions could be used to transform $f(x) = x^{3}$ into $F(x) = -\frac{1}{2}(2x - 7)^{3} + 1$?  What is the proper order of composition?
\begin{solution}
$(f \circ f)(-2)$ is undefined


\end{solution}

\end{question}

\begin{question}
Let $f(x) = 3x+1$ and let $g(x) =    \begin{mycases}  2x-1 &  \text{if $x \leq 3$} \\   4-x & \text{if $x > 3$} \\  \end{mycases}$.  Find expressions for $(f \circ g)(x)$ and $(g \circ f)(x)$.

\begin{solution}
For  $f(x) = \frac{x}{x+5}$ and $g(t) = \frac{2}{7-t^2}$,



\end{solution}

\end{question}

\begin{question}
The volume $V$ of a cube is a function of its side length $x$.  Let's assume that $x = t + 1$ is also a function of time $t$, where $x$ is measured in inches and $t$ is measured in minutes.  Find a formula for $V$ as a function of $t$.
\begin{solution}
$(g\circ f)(0) = \frac{2}{7}$
\end{solution}

\end{question}

\begin{question}
Suppose a local vendor charges $\$2$ per hot dog and that the number of hot dogs sold per hour $x$ is given by $x(t) = -4t^2+20t+92$, where $t$ is the number of hours since $10$ AM, $0 \leq t \leq 4$.

\begin{solution}
$(f\circ g)(-1) = \frac{1}{16}$
\end{solution}

\end{question}

\begin{question}
Find an expression for the revenue per hour $R$ as a function of $x$.
\begin{solution}
$(f \circ f)(2) = \frac{2}{37}$

\end{solution}

\end{question}

\begin{question}
Find and simplify $\left(R \circ x\right)(t)$.  What does this represent?
\begin{solution}
$(g\circ f)(-3) = \frac{8}{19}$
\end{solution}

\end{question}

\begin{question}
What is the revenue per hour at noon?
\begin{solution}
$(f\circ g)\left(\frac{1}{2}\right) = \frac{8}{143}$
\end{solution}

\end{question}

\begin{question}
Using Example \ref{surfaceareaex2chainrule} as a guide, verify $\frac{\Delta[R(x)]}{\Delta x} \cdot \frac{\Delta[x(t)]}{\Delta t} = \frac{\Delta[R(t)]}{\Delta t}$.
\begin{solution}
$(f \circ f)(-2) = -\frac{2}{13}$


\end{solution}

\end{question}

\begin{question}
Assume the volume of the conical pile, $V$,  is given by $V = \frac{1}{3} \, \pi r^2 h$ where $r$ is the radius of the base of the pile and $h$ is the height of the pile.  Given the pile is twice as tall as it is wide, show we can write $V = \frac{4}{3} \, \pi r^3$.
\begin{solution}
For  $f(x) = \frac{2x}{5-x^2}$ and $g(t) = \sqrt{4t+1}$,



\end{solution}

\end{question}

\begin{question}
Assuming a typical precalculus textbook is $0.10$ cubic feet $\left( \text{ft}^3 \right)$, use Theorem \ref{relatedratesaroc} to find the rate of change of the radius of the pile with respect to time as the radius changes from $2$ to $2.1$ feet. Be sure to include units on your answer.
\begin{solution}
$(g\circ f)(0) = 1$
\end{solution}

\end{question}

\end{document}