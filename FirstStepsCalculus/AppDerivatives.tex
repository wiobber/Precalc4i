\documentclass{ximera}

\begin{document}
	\author{Stitz-Zeager}
	\xmtitle{The Shape of Graphs}


\mfpicnumber{1}

\opengraphsfile{AppDerivatives}

\setcounter{footnote}{0}

\label{AppDerivatives}

We know if $f$ is differentiable at $x=a$ then the graph of  $f$ is \textbf{locally linear} at $x=a$ and $f'(a)$ is the \textbf{slope} of the tangent line at the point $(a, f(a))$.  In this section, we explore how local behavior near a point can be extrapolated to global behavior over an interval.  First, we review Definition \ref{incdeccnstdefn} from Section \ref{ConstantandLinearFunctions}:

\medskip

%% \colorbox{ResultColor}{\bbm

\begin{defnrecall}

Let $f$ be a function defined on an interval $I$.  Then $f$ is said to be:

\begin{itemize}

\item  \textbf{increasing} on $I$ if, whenever $a < b$, then $f(a) < f(b)$.   (i.e., as inputs increase, outputs \textbf{increase}.)

\textbf{NOTE:}  The graph of an increasing function  \textbf{rises} as one moves from left to right.

\item  \textbf{decreasing} on $I$ if, whenever $a < b$, then $f(a) > f(b)$.  (i.e., as inputs increase, outputs \textbf{decrease}.)

\textbf{NOTE:}  The graph of a decreasing function \textbf{falls} as one moves from left to right.

\item  \textbf{constant} on $I$ if $f(a) = f(b)$ for all $a$, $b$ in $I$.  (i.e., outputs don't change with inputs.)

\textbf{NOTE:}  The graph of a function that is constant over an interval is a horizontal line.

\end{itemize}

\end{defnrecall}

%% \ebm}


\medskip

Suppose a function satisfies $f'(x)  > 0$ for all $x$ in an open interval\footnote{We've defined derivatives  as two-sided limits, so an open interval here guarantees enough `room' on either side of any given number to take such a limit.} $I$.   Then we know that not only is the graph of $f$ locally linear on $I$, but the slopes of all of the tangent lines are positive.  This means that all of the tangent lines are increasing so it stands to reason that the function $f$ is likewise increasing on $I$. In other words, if a function is \textbf{locally} increasing on $I$,  then it is \textbf{globally} increasing on $I$ as well.

\medskip

We can apply the same reasoning above to situations where $f'(x)<0$ for all $x$ in $I$, which implies $f$ is decreasing on $I$ or $f'(x) = 0$ on $I$, which implies $f$ is constant on $I$.  In Calculus, you'll learn this fact is a consequence of the Mean Value Theorem.\footnote{which Carl thinks is the actual `Fundamental Theorem of Calculus' since it relates local and global behavior \ldots}  In this text, we'll just accept the following theorem is true and hope we've done enough hand-waving to deem it reasonable.

\medskip

%% \colorbox{ResultColor}{\bbm

\begin{theorem}  \label{firstderivatveandgraphs}   Suppose $f$ is differentiable on an open interval $I$:

\begin{itemize}

\item If $f'(x) > 0$ for all $x$ in $I$, then $f$ is increasing on $I$.  

\item If $f'(x) < 0$ for all $x$ in $I$, then $f$ is decreasing on $I$. 

\item If $f'(x) = 0$ for all $x$ in $I$, then $f$ is constant on $I$.


\end{itemize}

\end{theorem}
%% \ebm}

\pagebreak


Theorem \ref{firstderivatveandgraphs} may be visualized as follows:

\begin{itemize}

\item  $f'(x) > 0$  for all $x$ in $I$:

\begin{center}

\begin{multicols}{2}

\includegraphics[width=1.5in]{./AppDerivativesGraphics/IncCU.png} 

\includegraphics[width=1.5in]{./AppDerivativesGraphics/IncCD.png} 

\end{multicols}

\end{center}

\item  $f'(x) < 0$  for all $x$ in $I$:


\begin{center}

\begin{multicols}{2}

\includegraphics[width=1.5in]{./AppDerivativesGraphics/DecCU.png} 

\includegraphics[width=1.5in]{./AppDerivativesGraphics/DecCD.png} 

\end{multicols}

\end{center}


\item $f'(x) = 0$  for all $x$ in $I$:

\begin{center}

\includegraphics[width=1.5in]{./AppDerivativesGraphics/Constant.png} 

\end{center}

\end{itemize}

We can use Theorem \ref{firstderivatveandgraphs} to help us determine the (open) intervals over which a function $f$ is increasing, decreasing, and constant by making a sign diagram for the derivative $f'$. 

\medskip

In order to avoid us having to go through the (somewhat lengthy) process of finding $f'(x)$ using Definition \ref{derivativefcndefn}, we'll just use some properties of derivatives from Calculus behind the scenes and present you with both a function and its derivative.  It's time for an example.

\medskip

\begin{example}\label{polyincdec}  Let  $f(x) = x^3 - 3x^2 - 9x+5$.  Use the fact that $f'(x) = 3x^2-6x-9$  to find the open intervals over which $f$ is increasing, decreasing, and constant.  Check your answer graphically.

\medskip


{\bf Solution.}   To make use of  Theorem \ref{firstderivatveandgraphs}, we make a sign diagram for $f'(x)$.  Since $f'$ is a polynomial, $f'$ is continuous so the per the Intermediate Value Theorem, Theorem \ref{IVT}, $f'$ will only change sign on either side of zeros.  Hence, our first step is to solve  $f'(x) = 0$.  


\medskip

We are given  $f'(x) = 3x^2-6x-9$.   Solving $f'(x) =  3x^2-6x-9 = 0$  gives  $3(x^2-2x-3) = 0$ or   $3(x-3)(x+1) = 0$.  We get two solutions:  $x = -1$ and $x = 3$ which divides the $x$-axis into three regions:  $x< -1$, $-1<x<3$ and $x>3$. 

\medskip

Next we select a test value in each of these three regions to determine the sign of $f'(x)$.  For the interval $x<-1$, we select $x = -3$:   $f'(-3) = 3(-3)^2-6(-3)-9 = (+)$.  For $-1<x<3$, we select $x = 0$:  $f'(0) =  3(0)^2-6(0)-9 = (-)$. Finally, for $x>3$, we select $x = 4$:  $f'(4) = 3(4)^2-6(4)-9  = (+)$.

\medskip

Below on the left is a sign diagram for $f'(x)$ and on the right is what this means for the graph of $y=f(x)$:

\begin{center}

\begin{multicols}{2}

\begin{mfpic}[15]{-6}{6}{-2}{2}
\arrow \reverse \arrow \polyline{(-5,0),(5,0)}
\xmarks{-2,2}
\arrow \polyline{(-3.5,-1.5),(-3.5,-0.5)}
\arrow \polyline{(0,-1.5),(0,-0.5)}
\arrow \polyline{(3.5,-1.5),(3.5,-0.5)}
\tlpointsep{4pt}
\axislabels {x}{{$-1 \hspace{7pt}$} -2, {$3$} 2}
\tlabel[cc](-3.5,1){$(+)$}
\tlabel[cc](-2,1){$0$}
\tlabel[cc](0,1){$(-)$}
\tlabel[cc](2,1){$0$}
\tlabel[cc](3.5,1){$(+)$}
\tlabel[cc](-3.75,-2.25){$-3$}
\tlabel[cc](0,-2.25){$0$}
\tlabel[cc](3.6,-2.25){$4$}
\tlabel[cc](6,1){$f'(x)$}
\tlabel[cc](6,-1){$x$}
%\tlabel[cc](6,0){$\infty$}
%\tlabel[cc](-6,0){$-\infty$}
\end{mfpic}

\begin{mfpic}[15]{-6}{6}{-2}{2}
\arrow \reverse \arrow \polyline{(-5,0),(5,0)}
\xmarks{-2,2}
%\arrow \polyline{(-3.5,-1.5),(-3.5,-0.5)}
%\arrow \polyline{(0,-1.5),(0,-0.5)}
%\arrow \polyline{(3.5,-1.5),(3.5,-0.5)}
\tlpointsep{4pt}
\axislabels {x}{{$-1\hspace{7pt}$} -2, {$3$} 2}
\tlabel[cc](-3.5,1){$\nearrow$}
\tlabel[cc](-2,1){$\rightarrow$}
\tlabel[cc](0,1){$\searrow$}
\tlabel[cc](2,1){$\rightarrow$}
\tlabel[cc](3.5,1){$\nearrow$}
%\tlabel[cc](-3.75,-2.25){$-3$}
%\tlabel[cc](0,-2.25){$0$}
%\tlabel[cc](3.6,-2.25){$4$}
\tlabel[cc](6,1){$f(x)$}
\tlabel[cc](6,-1){$x$}
%\tlabel[cc](6,0){$\infty$}
%\tlabel[cc](-6,0){$-\infty$}
\end{mfpic}


\end{multicols}
\end{center}

We find $f$ is increasing on $(-\infty, -1)$ and again on $(3, \infty)$ while  $f$ is decreasing on $(-1,3)$.  At the points $x = -1$ and $x=3$, we have $f'(x) = 0$ so the graph of $f$ is locally flat there.  

\medskip

Since $f$ changes from increasing just to the left of $x=-1$ to decreasing just to the right of $x=-1$, it stands to reason that $f$ has a local maximum at $x=-1$.  This is indeed the case and we find that the local maximum value is  $f(-1) = (-1)^3 - 3(-1)^2 - 9(-1)+5 = 10$.

\medskip

Similarly, since $f$ changes from decreasing just to the left of $x=3$ to increasing just to the right of $x=3$, $f$ has a local minimum at $x=3$.  The local minimum value is $f(3) = (3)^3 - 3(3)^2 - 9(3)+5 = -22$.

\medskip


A quick check using desmos confirms our results.

\medskip

\centerline{ \includegraphics[width=4in]{./AppDerivativesGraphics/IncDecPoly.png}}

\hfill \qed

\end{example}

We generalize our observations about local extrema in the following result.

\medskip

%% \colorbox{ResultColor}{\bbm

\begin{theorem}  \label{firstderivatvetest}   \textbf{The (First)\footnote{Why `First'?  Stay tuned \ldots} Derivative Test for Local Extema:}  Let $f$ be continuous on an open interval $I$ containing a critical number $c$.\footnote{Recall this means $f'(c) = 0$ or $f'(c)$ does not exist.}  If $f$ is differentiable on $I$, except possibly at $c$, then  

\medskip

\begin{itemize}

\item If $f'(x)$ changes from $(+)$ for $x<c$ to $(-)$ for $x>c$, $f$ has a local maximum at $x=c$.

\item If $f'(x)$ changes from $(-)$ for $x<c$ to $(+)$ for $x>c$, $f$ has a local minimum at $x=c$.

\item If $f'(x)$ doesn't change sign going from $x<c$ to $x>c$, $f$ does not have a local extremum at $x=c$.

\end{itemize}

\end{theorem}
%% \ebm}

\medskip

\begin{example} \label{powerfcnincdec} Let $f(x) = x^{4/3} - 4x^{1/3}$. Use the fact that $f'(x) = \frac{4}{3} x^{1/3} - \frac{4}{3} x^{-2/3}$ to help you find:

\begin{enumerate}

\item the open intervals over which $f$ is increasing, decreasing, and constant.

\item the local extrema.

\end{enumerate}

{\bf Solution.}    \begin{enumerate}  \item In order to make a sign diagram for $f'(x)$, we rewrite $f'(x)$ as a single fraction:  \[f'(x) = \dfrac{4}{3} x^{1/3} - \dfrac{4}{3} x^{-2/3} =  \dfrac{4x^{1/3}}{3}  - \dfrac{4}{3x^{2/3}}  = \dfrac{4x^{1/3}}{3} \cdot \dfrac{x^{2/3}}{x^{2/3}} - \dfrac{4}{3x^{2/3}} = \dfrac{4x}{3x^{2/3}} - \dfrac{4}{3x^{2/3}}   = \dfrac{4x-4}{3x^{2/3}}.\]

\medskip

Unlike the derivative in Example \ref{polyincdec}, $f'(x) =  \frac{4x-4}{3x^{2/3}}$ is undefined when $3x^{2/3} = 0$, that is, when $x = 0$, so we need to record this on our sign diagram with the customary `\textinterrobang.'  

\medskip

Next, we solve  $f'(x) =  \frac{4x-4}{3x^{2/3}} = 0$ to get $4x-4 = 0$ or $x = 1$. The usual machinations produces the  sign diagram for $f'(x)$ below on the left.  We interpret what this means for $f$ below on the right.

\medskip

\begin{center}

\begin{multicols}{2}

\begin{mfpic}[15]{-6}{6}{-2}{2}
\arrow \reverse \arrow \polyline{(-5,0),(5,0)}
\xmarks{-2,2}
\arrow \polyline{(-3.5,-1.5),(-3.5,-0.5)}
\arrow \polyline{(0,-1.5),(0,-0.5)}
\arrow \polyline{(3.5,-1.5),(3.5,-0.5)}
\tlpointsep{4pt}
\axislabels {x}{{$0$} -2, {$1$} 2}
\tlabel[cc](-3.5,1){$(-)$}
\tlabel[cc](-2,1){\textinterrobang}
\tlabel[cc](0,1){$(-)$}
\tlabel[cc](2,1){$0$}
\tlabel[cc](3.5,1){$(+)$}
\tlabel[cc](-3.75,-2.25){$-1$}
\tlabel[cc](0,-2.25){$\frac{1}{2}$}
\tlabel[cc](3.6,-2.25){$2$}
\tlabel[cc](6,1){$f'(x)$}
\tlabel[cc](6,-1){$x$}
%\tlabel[cc](6,0){$\infty$}
%\tlabel[cc](-6,0){$-\infty$}
\end{mfpic}

\begin{mfpic}[15]{-6}{6}{-2}{2}
\arrow \reverse \arrow \polyline{(-5,0),(5,0)}
\xmarks{-2,2}
%\arrow \polyline{(-3.5,-1.5),(-3.5,-0.5)}
%\arrow \polyline{(0,-1.5),(0,-0.5)}
%\arrow \polyline{(3.5,-1.5),(3.5,-0.5)}
\tlpointsep{4pt}
\axislabels {x}{{$0$} -2, {$1$} 2}
\tlabel[cc](-3.5,1){$\searrow$}
\tlabel[cc](-2,1){\textinterrobang}
\tlabel[cc](0,1){$\searrow$}
\tlabel[cc](2,1){$\rightarrow$}
\tlabel[cc](3.5,1){$\nearrow$}
%\tlabel[cc](-3.75,-2.25){$-3$}
%\tlabel[cc](0,-2.25){$0$}
%\tlabel[cc](3.6,-2.25){$4$}
\tlabel[cc](6,1){$f(x)$}
\tlabel[cc](6,-1){$x$}
%\tlabel[cc](6,0){$\infty$}
%\tlabel[cc](-6,0){$-\infty$}
\end{mfpic}


\end{multicols}
\end{center}

We get $f$ is decreasing for $x<0$ as well as from $0 < x < 1$.  Since $0$ is in the domain of $f$, we splice the two intervals together so $f$ is decreasing from $(-\infty, 1)$.  We see $f$ is increasing from $(1, \infty)$.

\medskip

\item   We note that $f$ satisfies the conditions of Theorem \ref{firstderivatvetest} since $f$ is continuous everywhere and $f'$ exists for all $x \neq 0$.  Since $f$ changes from decreasing just to the left of $x=1$ to increasing just to the right of $x=1$,  the graph of $f$ has a local minimum at $x=1$.  The local minimum value is $f(1) = (1)^{4/3} - 4(1)^{1/3} = -3$. 

\medskip

What is happening at $x = 0$?  Since $f'(x)$ doesn't change sign on either side of $0$,  the graph of $f$ doesn't have a local extremum there.  The sign diagram indicates  $f$ is decreasing through that point.  A quick check using desmos reveals `unsual steepness' at $x = 0$, a phenomenon which is called a \index{vertical tangent}\index{tangent ! vertical}\textbf{vertical tangent}. This means the function locally resembles a vertical line.\footnote{See Example \ref{rootradicalfcnex} for another such example and discussion.}

\medskip

\centerline{ \includegraphics[width=3in]{./AppDerivativesGraphics/IncDecRoot.png}}

\end{enumerate}

\hfill \qed

\end{example}


\subsection{Concavity and the Second Derivative}
\label{concavity}

In section Section \ref{PowerFunctions}, we introduced the notion of \index{concavity}\textbf{concavity}.  In that section, we described curves as  being  \index{concave up}\index{concavity ! concave up}\textbf{concave up} over an interval if it resembles a  portion of a `$\smile$' shape and   \index{concave down}\index{concavity ! concave down}\textbf{concave down} over an interval if resembles part of a `$\frown$' shape. Now that we've had some exposure to Calculus, we can more precisely define these notions.

\medskip


%% \colorbox{ResultColor}{\bbm

\begin{definition}  \label{concavitydefn} Let $f$ be a differentiable function on an open interval  $I$.  Then $f$ is said to be:

\begin{itemize}

\item  \textbf{concave up} on $I$ if the tangent lines lie \textbf{below} the graph on $I$.

\item  \textbf{concave down} on $I$ if the tangent lines lie \textbf{above} the graph on $I$.

\smallskip

\end{itemize}

\end{definition}

%% \ebm}

\pagebreak

If we take the time to study a generic concave up curve, the `$\smile$' shape can be divided into a decreasing and increasing arc:

\begin{center}

\begin{multicols}{2}

\includegraphics[width=1.5in]{./AppDerivativesGraphics/DecCU.png} 

\includegraphics[width=1.5in]{./AppDerivativesGraphics/IncCU.png} 


\end{multicols}

\end{center}

\begin{center}

\begin{multicols}{2}

slopes are increasing towards $0$

slopes are increasing away from $0$

\end{multicols}

\end{center}

In both of these cases, the \textbf{slopes} of the tangent line are \textbf{increasing}.  

\medskip

Likewise, we can dissect a generic `$\frown$' shape curve into an increasing and decreasing arc:

\begin{center}

\begin{multicols}{2}

\includegraphics[width=1.5in]{./AppDerivativesGraphics/IncCD.png} 

\includegraphics[width=1.5in]{./AppDerivativesGraphics/DecCD.png} 

\end{multicols}

\end{center}

\begin{center}

\begin{multicols}{2}

slopes are decreasing towards $0$

slopes are decreasing away from $0$

\end{multicols}

\end{center}

Here, the \textbf{slopes} of the tangent line are \textbf{decreasing}.  

\medskip

We know from Theorem \ref{firstderivatveandgraphs} that the derivative of a function can tell us where that function is increasing and decreasing.  Since the function which gives us the slopes of tangent lines is the derivative, $f'(x)$, we could use the derivative of $f'(x)$ to determine where the slopes of the tangent lines were increasing and decreasing.  This leads us to define the \index{second derivative}\index{derivative ! second}\textbf{second derivative}, $f''(x)$  as the derivative of $f'(x)$.

\medskip

We present the following theorem without proof, but hopefully sufficiently motivated. 

\medskip

%% \colorbox{ResultColor}{\bbm

\begin{theorem}  \label{secondderivatveandgraphs}   Suppose $f$ is twice differentiable on an open interval $I$:

\begin{itemize}

\item If $f''(x) > 0$ for all $x$ in $I$, then \textbf{slopes} are \textbf{increasing} and  $f$ is \textbf{concave up} on $I$.  

\item If $f''(x) < 0$ for all $x$ in $I$, then \textbf{slopes} are \textbf{decreasing} and  $f$ is \textbf{concave down} on $I$. 

\end{itemize}

\end{theorem}

%% \ebm}


\pagebreak

\begin{example}\label{polyconcavity} Let $f(x) = x^3 - 3x^2 - 9x+5$.  Use the fact that $f''(x) = 6x-6$ to find the intervals over which the graph of $f$ is concave up and concave down.

\medskip

{\bf Solution.}  To analyze the concavity of the graph of $f$, we need to make a sign diagram for $f''(x)$.  

\medskip

Solving $f''(x) = 6x-6 = 0$ gives $x=1$. We find $f''(0) =  6(0)-6 = (-)$ and $f''(2) = 6(2) - 6=  (+)$.  

\medskip

We have our sign diagram below on the left and our interpretation below on the right.

\begin{center}

\begin{multicols}{2}

\begin{mfpic}[15]{-6}{6}{-2}{2}
\arrow \reverse \arrow \polyline{(-5,0),(5,0)}
\xmarks{0}
\arrow \polyline{(-2,-1.5),(-2,-0.5)}
\arrow \polyline{(2,-1.5),(2,-0.5)}
\tlpointsep{4pt}
\axislabels {x}{{$1$} 0}
\tlabel[cc](-2,1){$(-)$}
\tlabel[cc](0,1){$0$}
\tlabel[cc](2,1){$(+)$}
\tlabel[cc](-2,-2.25){$0$}
\tlabel[cc](2,-2.25){$2$}
\tlabel[cc](6,1){$f''(x)$}
\tlabel[cc](6,-1){$x$}
%\tlabel[cc](6,0){$\infty$}
%\tlabel[cc](-6,0){$-\infty$}
\end{mfpic}

\begin{mfpic}[15]{-6}{6}{-2}{2}
\arrow \reverse \arrow \polyline{(-5,0),(5,0)}
\xmarks{0}
%\arrow \polyline{(-2,-1.5),(-2,-0.5)}
%\arrow \polyline{(2,-1.5),(2,-0.5)}
\tlpointsep{4pt}
\axislabels {x}{{$1$} 0}
\tlabel[cc](-2,1){\huge $\frown$}
%\tlabel[cc](0,1){$0$}
\tlabel[cc](2,1){\huge $\smile$}
%\tlabel[cc](-2,-2.25){$0$}
%\tlabel[cc](2,-2.25){$2$}
\tlabel[cc](6,1){$f(x)$}
\tlabel[cc](6,-1){$x$}
%\tlabel[cc](6,0){$\infty$}
%\tlabel[cc](-6,0){$-\infty$}
\end{mfpic}


\end{multicols}
\end{center}

We find $f$ is concave down on $(-\infty, 1)$ and concave up on $(1, \infty)$.  

\medskip

At $x=1$, the concavity changes.  We find $f(1) = (1)^3 - 3(1)^2 - 9(1)+5 = -6$ and we call the point $(1, -6)$ an \textbf{inflection point}.  In this case since the concavity changes from concave down to concave up, the point $(1,-6)$ is the point on the graph of $y=f(x)$ were the slopes stop decreasing and start to increase.

\medskip

A quick check using desmos confirms our results.

\medskip

\centerline{ \includegraphics[width=4in]{./AppDerivativesGraphics/ConcavityPoly.png}}

\hfill \qed

\end{example}

\medskip

Note that we can use concavity to help us distinguish local extrema.  

\medskip

For the function above, both $f'(-1) = 0$ and $f'(3) = 0$.  Note that $f''(-1) < 0$ which means $f$ is concave down there. This forces $f$ to have a local maximum at $(-1,6)$.  Likewise, $f''(3) > 0$ which means $f$ is concave up there.  This forces $f$ to have a local minimum at $(3,-22)$.  We generalize this observation below.

\medskip



%% \colorbox{ResultColor}{\bbm

\begin{theorem}  \label{secondderivatvetest}   \textbf{The Second\footnote{Now you know why we titled Theorem \ref{firstderivatvetest} the `First' Derivative Test for Local Extrema.} Derivative Test for Local Extrema:}  Suppose $f$ is differentiable on an open interval $I$ containing $c$ and $f'(c) = 0$:

\begin{itemize}

\item If $f''(c) > 0$ then $f$ has a local minimum at $x=c$.

\item If $f''(c) < 0$ then $f$ has a local maximum at $x=c$.

\item If $f''(c) = 0$ then the test is inconclusive.  $f$ may or may not have a local extremum at $x=c$.  

(In this case, we would appeal to the first derivative test.)

\end{itemize}
\end{theorem}

%% \ebm}



\medskip

\begin{example} \label{powerfcnconcavity}  Let $f(x) = x^{4/3} - 4x^{1/3}$.   Use the fact that $f''(x) =  \frac{4}{9} x^{-2/3} + \frac{8}{9} x^{-5/3}$ to help you find:

\begin{enumerate}

\item  the open intervals over which the graph of $f$ is concave up and concave down.

\item  the inflection points in the graph.

\end{enumerate}

\medskip

{\bf Solution.}

\begin{enumerate} \item  As in Example \ref{powerfcnincdec}, our first step is to  rewrite  $f''(x) =  \frac{4}{9} x^{-2/3} + \frac{8}{9} x^{-5/3}$  as a single fraction:


\[ f''(x) = \dfrac{4}{9} x^{-2/3} + \dfrac{8}{9} x^{-5/3} = \dfrac{4}{9x^{2/3}} + \dfrac{8}{9x^{5/3}} =  \dfrac{4}{9x^{2/3}} \cdot \dfrac{x^{3/3}}{x^{3/3}}+ \dfrac{8}{9x^{5/3}}  =  \dfrac{4x}{9x^{5/3}} + \dfrac{8}{9x^{5/3}}   =  \dfrac{4x+8}{9x^{5/3}}  \]

\medskip

We see $f''(x) =  \frac{4x+8}{9x^{5/3}} $ is undefined when $9x^{5/3}= 0$, that is, when $x = 0$.  

\medskip

Solving $f''(x) =  \frac{4x+8}{9x^{5/3}}  = 0$ gives $4x+8 = 0$ so $x = -2$. 

\newpage

Going through the usual routine, we obtain our sign diagram for $f''(x)$ is below.

\medskip

\begin{center}

\begin{multicols}{2}

\begin{mfpic}[15]{-6}{6}{-2}{2}
\arrow \reverse \arrow \polyline{(-5,0),(5,0)}
\xmarks{-2,2}
\arrow \polyline{(-3.5,-1.5),(-3.5,-0.5)}
\arrow \polyline{(0,-1.5),(0,-0.5)}
\arrow \polyline{(3.5,-1.5),(3.5,-0.5)}
\tlpointsep{4pt}
\axislabels {x}{{$-2 \hspace{7 pt}$} -2, {$0$} 2}
\tlabel[cc](-3.5,1){$(+)$}
\tlabel[cc](-2,1){$0$}
\tlabel[cc](0,1){$(-)$}
\tlabel[cc](2,1){\textinterrobang}
\tlabel[cc](3.5,1){$(+)$}
\tlabel[cc](-3.75,-2.25){$-1$}
\tlabel[cc](0,-2.25){$\frac{1}{2}$}
\tlabel[cc](3.6,-2.25){$2$}
\tlabel[cc](6,1){$f''(x)$}
\tlabel[cc](6,-1){$x$}
%\tlabel[cc](6,0){$\infty$}
%\tlabel[cc](-6,0){$-\infty$}
\end{mfpic}

\begin{mfpic}[15]{-6}{6}{-2}{2}
\arrow \reverse \arrow \polyline{(-5,0),(5,0)}
\xmarks{-2,2}
%\arrow \polyline{(-3.5,-1.5),(-3.5,-0.5)}
%\arrow \polyline{(0,-1.5),(0,-0.5)}
%\arrow \polyline{(3.5,-1.5),(3.5,-0.5)}
\tlpointsep{4pt}
\axislabels {x}{{$-2 \hspace{7 pt}$} -2, {$0$} 2}
\tlabel[cc](-3.5,1){\Huge $\smile$}
%\tlabel[cc](-2,1){?}
\tlabel[cc](0,1){\Huge $\frown$}
%\tlabel[cc](2,1){?}
\tlabel[cc](3.5,1){\Huge $\smile$}
%\tlabel[cc](-3.75,-2.25){$-3$}
%\tlabel[cc](0,-2.25){$0$}
%\tlabel[cc](3.6,-2.25){$4$}
\tlabel[cc](6,1){$f(x)$}
\tlabel[cc](6,-1){$x$}
%\tlabel[cc](6,0){$\infty$}
%\tlabel[cc](-6,0){$-\infty$}
\end{mfpic}


\end{multicols}
\end{center}

We see $f$ is concave up on $(-\infty, -2)$ and again from $(0, \infty)$.  $f$ is concave down on $(-2,0)$.

\item  Since $f$ changes concavity at both $x=-2$ and $x=0$, we have inflection point at both of these values.  

\medskip

We find:  $f(-2) = (-2)^{4/3} - 4(-2)^{1/3} = 2 (2)^{1/3} + 4 (2)^{1/3} = 6 (2)^{1/3}$. So $\left(-2, 6 (2)^{1/3} \right)$ is one inflection point.  When $x = 0$, $f(0) = (0)^{4/3} - 4(0)^{1/3}  = 0$, so $(0,0)$ is the other inflection point.  

\medskip
Checking with desmos, it's not apparent that the graph of $y=f(x)$ is concave up for $x<-2$.  We invite the reader to graph $f$ and zoom out to see that characteristic of the graph.


\medskip

\centerline{ \includegraphics[width=3in]{./AppDerivativesGraphics/ConcavityRoot.png}}

\hfill \qed



\end{enumerate}

\end{example}

\medskip

Our last example offers a twist on these sorts of curve-sketching problems.

\pagebreak

\begin{example}\label{graphfromderivativegraphex} Below is the graph of the \textbf{derivative} of a function.  Assume as $x \rightarrow \pm \infty$, $f'(x) \rightarrow -\infty$.


\begin{center}

\includegraphics[width=4in]{./AppDerivativesGraphics/derivgraph.png}

The graph of $y=f'(x)$

\end{center}

\begin{enumerate}

\item  Use the graph of $y=f'(x)$ to determine the open intervals where $f$ is increasing and decreasing.  

\medskip

Find the $x$-coordinates of the local extrema.

\medskip

\item  Use the graph of $y=f'(x)$ make a sign diagram for $y=f''(x)$.

\medskip

\item  List the open intervals over which the graph of $f$ is concave up and concave down.  

\medskip

Find the $x$-coordinates of the inflection points. 

\medskip

\item  Sketch a possible graph of $y = f(x)$.

\end{enumerate}

\medskip

{\bf Solution.}

\begin{enumerate}

\item  Recall from algebra, the solutions to $f'(x) < 0$ are the $x$-values where the graph of $y = f'(x)$  is below the $x$-axis.  This happens on the intervals $(-\infty, 0)$ and $(4, \infty)$, so this means  $f$ is decreasing here.  

\medskip

Likewise,  the solutions to $f'(x) > 0$ are the $x$-values where $y=f'(x)$ is above the $x$-axis.  This happens on the interval $(0,4)$, so $f$ is increasing here.  


\medskip

Since $f$ goes from decreasing to the left of $x=0$ to increasing to the right of $x=0$, $f$ has a local minimum at $x=0$.  Since $f$ goes from increasing to the left of $x=4$ to decreasing to the right of $x=4$, $f$ has a local maximum at $x=4$.

\pagebreak


\item  Since $f''(x)$ is the derivative of $f'(x)$, we know $f''(x) > 0$ on $(-\infty, 2)$ since $f'(x)$ is increasing there.  We see $f''(2) = 0$ since $f'(x)$ is locally flat at $(2,4)$.  Lastly, we see $f''(x) < 0$ on $(2, \infty)$ since $f'(x)$ is decreasing there.  We put all this together in a sign diagram below.

\medskip

\begin{center}

\begin{multicols}{2}

\begin{mfpic}[15]{-6}{6}{-2}{2}
\arrow \reverse \arrow \polyline{(-5,0),(5,0)}
\xmarks{0}
%\arrow \polyline{(-2,-1.5),(-2,-0.5)}
%\arrow \polyline{(2,-1.5),(2,-0.5)}
\tlpointsep{4pt}
\axislabels {x}{{$2$} 0}
\tlabel[cc](-2,1){$(+)$}
\tlabel[cc](0,1){$0$}
\tlabel[cc](2,1){$(-)$}
%\tlabel[cc](-2,-2.25){$0$}
%\tlabel[cc](2,-2.25){$3$}
\tlabel[cc](6,1){$f''(x)$}
\tlabel[cc](6,-1){$x$}
%\tlabel[cc](6,0){$\infty$}
%\tlabel[cc](-6,0){$-\infty$}
\end{mfpic}

\begin{mfpic}[15]{-6}{6}{-2}{2}
\arrow \reverse \arrow \polyline{(-5,0),(5,0)}
\xmarks{0}
%\arrow \polyline{(-2,-1.5),(-2,-0.5)}
%\arrow \polyline{(2,-1.5),(2,-0.5)}
\tlpointsep{4pt}
\axislabels {x}{{$2$} 0}
\tlabel[cc](-2,1){\Huge $\smile$}
%\tlabel[cc](0,1){$\rightarrow$}
\tlabel[cc](2,1){\Huge $\frown$}
%\tlabel[cc](-2,-2.25){$0$}
%\tlabel[cc](2,-2.25){$2$}
\tlabel[cc](6,1){$f(x)$}
\tlabel[cc](6,-1){$x$}
%\tlabel[cc](6,0){$\infty$}
%\tlabel[cc](-6,0){$-\infty$}
\end{mfpic}


\end{multicols}

\end{center}


\item   We have $f$ is concave up on $(-\infty, 2)$ and concave down on $(2, \infty)$.

\medskip

Since $f$ changes concavity at $x=2$, there is an inflection point there.

\medskip

\item   A plausible graph of $y = f(x)$  is below.  We cannot determine any $y$-coordinates (why not?)


\begin{center}

 \includegraphics[width=4in]{./AppDerivativesGraphics/original.png}
 
 A possible graph of $y = f(x)$
 
 \end{center}


\end{enumerate}

\hfill \qed

\end{example}



%\subsection{Related Rates}
%\label{relatedrates}

%\subsection{Marginal Analysis}
%\label{marginals}




\newpage

\subsection{Exercises}
%% SKIPPED %% \documentclass{ximera}

\begin{document}
	\author{Stitz-Zeager}
	\xmtitle{TITLE}
\mfpicnumber{1} \opengraphsfile{ExercisesforAppDerivatives} % mfpic settings added 


\label{ExercisesforAppDerivatives}


In Exercises \ref{incdecderivativeexercisefirst} - \ref{incdecderivativeexerciselast},  use the given function $f$ and its (first) derivative $f'$ to help you find:

\begin{itemize}

\item the open intervals over which $f$ is increasing, decreasing, and constant.

\item the local extrema.

\end{itemize}

Check your answers using a graphing utility.

\begin{enumerate}

\item\label{incdecderivativeexercisefirst} $f(x) = 2x^{3}-3x^{2}-12x + 1$, $f'(x) = 6x^2-6x-12$ % $f''(x) = 12x - 6$

\smallskip

\item $f(x) = \dfrac{10x}{x^2+1}$, $f'(x) = \dfrac{10-10x^2}{\left(x^2+1\right)^2}$ % $f''(x) = \dfrac{20x^3-60x}{\left(x^2+1\right)^3}$

\smallskip

\item\label{incdecderivativeexerciselast} $f(x) = x \sqrt[3]{x-2}$, $f'(x)=\dfrac{4x-6}{3(x-2)^{\frac{2}{3}}}$ % $f''(x)=\frac{4(x-3)}{9(x-2)^{\frac{5}{3}}}$

\smallskip

\setcounter{HW}{\value{enumi}}
\end{enumerate}

In Exercises \ref{concavederivativeexercisefirst} - \ref{concavederivativeexerciselast},  use the given function $f$ and its second derivative $f''$ to help you find:

\begin{itemize}

\item  the open intervals over which the graph of $f$ is concave up and concave down.

\item  the inflection points in the graph.

\end{itemize}

Check your answers using a graphing utility.

\begin{enumerate}
\setcounter{enumi}{\value{HW}}


\item\label{concavederivativeexercisefirst}  $f(x) = 2x^{3}-3x^{2}-12x + 1$,  $f''(x) = 12x - 6$

\smallskip

\item $f(x) = \dfrac{10x}{x^2+1}$,  $f''(x) = \dfrac{20x^3-60x}{\left(x^2+1\right)^3}$

\smallskip

\item\label{concavederivativeexerciselast} $f(x) = x \sqrt[3]{x-2}$, $f''(x)=\dfrac{4(x-3)}{9(x-2)^{\frac{5}{3}}}$ 

\smallskip

\setcounter{HW}{\value{enumi}}
\end{enumerate}

\begin{enumerate}
\setcounter{enumi}{\value{HW}}

\item If $a \neq 0$, we showed in Exercise \ref{quadraticderivativeformulaexercise} in Section \ref{IntroductiontoDerivatives} that if $f(x) = ax^2 + bx + c$, then $f'(x) = 2ax + b$. Solving $f'(x) = 0$ produced $x = -\frac{b}{2a}$, the $x$-coordinate of the vertex of the parabola $y = f(x)$. This Exercise shows this is part of a pattern.

\begin{enumerate}  \item  If $a \neq 0$, show the $x$-coordinate of the $x$-intercept of the graph of $y = ax + b$ is $x = -\frac{b}{a} = -\frac{b}{1 \, a}$.

\item  If $a \neq 0$, for $f(x) = ax^3 + bx^2 + cx + d$ it turns out that $f''(x) = 6ax + 2b$.  Show $x = -\frac{b}{3a}$ is the $x$-coordinate of the inflection point of the graph of $y = f(x)$.

\end{enumerate}

\item\label{MinimizeAverageCostProofExercise}  In Exercise \ref{AverageCostMarginalCostExercise} in Section \ref{FunctionArithmetic}, we observed that average cost appeared to be minimized when average cost was approximately equal to marginal cost.  In this Exercise, we use Calculus and the tools from this section to show this. 

\smallskip

 Recall if $C(x)$ is the cost to produce $x$ items, the \index{average cost}\index{cost ! average}\textbf{average cost} is defined as $\overline{C}(x) = \frac{C(x)}{x}$, $x > 0$,  is the cost per item. 
\begin{enumerate}

\item\label{avgcostcostderivequal}  It turns out that $\overline{C}'(x) = \dfrac{x \, C'(x) - C(x)}{x^2}$.  Show  $\overline{C}'(x) = 0$ when $C'(x) = \overline{C}(x)$.

\smallskip

\item  It turns out that $\overline{C}''(x) = \dfrac{x^2 \, C''(x) - 2x\, C'(x) + 2C(x)}{x^3}$. 

\smallskip

Show we can rewrite this as: $\overline{C}''(x) = \dfrac{x \, C''(x) - 2\, C'(x) + 2 \overline{C}(x)}{x^2}$.

\smallskip

\item\label{reduceavgcostseconderiv}  Show that when $C'(x) = \overline{C}(x)$, then $\overline{C}''(x)  = \frac{C''(x)}{x}$.

\smallskip

\item  It is usually assumed in most economic settings that for cost functions,\footnote{Can you think of reasons why?} $C''(x) > 0$.   Use this and your results from parts \ref{avgcostcostderivequal} and \ref{reduceavgcostseconderiv} to prove that a minimum is produced  when $C'(x) = \overline{C}(x)$.  

\smallskip

\textbf{NOTE:} In Exercise \ref{MarginalCostDerivativeExercise} in Section \ref{IntroductiontoDerivatives}, we saw how $C'(x)$ can be used to approximate the marginal cost, $MC(x)$, so we have established that in order to minimize average cost, we should look where the average cost matches the marginal cost.


\end{enumerate}

\item  The complete graph of $y = f(x)$ is shown below. Use the graph to answer the following questions.

\smallskip


\textbf{NOTE:} Assume $(2.029, 1.82)$ is a local maximum and that $(1.077, 0.948)$ is an inflection point.


\smallskip

\centerline{\includegraphics[width = 5in]{./AppDerivativesGraphics/T04Graph.png}}

\smallskip

\begin{enumerate}

\item  Determine the $x$-values where:

\begin{multicols}{2}

 $f(x) = 0$:
 
  $f'(x) = 0$:

\end{multicols}

\smallskip

\item List the open intervals over which:

\begin{multicols}{2}

 $f(x) > 0$:
 
  $f(x) < 0$:

\end{multicols}

\smallskip

\begin{multicols}{2}

 $f'(x) > 0$:
 
  $f'(x) < 0$:

\end{multicols}

\smallskip

\begin{multicols}{2}

 $f''(x) > 0$:
 
  $f''(x) < 0$:

\end{multicols}

\end{enumerate}


\item  Below is the graph of $y = f'(x)$ for a \textbf{continuous} function $f$.  

\smallskip

Using Example \ref{graphfromderivativegraphex} as a guide,  sketch a probable graph of $y = f(x)$.  

\begin{center}

\centerline{\includegraphics[width = 5in]{./AppDerivativesGraphics/graphfromderivativeexercise.png}}
\end{center}


\newpage

\item  The graph below was taken from https://ohiohospitals.org/covid19data on January 28th, 2021:

\bigskip

\centerline{\includegraphics[width = 5in]{./AppDerivativesGraphics/COVIDPatients.png}}

\bigskip


With help from your classmate, highlight and label one segment on the graph which (roughly) represents the following scenarios.

\smallskip

For brevity, we'll use `patients' to mean `inpatient COVID positive patients' and `the rate of change' to mean `the rate of change of inpatient COVID positive patients with respect to time.'

\smallskip

\begin{enumerate}

\item  The number of patients is \textbf{decreasing}  \underline{and}  the rate of change is \textbf{decreasing}.  (Label this `a.')

\smallskip

\item  The number of patients is \textbf{decreasing}  \underline{and}  the rate of change  is \textbf{increasing}.  (Label this `b.')

\smallskip


\item  The number of patients is \textbf{increasing}  \underline{and}  the rate of change  is \textbf{increasing}.  (Label this `c.')

\smallskip


\item  The number of patients is \textbf{increasing}  \underline{and}  the rate of change is \textbf{decreasing}.  (Label this `d.')

\smallskip

\item  Discuss with your classmates what the phrase  `flatten the curve' could mean in terms of first and second derivatives.  (See below for an illustration.)


\smallskip

\centerline{\includegraphics[width = 4.5in]{./AppDerivativesGraphics/flattenthecurve.jpeg}}


\end{enumerate}




\setcounter{HW}{\value{enumi}}
\end{enumerate}

\newpage

\subsection{Answers}

\begin{enumerate}

\item increasing:  $(-\infty, -1)$, $(2, \infty)$;  decreasing:  $(-1,2)$;  local max:  $(-1,8)$;  local min:  $(2, -19)$.

\smallskip

\item increasing:  $(-1,1)$;  decreasing:  $(-\infty, -1)$, $(1, \infty)$;  local max:  $(1,5)$;  local min:  $(-1, -5)$.

\smallskip

\item increasing:  $\left( \frac{3}{2}, \infty\right)$; decreasing:  $\left( -\infty, \frac{3}{2}\right)$;  local (absolute) min:  $\left(\frac{3}{2}, -\frac{3}{2 \sqrt[3]{2}}\right)$

\smallskip

\setcounter{HW}{\value{enumi}}
\end{enumerate}

\begin{enumerate}
\setcounter{enumi}{\value{HW}}


\item  concave up: $\left(\frac{1}{2}, \infty\right)$;  concave down: $\left( - \infty, \frac{1}{2} \right)$; inflection point:  $\left(\frac{1}{2}, \frac{11}{2}\right)$

\smallskip

\item concave up:  $\left( -\sqrt{3}, 0 \right)$, $\left( \sqrt{3}, \infty \right)$;  concave down:    $\left(- \infty,  -\sqrt{3} \right)$, $\left(0,  \sqrt{3} \right)$;  inflection points:  $\left( -\sqrt{3}, -\frac{5 \sqrt{3}}{2} \right)$, $(0,0)$, $\left( \sqrt{3}, \frac{5 \sqrt{3}}{2} \right)$
\smallskip

\item  concave up:  $(-\infty, 2)$, $(3, \infty)$;  concave down:  $(2,3)$;  inflection points: $(2,0)$, $(3, 3)$. 

\smallskip

\setcounter{HW}{\value{enumi}}
\end{enumerate}

\begin{enumerate}
\setcounter{enumi}{\value{HW}}

\item  \begin{enumerate}  \item To find the $x$-intercept, we set $ax+b = 0$  and get $x = -\frac{b}{a}$ provided $a \neq 0$.

\item  Solving  $f''(x) = 6ax + 2b = 0$, we get $x = -\frac{2b}{6a} = - \frac{b}{3a}$, provided $a \neq 0$.  

\smallskip

The graph of $f''(x) =  6ax + 2b$ is a line so we know on one side of $x= - \frac{b}{3a}$, $f''(x) > 0$ and on the other side, $f''(x) < 0$.  

\smallskip

Hence, the graph of the original function $y = f(x)$ changes concavity at $x = -\frac{b}{3a}$.

\end{enumerate}

\item\begin{enumerate}  \item  To solve $\overline{C}'(x) = \dfrac{x \, C'(x) - C(x)}{x^2} = 0$, we set the numerator,  $x \, C'(x) - C(x) = 0$. We get  $x \, C'(x)  = C(x)$ so $C'(x) = \frac{C(x)}{x} = \overline{C}(x)$.

\smallskip

\item  Divide both numerator and denominator of  $\overline{C}''(x) = \dfrac{x^2 \, C''(x) - 2x\, C'(x) + 2C(x)}{x^3}$ by $x$: 

\smallskip

$\overline{C}''(x) = \dfrac{x \, C''(x) - 2 \, C'(x) + 2\frac{C(x)}{x}}{x^2}$ and substitute  $\frac{C(x)}{x} = \overline{C}(x)$.

\smallskip

\item  If  $C'(x) = \overline{C}(x)$, then:

\smallskip

 $\overline{C}''(x)  = \dfrac{x\, C''(x) - 2\, C'(x) + 2\overline{C}(x)}{x^2} =  \dfrac{x\, C''(x) - 2\overline{C} (x) + 2\overline{C}(x)}{x^2}= \dfrac{x\, C''(x)}{x^2} =  \dfrac{C''(x)}{x}$.

\smallskip

\item  When $C'(x) = \overline{C}(x)$, we have that $\overline{C}'(x) = 0$ and $\overline{C}''(x) > 0$.  By the Second Derivative Test for Local Extrema,  Theorem \ref{secondderivatvetest}, the average cost $\overline{C}(x)$ has a minimum when $C'(x) = \overline{C}(x)$ 

\smallskip



\end{enumerate}

\newpage

\item \begin{enumerate}  \item

\begin{multicols}{2}

 $f(x) = 0$:  $x = 0, \pi$
 
  $f'(x) = 0$: $x = 2.029$ ($f_{+}'(0) = 0$.,  too.)

\end{multicols}

\smallskip

\item  \begin{multicols}{2}

 $f(x) > 0$: $(0, \pi)$
 
  $f(x) < 0$: none

\end{multicols}

\smallskip

\begin{multicols}{2}

 $f'(x) > 0$: $(0, 2.029)$
 
  $f'(x) < 0$: $(2.029, \pi)$

\end{multicols}

\smallskip

\begin{multicols}{2}

 $f''(x) > 0$: $(0, 1.077)$
 
  $f''(x) < 0$: $(1.077, \pi)$.

\end{multicols}

\end{enumerate}

\item Answers vary.  below is a sketch of the key features that should be included:


\begin{center}

\centerline{\includegraphics[width = 5in]{./AppDerivativesGraphics/graphfromderivativeexerciseanswer.png}}
\end{center}


\setcounter{HW}{\value{enumi}}
\end{enumerate}




\end{document}


\closegraphsfile

\end{document}
