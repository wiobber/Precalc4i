\documentclass{ximera}

\begin{document}
	\author{Stitz-Zeager}
	\xmtitle{Exercises for Poly Arith}{}

\mfpicnumber{1} \opengraphsfile{ExercisesforAppPolyArith} % mfpic settings added 


\label{ExercisesforAppPolyArith}

\begin{question}
In Exercises \ref{polyarithexfirst} - \ref{polyarithexlast}, perform the indicated operations and simplify.

\begin{problem}\label{polyarithexfirst}
$(4-3x) + (3x^2 + 2x + 7)$

\begin{solution}
     $3x^2 - x + 11$
\end{solution}
\end{problem}

\begin{problem}
$t^2 + 4t - 2(3-t)$

\begin{solution}
    $t^2 + 6t-6$
\end{solution}
\end{problem}

\begin{problem}
$q(200-3q) - (5q + 500)$

\begin{solution}
     $-3q^2+195q-500$
\end{solution}
\end{problem}

\begin{problem}
$(3y-1)(2y+1)$

\begin{solution}
    $6y^2+y-1$
\end{solution}
\end{problem}

\begin{problem}
$\left(3-\frac{x}{2}\right)(2x+5)$

\begin{solution}
    $-x^2 + \frac{7}{2} x + 15$
\end{solution}
\end{problem}

\begin{problem}
$-(4t+3)(t^2-2)$

\begin{solution}
    $-4t^3-3t^2+8t+6$
\end{solution}
\end{problem}

\begin{problem}
$2w(w^3-5)(w^3+5)$

\begin{solution}
    $2w^7 - 50w$
\end{solution}
\end{problem}

\begin{problem}
$(5a^2 - 3)(25a^4 + 15a^2 + 9)$

\begin{solution}
    $125a^6 - 27$
\end{solution}
\end{problem}

\begin{problem}
$(x^2-2x+3)(x^2+2x+3)$

\begin{solution}
    $x^4+2x^2+9$
\end{solution}
\end{problem}

\begin{problem}
$(\sqrt{7} - z)(\sqrt{7} + z)$

\begin{solution}
    $7-z^2$
\end{solution}
\end{problem}

\begin{problem}
$(x - \sqrt[3]{5})^3$

\begin{solution}
    $x^3 - 3x^2\sqrt[3]{5} + 3x\sqrt[3]{25} - 5$
\end{solution}
\end{problem}

\begin{problem}
$(x - \sqrt[3]{5})(x^2 + x\sqrt[3]{5} + \sqrt[3]{25})$

\begin{solution}
    $x^3 - 5$
\end{solution}
\end{problem}

\begin{problem}
$(w-3)^2 - (w^2 + 9)$

\begin{solution}
    $-6w$
\end{solution}
\end{problem}

\begin{problem}
$(x+h)^2 - 2(x+h) - (x^2 - 2x)$

\begin{solution}
    $h^2 + 2xh - 2h$
\end{solution}
\end{problem}

\begin{problem}\label{polyarithexlast}
$(x-[2+\sqrt{5}])(x-[2-\sqrt{5}])$

\begin{solution}
    $x^2 - 4x - 1$
\end{solution}
\end{problem}
\end{question}

\begin{question}
In Exercises \ref{polydivexfirst} - \ref{polydivexlast}, perform the indicated division.  Check your answer by showing \[\text{dividend} = (\text{divisor})( \text{quotient}) + \text{remainder}\]

\begin{problem}\label{polydivexfirst}
$(5x^2 - 3x + 1) \div (x + 1)$

quotient: 
$\answer{5}x + \answer{-8}$

remainder: 
$\answer{9}$ 
\end{problem}

\begin{problem}
$(3y^2 + 6y - 7) \div (y-3)$

quotient: 
$\answer{3}y + \answer{15}$ 

remainder: 
$\answer{38}$
\end{problem}

\begin{problem}
$(6w - 3) \div (2w+5)$

    quotient: 
    $\answer{3}$ 
    
    remainder: 
    $\answer{18}$
\end{problem}

\begin{problem}
$(2x+1) \div (3x-4)$

quotient: 
$\answer{\frac{2}{3}}$

remainder: 
$\answer{\frac{11}{3}}$
\end{problem}

\begin{problem}
$(t^2 - 4) \div (2t + 1)$

\begin{solution}
    quotient: $\frac{t}{2} - \frac{1}{4}$
    
    remainder: $-\frac{15}{4}$
\end{solution}

\end{problem}

\begin{problem}
$(w^3 - 8) \div (5w-10)$

\begin{solution}
    quotient: $\frac{w^2}{5} + \frac{2w}{5} + \frac{4}{5}$
    
    remainder: $0$
\end{solution}
\end{problem}

\begin{problem}
$(2x^2 - x + 1) \div (3x^2 + 1)$

\begin{solution}
     quotient: $\frac{2}{3}$
     
     remainder: $-x + \frac{1}{3}$
\end{solution}
\end{problem}

\begin{problem}
$(4y^4+3y^2+1) \div (2y^2-y+1)$

\begin{solution}
     quotient:  $2y^2+y+1$
     
     remainder: $0$
\end{solution}
\end{problem}

\begin{problem}
$w^4 \div (w^3 - 2)$

\begin{solution}
    quotient: $w$
    
    remainder: $2w$
\end{solution}
\end{problem}

\begin{problem}
$(5t^3 - t + 1) \div (t^2 + 4)$

\begin{solution}
    quotient: $5t$
    
    remainder: $-21t + 1$
\end{solution}
\end{problem}

\begin{problem}
$(t^3 - 4) \div (t - \sqrt[3]{4})$

\begin{solution}
    quotient:\footnote{Note: $\sqrt[3]{16} = 2\sqrt[3]{2}$.} $t^2 + t \sqrt[3]{4} + 2\sqrt[3]{2}$
    
    remainder: $0$
\end{solution}
\end{problem}

\begin{problem}\label{polydivexlast}
$(x^2-2x-1) \div (x-[1-\sqrt{2}])$

\begin{solution}
    quotient: $x -1 - \sqrt{2}$
    
    remainder: 0 
\end{solution}
\end{problem}

\end{question}

\begin{question}
In Exercises \ref{specialformexfirst} - \ref{specialformexlast} verify the given formula by showing the left hand side of the equation simplifies to the right hand side of the equation.

\begin{problem}\label{specialformexfirst}
\textbf{Perfect Cube:} $(a+b)^3 = a^3 + 3a^2b + 3ab^2 + b^3$
\end{problem}

\begin{problem}
\textbf{Difference of Cubes:} $(a - b)(a^2 + ab + b^2) = a^3 - b^3$
\end{problem}

\begin{problem}
\textbf{Sum of Cubes:} $(a + b)(a^2 - ab + b^2) = a^3 + b^3$
\end{problem}

\begin{problem}
\textbf{Perfect Quartic:} $(a+b)^4 = a^4 + 4a^3b + 6a^2b^2 + 4ab^3 + b^4$
\end{problem}

\begin{problem}
\textbf{Difference of Quartics:} $(a-b)(a+b)(a^2+b^2) = a^4 - b^4$  
\end{problem}

\begin{problem}\label{specialformexlast}
\textbf{Sum of Quartics:}  $(a^2 + ab \sqrt{2} + b^2)(a^2 - ab \sqrt{2} + b^2) = a^4 + b^4$
\end{problem}
 

\end{question}

\begin{problem}
With help from your classmates, determine under what conditions $(a+b)^2 = a^2 + b^2$.  What about $(a+b)^3 = a^3 + b^3$? In general, when does $(a+b)^n = a^n + b^n$ for a natural number $n \geq 2$?
\end{problem}

\end{document}
