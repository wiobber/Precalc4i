\documentclass{ximera}

\begin{document}
	\author{Stitz-Zeager}
	\xmtitle{Exercises for Lines}{}

\mfpicnumber{1} \opengraphsfile{ExercisesforAppLines}\label{ExercisesforAppLines}

\begin{question}
In Exercises \ref{pointslopegivenlinefirst} - \ref{pointslopegivenlinelast}, find both the point-slope form and the slope-intercept form of the line with the given slope which passes through the given point.

\begin{problem}\label{pointslopegivenlinefirst}
    $m = 3, \;\; P(3, -1)$ 

The point slope form is ...

\begin{solution}
$y+1=3(x-3)$  
\end{solution}

The slope-intercept form is

 $y = \answer{3}x + \answer{-10}$

\end{problem} 

\begin{problem}
    $m = -2, \;\; P(-5, 8)$

The point slope form is ...

\begin{solution}
    $y-8 = -2(x+5)$
\end{solution}

The slope-intercept form is
 
 $y = \answer{-2}x + \answer{-2}$
    
\end{problem}

\begin{problem}
    $m = -1, \;\; P(-7, -1)$

The point slope form is ...

\begin{solution}
    $y + 1 = -(x+7)$
\end{solution}

The slope-intercept form is

 $y = \answer{-1}x + \answer{-8}$
    
\end{problem}

\begin{problem}
    $m = \frac{2}{3}, \;\; P(-2, 1)$
    
The point slope form is ...

\begin{solution}
    $y - 1 = \frac{2}{3} (x+2)$
\end{solution}

The slope-intercept form is

 $y = \answer{\frac{2}{3}} x + \answer{\frac{7}{3}}$
    
\end{problem}

\begin{problem}
$m = -\frac{1}{5}, \;\; P(10, 4)$

The point slope form is ...

\begin{solution}
    $y - 4 = -\frac{1}{5} (x-10)$
\end{solution}

The slope-intercept form is

$y = \answer{-\frac{1}{5}}x + \answer{6}$


\end{problem}

\begin{problem}
$m = \frac{1}{7}, \;\; P(-1, 4)$

The point slope form is ...

\begin{solution}
     $y - 4 = \frac{1}{7}(x + 1)$
\end{solution}

The slope-intercept form is

$y = \answer{\frac{1}{7}}x + \answer{\frac{29}{7}}$

\end{problem}

\begin{problem}
$m = 0, \;\; P(3, 117)$

The point slope form is ...

\begin{solution}
    $y - 117 = 0$
\end{solution}

The slope-intercept form is

$y = \answer{117}$

\end{problem}

\begin{problem}
$m = -\sqrt{2}, \;\; P(0, -3)$

The point slope form is ...

\begin{solution}
    $y + 3 = -\sqrt{2}(x - 0)$
\end{solution}

The slope-intercept form is

$y = \answer{-\sqrt{2}}x + \answer{-3}$

\end{problem}

\begin{problem}
$m = -5, \;\; P(\sqrt{3}, 2\sqrt{3})$

The point slope form is ...

\begin{solution}
    $y - 2\sqrt{3} = -5(x - \sqrt{3})$
\end{solution}

The slope-intercept form is

$y = \answer{-5}x + \answer{7\sqrt{3}}$ 


\end{problem}

\begin{problem}\label{pointslopegivenlinelast}
$m = 678, \;\; P(-1, -12)$

The point slope form is ...

\begin{solution}
    $y + 12 = 678(x + 1)$
\end{solution}

The slope-intercept form is

$y = \answer{678}x + \answer{666}$

\end{problem}

\end{question}

\begin{question}
In Exercises \ref{twopointsgivenlinefirst} - \ref{twopointsgivenlinelast}, find the slope-intercept form of the line which passes through the given points.

\begin{problem}\label{twopointsgivenlinefirst}
$P(0, 0), \; Q(-3, 5)$
\end{problem}

\begin{problem}
$P(-1, -2), \; Q(3, -2)$

$y = \answer{-2}$

\end{problem}

\begin{problem}
$P(5, 0), \; Q(0, -8)$
\end{problem}

\begin{problem}
$P(3, -5), \; Q(7, 4)$

\begin{solution}
$y = \answer{\frac{9}{4}}x + \answer{\frac{-47}{4}}$
\end{solution}
\end{problem}

\begin{problem}
$P(-1,5), \; Q(7, 5)$
\end{problem}

\begin{problem}
$P(4, -8), \; Q(5, -8)$
\end{problem}

\begin{problem}
$P\left(\frac{1}{2}, \frac{3}{4} \right), \; Q\left(\frac{5}{2}, -\frac{7}{4} \right)$
\end{problem}

\begin{problem}
$P\left(\frac{2}{3}, \frac{7}{2} \right), \; Q\left(-\frac{1}{3}, \frac{3}{2} \right)$
\end{problem}

\begin{problem}
$P\left(\sqrt{2}, -\sqrt{2} \right), \; Q\left(-\sqrt{2}, \sqrt{2} \right)$
\end{problem}

\begin{problem}\label{twopointsgivenlinelast}
$P\left(-\sqrt{3}, -1 \right), \; Q\left(\sqrt{3}, 1 \right)$
\end{problem}

\end{question}

\begin{question}
In Exercises \ref{graphlineexerfirst} - \ref{graphlineexerlast}, graph the line.  Find the slope, $y$-intercept and $x$-intercept, if any exist.

\begin{problem}\label{graphlineexerfirst}
$y = 2x - 1$ 
\end{problem}

\begin{problem}
$y = 3 - x$

\begin{solution}
\desmos{quutwlxctv}{800}{600}

Slope $=-1$, y-intercept is $(0,3)$, x-intercept is $(3,0)$
\end{solution}
\end{problem}

\begin{problem}
$y = 3$
\end{problem}

\begin{problem}
$y = 0$

\begin{solution}
\desmos{ufq5a7bsqt}{800}{600}

Slope $=0$, y-intercept is $(0,0)$, x-intercept is $(0,0)$
\end{solution}
\end{problem}

\begin{problem}
$y = \frac{2}{3} x + \frac{1}{3}$
\end{problem}

\begin{problem}\label{graphlineexerlast}
$y = \dfrac{1-x}{2}$

\begin{solution}
\desmos{uioah0eu3c}{800}{600}

Slope $=\frac{-1}{2}$, y-intercept is $\frac{1}{2}$, x-intercept is $(1,0)$
\end{solution}
\end{problem}
    
\end{question}

\begin{problem}
Graph $3v + 2w = 6$ on both the $vw$- and $wv$-axes.  What characteristics to both graphs share?  What's different? 
\end{problem} 

\begin{problem}
Find all of the points on the line $y=2x+1$ which are $4$ units from the point $(-1,3)$.
\end{problem}

\begin{question}
In Exercises \ref{parallelfirst} - \ref{parallellast}, you are given a line and a point which is not on that line.  Find the line parallel to the given line which passes through the given point.


\begin{problem}\label{parallelfirst}
$y = 3x + 2, \; P(0, 0)$ 
\end{problem}

\begin{problem}
$y = -6x + 5, \; P(3, 2)$
\end{problem}

\begin{problem}
$y = \frac{2}{3} x - 7, \; P(6, 0)$
\end{problem}

\begin{problem}
$y = \dfrac{4-x}{3}, \; P(1, -1)$
\end{problem}

\begin{problem}
$y = 6, \; P(3, -2)$
\end{problem}

\begin{problem}\label{parallellast}
$x=1, \; P(-5,0)$ 
\end{problem}

\end{question}

\begin{question}\label{perpendicularlines}
In Exercises \ref{perpendlinefirst} - \ref{perpendlinelast}, you are given a line and a point which is not on that line.  Find the line perpendicular to the given line which passes through the given point.  

\begin{problem}\label{perpendlinefirst}
$y = \frac{1}{3}x + 2, \; P(0, 0)$ 
\end{problem}

\begin{problem}
$y = -6x + 5, \; P(3, 2)$
\end{problem}

\begin{problem}
$y = \frac{2}{3} x - 7, \; P(6, 0)$
\end{problem}

\begin{problem}
$y = \dfrac{4-x}{3}, \; P(1, -1)$
\end{problem}

\begin{problem}
$y = 6, \; P(3, -2)$
\end{problem}

\begin{problem}\label{perpendlinelast}
$x=1, \; P(-5,0)$ 
\end{problem}

\end{question}

\begin{problem}\label{perpendicularlineproof}
We shall now prove that $y = m_1x + b_1$ is perpendicular to $y = m_2x + b_2$ if and only if $m_1 \cdot m_2 = -1$.  To make our lives easier we shall assume that $m_1 > 0$ and $m_2 < 0$.  We can also ``move'' the lines so that their point of intersection is the origin without messing things up, so we'll assume $b_1 = b_2 = 0.$  (Take a moment with your classmates to discuss why this is okay.)  Graphing the lines and plotting the points $O(0, 0)\;$, $P(1, m_1)\;$ and $Q(1, m_2)$ gives us the following set up.

\begin{center}
\begin{tikzpicture}[scale=0.9]

% Axes
\draw[->] (-5,0) -- (5.2,0);
\draw[->] (0,-5) -- (0,5.2);

\node at (5,-0.5) {\scriptsize $x$};
\node at (0.5,5) {\scriptsize $y$};

% Points
\fill (0,0) circle (2pt);
\fill (1.5,0.75) circle (2pt);
\fill (1.5,-3) circle (2pt);

% Connecting vertical segment
\draw (1.5,0.75) -- (1.5,-3);

% Diagonal arrows
\draw[<->] (-4,-2) -- (4,2);
\draw[<->] (-2,4) -- (2,-4);

% Labels
\node at (1.2,1) {\scriptsize $P$};
\node at (-0.5,-0.6) {\scriptsize $O$};
\node at (1.2,-3.55) {\scriptsize $Q$};

\end{tikzpicture}
\end{center}

The line $y = m_1x$ will be perpendicular to the line $y = m_2x$ if and only if $\bigtriangleup OPQ$ is a right triangle.  Let $d_1$ be the distance from $O$ to $P$, let $d_2$ be the distance from $O$ to $Q$ and let $d_3$ be the distance from $P$ to $Q$.  Use the Pythagorean Theorem to show that $\bigtriangleup OPQ$ is a right triangle if and only if $m_1 \cdot m_2 = -1$ by showing $d_1^{2} + d_2^{2} = d_3^2$ if and only if $m_1 \cdot m_2 = -1$. 

\end{problem}

\end{document}
