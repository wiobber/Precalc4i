\documentclass{ximera}

\begin{document}
	\author{Stitz-Zeager}
	\xmtitle{Exercises for App Lines}{}

\mfpicnumber{1} \opengraphsfile{ExercisesforAppLines} % mfpic settings added 


\label{ExercisesforAppLines}

In Exercises \ref{pointslopegivenlinefirst} - \ref{pointslopegivenlinelast}, find both the point-slope form and the slope-intercept form of the line with the given slope which passes through the given point.

\begin{multicols}{2}
\begin{enumerate}

\item $m = 3, \;\; P(3, -1)$ \label{pointslopegivenlinefirst}
\item $m = -2, \;\; P(-5, 8)$

\setcounter{HW}{\value{enumi}}
\end{enumerate}
\end{multicols}

\begin{multicols}{2}
\begin{enumerate}
\setcounter{enumi}{\value{HW}}

\item $m = -1, \;\; P(-7, -1)$
\item $m = \frac{2}{3}, \;\; P(-2, 1)$

\setcounter{HW}{\value{enumi}}
\end{enumerate}
\end{multicols}

\begin{multicols}{2}
\begin{enumerate}
\setcounter{enumi}{\value{HW}}

\item $m = -\frac{1}{5}, \;\; P(10, 4)$
\item $m = \frac{1}{7}, \;\; P(-1, 4)$

\setcounter{HW}{\value{enumi}}
\end{enumerate}
\end{multicols}

\begin{multicols}{2}
\begin{enumerate}
\setcounter{enumi}{\value{HW}}

\item $m = 0, \;\; P(3, 117)$
\item $m = -\sqrt{2}, \;\; P(0, -3)$

\setcounter{HW}{\value{enumi}}
\end{enumerate}
\end{multicols}

\begin{multicols}{2}
\begin{enumerate}
\setcounter{enumi}{\value{HW}}

\item $m = -5, \;\; P(\sqrt{3}, 2\sqrt{3})$
\item $m = 678, \;\; P(-1, -12)$ \label{pointslopegivenlinelast}

\setcounter{HW}{\value{enumi}}
\end{enumerate}
\end{multicols}

In Exercises \ref{twopointsgivenlinefirst} - \ref{twopointsgivenlinelast}, find the slope-intercept form of the line which passes through the given points.

\begin{multicols}{2}
\begin{enumerate}
\setcounter{enumi}{\value{HW}}

\item $P(0, 0), \; Q(-3, 5)$ \label{twopointsgivenlinefirst}
\item $P(-1, -2), \; Q(3, -2)$

\setcounter{HW}{\value{enumi}}
\end{enumerate}
\end{multicols}

\begin{multicols}{2}
\begin{enumerate}
\setcounter{enumi}{\value{HW}}

\item $P(5, 0), \; Q(0, -8)$
\item $P(3, -5), \; Q(7, 4)$

\setcounter{HW}{\value{enumi}}
\end{enumerate}
\end{multicols}

\begin{multicols}{2}
\begin{enumerate}
\setcounter{enumi}{\value{HW}}

\item $P(-1,5), \; Q(7, 5)$
\item $P(4, -8), \; Q(5, -8)$

\setcounter{HW}{\value{enumi}}
\end{enumerate}
\end{multicols}

\begin{multicols}{2}
\begin{enumerate}
\setcounter{enumi}{\value{HW}}

\item $P\left(\frac{1}{2}, \frac{3}{4} \right), \; Q\left(\frac{5}{2}, -\frac{7}{4} \right)$
\item $P\left(\frac{2}{3}, \frac{7}{2} \right), \; Q\left(-\frac{1}{3}, \frac{3}{2} \right)$

\setcounter{HW}{\value{enumi}}
\end{enumerate}
\end{multicols}

\begin{multicols}{2}
\begin{enumerate}
\setcounter{enumi}{\value{HW}}

\item $P\left(\sqrt{2}, -\sqrt{2} \right), \; Q\left(-\sqrt{2}, \sqrt{2} \right)$
\item $P\left(-\sqrt{3}, -1 \right), \; Q\left(\sqrt{3}, 1 \right)$ \label{twopointsgivenlinelast}

\setcounter{HW}{\value{enumi}}
\end{enumerate}
\end{multicols}

In Exercises \ref{graphlineexerfirst} - \ref{graphlineexerlast}, graph the line.  Find the slope, $y$-intercept and $x$-intercept, if any exist.

\begin{multicols}{2}
\begin{enumerate}
\setcounter{enumi}{\value{HW}}

\item $y = 2x - 1$ \label{graphlineexerfirst}
\item $y = 3 - x$

\setcounter{HW}{\value{enumi}}
\end{enumerate}
\end{multicols}

\begin{multicols}{2}
\begin{enumerate}
\setcounter{enumi}{\value{HW}}

\item $y = 3$
\item $y = 0$

\setcounter{HW}{\value{enumi}}
\end{enumerate}
\end{multicols}

\begin{multicols}{2}
\begin{enumerate}
\setcounter{enumi}{\value{HW}}

\item $y = \frac{2}{3} x + \frac{1}{3}$ \vphantom{$\dfrac{1-x}{2}$}
\item $y = \dfrac{1-x}{2}$ \label{graphlineexerlast}

\setcounter{HW}{\value{enumi}}
\end{enumerate}
\end{multicols}




\begin{enumerate}
\setcounter{enumi}{\value{HW}}

\item  Graph $3v + 2w = 6$ on both the $vw$- and $wv$-axes.  What characteristics to both graphs share?  What's different?

\item  Find all of the points on the line $y=2x+1$ which are $4$ units from the point $(-1,3)$.

\setcounter{HW}{\value{enumi}}
\end{enumerate}

In Exercises \ref{parallelfirst} - \ref{parallellast}, you are given a line and a point which is not on that line.  Find the line parallel to the given line which passes through the given point.


\begin{multicols}{2}
\begin{enumerate}
\setcounter{enumi}{\value{HW}}

\item $y = 3x + 2, \; P(0, 0)$ \label{parallelfirst}
\item $y = -6x + 5, \; P(3, 2)$

\setcounter{HW}{\value{enumi}}
\end{enumerate}
\end{multicols}


\begin{multicols}{2}
\begin{enumerate}
\setcounter{enumi}{\value{HW}}

\item $y = \frac{2}{3} x - 7, \; P(6, 0)$
\item $y = \dfrac{4-x}{3}, \; P(1, -1)$


\setcounter{HW}{\value{enumi}}
\end{enumerate}
\end{multicols}


\begin{multicols}{2}
\begin{enumerate}
\setcounter{enumi}{\value{HW}}

\item $y = 6, \; P(3, -2)$
\item $x=1, \; P(-5,0)$ \label{parallellast}


\setcounter{HW}{\value{enumi}}
\end{enumerate}
\end{multicols}


\phantomsection
\label{perpendicularlines}

In Exercises \ref{perpendlinefirst} - \ref{perpendlinelast}, you are given a line and a point which is not on that line.  Find the line perpendicular to the given line which passes through the given point.

\begin{multicols}{2}
\begin{enumerate}
\setcounter{enumi}{\value{HW}}


\item $y = \frac{1}{3}x + 2, \; P(0, 0)$ \label{perpendlinefirst}
\item $y = -6x + 5, \; P(3, 2)$

\setcounter{HW}{\value{enumi}}
\end{enumerate}
\end{multicols}

\begin{multicols}{2}
\begin{enumerate}
\setcounter{enumi}{\value{HW}}

\item $y = \frac{2}{3} x - 7, \; P(6, 0)$
\item $y = \dfrac{4-x}{3}, \; P(1, -1)$


\setcounter{HW}{\value{enumi}}
\end{enumerate}
\end{multicols}

\begin{multicols}{2}
\begin{enumerate}
\setcounter{enumi}{\value{HW}}

\item $y = 6, \; P(3, -2)$
\item $x=1, \; P(-5,0)$ \label{perpendlinelast}


\setcounter{HW}{\value{enumi}}
\end{enumerate}
\end{multicols}


\begin{enumerate}
\setcounter{enumi}{\value{HW}}

\item We shall now prove that $y = m_1x + b_1$ is perpendicular to $y = m_2x + b_2$ if and only if $m_1 \cdot m_2 = -1$.  To make our lives easier we shall assume that $m_1 > 0$ and $m_2 < 0$.  We can also ``move'' the lines so that their point of intersection is the origin without messing things up, so we'll assume $b_1 = b_2 = 0.$  (Take a moment with your classmates to discuss why this is okay.)  Graphing the lines and plotting the points $O(0, 0)\;$, $P(1, m_1)\;$ and $Q(1, m_2)$ gives us the following set up. \label{perpendicularlineproof}

\begin{center}

\begin{mfpic}[18]{-5}{5}{-5}{5}
\point[3pt]{(0, 0), (1.5, 0.75), (1.5, -3)}
\arrow \reverse \arrow \polyline{( -4, -2), (4, 2)}
\arrow \reverse \arrow \polyline{( -2, 4), (2, -4)}
\polyline{(1.5, 0.75), (1.5, -3)}
\tlabel(1.2, 1){\scriptsize $P$}
\tlabel(-.5,-.6){\scriptsize $O$}
\tlabel(1.2,-3.55){\scriptsize $Q$}
\axes
\tlabel[cc](5,-0.5){\scriptsize $x$}
\tlabel[cc](0.5,5){\scriptsize $y$}
\end{mfpic}

\end{center}

The line $y = m_1x$ will be perpendicular to the line $y = m_2x$ if and only if $\bigtriangleup OPQ$ is a right triangle.  Let $d_1$ be the distance from $O$ to $P$, let $d_2$ be the distance from $O$ to $Q$ and let $d_3$ be the distance from $P$ to $Q$.  Use the Pythagorean Theorem to show that $\bigtriangleup OPQ$ is a right triangle if and only if $m_1 \cdot m_2 = -1$ by showing $d_1^{2} + d_2^{2} = d_3^2$ if and only if $m_1 \cdot m_2 = -1$.  


\end{enumerate}

\end{document}
