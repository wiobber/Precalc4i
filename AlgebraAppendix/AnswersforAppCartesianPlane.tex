\documentclass{ximera}

\begin{document}
	\author{Stitz-Zeager}
	\xmtitle{Answers for App Cartesian Plane}{}

\mfpicnumber{1} \opengraphsfile{ExercisesforAppCartesianPlane} % mfpic settings added 


\label{AnswersforAppCartesianPlane}


\begin{enumerate} 
%\setcounter{enumi}{\value{HW}}
\item The required points $\;A(-3, -7)$, $\;B(1.3, -2)$, $\;C(\pi, \sqrt{10})$, $\;D(0, 8)$, $\;E(-5.5, 0)$, $\;F(-8, 4)$, $\;G(9.2, -7.8)$, and $H(7, 5)$ are plotted in the Cartesian Coordinate Plane below. 

\begin{center}

\begin{mfpic}[20]{-10}{10}{-10}{10}
\axes
\tlabel[cc](10,-0.5){\scriptsize $x$}
\tlabel[cc](0.5,10){\scriptsize $y$}
\xmarks{-9,-8,-7,-6,-5,-4,-3,-2,-1,1,2,3,4,5,6,7,8,9}
\ymarks{-9,-8,-7,-6,-5,-4,-3,-2,-1,1,2,3,4,5,6,7,8,9}
\gfill \circle{(-3, -7),0.1}
\tlabel[cc](-3, -7.75){$A(-3,-7)$}
\gfill \circle{(1.3,-2),0.1}
\tlabel[cc](1.5, -2.5){$B(1.3, -2)$}
\gfill \circle{(3.14159, 3.16228),0.1}
\tlabel[cc](3.14, 2.7){$C(\pi, \sqrt{10})$}
\gfill \circle{(0, 8),0.1}
\tlabel[cc](1.25, 8){$D(0, 8)$}
\gfill \circle{(-5.5,0),0.1}
\tlabel[cc](-5.5, 0.5){$E(-5.5,0)$}
\gfill \circle{(-8,4),0.1}
\tlabel[cc](-8, 3.5){$F(-8, 4)$}
\gfill \circle{(9.2,-7.8),0.1}
\tlabel[cc](9.2, -8.3){$G(9.2, -7.8)$}
\gfill \circle{(7 ,5),0.1}
\tlabel[cc](7, 5.5){$H(7, 5)$}
\tlpointsep{5pt}
\scriptsize
\axislabels {x}{{$-9 \hspace{7pt}$} -9, {$-8 \hspace{7pt}$} -8, {$-7 \hspace{7pt}$} -7, {$-6 \hspace{7pt}$} -6, {$-5 \hspace{7pt}$} -5, {$-4 \hspace{7pt}$} -4, {$-3 \hspace{7pt}$} -3, {$-2 \hspace{7pt}$} -2, {$-1 \hspace{7pt}$} -1, {$1$} 1, {$2$} 2, {$3$} 3, {$4$} 4, {$5$} 5, {$6$} 6, {$7$} 7, {$8$} 8, {$9$} 9}
\axislabels {y}{{$-9$} -9, {$-8$} -8, {$-7$} -7, {$-6$} -6, {$-5$} -5, {$-4$} -4, {$-3$} -3, {$-2$} -2, {$-1$} -1, {$1$} 1, {$2$} 2, {$3$} 3, {$4$} 4, {$5$} 5, {$6$} 6, {$7$} 7, {$8$} 8, {$9$} 9}
\normalsize
\end{mfpic}

\end{center}

\pagebreak

\small %In order to fit everything on one page, we made it smaller.

\item \begin{multicols}{2}

\begin{enumerate}

\item The point $A(-3, -7)$ is 

\begin{itemize}

\item in Quadrant III
\item symmetric about $x$-axis with $(-3, 7)$
\item symmetric about $y$-axis with $(3, -7)$
\item symmetric about origin with $(3, 7)$

\end{itemize}

\item The point $B(1.3, -2)$ is 

\begin{itemize}

\item in Quadrant IV
\item symmetric about $x$-axis with $(1.3, 2)$
\item symmetric about $y$-axis with $(-1.3, -2)$
\item symmetric about origin with $(-1.3, 2)$

\end{itemize}

\setcounter{HWindent}{\value{enumii}}
\end{enumerate}
\end{multicols}

\begin{multicols}{2}
\begin{enumerate}
\setcounter{enumii}{\value{HWindent}}

\item The point $C(\pi, \sqrt{10})$ is 

\begin{itemize}

\item in Quadrant I
\item symmetric about $x$-axis with {\small $(\pi, -\sqrt{10})$}
\item symmetric about $y$-axis with {\small $(-\pi, \sqrt{10})$}
\item symmetric about origin with {\scriptsize $(-\pi, -\sqrt{10})$}

\end{itemize}

\item The point $D(0, 8)$ is 

\begin{itemize}

\item on the positive $y$-axis
\item symmetric about $x$-axis with $(0, -8)$
\item symmetric about $y$-axis with $(0, 8)$
\item symmetric about origin with $(0, -8)$

\end{itemize}


\setcounter{HWindent}{\value{enumii}}
\end{enumerate}
\end{multicols}

\begin{multicols}{2}
\begin{enumerate}
\setcounter{enumii}{\value{HWindent}}

\item The point $E(-5.5, 0)$ is 

\begin{itemize}

\item on the negative $x$-axis
\item symmetric about $x$-axis with $(-5.5, 0)$
\item symmetric about $y$-axis with $(5.5, 0)$
\item symmetric about origin with $(5.5, 0)$

\end{itemize}

\item The point $F(-8, 4)$ is 

\begin{itemize}

\item in Quadrant II
\item symmetric about $x$-axis with $(-8, -4)$
\item symmetric about $y$-axis with $(8, 4)$
\item symmetric about origin with $(8, -4)$

\end{itemize}

\setcounter{HWindent}{\value{enumii}}
\end{enumerate}
\end{multicols}

\begin{multicols}{2}
\begin{enumerate}
\setcounter{enumii}{\value{HWindent}}

\item The point $G(9.2, -7.8)$ is 

\begin{itemize}

\item in Quadrant IV
\item symmetric about $x$-axis with $(9.2, 7.8)$
\item symmetric about $y$-axis with {\scriptsize $(-9.2, -7.8)$}
\item symmetric about origin with $(-9.2, 7.8)$

\end{itemize}

\item The point $H(7, 5)$ is 

\begin{itemize}

\item in Quadrant I
\item symmetric about $x$-axis with $(7, -5)$
\item symmetric about $y$-axis with $(-7, 5)$
\item symmetric about origin with $(-7, -5)$

\end{itemize}

\end{enumerate}
\end{multicols}
\setcounter{HW}{\value{enumi}}
\end{enumerate}


\begin{multicols}{2}
\begin{enumerate}
\setcounter{enumi}{\value{HW}}

\item $d = 5$ units, $M = \left(-1, \frac{7}{2} \right)$
\item $d = 4 \sqrt{10}$ units, $M = \left(1, -4 \right)$

\setcounter{HW}{\value{enumi}}
\end{enumerate}
\end{multicols}

\begin{multicols}{2}
\begin{enumerate}
\setcounter{enumi}{\value{HW}}

\item $d = \sqrt{26}$ units, $M = \left(1, \frac{3}{2} \right)$
\item $d= \frac{\sqrt{37}}{2}$ units, $M = \left(\frac{5}{6}, \frac{7}{4} \right)$

\setcounter{HW}{\value{enumi}}
\end{enumerate}
\end{multicols}

\begin{multicols}{2}
\begin{enumerate}
\setcounter{enumi}{\value{HW}}

\item  $d = \sqrt{74}$ units, $M = \left(\frac{13}{10}, -\frac{13}{10} \right)$ \vphantom{$\left( \frac{\sqrt{3}}{2} \right)$}
\item $d= 3\sqrt{5}$ units, $M = \left(-\frac{\sqrt{2}}{2}, -\frac{\sqrt{3}}{2} \right)$

\setcounter{HW}{\value{enumi}}
\end{enumerate}
\end{multicols}

\begin{multicols}{2}
\begin{enumerate}
\setcounter{enumi}{\value{HW}}

\item  $d = \sqrt{83}$ units, $M = \left(4 \sqrt{5}, \frac{5 \sqrt{3}}{2} \right)$
\item $d = 2$ units, $M = \left( 0, 0\right)$ \vphantom{$\left( \frac{\sqrt{3}}{2} \right)$}

\setcounter{HW}{\value{enumi}}
\end{enumerate}
\end{multicols}


\begin{enumerate}
\setcounter{enumi}{\value{HW}}

\item $(-3, -4)$, $5$ miles, $(4, -4)$


\addtocounter{enumi}{2}

\item  \begin{enumerate}  

\item  The distance from $A$ to $B$ is $|AB| = \sqrt{13}$, the distance from $A$ to $C$ is $|AC| = \sqrt{52}$, and the distance from $B$ to $C$ is $|BC| = \sqrt{65}$.  Since $\left(\sqrt{13}\right)^2 + \left( \sqrt{52} \right)^2 = \left( \sqrt{65} \right)^2$, we are guaranteed by the \href{http://en.wikipedia.org/wiki/Pythagorean_theorem#Converse}{\underline{converse of the Pythagorean Theorem}} that the triangle is a right triangle.
\item Show that $|AC|^{2} + |BC|^{2} = |AB|^{2}$

\end{enumerate}

\end{enumerate}

\normalsize

\end{document}
