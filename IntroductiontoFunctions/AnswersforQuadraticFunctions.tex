\documentclass{ximera}

\begin{document}
	\author{Stitz-Zeager}
	\xmtitle{Answers for Quadratic Functions}{}

\mfpicnumber{1} \opengraphsfile{ExercisesforQuadraticFunctions} % mfpic settings added 


\begin{enumerate}

\item \begin{multicols}{2} \raggedcolumns
$f(x) = x^{2} + 2$ (this is both forms!) \\
No $x$-intercepts \\
$y$-intercept $(0, 2)$\\
Domain: $(-\infty, \infty)$ \\
Range: $[2, \infty)$ \\
Decreasing on $(-\infty, 0]$ \\
Increasing on $[0, \infty)$ \\
Vertex $(0, 2)$ is a minimum \\
Axis of symmetry $x = 0$ \\

\begin{mfpic}[15][10]{-3}{3}{-1}{11}
\axes
\tlabel[cc](3,-0.5){\scriptsize $x$}
\tlabel[cc](0.5,11){\scriptsize $y$}
\xmarks{-2,-1,1,2}
\ymarks{1 step 1 until 10}
\tlpointsep{4pt}
\scriptsize
\axislabels {x}{{$-2 \hspace{6pt}$} -2, {$-1 \hspace{6pt}$} -1, {$1$} 1, {$2$} 2}
\axislabels {y}{{$1$} 1, {$2$} 2, {$3$} 3, {$4$} 4, {$5$} 5, {$6$} 6, {$7$} 7, {$8$} 8, {$9$} 9, {$10$} 10}
\normalsize
\point[4pt]{(0,2)}
\penwd{1.25pt}
\arrow \reverse \arrow \function{-3,3,0.1}{x**2 + 2}
\end{mfpic}

\end{multicols}

\item \begin{multicols}{2} \raggedcolumns
$f(x) = -(x + 2)^{2} = -x^2-4x-4$\\
$x$-intercept $(-2, 0)$ \\
$y$-intercept $(0, -4)$\\
Domain: $(-\infty, \infty)$ \\
Range: $(-\infty, 0]$ \\
Increasing on $(-\infty, -2]$ \\
Decreasing on $[-2, \infty)$ \\
Vertex $(-2, 0)$ is a maximum \\
Axis of symmetry $x = -2$ \\

\begin{mfpic}[15][10]{-5}{1}{-9}{1}
\axes
\tlabel[cc](1,-0.5){\scriptsize $x$}
\tlabel[cc](0.5,1){\scriptsize $y$}
\xmarks{-4,-3,-2,-1}
\ymarks{-8 step 1 until -1}
\tlpointsep{4pt}
\scriptsize
\axislabels {x}{{$-4 \hspace{6pt}$} -4, {$-3 \hspace{6pt}$} -3, {$-2 \hspace{6pt}$} -2, {$-1 \hspace{6pt}$} -1}
\axislabels {y}{{$-8$} -8, {$-7$} -7, {$-6$} -6, {$-5$} -5, {$-4$} -4, {$-3$} -3, {$-2$} -2, {$-1$} -1}
\normalsize
\point[4pt]{(-2,0), (0,-4)}
\penwd{1.25pt}
\arrow \reverse \arrow \function{-5,1,0.1}{-((x + 2)**2)}
\end{mfpic}

\end{multicols}

\item \begin{multicols}{2} \raggedcolumns
$f(x) = x^{2} - 2x - 8 = (x - 1)^{2} - 9$\\
$x$-intercepts $(-2, 0)$ and $(4, 0)$\\
$y$-intercept $(0, -8)$\\
Domain: $(-\infty, \infty)$ \\
Range: $[-9, \infty)$ \\
Decreasing on $(-\infty, 1]$ \\
Increasing on $[1, \infty)$ \\
Vertex $(1, -9)$ is a minimum \\
Axis of symmetry $x = 1$ \\

\begin{mfpic}[15][10]{-3}{5}{-10}{3}
\axes
\tlabel[cc](5,-0.5){\scriptsize $x$}
\tlabel[cc](0.5,3){\scriptsize $y$}
\xmarks{-2 step 1 until 4}
\ymarks{-9 step 1 until 2}
\tlpointsep{4pt}
\scriptsize
\axislabels {x}{{$-2 \hspace{6pt}$} -2, {$-1 \hspace{6pt}$} -1, {$1$} 1, {$2$} 2, {$3$} 3, {$4$} 4}
\axislabels {y}{{$-9$} -9, {$-8$} -8, {$-7$} -7, {$-6$} -6, {$-5$} -5, {$-4$} -4, {$-3$} -3, {$-2$} -2, {$-1$} -1, {$1$} 1, {$2$} 2}
\normalsize
\point[4pt]{(-2,0),(0,-8),(1,-9),(4,0)}
\penwd{1.25pt}
\arrow \reverse \arrow \function{-2.4,4.4,0.1}{x**2 - 2*x - 8}
\end{mfpic}

\end{multicols}

\item \begin{multicols}{2} \raggedcolumns
$g(t) = -2(t + 1)^{2} + 4 = -2t^2-4t+2$\\
$t$-intercepts {\small $(-1 - \sqrt{2}, 0)$ and $(-1 + \sqrt{2}, 0)$}\\
$y$-intercept $(0, 2)$\\
Domain: $(-\infty, \infty)$ \\
Range: $(-\infty, 4]$ \\
Increasing on $(-\infty, -1]$ \\
Decreasing on $[-1, \infty)$ \\
Vertex $(-1, 4)$ is a maximum \\
Axis of symmetry $t = -1$ \\

\begin{mfpic}[20][10]{-3.5}{2}{-5}{5}
\axes
\tlabel[cc](2,-0.5){\scriptsize $t$}
\tlabel[cc](0.5,5){\scriptsize $y$}
\xmarks{-3 step 1 until 1}
\ymarks{-4 step 1 until 4}
\tlpointsep{4pt}
\scriptsize
\axislabels {x}{{$-3 \hspace{6pt}$} -3, {$-2 \hspace{6pt}$} -2, {$-1 \hspace{6pt}$} -1, {$1$} 1}
\axislabels {y}{{$-4$} -4, {$-3$} -3, {$-2$} -2, {$-1$} -1, {$1$} 1, {$2$} 2, {$3$} 3, {$4$} 4}
\normalsize
\point[4pt]{(-2.4142,0),(0,2),(-1,4),(.4142,0)}
\penwd{1.25pt}
\arrow \reverse \arrow \function{-3.1,1.1,0.1}{4- 2*((x + 1)**2)}
\end{mfpic}

\end{multicols}



\item \begin{multicols}{2} \raggedcolumns
$g(t) = 2t^2-tx-1 = 2(t-1)^2-3$\\
$t$-intercepts {\small $\left(\frac{2-\sqrt{6}}{2}, 0\right)$ and $\left(\frac{2+\sqrt{6}}{2}, 0\right)$}\\
$y$-intercept $(0, -1)$\\
Domain: $(-\infty, \infty)$ \\
Range: $[-3, \infty)$ \\
Increasing on $[1,\infty)$ \\
Decreasing on $(-\infty,1]$ \\
Vertex $(1, -3)$ is a minimum \\
Axis of symmetry $t = 1$ \\

\begin{mfpic}[15]{-2}{4}{-4}{5}
\axes
\tlabel[cc](4,-0.5){\scriptsize $t$}
\tlabel[cc](0.5,5){\scriptsize $y$}
\xmarks{-1 step 1 until 3}
\ymarks{-3 step 1 until 4}
\tlpointsep{4pt}
\scriptsize
\axislabels {x}{{$-1 \hspace{6pt}$} -1, {$1$} 1, {$2$} 2, {$3$} 3}
\axislabels {y}{{$-3$} -3, {$-2$} -2, {$-1$} -1, {$1$} 1, {$2$} 2, {$3$} 3, {$4$} 4}
\normalsize
\point[4pt]{(-0.2247,0),(0,-1),(1,-3),(2.2247,0)}
\penwd{1.25pt}
\arrow \reverse \arrow \function{-0.8,2.8,0.1}{2*(x**2)-4*x-1}
\end{mfpic}

\end{multicols}


\item \begin{multicols}{2} \raggedcolumns 
$g(t) = -3t^{2} + 4t - 7 = -3\left(t - \frac{2}{3} \right)^{2} - \frac{17}{3}$\\
No $t$-intercepts \\
$y$-intercept $(0, -7)$\\
Domain: $(-\infty, \infty)$ \\
Range: $\left(-\infty, -\frac{17}{3}\right]$ \\
Increasing on $\left(-\infty, \frac{2}{3}\right]$ \\
Decreasing on $\left[\frac{2}{3}, \infty\right)$ \\
Vertex $\left(\frac{2}{3}, -\frac{17}{3}\right)$ is a maximum \\
Axis of symmetry $t = \frac{2}{3}$ \\

\begin{mfpic}[20][10]{-1}{3}{-15}{1}
\axes
\tlabel[cc](3,-0.5){\scriptsize $t$}
\tlabel[cc](0.25,1){\scriptsize $y$}
\xmarks{1,2}
\ymarks{-14 step 1 until -1}
\tlpointsep{4pt}
\scriptsize
\axislabels {x}{{$1$} 1, {$2$} 2}
\axislabels {y}{{$-14$} -14, {$-13$} -13, {$-12$} -12, {$-11$} -11, {$-10$} -10, {$-9$} -9, {$-8$} -8, {$-7$} -7, {$-6$} -6, {$-5$} -5, {$-4$} -4, {$-3$} -3, {$-2$} -2, {$-1$} -1}
\normalsize
\point[4pt]{(0,-7),(.66667,-5.66667)}
\penwd{1.25pt}
\arrow \reverse \arrow \function{-1,2.33,0.1}{-3*(x**2) + 4*x - 7}
\end{mfpic}

\end{multicols}

\item \begin{multicols}{2} \raggedcolumns 
$h(s) = s^2+s+1 = \left(s + \frac{1}{2}\right)^{2} + \frac{3}{4}$\\
No $s$-intercepts \\
$y$-intercept $(0, 1)$\\
Domain: $(-\infty, \infty)$ \\
Range: $\left[ \frac{3}{4}, \infty\right)$ \\
Increasing on $\left[-\frac{1}{2}, \infty\right)$ \\
Decreasing on $\left(-\infty, -\frac{1}{2}\right]$ \\
Vertex $\left(-\frac{1}{2}, \frac{3}{4}\right)$ is a minimum \\
Axis of symmetry $s = -\frac{1}{2}$ \\

\begin{mfpic}[18]{-3}{2}{-1}{5}
\axes
\tlabel[cc](2,-0.5){\scriptsize $s$}
\tlabel[cc](0.5,5){\scriptsize $y$}
\xmarks{-2,-1,1}
\ymarks{1,2,3,4}
\tlpointsep{4pt}
\scriptsize
\axislabels {x}{{$-2 \hspace{6pt}$} -2,{$-1 \hspace{6pt}$} -1,{$1$} 1}
\axislabels {y}{{$1$} 1, {$2$} 2, {$3$} 3, {$4$} 4}
\normalsize
\point[4pt]{(0,1),(-0.5,0.75)}
\penwd{1.25pt}
\arrow \reverse \arrow \function{-2.5,1.5,0.1}{(x**2)+x+1}
\end{mfpic}

\end{multicols}

\pagebreak

\item \begin{multicols}{2} \raggedcolumns
$h(s) = -3s^2+5s+4 = -3\left(s-\frac{5}{6}\right)^2 + \frac{73}{12}$\\
$s$-intercepts {\small $\left(\frac{5 - \sqrt{73}}{6}, 0\right)$ and $\left(\frac{5+\sqrt{73}}{6}, 0\right)$}\\
$y$-intercept $(0, 4)$\\
Domain: $(-\infty, \infty)$ \\
Range: $\left(-\infty,  \frac{73}{12} \right]$ \\
Increasing on $\left(-\infty, \frac{5}{6}\right]$ \\
Decreasing on $\left[ \frac{5}{6}, \infty\right)$ \\
Vertex $\left(\frac{5}{6}, \frac{73}{12} \right)$ is a maximum \\
Axis of symmetry $s = \frac{5}{6}$ \\

\begin{mfpic}[15]{-2}{4}{-4}{7}
\axes
\tlabel[cc](4,-0.5){\scriptsize $s$}
\tlabel[cc](0.5,7){\scriptsize $y$}
\xmarks{-1 step 1 until 3}
\ymarks{-3 step 1 until 6}
\tlpointsep{4pt}
\scriptsize
\axislabels {x}{{$-1 \hspace{6pt}$} -1, {$1$} 1, {$2$} 2, {$3$} 3}
\axislabels {y}{{$-3$} -3, {$-2$} -2, {$-1$} -1, {$1$} 1, {$2$} 2, {$3$} 3, {$4$} 4, {$5$} 5, {$6$} 6}
\normalsize
\point[4pt]{(-0.5907,0),(0,4),(0.8333,6.0833),(2.2573,0)}
\penwd{1.25pt}
\arrow \reverse \arrow \function{-1,2.62,0.1}{0-3*(x**2)+5*x+4}
\end{mfpic}

\end{multicols}

\item \begin{multicols}{2} \raggedcolumns
$h(s) = s^{2} - \frac{1}{100} s - 1 = \left(s - \frac{1}{200}\right)^{2} - \frac{40001}{40000}$\\
$s$-intercepts $\left(\frac{1 + \sqrt{40001}}{200}\right)$ and $\left(\frac{1 - \sqrt{40001}}{200}\right)$\\
$y$-intercept $(0, -1)$\\
Domain: $(-\infty, \infty)$ \\
Range: $\left[-\frac{40001}{40000}, \infty \right)$ \\
Decreasing on $\left(-\infty, \frac{1}{200}\right]$ \\
Increasing on $\left[\frac{1}{200}, \infty \right)$ \\
Vertex $\left(\frac{1}{200}, -\frac{40001}{40000}\right)$ is a minimum\footnote{You'll need to use your calculator to zoom in far enough to see that the vertex is not the $y$-intercept.} \\
Axis of symmetry $s = \frac{1}{200}$ \\

\begin{mfpic}[15][10]{-3}{3}{-2}{9}
\axes
\tlabel[cc](3,-0.5){\scriptsize $s$}
\tlabel[cc](0.5,9){\scriptsize $y$}
\xmarks{-2,-1,1,2}
\ymarks{1 step 1 until 8}
\tlpointsep{4pt}
\scriptsize
\axislabels {x}{{$-2 \hspace{6pt}$} -2, {$-1 \hspace{6pt}$} -1, {$1$} 1, {$2$} 2}
\axislabels {y}{{$1$} 1, {$2$} 2, {$3$} 3, {$4$} 4, {$5$} 5, {$6$} 6, {$7$} 7, {$8$} 8}
\normalsize
\point[4pt]{(0,-1), (0.005, -1.000025)}
\penwd{1.25pt}
\arrow \reverse \arrow \function{-3,3,0.1}{x**2 - (x/100) - 1}
\end{mfpic}

\end{multicols}
\setcounter{HW}{\value{enumi}}
\end{enumerate}

\begin{multicols}{2}
\begin{enumerate}
\setcounter{enumi}{\value{HW}}

\item $F(x) = (x+2)^2-3$  \vphantom{$F(x) = \frac{1}{2}(x-2)^2-1$}

\item $F(x) = \frac{1}{2}(x-2)^2-1$

\setcounter{HW}{\value{enumi}}
\end{enumerate}
\end{multicols}

\begin{multicols}{2}
\begin{enumerate}
\setcounter{enumi}{\value{HW}}

\item $F(x) = -x^2+4$  

\item $F(x) =-2(x-1.5)^2+4.5$

\setcounter{HW}{\value{enumi}}
\end{enumerate}
\end{multicols}

%HERE

\begin{multicols}{2}
\begin{enumerate}
\setcounter{enumi}{\value{HW}}

\item $f(x) = x^2 - 3$ 

\item $F(x) = (x+1)^2-4 = x^2+2x-3$

\setcounter{HW}{\value{enumi}}
\end{enumerate}
\end{multicols}

\begin{multicols}{2}
\begin{enumerate}
\setcounter{enumi}{\value{HW}}

\item $F(s) = -(s+1)^2-1 = -s^2-2s-2$ \vphantom{$s(t) = -\frac{1}{3}(t-2)^2 + \frac{4}{3}= -\frac{1}{3} t^2 + \frac{4}{3} t$}

\item $s(t) = -\frac{1}{3}(t-2)^2 + \frac{4}{3}= -\frac{1}{3} t^2 + \frac{4}{3} t$

\setcounter{HW}{\value{enumi}}
\end{enumerate}
\end{multicols}


\begin{multicols}{2}
\begin{enumerate}
\setcounter{enumi}{\value{HW}}

\item $(-\infty, -3] \cup [1, \infty)$

\item  $\left(-\infty, -\frac{1}{4}\right) \cup \left(-\frac{1}{4}, \infty \right)$

\setcounter{HW}{\value{enumi}}
\end{enumerate}
\end{multicols}

\begin{multicols}{2}
\begin{enumerate}
\setcounter{enumi}{\value{HW}}

\item  No solution
\item  $(-\infty, \infty)$


\setcounter{HW}{\value{enumi}}
\end{enumerate}
\end{multicols}

\begin{multicols}{2}
\begin{enumerate}
\setcounter{enumi}{\value{HW}}

\item  $\left\{2 \right\}$
\item No solution


\setcounter{HW}{\value{enumi}}
\end{enumerate}
\end{multicols}

\begin{multicols}{2}
\begin{enumerate}
\setcounter{enumi}{\value{HW}}

\item $\left[-\frac{1}{3}, 4 \right]$
\item $(0, 1)$

\setcounter{HW}{\value{enumi}}
\end{enumerate}
\end{multicols}

\begin{multicols}{2}
\begin{enumerate}
\setcounter{enumi}{\value{HW}}


\item  $\left(-\infty, 1-\frac{\sqrt{6}}{2} \right) \cup \left(1+\frac{\sqrt{6}}{2}, \infty \right)$

\item  $\left(-\infty, \frac{5 - \sqrt{73}}{6} \right] \cup \left[\frac{5 + \sqrt{73}}{6}, \infty \right)$


\setcounter{HW}{\value{enumi}}
\end{enumerate}
\end{multicols}

\begin{multicols}{2}
\begin{enumerate}
\setcounter{enumi}{\value{HW}}

\item {\scriptsize $\left(-3\sqrt{2}, -\sqrt{11} \right] \cup \left[-\sqrt{7}, 0 \right) \cup \left(0, \sqrt{7} \right] \cup \left[\sqrt{11}, 3\sqrt{2} \right)$}
\item $\left[-2-\sqrt{7}, -2+\sqrt{7} \right] \cup [1, 3]$


\setcounter{HW}{\value{enumi}}
\end{enumerate}
\end{multicols}



\begin{multicols}{2}
\begin{enumerate}
\setcounter{enumi}{\value{HW}}

\item $(-\infty, \infty)$
\item  $(-\infty, -1] \cup \left\{ 0 \right\} \cup [1,\infty)$

\setcounter{HW}{\value{enumi}}
\end{enumerate}
\end{multicols}


\begin{multicols}{2}
\begin{enumerate}
\setcounter{enumi}{\value{HW}}


\item  $[-6,-3] \cup [-2, \infty)$

\item  $(-\infty, 1) \cup \left(2, \frac{3+\sqrt{17}}{2}\right)$


\setcounter{HW}{\value{enumi}}
\end{enumerate}
\end{multicols}


\begin{enumerate}
\setcounter{enumi}{\value{HW}}

\item \begin{itemize}

\item $P(x) = -2x^2+28x-26$, for $0 \leq x \leq 15$.

\item $7$ T-shirts should be made and  sold to maximize profit. 

\item The maximum profit is $\$72$. 

\item The price per T-shirt should be set at $\$16$ to maximize profit. 

\item The break even points are $x=1$ and $x=13$, so to make a profit, between 1 and 13 T-shirts need to be made and sold.

\end{itemize}

\item  \begin{itemize}

\item   $P(x) = -x^2+25x-100$, for $0 \leq x \leq 35$

\item  Since the vertex occurs at $x=12.5$, and it is impossible to make or sell $12.5$ bottles of tonic, maximum profit occurs when either $12$ or $13$ bottles of tonic are made and sold.

\item  The maximum profit is $\$56$.

\item  The price per bottle can be either $\$23$ (to sell 12 bottles) or $\$22$ (to sell 13 bottles.)  Both will result in the maximum profit.

\item The break even points are $x=5$ and $x=20$, so to make a profit, between 5 and 20 bottles of tonic need to be made and sold.

\end{itemize}



\item \begin{itemize}

\item  $P(x) = -3x^2+72x-240$, for $0 \leq x \leq 30$

\item  $12$ cups of lemonade need to be made and sold to maximize profit.

\item  The maximum profit is $192$\textcent \, or $\$1.92$.

\item  The price per cup should be set at $54$\textcent \, per cup to maximize profit.

\item  The break even points are $x=4$ and $x=20$, so to make a profit, between 4 and 20 cups of lemonade need to be made and sold.


\end{itemize}


\item \begin{itemize}

\item $P(x) = -0.5 x^2+9x-36$, for $0 \leq x \leq 24$

\item  $9$ pies should be made and sold to maximize the daily profit.

\item The maximum daily profit is $\$4.50$.

\item  The price per pie should be set at $\$7.50$ to maximize profit.

\item  The break even points are $x=6$ and $x=12$, so to make a profit, between 6 and 12 pies  need to be made and sold daily.

\end{itemize}

\item \begin{itemize}

\item  $P(x) = -2x^2+120x-1000$, for $0 \leq x \leq 70$

\item  $30$ scooters need to be made and sold to maximize profit.

\item  The maximum monthly profit is $800$ hundred dollars, or $\$80,\!000$.

\item The price per scooter should be set at $80$ hundred dollars, or $\$8000$ per scooter.

\item  The break even points are $x=10$ and $x=50$, so to make a profit, between 10 and 50 scooters  need to be made and sold monthly.

\end{itemize}

\setcounter{HW}{\value{enumi}}
\end{enumerate}

\begin{enumerate}
\setcounter{enumi}{\value{HW}}


\item 495 cookies

\item The vertex is (approximately) $(29.60, 22.66)$, which corresponds to a maximum fuel economy of 22.66 miles per gallon, reached sometime between 2009 and 2010 (29 -- 30 years after 1980.)  Unfortunately, the model is only valid up until 2008 (28 years after 1908.)  So, at this point, we are using the model to \textit{predict} the maximum fuel economy.



\item  $64^{\circ}$ at 2 PM (8 hours after 6 AM.)

\item  5000 pens should be produced for a cost of $\$200$.

\item 8 feet by 16 feet; maximum area is 128 square feet.

\item 50 feet by 50 feet;  maximum area is 2500 feet;  he can raise 100 average alpacas. 

\item The largest rectangle has area $12.25$ square inches.


\item  $2$ seconds.


\item  The rocket reaches its maximum height of $500$ feet $10$ seconds after lift-off.


\item  The hammer reaches a maximum height of approximately $13.62$ feet.  The hammer is in the air approximately $1.61$ seconds.  

\setcounter{HW}{\value{enumi}}
\end{enumerate}


\begin{enumerate}
\setcounter{enumi}{\value{HW}}


\item \begin{enumerate}

\item The applied domain is $[0, \infty)$.

\addtocounter{enumii}{2}

\item The height function is this case is $s(t) = -4.9t^{2} + 15t$.  The vertex of this parabola is approximately $(1.53, 11.48)$ so the maximum height reached by the marble is $11.48$ meters.  It hits the ground again when $t \approx 3.06$ seconds.

\item The revised height function is $s(t) = -4.9t^{2} + 15t + 25$ which has zeros at $t \approx -1.20$ and $t \approx 4.26$.  We ignore the negative value and claim that the marble will hit the ground after $4.26$ seconds.

\item Shooting down means the initial velocity is negative so the height functions becomes $s(t) = -4.9t^{2} - 15t + 25$.

\end{enumerate}

\item Make the vertex of the parabola $(0, 10)$ so that the point on the top of the left-hand tower where the cable connects is $(-200, 100)$ and the point on the top of the right-hand tower is $(200, 100)$.  Then the parabola is given by $p(x) = \frac{9}{4000}x^{2} + 10$.  Standing $50$ feet to the right of the left-hand tower means you're standing at $x= -150$ and $p(-150) = 60.625$.  So the cable is 60.625 feet above the bridge deck there.


\setcounter{HW}{\value{enumi}}
\end{enumerate}

\begin{enumerate}
\setcounter{enumi}{\value{HW}}

\item \begin{enumerate}

\item The line for the Thursday data is $y = -.12x + 237.69$.  We have $r = -.9568$ and $r^{2} = .9155$ so this is a really good fit.

\item The line for the Saturday data is $y = -0.000693x + 235.94$.  We have $r = -0.008986$ and $r^{2} = 0.0000807$ which is horrible.  This data is not even close to linear.  

\item The parabola for the Saturday data is $y = 0.003x^{2} - 0.21x + 238.30$.  We have $R^{2} = .47497$ which isn't good.  Thus the data isn't modeled well by a quadratic function, either.

\item The Thursday linear model had my weight on January 1, 2010 at 193.77 pounds.  The Saturday models give 235.69 and 563.31 pounds, respectively.  The Thursday line has my weight going below 0 pounds in about five and a half years, so that's no good.  The quadratic has a positive leading coefficient which would mean unbounded weight gain for the rest of my life.  The Saturday line, which mathematically does not fit the data at all, yields a plausible weight prediction in the end.  I think this is why grown-ups talk about ``Lies, Damned Lies and Statistics.''

\end{enumerate}

\item \begin{enumerate}

\item The quadratic model for the cats in Portage county is $y = 1917803.54x^{2} - 16036408.29x + 24094857.7$.  Although $R^{2} = .70888$ this is not a good model because it's so far off for small values of $x$.  The model gives us 24,094,858 cats when $x = 0$ but we know $N(0) = 2$.

\item The quadratic model for the hours of daylight in Fairbanks, Alaska is $y = .51x^{2} + 6.23x - .36$.  Even with $R^{2} = .92295$ we should be wary of making predictions beyond the data.  Case in point, the model gives $-4.84$ hours of daylight when $x = 13$.  So January 21, 2010 will be ``extra dark''?  Obviously a parabola pointing down isn't telling us the whole story.

\end{enumerate}

\setcounter{HW}{\value{enumi}}
\end{enumerate}

\begin{multicols}{2}
\begin{enumerate}
\setcounter{enumi}{\value{HW}}
\addtocounter{enumi}{1}
\item $y = |1 -x^{2}|$

\begin{mfpic}[12]{-3}{3}{-1}{8}
\axes
\tlabel[cc](3,-0.5){\scriptsize $x$}
\tlabel[cc](0.5,8){\scriptsize $y$}
\penwd{1.25pt}
\arrow \reverse \function{-3,-1,0.1}{x**2 - 1}
\function{-1,1,0.1}{1-x**2}
\arrow \function{1,3,0.1}{x**2-1}
\point[4pt]{(-1,0),(1,0),(0,1)}
\xmarks{-2 step 1 until 2}
\ymarks{1,2,3,4,5,6,7}
\scriptsize
\tlpointsep{4pt}
\axislabels {x}{{$-2 \hspace{6pt}$} -2, {$-1 \hspace{6pt}$} -1, {$1$} 1, {$2$} 2}
\axislabels {y}{{$1$} 1, {$2$} 2, {$3$} 3, {$4$} 4, {$5$} 5, {$6$} 6, {$7$} 7}
\normalsize
\end{mfpic}

\item $\left(\dfrac{3 - \sqrt{7}}{2}, \dfrac{-1 + \sqrt{7}}{2} \right)$, $\left(\dfrac{3 + \sqrt{7}}{2}, \dfrac{-1 - \sqrt{7}}{2} \right)$

\setcounter{HW}{\value{enumi}}
\end{enumerate}
\end{multicols}

\begin{enumerate}
\setcounter{enumi}{\value{HW}}


\item $D(x) = x^2 + (2x+1)^2 = 5x^2+4x+1$ is minimized when $x=-\frac{2}{5}$.  Hence to find the  point on $y=2x+1$ closest to $(0,0)$ we substitute $x = -\frac{2}{5}$ into  $y=2x+1$ to get $\left(-\frac{2}{5}, \frac{1}{5}\right)$.

\setcounter{HW}{\value{enumi}}
\end{enumerate}

\begin{multicols}{3}
\begin{enumerate}
\setcounter{enumi}{\value{HW}}
\addtocounter{enumi}{2}

\item $x = \pm y\sqrt{10}$ \vphantom{$\dfrac{m \pm \sqrt{m^{2} + 4}}{2}$}
\item $x = \pm (y - 2) $ \vphantom{$\dfrac{m \pm \sqrt{m^{2} + 4}}{2}$}
\item $x = \dfrac{m \pm \sqrt{m^{2} + 4}}{2}$

\setcounter{HW}{\value{enumi}}
\end{enumerate}
\end{multicols}

\begin{multicols}{3}
\begin{enumerate}
\setcounter{enumi}{\value{HW}}


\item $y = \dfrac{3 \pm \sqrt{16x + 9}}{2}$ \vphantom{$\dfrac{m \pm \sqrt{m^{2} + 4}}{2g}$}
\item $y = 2 \pm x$ \vphantom{$\dfrac{m \pm \sqrt{m^{2} + 4}}{2g}$}
\item $t = \dfrac{v_{\mbox{\scriptsize $0$}} \pm \sqrt{v_{\mbox{\scriptsize $0$}}^{2} + 4gs_{\mbox{\scriptsize $0$}}}}{2g}  $

\setcounter{HW}{\value{enumi}}
\end{enumerate}
\end{multicols}

\begin{enumerate}
\setcounter{enumi}{\value{HW}}

\item
\begin{enumerate}

\item \begin{multicols}{3}

\begin{enumerate}

\item   $L(x) = x^2$

\item   $L(x) = 2x^2+x$

\item  $L(x) = -x^2+5x+1$

\end{enumerate}

\end{multicols}

\addtocounter{enumii}{1}

\vspace{-.1in}

\item The three points lie on the same line and we get $L(x) = -x+5$.

\item  To obtain a quadratic function, we require that the points are not collinear (i.e., they do not all lie on the same line.)


\end{enumerate}

\setcounter{HW}{\value{enumi}}
\end{enumerate}



\end{document}
