\documentclass{ximera}

\begin{document}
	\author{Stitz-Zeager}
	\xmtitle{TITLE}
\mfpicnumber{1} \opengraphsfile{ExercisesforConstantandLinearFunctions} % mfpic settings added 


In Exercises \ref{graphlinearfunctionfirsta} - \ref{graphlinearfunctionlasta}, graph the function.  Find the slope and  axis intercepts, if any.

\begin{multicols}{2}
\begin{enumerate}


\item $f(x) = 2x - 1$ \label{graphlinearfunctionfirsta}
\item $g(t) = 3 - t$

\setcounter{HW}{\value{enumi}}
\end{enumerate}
\end{multicols}

\begin{multicols}{2}
\begin{enumerate}
\setcounter{enumi}{\value{HW}}

\item $F(w) = 3$
\item $G(s) = 0$

\setcounter{HW}{\value{enumi}}
\end{enumerate}
\end{multicols}

\begin{multicols}{2}
\begin{enumerate}
\setcounter{enumi}{\value{HW}}

\item $h(t) = \frac{2}{3} t + \frac{1}{3}$ \vphantom{$\dfrac{1-x}{2}$}
\item $j(w) = \dfrac{1-w}{2}$  \label{graphlinearfunctionlasta}

\setcounter{HW}{\value{enumi}}
\end{enumerate}
\end{multicols}

In Exercises \ref{graphpwiseexerfirst} - \ref{graphpwiseexerlast}, graph the function.  Find the domain, range, and axis intercepts, if any.

\enlargethispage{0.25in}

\begin{multicols}{2}
\begin{enumerate}
\setcounter{enumi}{\value{HW}}

\item ${\displaystyle f(x) = \left\{ \begin{array}{rcl} 4-x & \mbox{ if } &  x \leq 3 \\
                                                            2 & \mbox{ if } & x > 3
                                     \end{array} \right. }$   \label{graphpwiseexerfirst}

\item ${\displaystyle g(x) = \left\{ \begin{array}{rcl} 2-x & \mbox{ if } &  x < 2 \\
                                                            x-2 & \mbox{ if } & x \geq  2
                                     \end{array} \right. }$


\setcounter{HW}{\value{enumi}}
\end{enumerate}
\end{multicols}


\begin{multicols}{2}
\begin{enumerate}
\setcounter{enumi}{\value{HW}}

\item ${\displaystyle F(t) = \left\{ \begin{array}{rcl} -2t - 4 & \mbox{ if } &  t < 0 \\
                                                             3t & \mbox{ if } & t \geq 0
                                     \end{array} \right. }$


\item ${\displaystyle G(t) = \left\{ \begin{array}{rcl}  -3 & \mbox{ if } & t < 0 \\
                                                        2t-3 & \mbox{ if } & 0 < t < 3 \\
                                                            3 & \mbox{ if } & t > 3
                                     \end{array} \right. }$  \label{graphpwiseexerlast}



 \label{graphlineexerlast}

\setcounter{HW}{\value{enumi}}
\end{enumerate}
\end{multicols}

\begin{enumerate}
\setcounter{enumi}{\value{HW}}

\item  \label{unitstepexercise} The \index{unit step function}\textbf{unit step function} is defined as $U(t) = \begin{cases}
    0 &  \text{if $t<0$, } \\
    1  & \text{if $t \geq 0$.} \\
   \end{cases}$

\begin{enumerate}

\item  Graph $y = U(t)$.

\item  State the domain and range of $U$.

\item  List the interval(s) over which $U$ is increasing, decreasing, and/or constant.

\item  Write $U(t-2)$ as a piecewise defined function and graph.

\end{enumerate}


\setcounter{HW}{\value{enumi}}
\end{enumerate}


In Exercises \ref{findformulalinearfirstex} - \ref{findformulalinearlastex}, find a formula for the function.

\begin{multicols}{2}

\begin{enumerate}

\setcounter{enumi}{\value{HW}}

\item $~$   \label{findformulalinearfirstex}

\begin{mfpic}[15]{-5}{5}{-5}{5}
\axes
\tlabel[cc](5,-0.5){\scriptsize $x$}
\tlabel[cc](0.5,5){\scriptsize $y$}
\tlabel[cc](1, -2.75){\scriptsize $(0,-3)$}
\xmarks{-4,-3,-2,-1,1,2,3,4}
\ymarks{-4,-3,-2, -1, 1,2,3,4}
\tlpointsep{4pt}
\scriptsize
\axislabels {x}{ {$-4 \hspace{7pt}$} -4,{$-3 \hspace{7pt}$} -3, {$-2 \hspace{7pt}$} -2, {$-1 \hspace{7pt}$} -1, {$1$} 1, {$2$} 2, {$3$} 3, {$4$} 4}
\axislabels {y}{{$-4$} -4,{$-2$} -2, {$-1$} -1,{$1$} 1, {$2$} 2, {$3$} 3, {$4$} 4}
\penwd{1.25pt}
\arrow \reverse \arrow \polyline{( -5,-3), (5,-3)}
\point[4pt]{(0,-3)}
\tcaption{ \scriptsize$y = f(x)$}
\normalsize
\end{mfpic}



\item $~$


\begin{mfpic}[15]{-5}{5}{-5}{5}
\axes
\tlabel[cc](5,-0.5){\scriptsize $t$}
\tlabel[cc](0.5,5){\scriptsize $s$}
\tlabel[cc](1, 1.5){\scriptsize $(1,2)$}
\tlabel[cc](1, -3.5){\scriptsize $(1,-3)$}
\tlabel[cc](3, -2.5){\scriptsize $(3,-3)$}
\tlabel[cc](3, 3.5){\scriptsize $(3,4)$}
\xmarks{-4,-3,-2,-1,1,2,3,4}
\ymarks{-4,-3,-2, -1, 1,2,3,4}
\tlpointsep{4pt}
\scriptsize
\axislabels {x}{ {$-4 \hspace{7pt}$} -4,{$-3 \hspace{7pt}$} -3, {$-2 \hspace{7pt}$} -2, {$-1 \hspace{7pt}$} -1, {$1$} 1, {$2$} 2, {$3$} 3, {$4$} 4}
\axislabels {y}{{$-4$} -4,{$-3$} -3,{$-2$} -2, {$-1$} -1,{$1$} 1,  {$3$} 3, {$4$} 4}
\penwd{1.25pt}
\arrow \polyline{(1,2), (-5,2)}
 \polyline{(1,-3), (3,-3)}
 \arrow \polyline{(3,4), (5,4)}
\tcaption{ \scriptsize$s = F(t)$}
\point[4pt]{(1,2), (3,-3)}
\pointfillfalse
\point[4pt]{(1,-3), (3,4)}
\normalsize
\end{mfpic}


\setcounter{HW}{\value{enumi}}

\end{enumerate}

\end{multicols}

\begin{multicols}{2}

\begin{enumerate}

\setcounter{enumi}{\value{HW}}

\item $~$


\begin{mfpic}[15]{-5}{5}{-5}{5}
\axes
\tlabel[cc](5,-0.5){\scriptsize $x$}
\tlabel[cc](0.5,5){\scriptsize $y$}
\tlabel[cc](-1, 0.75){\scriptsize $(0,1)$}
\tlabel[cc](1.25, -0.75){\scriptsize $\left( \frac{5}{3}, 0 \right)$}
\xmarks{-4,-3,-2,-1,1,2,3,4}
\ymarks{-4,-3,-2, -1, 1,2,3,4}
\tlpointsep{4pt}
\scriptsize
\axislabels {x}{ {$-4 \hspace{7pt}$} -4,{$-3 \hspace{7pt}$} -3, {$-2 \hspace{7pt}$} -2, {$-1 \hspace{7pt}$} -1, {$3$} 3, {$4$} 4}
\axislabels {y}{{$-4$} -4,{$-3$} -3,{$-2$} -2, {$-1$} -1, {$2$} 2, {$3$} 3, {$4$} 4}
\penwd{1.25pt}
\arrow \reverse \arrow \polyline{( -5,4), (5,-2)}
\point[4pt]{(0,1), (1.66,0)}
\tcaption{ \scriptsize$y = L(x)$}
\normalsize
\end{mfpic}



\item $~$  \label{findformulalinearlastex}


\begin{mfpic}[15]{-5}{5}{-5}{5}
\axes
\tlabel[cc](5,-0.5){\scriptsize $v$}
\tlabel[cc](0.5,5){\scriptsize $w$}
\tlabel[cc](-3.25, -4.5){\scriptsize $(-3,-4)$}
\tlabel[cc](3, 2.5){\scriptsize $(3,2)$}
\tlabel[cc](-1.5, 2.5){\scriptsize $(-1,2)$}
\xmarks{-4,-3,-2,-1,1,2,3,4}
\ymarks{-4,-3,-2, -1, 1,2,3,4}
\tlpointsep{4pt}
\scriptsize
\axislabels {x}{ {$-4 \hspace{7pt}$} -4,{$-3 \hspace{7pt}$} -3, {$-2 \hspace{7pt}$} -2, {$-1 \hspace{7pt}$} -1, {$1$} 1,{$2$} 2, {$3$} 3, {$4$} 4}
\axislabels {y}{{$-4$} -4,{$-3$} -3,{$-2$} -2, {$-1$} -1, {$1$} 1, {$3$} 3, {$4$} 4}
\penwd{1.25pt}
\polyline{( -3,-4), (-1, 2), (3,2)}
\point[4pt]{(-3,-4),  (3,2)}
\pointfillfalse
\point[4pt]{(-1,2)}
\tcaption{ \scriptsize$w = g(v)$}
\normalsize
\end{mfpic}

\setcounter{HW}{\value{enumi}}

\end{enumerate}

\end{multicols}



\begin{enumerate}

\setcounter{enumi}{\value{HW}}



\item \label{piecewiseordering} For $n$ copies of the book \textit{Me and my Sasquatch}, a print on-demand company charges $C(n)$ dollars, where $C(n)$ is determined by the formula \[{\displaystyle C(n) = \left\{ \begin{array}{rcl}  15n & \mbox{ if } & 1 \leq n \leq 25  \\
                                                            13.50n  & \mbox{ if } & 25 < n \leq 50 \\
                                                            12n & \mbox{ if } & n > 50 \\
                                     \end{array} \right. }\]


\begin{enumerate}

\item  Find and interpret $C(20)$.

\item  \label{50vs51} How much does it cost to order 50 copies of the book?  What about 51 copies?

\item  Your answer to \ref{50vs51} should get you thinking. Suppose a bookstore estimates it will sell 50 copies of the book.  How many books can, in fact, be ordered for the same price as those 50 copies? (Round your answer to a  whole number of books.)

\end{enumerate}

\item \label{piecewiseshipping} An on-line comic book retailer charges shipping costs according to the following formula \[{\displaystyle S(n) = \left\{ \begin{array}{rcl}  1.5 n + 2.5 & \mbox{ if } & 1 \leq n \leq 14  \\
                                                            0  & \mbox{ if } & n \geq 15
                                     \end{array} \right. }\]

where $n$ is the number of  comic books purchased and $S(n)$ is the shipping cost in dollars.

\begin{enumerate}

\item  What is the cost to ship 10 comic books?

\item  What is the significance of the formula $S(n) = 0$ for $n \geq 15$?

\end{enumerate}

\newpage

\item  \label{piecewisemobile} The cost in dollars $C(m)$  to talk $m$ minutes a month on a mobile phone plan is modeled by   \[{\displaystyle C(m) = \left\{ \begin{array}{rcl} 25 & \mbox{ if } & 0 \leq m \leq 1000 \\
                                                            25+0.1(m-1000) & \mbox{ if } & m > 1000
                                     \end{array} \right. }\]

\begin{enumerate}

\item  How much does it cost to talk $750$ minutes per month with this plan?

\item  How much does it cost to talk $20$ hours a month with this plan?

\item  Explain the terms of the plan verbally.
\end{enumerate}


\setcounter{HW}{\value{enumi}}
\end{enumerate}


\begin{enumerate}

\setcounter{enumi}{\value{HW}}


\item  Jeff can walk comfortably at $3$ miles per hour.  Find an expression for a linear function $d(t)$ that represents the total distance Jeff can walk in $t$ hours, assuming he doesn't take any breaks.

\item  Carl can stuff $6$ envelopes per \textit{minute}.  Find an expression for a linear function $E(t)$ that represents the total number of envelopes Carl can stuff after $t$ \textit{hours}, assuming he doesn't take any breaks.

\item  A landscaping company charges $\$45$ per cubic yard of mulch plus a delivery charge of $\$20$.  Find an expression for a  linear function $C(x)$ which computes the total cost in dollars  to deliver $x$ cubic yards of mulch.

\item  A plumber charges $\$50$ for a service call plus $\$80$ per hour.  If she spends no longer than 8 hours a day at any one site, find an expression for a  linear function $C(t)$ that computes her total daily charges in dollars as a function of the amount of time spent in hours, $t$  at any one given location.

\item A salesperson is paid \$200 per week plus 5\% commission on her weekly sales of $x$ dollars.  Find an expression for a  linear function $W(x)$ which computes her total weekly pay in dollars as a function of $x$.  What must her weekly sales be in order for her to earn \$475.00 for the week?


\item  An on-demand publisher charges $\$22.50$ to print a 600 page book and $\$15.50$ to print a 400 page book.  Find an expression for a linear function which models the cost of a book in dollars $C(p)$ as a function of the number of pages $p$.  Find and interpret both the slope of the linear function  and $C(0)$.

\item The Topology Taxi Company charges $\$2.50$ for the first fifth of a mile and $\$0.45$ for each additional fifth of a mile.  Find an expression for a  linear function which models the taxi fare $F(m)$ as a function of the number of miles driven, $m$.  Find and interpret both the slope of the linear function  and $F(0)$.

\item Water freezes at $0^{\circ}$ Celsius and $32^{\circ}$ Fahrenheit and it boils at $100^{\circ}$C and $212^{\circ}$F.
\label{celsiustofahr}

\begin{enumerate}

\item Find an expression for a  linear function $F(T)$ that computes temperature in the Fahrenheit scale as a function of  the temperature $T$ given in degrees Celsius.  Use this function to convert $20^{\circ}$C into Fahrenheit.

\item Find an expression for a  linear function $C(T)$ that computes temperature in the Celsius scale as a function of  the temperature $T$ given in degrees Fahrenheit.  Use this function to convert $110^{\circ}$F into Celsius.

\item Is there a temperature $T$ such that $F(T) = C(T)$?

\end{enumerate}

\enlargethispage{0.5in}

\item Legend has it that a bull Sasquatch in rut will howl approximately 9 times per hour when it is $40^{\circ}F$ outside and only 5 times per hour if it's $70^{\circ}F$.  Assuming that the number of howls per hour, $N$, can be represented by a linear function of temperature Fahrenheit, find the number of howls per hour he'll make when it's only $20^{\circ}F$ outside. What troubles do you encounter when trying to determine a reasonable applied domain?

\item \label{exerredoportaboy} Economic forces have changed the cost function for PortaBoys to $C(x) = 105x + 175$.  Rework Example \ref{PortaBoyCost} with this new cost function.

\item In response to the economic forces in Exercise \ref{exerredoportaboy} above, the local retailer sets the selling price of a PortaBoy at \$250.  Remarkably, 30 units were sold each week.  When the systems went on sale for \$220, 40 units per week were sold.  Rework Example \ref{PortaBoyDemand}  with this new data.

\item A local pizza store offers medium two-topping pizzas delivered for $\$6.00$ per pizza plus a $\$1.50$ delivery charge per order.  On weekends, the store runs a `game day' special:  if six or more medium two-topping pizzas are ordered, they are $\$5.50$ each with no delivery charge.  Write a piecewise-defined linear function which calculates the cost in dollars $C(p)$ of  $p$ medium two-topping pizzas delivered during a weekend.

\item  A restaurant offers a buffet which costs $\$15$ per person.  For parties of $10$ or more people, a group discount applies, and the cost is $\$12.50$ per person.   Write a piecewise-defined linear function which calculates the total bill $T(n)$ of a party of $n$ people who all choose the buffet.

\item  A mobile plan charges a base monthly rate of $\$10$ for the first $500$ minutes of air time plus a charge of $15$\textcent \, for each additional minute.  Write a piecewise-defined linear function which calculates the monthly cost in dollars  $C(m)$  for using $m$ minutes of air time.

\textbf{HINT:}  You may wish to refer to number \ref{piecewisemobile} for inspiration.


\item  The local pet shop charges $12$\textcent \, per cricket up to 100 crickets, and $10$\textcent \, per cricket thereafter.  Write a piecewise-defined linear function which calculates the price in dollars $P(c)$ of purchasing $c$ crickets.

\item  The cross-section of a swimming pool is below.  Write a piecewise-defined linear function which describes the depth of the pool, $D$ (in feet) as a function of:

\begin{enumerate}

\item  the distance (in feet) from the edge of the shallow end of the pool, $d$.

\item  the distance (in feet) from the edge of the deep end of the pool, $s$.

\item  Graph each of the functions in (a) and (b).  Discuss with your classmates how to transform one into the other and how they relate to the diagram of the pool.

\end{enumerate}

\begin{center}

\begin{mfpic}[25]{-1}{13}{-1}{4}
\point[2pt]{(0,4), (12,4)}
\arrow \polyline{(0,4), (4,4)}
\arrow \polyline{(12,4), (8,4)}
\arrow \reverse \arrow \polyline{(-0.25,0), (-0.25,3)}
\arrow \reverse \arrow \polyline{(0,-0.35), (5,-0.35)}
\arrow \reverse \arrow \polyline{(0,3.35), (12,3.35)}
\arrow \reverse \arrow \polyline{(9,1.65), (12,1.65)}
\arrow \reverse \arrow \polyline{(12.25,2), (12.25,3)}
\gclear \tlabelrect(2, 4){\,$d$ ft.}
\gclear \tlabelrect(10, 4){\, $s$ ft.}
\gclear \tlabelrect(6, 3.35){37 ft.}
\gclear \tlabelrect(2.5, -0.35){15 ft.}
\gclear \tlabelrect(10.5, 1.65){10 ft.}
\gclear \tlabelrect(-1, 1.5){8 ft.}
\gclear \tlabelrect(13, 2.5){2 ft.}
\penwd{1.5pt}
\polyline{(0,0), (5,0), (9, 2), (12, 2), (12,3), (0,3), (0,0)}
\end{mfpic}

\end{center}

\setcounter{HW}{\value{enumi}}
\end{enumerate}


\begin{enumerate}
\setcounter{enumi}{\value{HW}}

\item \label{identityexercise} The function defined by $I(x) = x$ is called the \index{function ! identity} Identity Function.   Thinking from a procedural perspective, explain a possible origin of this name.

\item  \label{onlyoneyintexercise} Why must the graph of a function $y = f(x)$ have at most one $y$-intercept?

\textbf{HINT:}  Consider what would happen graphically if there were more than one \ldots

\item  Why is a discussion of vertical lines omitted when discussing functions?

\item  \label{xinterceptoflinear} Find a formula for the $x$-intercept of the graph of $f(x) = mx + b$.  Assume $m \neq 0$.

\item  \label{xinterceptformoflinear} Suppose $(c,0)$ is the $x$-intercept of a linear function $f$.  Use the point-slope form of a liner function, Equation \ref{linearfunctionpointslope} to show $f(x) = m(x-c)$.  This is the `slope $x$-intercept' form of the linear function.

\item Prove that for all linear functions $L$ with with slope $3$, $L(120) = L(100) + 60$.

\setcounter{HW}{\value{enumi}}
\end{enumerate}

\begin{enumerate}

\setcounter{enumi}{\value{HW}}

\item  Find the slopes between the following points from the data set given in Example \ref{timetempregressionex} and compare them with the slope of the corresponding regression line:

\begin{multicols}{4}

\begin{enumerate}

\item  $(0, 64)$, $(4, 75)$

\item  $(4, 75)$, $(8, 83)$

\item  $(8, 83)$, $(10, 83)$

\item  $(10, 83)$, $(12, 82)$

\end{enumerate}

\end{multicols}

\item According to this \href{http://www.ohiobiz.com/census/Lake.pdf}{\underline{website}}\footnote{\href{http://www.ohiobiz.com/census/Lake.pdf}{\underline{http://www.ohiobiz.com/census/Lake.pdf}}}, the census data for Lake County, Ohio is:

\noindent \begin{tabular}{|l|r|r|r|r|} \hline
Year & 1970 & 1980 & 1990 & 2000 \\
\hline
Population & 197200 & 212801 & 215499 & 227511 \\ \hline
\end{tabular}

\begin{enumerate}


\item  Find the least squares regression line for these data and comment on the goodness of fit.\footnote{We'll develop more sophisticated models for the growth of populations in Chapter \ref{ExponentialandLogarithmicFunctions}.  For the moment, we use a theorem from Calculus to approximate those functions with lines.} Interpret the slope of the line of best fit.

\item  Use the regression line to predict the population of Lake County in 2010.  (The recorded figure from the 2010 census is $230,\!041$)

\item  Use the regression line to predict when the population of Lake County will reach $250,\!000$.

\end{enumerate}


\item According to this \href{http://www.ohiobiz.com/census/Lorain.pdf}{\underline{website}}\footnote{\href{http://www.ohiobiz.com/census/Lorain.pdf}{\underline{http://www.ohiobiz.com/census/Lorain.pdf}}}, the census data for Lorain County, Ohio is:

\noindent \begin{tabular}{|l|r|r|r|r|} \hline
Year & 1970 & 1980 & 1990 & 2000 \\
\hline
Population & 256843 & 274909 & 271126 & 284664 \\ \hline
\end{tabular}

\begin{enumerate}


\item  Find the least squares regression line for these data and comment on the goodness of fit. Interpret the slope of the line of best fit.

\item  Use the regression line to predict the population of Lorain County in 2010.  (The recorded figure from the 2010 census is $301,\!356$)

\item  Use the regression line to predict when the population of Lake County will reach $325,\!000$.

\end{enumerate}

\item The chart below contains a portion of the fuel consumption information for a 2002 Toyota Echo that Jeffrey used to own.  The first row is the cumulative number of gallons of gasoline that I had used and the second row is the odometer reading when I refilled the gas tank.  So, for example, the fourth entry is the point (28.25, 1051) which says that I had used a total of 28.25 gallons of gasoline when the odometer read 1051 miles.

\medskip



\noindent \begin{tabular}{|l|r|r|r|r|r|r|r|r|r|r|r|} \hline
Gasoline Used & & & & & & & & & & & \\
(Gallons)  & 0 & 9.26 & 19.03 & 28.25 & 36.45 & 44.64 & 53.57 & 62.62 & 71.93 & 81.69 & 90.43\\
\hline
Odometer & & & & & & & & & & & \\
(Miles) & 41 & 356 & 731 & 1051 & 1347 & 1631 & 1966 & 2310 & 2670 & 3030 & 3371\\ \hline
\end{tabular}

\normalsize

\medskip

\noindent Find the least squares line for this data.  Is it a good fit?  What does the slope of the line represent?  Do you and your classmates believe this model would have held for ten years had I not crashed the car on the Turnpike a few years ago?


\item Using the energy production data given below

\noindent \begin{tabular}{|l|r|r|r|r|r|r|} \hline
Year & 1950 & 1960 & 1970 & 1980 & 1990 & 2000 \\
\hline
Production & & & & & & \\
(in Quads) & 35.6 & 42.8 & 63.5 & 67.2 & 70.7 & 71.2 \\ \hline
\end{tabular}

\begin{enumerate}

\item  Plot the data using a graphing utility and explain why it does not appear to be linear.

\item  Discuss with your classmates why ignoring the first two data points may be justified from a historical perspective.

\item Find the least squares regression line for the last four data points and comment on the goodness of fit. Interpret the slope of the line of best fit.

\item  Use the regression line to predict the annual US energy production in the year $2010$.

\item  Use the regression line to predict when the annual US energy production will reach $100$ Quads.

\end{enumerate}



\setcounter{HW}{\value{enumi}}
\end{enumerate}


In Exercises \ref{averagerateexerfirst} - \ref{averagerateexerlast}, compute the average rate of change of the  function over the specified interval.

\begin{multicols}{2}
\begin{enumerate}
\setcounter{enumi}{\value{HW}}

\item $f(x) = x^{3}, \; [-1, 2]$ \vphantom{$\dfrac{1}{x}$} \label{averagerateexerfirst}
\item $g(x) = \dfrac{1}{x}, \; [1, 5]$

\setcounter{HW}{\value{enumi}}
\end{enumerate}
\end{multicols}

\begin{multicols}{2}
\begin{enumerate}
\setcounter{enumi}{\value{HW}}

\item $f(t) = \sqrt{t}, \; [0, 16]$
\item $g(t) = x^{2}, \; [-3, 3]$

\setcounter{HW}{\value{enumi}}
\end{enumerate}
\end{multicols}

\begin{multicols}{2}
\begin{enumerate}
\setcounter{enumi}{\value{HW}}

\item $F(s) = \dfrac{s + 4}{s - 3}, \; [5, 7]$
\item $G(s) = 3s^{2} + 2s - 7, \; [-4, 2]$ \vphantom{$\dfrac{4}{x}$} \label{averagerateexerlast}

\setcounter{HW}{\value{enumi}}
\end{enumerate}
\end{multicols}



\begin{enumerate}
\setcounter{enumi}{\value{HW}}

\item  The height of an object dropped from the roof of a building is modeled by:  $h(t) = -16t^2 + 64$, for $0 \leq t \leq 2$. Here,  $h(t)$ is the height of the object off the ground in feet $t$ seconds after the object is dropped.  Find and interpret the average rate of change of $h$ over the interval $[0,2]$.

\item Using data from \href{http://www.bts.gov/publications/national_transportation_statistics/html/table_04_23.html}{\underline{Bureau of Transportation Statistics}}, the average fuel economy $F(t)$ in miles per gallon for passenger cars in the US can be modeled by  $F(t) = -0.0076t^2+0.45t + 16$, $0 \leq t \leq 28$, where $t$ is the number of years since $1980$. Find and interpret the average rate of change of $F$ over the interval $[0,28]$.



\item  The temperature $T(t)$ in degrees Fahrenheit $t$ hours after 6 AM is given by:

\[ T(t) = -\frac{1}{2} t^2 + 8t+32, \quad 0 \leq t \leq 12\]

\begin{enumerate}

\item  Find and interpret $T(4)$, $T(8)$ and $T(12)$.

\item  Find and interpret the average rate of change of $T$ over the interval $[4,8]$.

\item  Find and interpret the average rate of change of $T$ from $t=8$ to $t=12$.

\item  Find and interpret the average rate of temperature change between 10 AM and 6 PM.

\end{enumerate}

\item  Suppose $C(x) = x^2-10x+27$ represents the costs, in \textit{hundreds}, to produce $x$ \textit{thousand} pens.  Find and interpret the average rate of change as production is increased from making 3000 to 5000 pens.


\item \label{IRCRocketExercise} Recall from Example \ref{ARCRocketExample} The formula $s(t) = -5t^2+100t$ for $0 \leq t \leq 20$ gives the height, $s(t)$, measured in feet, of a model rocket above the Moon's surface as a function of the time after lift-off, $t$, in seconds.

\begin{enumerate}

\item  Find and interpret the average rate of change of $s$ over the following intervals:

\begin{multicols}{4}

\begin{enumerate}

\item $[14.9, 15]$

\item  $[15, 15.1]$

\item  $[14.99, 15]$

\item  $[15, 15.01]$

\end{enumerate}

\end{multicols}

\item  What value does the average rate of change appear to be approaching as the interval shrinks closer to the value $t=15$?

\item  Find the equation of the line containing $(15, 375)$ with slope $m = -50$ and graph it along with $s$ on the same set of axes using a graphing utility.  What happens as you zoom in near $(15, 375)$?

\end{enumerate}


\item  \label{lineshaveconstantratesofchange} Show the average rate of change of a function of the form $f(x) = mx+b$ over \textit{any} interval is $m$.

\item Why doesn't the graph of the vertical line $x = b$ in the $xy$-plane represent $y$ as a function of $x$?\label{whynoverticallineshere}

\item With help from a graphing utility, graph the following pairs of functions on the same set of axes:\footnote{See Example \ref{greatestintegerdefn} for the definition of $\lfloor x \rfloor$.}

\begin{multicols}{2}

\begin{itemize}


\item  $f(x) = 2-x$ and $g(x) = \lfloor 2-x \rfloor$

\item  $f(x) = x^2-4$ and $g(x) = \lfloor x^2 -4\rfloor$

\end{itemize}

\end{multicols}


\begin{multicols}{2}

\begin{itemize}

\item  $f(x) = x^3$ and $g(x) = \lfloor x^3 \rfloor$

\item  $f(x) = \sqrt{x}-4$ and $g(x) = \lfloor \sqrt{x} -4  \rfloor$

\end{itemize}

\end{multicols}

Choose more functions $f(x)$ and graph $y = f(x)$ alongside $y = \lfloor f(x) \rfloor$ until you can explain how, in general, one would obtain the graph of $y = \lfloor f(x) \rfloor$ given the graph of $y = f(x)$.


\item \label{LagrangeLinearExercise} The \href{https://en.wikipedia.org/wiki/Lagrange_polynomial}{\underline{Lagrange Interpolate}} function $L$ for two points $(x_{0}, y_{0})$ and $(x_{1}, y_{1})$ where $x_{0} \neq x_{1}$  is given by: \[L(x) = y_{0}  \dfrac{x - x_{1}}{x_{0} - x_{1}}+ y_{1}\dfrac{x - x_{0}}{x_{1} - x_{0}}\]

\begin{enumerate}

\item For each of the following pairs of points,  find  $L(x)$ using the formula above and verify each of the points lies on the graph of $y = L(x)$.

\begin{multicols}{4}

\begin{enumerate}

\item  $(-1,3)$, $(2,3)$

\item  $(-3,-2)$,  $(5,-2)$

\item  $(-3,-2)$, $(0,1)$

\item  $(-1,5)$, $(2,-1)$

\end{enumerate}

\end{multicols}

\item  Verify that, in general, $L(x_{0}) = y_{0}$ and $L(x_{1}) = y_{1}$.

\item  Show the point-slope form of a linear function, Equation \ref{linearfunctionpointslope} is equivalent to the formula given for $L(x)$ after making the identifications:  $f(x_{0}) = y_{0}$ and $m = \dfrac{y_{1} - y_{0}}{x_{1} - x_{0}}$.

\end{enumerate}




\setcounter{HW}{\value{enumi}}
\end{enumerate}


\newpage

\subsection{Answers}


\begin{enumerate}

\item \begin{multicols}{2} \raggedcolumns

$f(x) =2x-1$

slope: $m = 2$

$y$-intercept:  $(0,-1)$

$x$-intercept: $\left(\frac{1}{2}, 0 \right)$

\vfill

\columnbreak

\begin{mfpic}[15]{-3}{3}{-4}{4}
\point[4pt]{(0,-1), (0.5,0)}
\axes
\tlabel[cc](3,-0.5){\scriptsize $x$}
\tlabel[cc](0.5,4){\scriptsize $y$}
\xmarks{-2,-1,1,2}
\ymarks{-3,-2,-1,1,2,3}
\tlpointsep{4pt}
\tiny
\axislabels {x}{{$-2 \hspace{6pt}$} -2,{$-1 \hspace{6pt}$} -1, {$1$} 1, {$2$} 2}
\axislabels {y}{{$-3$} -3,{$-2$} -2,{$-1$} -1, {$1$} 1, {$2$} 2, {$3$} 3}
\normalsize
\penwd{1.25pt}
\arrow \reverse \arrow \function{-1,2, 0.1}{2*x-1}
\end{mfpic}

\end{multicols}

\item \begin{multicols}{2} \raggedcolumns

$g(t) =3-t$

slope: $m = -1$

$y$-intercept:  $(0,3)$

$t$-intercept: $(3, 0)$

\vfill

\columnbreak

\begin{mfpic}[15]{-2}{5}{-2}{5}
\point[4pt]{(0,3), (3,0)}
\axes
\tlabel[cc](5,-0.5){\scriptsize $t$}
\tlabel[cc](0.5,5){\scriptsize $y$}
\xmarks{-1,1,2,3,4}
\ymarks{-1,1,2,3,4}
\tlpointsep{4pt}
\tiny
\axislabels {x}{{$-1 \hspace{6pt}$} -1, {$1$} 1, {$2$} 2, {$3$} 3, {$4$} 4}
\axislabels {y}{{$-1$} -1, {$1$} 1, {$2$} 2, {$3$} 3, {$4$} 4}
\normalsize
\penwd{1.25pt}
\arrow \reverse \arrow \function{-1,4, 0.1}{3-x}
\end{mfpic}

\end{multicols}


\item \begin{multicols}{2} \raggedcolumns

$F(w) = 3$

slope: $m =0$

$y$-intercept:  $(0,3)$

$w$-intercept: none

\vfill

\columnbreak

\begin{mfpic}[15]{-3}{3}{-1}{5}
\point[4pt]{(0,3)}
\axes
\tlabel[cc](3,-0.5){\scriptsize $w$}
\tlabel[cc](0.5,5){\scriptsize $y$}
\xmarks{-2,-1,1,2}
\ymarks{1,2,3,4}
\tlpointsep{4pt}
\tiny
\axislabels {x}{{$-2 \hspace{6pt}$} -2,{$-1 \hspace{6pt}$} -1, {$1$} 1, {$2$} 2}
\axislabels {y}{{$1$} 1, {$2$} 2, {$3$} 3, {$4$} 4}
\normalsize
\penwd{1.25pt}
\arrow \reverse \arrow \function{-3,3, 0.1}{3}
\end{mfpic}

\end{multicols}

\item \begin{multicols}{2} \raggedcolumns

$G(s) = 0$

slope: $m =0$

$y$-intercept:  $(0,0)$

$s$-intercept: $\{ (s,0) \, | \, \text{$s$ is a real number} \}$

\vfill

\columnbreak

\begin{mfpic}[15]{-3}{3}{-2}{2}

\arrow \polyline{(0,-2), (0,2)}
\tlabel[cc](3,-0.5){\scriptsize $s$}
\tlabel[cc](0.5,2){\scriptsize $y$}
\xmarks{-2,-1,1,2}
\ymarks{-1,1}
\tlpointsep{4pt}
\tiny
\axislabels {x}{{$-2 \hspace{6pt}$} -2,{$-1 \hspace{6pt}$} -1, {$1$} 1, {$2$} 2}
\axislabels {y}{{$-1$} -1,{$1$} 1}
\normalsize
\penwd{1.25pt}
\arrow \reverse \arrow \function{-3,3, 0.1}{0}
\end{mfpic}

\end{multicols}

\newpage


\item \begin{multicols}{2} \raggedcolumns

$h(t) = \frac{2}{3} x + \frac{1}{3}$

slope: $m = \frac{2}{3}$

$y$-intercept:  $\left(0, \frac{1}{3}\right)$

$t$-intercept:  $\left(-\frac{1}{2}, 0\right)$

\vfill

\columnbreak

\begin{mfpic}[15]{-3}{3}{-2}{3}
\point[4pt]{(0,0.33333), (-0.5,0)}
\axes
\tlabel[cc](3,-0.5){\scriptsize $t$}
\tlabel[cc](0.5,3){\scriptsize $y$}
\xmarks{-2,-1,1,2}
\ymarks{-1,1,2}
\tlpointsep{4pt}
\tiny
\axislabels {x}{{$-2 \hspace{6pt}$} -2, {$1$} 1, {$2$} 2}
\axislabels {y}{{$-1$} -1, {$1$} 1, {$2$} 2}
\normalsize
\penwd{1.25pt}
\arrow \reverse \arrow \function{-3,3, 0.1}{0.66667*x+0.33333}
\end{mfpic}

\end{multicols}

\item \begin{multicols}{2} \raggedcolumns

$j(w)= \dfrac{1-w}{2}$

slope: $m = -\frac{1}{2}$

$y$-intercept:  $\left(0, \frac{1}{2}\right)$

$w$-intercept:  $\left(1, 0\right)$

\vfill

\columnbreak

\begin{mfpic}[15]{-3}{3}{-2}{3}
\point[4pt]{(0,0.5), (1,0)}
\axes
\tlabel[cc](3,-0.5){\scriptsize $w$}
\tlabel[cc](0.5,3){\scriptsize $y$}
\xmarks{-2,-1,1,2}
\ymarks{-1,1,2}
\tlpointsep{4pt}
\tiny
\axislabels {x}{{$-2 \hspace{6pt}$} -2,{$-1 \hspace{6pt}$} -1, {$1$} 1, {$2$} 2}
\axislabels {y}{{$-1$} -1, {$1$} 1, {$2$} 2}
\normalsize
\penwd{1.25pt}
\arrow \reverse \arrow \function{-3,3, 0.1}{0.5-0.5*x}
\end{mfpic}

\end{multicols}

\setcounter{HW}{\value{enumi}}
\end{enumerate}


\begin{enumerate}
\setcounter{enumi}{\value{HW}}

\item $~$

\begin{multicols}{2} \raggedcolumns

domain: $(-\infty, \infty)$

range:  $[1, \infty)$

$y$-intercept:  $(0,4)$

$x$-intercept: none

\vfill

\begin{mfpic}[10]{-2}{8}{-1}{6}
\axes
\tlabel[cc](0.5,6){\scriptsize $y$}
\tlabel[cc](8,-0.5){\scriptsize $x$}

\ymarks{1, 2, 3, 4, 5}
\xmarks{-1,1,2,3,4,5,6,7}
\tlpointsep{4pt}
\axislabels {y}{{\tiny $1$} 1, {\tiny $2$} 2, {\tiny $3$} 3, {\tiny $4$} 4, {\tiny $5$} 5}
\axislabels {x}{{\tiny $-1$ \hspace{7pt}} -1, {\tiny $1$} 1,{\tiny $2$} 2, {\tiny $3$} 3, {\tiny $4$} 4, {\tiny $5$} 5, {\tiny $6$} 6, {\tiny $7$} 7}
\penwd{1.25pt}
\arrow \polyline{(3,1), (-2,6)}
\arrow \polyline{(3,2), (8,2)}
\point[4pt]{(3,1), (0,4)}
\pointfillfalse
\point[4pt]{(3,2)}
\end{mfpic}

\end{multicols}

\item $~$

\begin{multicols}{2} \raggedcolumns

domain: $(-\infty, \infty)$

range:  $[0, \infty)$

$y$-intercept:  $(0,2)$

$x$-intercept: $(2,0)$

\vfill

\begin{mfpic}[10]{-2}{8}{-1}{6}
\axes
\tlabel[cc](0.5,6){\scriptsize $y$}
\tlabel[cc](8,-0.5){\scriptsize $x$}

\ymarks{1, 2, 3, 4, 5}
\xmarks{-1,1,2,3,4,5,6,7}
\tlpointsep{4pt}
\axislabels {y}{{\tiny $1$} 1, {\tiny $2$} 2, {\tiny $3$} 3, {\tiny $4$} 4, {\tiny $5$} 5}
\axislabels {x}{{\tiny $-1$ \hspace{7pt}} -1, {\tiny $1$} 1,{\tiny $2$} 2, {\tiny $3$} 3, {\tiny $4$} 4, {\tiny $5$} 5, {\tiny $6$} 6, {\tiny $7$} 7}
\penwd{1.25pt}
\arrow \reverse \arrow \polyline{(-2,4), (2,0), (6,4)}
\point[4pt]{(2,0), (0,2)}
\end{mfpic}

\end{multicols}

\item $~$

\begin{multicols}{2} \raggedcolumns

domain: $(-\infty, \infty)$

range:  $(-4, \infty)$

$y$-intercept:  $(0,0)$

$t$-intercepts: $(-2,0)$, $(0,0)$

\vfill

\begin{mfpic}[15]{-3}{2}{-4.3}{4}
\axes
\tlabel[cc](2,-0.5){\scriptsize $t$}
\tlabel[cc](0.5,3.75){\scriptsize $y$}
\xmarks{-2,-1,1}
\ymarks{-4,-3,-2,-1,1,2,3}
\tlpointsep{4pt}
\tiny
\axislabels {x}{{$-2 \hspace{6pt}$} -2, {$-1 \hspace{6pt}$} -1, {$1$} 1}
\axislabels {y}{{$-4$} -4,{$-3$} -3,{$-2$} -2,{$-1$} -1, {$1$} 1, {$2$} 2, {$3$} 3}
\normalsize
\penwd{1.25pt}
\arrow \reverse \function{-2.5, 0, 0.1}{-2*x - 4}
\arrow \function{0, 1.2, 0.1}{3*x}
\point[4pt]{(0,0), (-2,0)}
\pointfillfalse
\point[4pt]{(0,-4)}

\end{mfpic}
\end{multicols}

\item $~$

\begin{multicols}{2} \raggedcolumns

domain: $(-\infty, \infty)$

range:  $[-3, 3]$

$y$-intercept:  $(0,-3)$

$t$-intercept: $\left(\frac{3}{2}, 0 \right) = (1.5,0)$

\vfill

\begin{mfpic}[10]{-5}{5}{-4}{4}
\axes
\tlabel[cc](0.5,4){\scriptsize $y$}
\tlabel[cc](5,-0.5){\scriptsize $t$}
\ymarks{-3,-2,-1,1, 2, 3}
\xmarks{-4,-3,-2,-1,1,2,3,4}
\tlpointsep{4pt}
\axislabels {y}{{\tiny $-2$} -2,{\tiny $-1$} -1,{\tiny $1$} 1, {\tiny $2$} 2, {\tiny $3$} 3}
\axislabels {x}{{\tiny $-4$ \hspace{7pt}} -4, {\tiny $-3$ \hspace{7pt}} -3, {\tiny $-2$ \hspace{7pt}} -2,{\tiny $-1$ \hspace{7pt}} -1,{\tiny $1$} 1,{\tiny $2$} 2, {\tiny $3$} 3, {\tiny $4$} 4}
\penwd{1.25pt}
\arrow \polyline{(0,-3), (-5,-3)}
\arrow \polyline{(3,3), (5,3)}
\polyline{(0,-3), (3,3)}
\point[4pt]{(3,3), (0,-3), (1.5,0)}

\end{mfpic}

\end{multicols}

\setcounter{HW}{\value{enumi}}
\end{enumerate}


\begin{enumerate}
\setcounter{enumi}{\value{HW}}

\item

\begin{enumerate}

\item  $~$

\begin{multicols}{2} \raggedcolumns

\begin{mfpic}[15]{-5}{5}{-1}{2}
\axes
\tlabel[cc](5,-0.5){\scriptsize $t$}
\tlabel[cc](0.5,2){\scriptsize $y$}
\tlabel[cc](-1, 1){\scriptsize $(0,1)$}
\xmarks{-4,-3,-2,-1,1,2,3,4}
\ymarks{1}
\tlpointsep{4pt}
\scriptsize
\axislabels {x}{ {$-4 \hspace{7pt}$} -4,{$-3 \hspace{7pt}$} -3, {$-2 \hspace{7pt}$} -2, {$-1 \hspace{7pt}$} -1, {$1$} 1,{$2$} 2, {$3$} 3, {$4$} 4}
\penwd{1.25pt}
\arrow \reverse \polyline{( -5,0), (0,0)}
\arrow \polyline{( 0,1), (5,1)}
\point[4pt]{(0,1)}
\pointfillfalse
\point[4pt]{(0,0)}
\tcaption{ \scriptsize$y = U(t)$}
\normalsize
\end{mfpic}


\item  domain: $(-\infty, \infty)$, range:  $\{ 0, 1\}$

\item  $U$ is constant on $(-\infty, 0)$ and $[0, \infty)$.

\end{multicols}

 \item

 \begin{multicols}{2} \raggedcolumns


 $U(t-2) = \begin{cases}
    0 &  \text{if $t<2$, } \\
    1  & \text{if $t \geq 2$.} \\
   \end{cases}$


 \vfill

 \begin{mfpic}[15]{-5}{5}{-1}{2}
\axes
\tlabel[cc](5,-0.5){\scriptsize $t$}
\tlabel[cc](0.5,2){\scriptsize $y$}
\tlabel[cc](-1, 1){\scriptsize $(0,1)$}
\xmarks{-4,-3,-2,-1,1,2,3,4}
\ymarks{1}
\tlpointsep{4pt}
\scriptsize
\axislabels {x}{ {$-4 \hspace{7pt}$} -4,{$-3 \hspace{7pt}$} -3, {$-2 \hspace{7pt}$} -2, {$-1 \hspace{7pt}$} -1, {$1$} 1,{$2$} 2, {$3$} 3, {$4$} 4}
\penwd{1.25pt}
\arrow \reverse \polyline{( -5,0), (2,0)}
\arrow \polyline{(2,1), (5,1)}
\point[4pt]{(2,1)}
\pointfillfalse
\point[4pt]{(2,0)}
\tcaption{ \scriptsize$y = U(t-2)$}
\normalsize
\end{mfpic}

 \end{multicols}

\end{enumerate}


\setcounter{HW}{\value{enumi}}
\end{enumerate}


\begin{multicols}{2}

\begin{enumerate}

\setcounter{enumi}{\value{HW}}

\item $f(x) = -3$  \vphantom{$F(t) = \begin{cases}
  \hphantom{\text{$-$}}2 &  \text{if $t \leq 1$, } \\
  -3  & \text{if $1 < t \leq 3$,} \\
  \hphantom{\text{$-$}}4 & \text{if $t>3$.} \\
 \end{cases}$}

\item $F(t) = \begin{cases}
  \hphantom{\text{$-$}}2 &  \text{if $t \leq 1$, } \\
  -3  & \text{if $1 < t \leq 3$,} \\
  \hphantom{\text{$-$}}4 & \text{if $t>3$.} \\
 \end{cases}$

\setcounter{HW}{\value{enumi}}

\end{enumerate}

\end{multicols}

\begin{multicols}{2}

\begin{enumerate}

\setcounter{enumi}{\value{HW}}

\item $L(x) = -\frac{3}{5} x + 1$ \vphantom{ $g(v) = \begin{cases}
   3v+5 &  \text{if $v \leq -1$, } \\
  \hphantom{\text{$3v+$}}2  & \text{if $-1 <  v \leq 3$,} \\
    \end{cases}$}

\item $g(v) = \begin{cases}
   3v+5 &  \text{if $-3 \leq v < -1$, } \\
  \hphantom{\text{$3v+$}}2  & \text{if $-1 <  v \leq 3$,} \\
    \end{cases}$

\setcounter{HW}{\value{enumi}}

\end{enumerate}

\end{multicols}



\begin{enumerate}

\setcounter{enumi}{\value{HW}}


\item


\begin{enumerate}

\item   $C(20) = 300$.  It costs $\$300$ for 20 copies of the book.

\item $C(50) = 675$, $\$ 675$.  $C(51) = 612$, $\$ 612$.

\item   56 books.

\end{enumerate}

\item

\begin{enumerate}

\item   $S(10) = 17.5$, $\$ 17.50$.

\item  There is free shipping on orders of $15$ or more comic books.

\end{enumerate}

\item

\begin{enumerate}

\item   $C(750) = 25$, $\$ 25$.

\item   $C(1200) = 45$, $\$ 45$.

\item  It costs $\$25$ for up to $1000$ minutes and $10$ cents per minute for each minute over $1000$ minutes.

\end{enumerate}

\setcounter{HW}{\value{enumi}}
\end{enumerate}


\begin{multicols}{2}
\begin{enumerate}
\setcounter{enumi}{\value{HW}}

\item  $d(t) = 3t$, $t \geq 0$.
\item  $E(t) = 360t$, $t \geq 0$.

\setcounter{HW}{\value{enumi}}
\end{enumerate}
\end{multicols}

\begin{multicols}{2}
\begin{enumerate}
\setcounter{enumi}{\value{HW}}


\item  $C(x) = 45x+20$, $x \geq 0$.
\item  $C(t) = 80t + 50$,  $0 \leq t \leq 8$.
\setcounter{HW}{\value{enumi}}
\end{enumerate}
\end{multicols}


\begin{enumerate}
\setcounter{enumi}{\value{HW}}


\item  $W(x) = 200 + .05x,\, x \geq 0\;\;$ She must make \$5500 in weekly sales.

\setcounter{HW}{\value{enumi}}
\end{enumerate}


\begin{enumerate}
\setcounter{enumi}{\value{HW}}

\item  $C(p) = 0.035p + 1.5 \;$  The slope $0.035$ means it costs $3.5$\textcent \, per page.  $C(0) = 1.5$ means there is a fixed, or start-up, cost of $\$1.50$ to make each book.

\item $F(m) = 2.25m + 2.05 \;$  The slope $2.25$ means it costs an additional $\$2.25$ for each mile beyond the first 0.2 miles.  $F(0) = 2.05$, so according to the model, it would cost $\$2.05$ for a trip of $0$ miles.  Would this ever really happen?  Depends on the driver and the passenger, we suppose.


\item   \begin{multicols}{2}

\begin{enumerate}

\item $F(T) = \frac{9}{5}T + 32$
\item $C(T) = \frac{5}{9}(T - 32) = \frac{5}{9}T - \frac{160}{9}$

\setcounter{HWindent}{\value{enumii}}

\end{enumerate}

\end{multicols}

\begin{enumerate}
\setcounter{enumii}{\value{HWindent}}

\item $F(-40) = -40 = C(-40)$.

\end{enumerate}



\item $N(T) = -\frac{2}{15}T + \frac{43}{3}$  and $N(20) = \frac{35}{3} \approx 12$ howls per hour.

Having a negative number of howls makes no sense and since $N(107.5) = 0$ we can put an upper bound of $107.5^{\circ}F$ on the domain.  The lower bound is trickier because there's nothing other than common sense to go on.  As it gets colder, he howls more often.  At some point it will either be so cold that he freezes to death or he's howling non-stop.  So we're going to say that he can withstand temperatures no lower than $-42^{\circ}F$ so that the applied domain is $[-42, 107.5]$.

\item \begin{enumerate}

\item  $C(0) = 175$, so our start-up costs are $\$ 175$.  $C(5) = 700$, so to produce $5$ systems, it costs $\$ 700$.

\begin{center}

\begin{mfpic}[15]{-1}{7}{-1}{8.5}
\axes
\tlabel[cc](7,-0.5){\scriptsize $x$}
\tlabel[cc](0.5,8.5){\scriptsize $y$}
\tlabel[cc](1,1.25){\scriptsize $(0, 175)$}
\tlabel[cc](5.5,6.25){\scriptsize $(5, 700)$}
\xmarks{1,2,3,4,5,6}
\ymarks{1,2,3,4,5,6}
\tlpointsep{4pt}
\scriptsize
\axislabels {x}{  {$1$} 1, {$2$} 2, {$3$} 3, {$4$} 4, {$5$} 5, {$6$} 6}
\axislabels {y}{{$100$} 1, {$200$} 2, {$300$} 3, {$400$} 4,  {$500$} 5,  {$600$} 6,  {$700$} 7,  {$800$} 8}
\penwd{1.25pt}
\arrow  \polyline{(0,1.75), (6,8.05)}
\point[4pt]{(0,1.75), (5,7)}
\tcaption{ \scriptsize $y = C(x)$}
\normalsize
\end{mfpic}

\end{center}

\item   Since we can't make a negative number of game systems, $x \geq 0$.


\item The slope is $m = 105$ so for each additional system produced, it costs an additional $\$105$.

\item  Solving $C(x) = 15000$ gives $x \approx 141.19$ so  $141$ can be produced for $\$ 15, \! 000$.
\end{enumerate}

\newpage

\item \begin{enumerate}

\item  $p(x) = -3x+340$, $0 \leq x \leq 113$.

\smallskip

\begin{center}

\begin{mfpic}[15]{-1}{7}{-1}{8}
\axes
\tlabel[cc](7.5,0){\scriptsize $x$}
\tlabel[cc](0.5,8){\scriptsize $y$}
\tlabel[cc](1,6.75){\scriptsize $(0, 340)$}
\tlabel[cc](6,0.75){\scriptsize $(113, 1)$}
\xmarks{1,2,3,4,5,6}
\ymarks{1,2,3,4,5,6}
\tlpointsep{4pt}
\scriptsize
\axislabels {x}{  {$20$} 1, {$40$} 2, {$60$} 3, {$80$} 4, {$100$} 5, {$120$} 6}
\axislabels {y}{{$50$} 1, {$100$} 2, {$150$} 3, {$200$} 4,  {$250$} 5,  {$300$} 6}
\penwd{1.25pt}
\polyline{(0,6.8), (5.65,0.02)}
\point[4pt]{(0,6.8), (5.65,0.02)}
\tcaption{ \scriptsize $y = p(x)$}
\normalsize
\end{mfpic}


\end{center}



\item The slope is $m = -3$ so for each $\$3$ drop in price, we sell one additional game system.

\item   Since $x = 150$ is not in the domain of $p$, $p(150)$ is not defined.  (In other words, under these conditions, it is impossible to sell 150 game systems.)

\item Solving $p(x) = 150$ gives $x \approx 63.33$ so if the price  $\$150$ per system, we would sell $63$ systems.

\end{enumerate}



\item ${\displaystyle C(p) = \left\{ \begin{array}{rcl} 6p + 1.5 & \mbox{ if } & 1 \leq p \leq 5 \\
                                                            5.5p & \mbox{ if } & p\geq 6
                                     \end{array} \right. }$



\item  ${\displaystyle T(n) = \left\{ \begin{array}{rcl} 15n & \mbox{ if } & 1 \leq n \leq 9 \\
                                                            12.5n & \mbox{ if } & n \geq 10 \\
                                     \end{array} \right. }$


\item ${\displaystyle C(m) = \left\{ \begin{array}{rcl} 10 & \mbox{ if } & 0 \leq m \leq 500 \\
                                                            10+0.15(m-500) & \mbox{ if } & m > 500
                                     \end{array} \right. }$

\item ${\displaystyle P(c) = \left\{ \begin{array}{rcl} 0.12c & \mbox{ if } & 1 \leq c \leq 100 \\
                                                            12 + 0.1(c-100) & \mbox{ if } & c > 100
                                     \end{array} \right. }$

\item

\begin{enumerate}

\item \[{\displaystyle D(d) = \left\{ \begin{array}{rcl} 8 & \mbox{ if } & 0 \leq d \leq 15 \\
                                       -\frac{1}{2} \, d + \frac{31}{2} & \mbox{ if } & 15 \leq d \leq 27 \\
                                       2 & \mbox{ if } & 27 \leq d \leq 37  \\
                                     \end{array} \right. }\]

\item \[{\displaystyle D(s) = \left\{ \begin{array}{rcl} 2 & \mbox{ if } & 0 \leq s \leq 10 \\
                                       \frac{1}{2} \, s -3 & \mbox{ if } & 10 \leq s \leq 22 \\
                                       8 & \mbox{ if } & 22 \leq s \leq 37  \\
                                     \end{array} \right. }\]


\newpage

\item  $~$

\begin{center}
\begin{tabular}{cc}

\begin{mfpic}[10][15]{-1}{13}{-1}{4}
\axes
\point[4pt]{(0,3), (5,3), (9,1), (12,1)}
\xmarks{5,9,12}
\ymarks{0,1,3}
\tlpointsep{5pt}
\axislabels{x}{ {$15$} 5, {$27$} 9, {$37$} 12}
\axislabels{y}{ {$2$} 1, {$8$} 3}
\tcaption{$y = D(d)$}
\penwd{1.25pt}
\polyline{(0,3), (5,3), (9,1), (12,1)}
\end{mfpic}


&

\hspace{.5in}

\begin{mfpic}[10][15]{-1}{13}{-1}{4}
\axes
\point[4pt]{(12,3), (7,3), (3,1), (0,1)}
\xmarks{3,7,12}
\ymarks{0,1,3}
\tlpointsep{5pt}
\axislabels{x}{ {$10$} 3, {$22$} 7, {$37$} 12}
\axislabels{y}{ {$2$} 1, {$8$} 3}
\tcaption{$y = D(s)$}
\penwd{1.25pt}
\polyline{(12,3), (7,3), (3,1), (0,1)}
\end{mfpic}  \\

\end{tabular}

\end{center}

\end{enumerate}

\setcounter{HW}{\value{enumi}}
\end{enumerate}


\begin{enumerate}
\setcounter{enumi}{\value{HW}}

\item  Since $I(x) = x$ for all real numbers $x$, the function $I$ doesn't change the `identity' of the input at all.


\item If a graph contains more than one $y$-intercept, it would violate the Vertical Line Test since $x=0$ would be matched with (at least) two different $y$-values.

\item  Vertical Lines fail the Vertical Line Test.

\item  $\left(- \frac{b}{m}, 0 \right)$.  (Note the importance here of $m \neq 0$.)

\item Plugging in $(c,0)$ for $(x_{0}, f(x_{0}))$, we get $f(x) = f(x_{0}) + m (x - x_{0}) = 0 + m(x-c)$ or $f(x) = m(x-c)$.

\item Since $L$ is linear with slope $3$, $L(x) = L(x_{0}) + m \Delta x = L(100) + (3)(120-100) = L(100)+60$.

\setcounter{HW}{\value{enumi}}
\end{enumerate}


\begin{enumerate}

\setcounter{enumi}{\value{HW}}

\item

\begin{multicols}{2}

 \begin{enumerate}

\item $m = \frac{75-64}{4-0} = 2.75$

\item $m = \frac{83-75}{8-4} = 2$

\end{enumerate}

\end{multicols}

\begin{multicols}{2}

\begin{enumerate}

\addtocounter{enumii}{2}

\item  $m = \frac{83-83}{10-8} = 0$

\item  $m = \frac{82-83}{12-10} = -0.5$

\end{enumerate}

\end{multicols}

The first two points contributed to a regression line slope of $m = 2.55$;  the last two points contributed to a regression line slope of $m=-0.25$.

\item  \begin{enumerate}


\item  $y = 936.31x - 1645322.6$ with $r=0.9696$ which indicates a good fit.  The slope $936.31$ indicates Lake County's population is increasing at a rate of (approximately) 936 people per year.

\item  According to the model, the population in 2010 will be $236, \!660$.

\item  According to the model, the population of Lake County will reach $250,\!000$ sometime between 2024 and 2025.

\end{enumerate}

\item  \begin{enumerate}


\item  $y = 796.8x - 1309762.5$ with $r=0.8916$ which indicates a reasonable fit.  The slope $796.8$ indicates Lorain County's population is increasing at a rate of (approximately) 797 people per year.

\item  According to the model, the population in 2010 will be $291, \! 805$.

\item  According to the model, the population of Lake County will reach $325,\!000$ sometime between 2051 and 2052.

\end{enumerate}

\item The regression line is $y = 36.8x + 16.39$ with  $r = .99987$, so this is an excellent fit.  The slope $36.8$ represents mileage in miles per gallon.

\item \begin{enumerate}

\setcounter{enumii}{2}

\item $y = 0.266x - 459.86$ with $r = 0.9607$ which indicates a good fit.  The slope $0.266$ indicates the country's energy production is increasing at a rate of $0.266$ Quad per year.

\item According to the model, the production in 2010 will be $74.8$ Quad.

\item According to the model, the production will reach $100$ Quad in the year 2105.

\end{enumerate}



\setcounter{HW}{\value{enumi}}
\end{enumerate}

\vspace{-0.1in}

\begin{multicols}{4}
\begin{enumerate}
\setcounter{enumi}{\value{HW}}

\item $\frac{2^{3} - (-1)^{3}}{2 - (-1)} = 3$
\item $\frac{\frac{1}{5} - \frac{1}{1}}{5 - 1} = -\frac{1}{5}$




\item $\frac{\sqrt{16} - \sqrt{0}}{16 - 0} = \frac{1}{4}$
\item $\frac{3^{2} - (-3)^{2}}{3 - (-3)} = 0$

\setcounter{HW}{\value{enumi}}
\end{enumerate}
\end{multicols}

\begin{multicols}{2}
\begin{enumerate}
\setcounter{enumi}{\value{HW}}


\item $\dfrac{\frac{7 + 4}{7 - 3} - \frac{5 + 4}{5 - 3}}{7 - 5} = -\frac{7}{8}$
\item \scriptsize $\dfrac{(3(2)^{2}+2(2)-7)-(3(-4)^{2}+2(-4)-7)}{2-(-4)}=-4$ \normalsize



\setcounter{HW}{\value{enumi}}
\end{enumerate}
\end{multicols}


\begin{enumerate}
\setcounter{enumi}{\value{HW}}


\item The average rate of change is $\frac{h(2) - h(0)}{2-0}=-32$.  During the first two seconds after it is dropped, the object has fallen at an average rate of $32$ feet per second.

\item The average rate of change is $\frac{F(28) - F(0)}{28-0}=0.2372$.  From 1980 to 2008, the average fuel economy of passenger cars in the US increased, on average, at a rate of $0.2372$ miles per gallon per year.

\item
\begin{enumerate}

\item  $T(4) = 56$, so at 10 AM (4 hours after 6 AM), it is $56^{\circ}$F.  $T(8) = 64$, so at 2 PM (8 hours after 6 AM), it is $64^{\circ}$F.  $T(12) = 56$, so at 6 PM (12 hours after 6 AM), it is $56^{\circ}$F.

\item  The average rate of change is $\frac{T(8)-T(4)}{8-4}=2$.  Between 10 AM and 2 PM, the temperature increases, on average, at a rate of $2^{\circ}$F per hour.

\item  The average rate of change is $\frac{T(12)-T(8)}{12-8}=-2$.  Between 2 PM and 6 PM, the temperature decreases, on average, at a rate of $2^{\circ}$F per hour.

\item  The average rate of change is $\frac{T(12)-T(4)}{12-4}=0$.  Between 10 AM and 6 PM, the temperature, on average, remains constant.

\end{enumerate}


\item The average rate of change is $\frac{C(5)-C(3)}{5-3}=-2$.  As production is increased from 3000 to 5000 pens, the cost decreases at an average rate of  $\$200$ per 1000 pens produced (20\textcent \, per pen.)


\item

\begin{enumerate}

\item

\begin{enumerate}

\item  $-49.5$ so the average velocity of the rocket between $14.9$ and $15$ seconds after lift off is $-49.5$ feet per second ($49.5$ feet per second directed \textit{downwards}.)

\item  $-50.5$ so the average velocity of the rocket between $14$ and $15.1$ seconds after lift off is $-50.5$ feet per second. ($50.5$ feet per second directed \textit{downwards}.)

\item  $-49.95$ so the average velocity of the rocket between $14.99$ and $15$ seconds after lift off is $-49.95$ feet per second. ($49.95$ feet per second directed \textit{downwards}.)

\item   $-50.05$ so the average velocity of the rocket between $15.01$ and $15$ seconds after lift off is $-50.05$ feet per second. ($50.05$ feet per second directed \textit{downwards}.)
\end{enumerate}


\item  The average rate of change seem to be approaching $-50$.

\item  Line:  $y = -50(t-15) + 375$ or $y =   -50t + 1125$.  Graphing this line along with the $s$ on a graphing utility we find the two graphs become indistinguishable as we zoom in near $(15, 375)$.

\end{enumerate}

\enlargethispage{0.5in}

\addtocounter{enumi}{3}

\item \begin{enumerate}

 \item

 \begin{multicols}{4}

  \begin{enumerate}

  \item $L(x) = 3$

  \item $L(x) = -2$

  \item $L(x) = x+1$

  \item $L(x) = -2x+3$

  \end{enumerate}

  \end{multicols}

  \end{enumerate}
\setcounter{HW}{\value{enumi}}
\end{enumerate}


\end{document}
