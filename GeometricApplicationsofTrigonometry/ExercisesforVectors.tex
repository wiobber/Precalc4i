\documentclass{ximera}

\begin{document}
	\author{Stitz-Zeager}
	\xmtitle{Exercises for Vectors}{}

\mfpicnumber{1} \opengraphsfile{ExercisesforVectors} % mfpic settings added 


In Exercises \ref{vectorbasicfirst} - \ref{vectorbasiclast}, use the given pair of vectors $\vec{v}$ and $\vec{w}$ to find the following quantities.  State whether the result is a vector or a scalar.  

\medskip

\hspace{.15in} $\text{\tiny $\bullet$} \, \vec{v} + \vec{w} \;\;\;$ \hfill $\text{\tiny $\bullet$} \, \vec{w}  - 2\vec{v} \;\;\;$ \hfill $\text{\tiny $\bullet$} \, \| \vec{v} + \vec{w} \| \;\;\;$ \hfill $\text{\tiny $\bullet$} \, \| \vec{v} \| + \| \vec{w} \|$ \hfill $\text{\tiny $\bullet$} \, \| \vec{v} \| \vec{w} - \| \vec{w} \| \vec{v}$ \hfill $\text{\tiny $\bullet$} \, \|\vec{w}\| \hat{v}$

\medskip

Finally, verify that the vectors satisfy the \href{http://en.wikipedia.org/wiki/Parallelogram_law}{\underline{\textbf{Parallelogram Law}}}

\[ \|\vec{v}\|^2 + \|\vec{w}\|^2 = \dfrac{1}{2}\left[ \| \vec{v} + \vec{w}\|^2 + \|\vec{v} - \vec{w}\|^2\right] \]

\begin{multicols}{2}

\begin{enumerate}

\item  $\vec{v} = \left<12, -5\right>$, $\vec{w} = \left<3, 4\right>$ \label{vectorbasicfirst}
\item $\vec{v} = \left<-7, 24 \right>$, $\vec{w} = \left<-5, -12\right>$

\setcounter{HW}{\value{enumi}}

\end{enumerate}

\end{multicols}

\begin{multicols}{2}

\begin{enumerate}

\setcounter{enumi}{\value{HW}}

\item $\vec{v} = \left<2, -1 \right>$, $\vec{w} = \left<-2, 4 \right>$
\item $\vec{v} = \left<10, 4 \right>$, $\vec{w} = \left<-2, 5 \right>$

\setcounter{HW}{\value{enumi}}

\end{enumerate}

\end{multicols}

\begin{multicols}{2}

\begin{enumerate}

\setcounter{enumi}{\value{HW}}

\item $\vec{v} = \left<-\sqrt{3}, 1\right>$, $\vec{w} = \left<2\sqrt{3}, 2\right>$
\item  $\vec{v} = \left<\frac{3}{5}, \frac{4}{5}\right>$, $\vec{w} = \left<-\frac{4}{5}, \frac{3}{5}\right>$

\setcounter{HW}{\value{enumi}}

\end{enumerate}

\end{multicols}

\begin{multicols}{2}

\begin{enumerate}

\setcounter{enumi}{\value{HW}}

\item $\vec{v} = \left<\frac{\sqrt{2}}{2}, -\frac{\sqrt{2}}{2}\right>$, $\vec{w} = \left<-\frac{\sqrt{2}}{2}, \frac{\sqrt{2}}{2} \right>$
\item $\vec{v} = \left<\frac{1}{2}, \frac{\sqrt{3}}{2}  \right>$, $\vec{w} =  \left< -1, -\sqrt{3} \right>$

\setcounter{HW}{\value{enumi}}

\end{enumerate}

\end{multicols}

\begin{multicols}{2}

\begin{enumerate}

\setcounter{enumi}{\value{HW}}

\item $\vec{v} = 3\bm\hat{\text{i}} + 4\bm\hat{\text{j}}$, $\vec{w} = -2\bm\hat{\text{j}}$
\item $\vec{v} =\frac{1}{2} \left(\bm\hat{\text{i}} + \bm\hat{\text{j}}\right)$, $\vec{w} = \frac{1}{2} \left(\bm\hat{\text{i}} - \bm\hat{\text{j}}\right)$ \label{vectorbasiclast}

\setcounter{HW}{\value{enumi}}

\end{enumerate}

\end{multicols}

In Exercises \ref{vectorcompfirst} - \ref{vectorcomplast}, find the component form of the vector $\vec{v}$ using the information given about its magnitude and direction.  Give exact values.

\begin{enumerate}

\setcounter{enumi}{\value{HW}}

\item $\|\vec{v}\| = 6$; when drawn in standard position $\vec{v}$ lies in Quadrant I and makes a $60^{\circ}$ angle with the positive $x$-axis \label{vectorcompfirst}

\item $\|\vec{v}\| = 3$; when drawn in standard position $\vec{v}$ lies in Quadrant I and makes a $45^{\circ}$ angle with the positive $x$-axis

\item $\|\vec{v}\| = \frac{2}{3}$; when drawn in standard position $\vec{v}$ lies in Quadrant I and makes a $60^{\circ}$ angle with the positive $y$-axis

\item $\|\vec{v}\| = 12$; when drawn in standard position $\vec{v}$ lies along the positive $y$-axis

\item $\|\vec{v}\| = 4$; when drawn in standard position $\vec{v}$ lies in Quadrant II and makes a $30^{\circ}$ angle with the negative $x$-axis

\item $\|\vec{v}\| = 2\sqrt{3}$; when drawn in standard position $\vec{v}$ lies in Quadrant II and makes a $30^{\circ}$ angle with the positive $y$-axis

\item $\|\vec{v}\| = \frac{7}{2}$; when drawn in standard position $\vec{v}$ lies along the negative $x$-axis

\item $\|\vec{v}\| = 5\sqrt{6}$; when drawn in standard position $\vec{v}$ lies in Quadrant III and makes a $45^{\circ}$ angle with the negative $x$-axis

\item $\|\vec{v}\| = 6.25$; when drawn in standard position $\vec{v}$ lies along the negative $y$-axis

\item $\|\vec{v}\| = 4\sqrt{3}$; when drawn in standard position $\vec{v}$ lies in Quadrant IV and makes a $30^{\circ}$ angle with the positive $x$-axis

\item $\|\vec{v}\| = 5\sqrt{2}$; when drawn in standard position $\vec{v}$ lies in Quadrant IV and makes a $45^{\circ}$ angle with the negative $y$-axis

\item $\| \vec{v}\| = 2\sqrt{5}$; when drawn in standard position $\vec{v}$ lies in Quadrant I and makes an angle measuring $\arctan(2)$ with the positive $x$-axis

\item $\| \vec{v}\| = \sqrt{10}$; when drawn in standard position $\vec{v}$ lies in Quadrant II and makes an angle measuring $\arctan(3)$ with the negative $x$-axis

\item $\| \vec{v}\| = 5$; when drawn in standard position $\vec{v}$ lies in Quadrant III and makes an angle measuring $\arctan\left(\frac{4}{3}\right)$ with the negative $x$-axis

\item $\| \vec{v}\| = 26$; when drawn in standard position $\vec{v}$ lies in Quadrant IV and makes an angle measuring $\arctan\left(\frac{5}{12}\right)$ with the positive $x$-axis \label{vectorcomplast}

\setcounter{HW}{\value{enumi}}

\end{enumerate}

In Exercises \ref{vectorcompcalcfirst} - \ref{vectorcompcalclast}, approximate the component form of the vector $\vec{v}$ using the information given about its magnitude and direction.  Round your approximations to two decimal places.

\begin{enumerate}

\setcounter{enumi}{\value{HW}}

\item $\|\vec{v}\| = 392$; when drawn in standard position $\vec{v}$ makes a $117^{\circ}$ angle with the positive $x$-axis \label{vectorcompcalcfirst}

\item $\|\vec{v}\| = 63.92$; when drawn in standard position $\vec{v}$ makes a $78.3^{\circ}$ angle with the positive $x$-axis

\item $\|\vec{v}\| = 5280$; when drawn in standard position $\vec{v}$ makes a $12^{\circ}$ angle with the positive $x$-axis 

\item $\|\vec{v}\| = 450$; when drawn in standard position $\vec{v}$ makes a $210.75^{\circ}$ angle with the positive $x$-axis 

\item $\|\vec{v}\| = 168.7$; when drawn in standard position $\vec{v}$ makes a $252^{\circ}$ angle with the positive $x$-axis

\item $\| \vec{v}\| = 26$; when drawn in standard position $\vec{v}$ makes a $304.5^{\circ}$ angle with the positive $x$-axis \label{vectorcompcalclast}

\setcounter{HW}{\value{enumi}}

\end{enumerate}

In Exercises \ref{findmaganglefirst} - \ref{findmaganglelast}, for the given vector $\vec{v}$, find the magnitude $\|\vec{v}\|$ and an angle $\theta$ with $0 \leq \theta < 360^{\circ}$ so that $\vec{v} = \|\vec{v}\| \left<\cos(\theta), \sin(\theta) \right>$ (See Definition \ref{polarformvector}.)  Round approximations to two decimal places.

\begin{multicols}{3}

\begin{enumerate}

\setcounter{enumi}{\value{HW}}

\item  $\vec{v} = \left<1,\sqrt{3}\right>$ \label{findmaganglefirst} 
\item $\vec{v} = \left<5,5\right>$
\item $\vec{v} = \left<-2\sqrt{3}, 2 \right>$

\setcounter{HW}{\value{enumi}}

\end{enumerate}

\end{multicols}

\begin{multicols}{3}

\begin{enumerate}

\setcounter{enumi}{\value{HW}}

\item $\vec{v} = \left<-\sqrt{2}, \sqrt{2} \right>$
\item $\vec{v} = \left<-\frac{\sqrt{2}}{2}, -\frac{\sqrt{2}}{2}\right>$
\item $\vec{v} = \left<-\frac{1}{2}, -\frac{\sqrt{3}}{2}  \right>$

\setcounter{HW}{\value{enumi}}

\end{enumerate}

\end{multicols}

\begin{multicols}{3}

\begin{enumerate}

\setcounter{enumi}{\value{HW}}

\item $\vec{v} = \left<6, 0\right>$
\item $\vec{v} = \left<-2.5, 0\right>$
\item $\vec{v} = \left<0, \sqrt{7} \right>$

\setcounter{HW}{\value{enumi}}

\end{enumerate}

\end{multicols}

\begin{multicols}{3}

\begin{enumerate}

\setcounter{enumi}{\value{HW}}

\item  $\vec{v} = -10 \bm\hat{\text{j}}$
\item  $\vec{v} = \left< 3,4\right>$
\item  $\vec{v} = \left<12, 5\right>$

\setcounter{HW}{\value{enumi}}

\end{enumerate}

\end{multicols}

\begin{multicols}{3}

\begin{enumerate}

\setcounter{enumi}{\value{HW}}

\item $\vec{v} = \left<-4, 3 \right>$
\item  $\vec{v} = \left<-7, 24\right>$
\item $\vec{v} = \left<-2, -1 \right>$

\setcounter{HW}{\value{enumi}}

\end{enumerate}

\end{multicols}

\begin{multicols}{3}

\begin{enumerate}

\setcounter{enumi}{\value{HW}}

\item  $\vec{v} = \left<-2, -6\right>$
\item  $\vec{v} = \bm\hat{\text{i}} + \bm\hat{\text{j}}$
\item  $\vec{v} = \bm\hat{\text{i}} - 4\bm\hat{\text{j}}$

\setcounter{HW}{\value{enumi}}

\end{enumerate}

\end{multicols}

\begin{multicols}{3}

\begin{enumerate}

\setcounter{enumi}{\value{HW}}

\item  $\vec{v} = \left<123.4, -77.05\right>$
\item  $\vec{v} = \left<965.15, 831.6\right>$
\item  $\vec{v} = \left<-114.1, 42.3\right>$ \label{findmaganglelast}

\setcounter{HW}{\value{enumi}}

\end{enumerate}

\end{multicols}

\begin{enumerate}

\setcounter{enumi}{\value{HW}}

\item A small boat leaves the dock at Camp DuNuthin and heads across the Nessie River at 17 miles per hour (that is, with respect to the water) at a bearing of  S$68^{\circ}$W.   The river is flowing due east at 8 miles per hour.  What is the boat's true speed and heading?  Round the speed to the nearest mile per hour and express the heading as a bearing, rounded to the nearest tenth of a degree.  

\item \label{HMSSasquatchVectorBearing} The HMS Sasquatch leaves port with bearing S$20^{\circ}$E maintaining a speed of 42 miles per hour (that is, with respect to the water).  If the ocean current is 5 miles per hour with a bearing of N$60^{\circ}$E, find the HMS Sasquatch's true speed and bearing.  Round the speed to the nearest mile per hour and express the heading as a bearing, rounded to the nearest tenth of a degree. 

\item If the captain of the HMS Sasquatch in Exercise \ref{HMSSasquatchVectorBearing} wishes to reach Chupacabra Cove, an island 100 miles away at a bearing of  S$20^{\circ}$E from port, in three hours, what speed and heading should she set to take into account the ocean current?   Round the speed to the nearest mile per hour and express the heading as a bearing, rounded to the nearest tenth of a degree.  

\textbf{HINT:}  If $\vec{v}$ denotes the velocity of the HMS Sasquatch and $\vec{w}$ denotes the velocity of the current, what does $\vec{v} + \vec{w}$ need to be to reach Chupacabra Cove in three hours?

\item In calm air, a plane flying from the Pedimaxus International Airport can reach Cliffs of Insanity Point in two hours by following a bearing of N$8.2^{\circ}$E at 96 miles an hour.  (The distance between the airport and the cliffs is 192 miles.)  If the wind is blowing from the southeast at 25 miles per hour, what speed and bearing should the pilot take so that she makes the trip in two hours along the original heading?  Round the speed to the nearest hundredth of a mile per hour and your angle to the nearest tenth of a degree.

\item  The SS Bigfoot leaves Yeti Bay on a course of N$37^{\circ}$W at a speed of 50 miles per hour.  After traveling half an hour, the captain determines he is 30 miles from the bay and his bearing back to the bay is S$40^{\circ}$E.  What is the speed and bearing of the ocean current?  Round the speed to the nearest mile per hour and express the heading as a bearing, rounded to the nearest tenth of a degree.  

\item  A $600$ pound Sasquatch statue is suspended by two cables from a gymnasium ceiling.  If  each cable makes a $60^{\circ}$ angle with the ceiling, find the tension on each cable.  Round your answer to the nearest pound.

\item  Two cables are to support an object hanging from a ceiling.  If the cables are each to make a $42^{\circ}$ angle with the ceiling, and each cable is rated to withstand a maximum tension of $100$ pounds, what is the heaviest object that can be supported?  Round your answer down to the nearest pound.

\item A $300$ pound metal star is hanging on two cables which are attached to the ceiling.  The left hand cable makes a $72^{\circ}$ angle with the ceiling while the right hand cable makes a $18^{\circ}$ angle with the ceiling.  What is the tension on each of the cables?  Round your answers to three decimal places.

\item Two drunken college students have filled an empty beer keg with rocks and tied ropes to it in order to drag it down the street in the middle of the night.  The stronger of the two students pulls with a force of 100 pounds at a heading of N$77^{\circ}$E and the other pulls at a heading of S$68^{\circ}$E.  What force should the weaker student apply to his rope so that the keg of rocks heads due east?  What resultant force is applied to the keg?  Round your answer to the nearest pound.
\label{kegpull}

\item Emboldened by the success of their late night keg pull in Exercise \ref{kegpull} above, our intrepid young scholars have decided to pay homage to the chariot race scene from the movie `Ben-Hur' by tying three ropes to a couch, loading the couch with all but one of their friends and pulling it due west down the street. The first rope points N$80^{\circ}$W, the second points due west and the third points S$80^{\circ}$W.  The force applied to the first rope is 100 pounds, the force applied to the second rope is 40 pounds and the force applied (by the non-riding friend) to the third rope is 160 pounds.  They need the resultant force to be at least 300 pounds otherwise the couch won't move.  Does it move?  If so, is it heading due west?

\item Let $\vec{v} = \langle v_{\text{\tiny $1$}}, v_{\text{\tiny $2$}} \rangle$ be any non-zero vector. Show that $\dfrac{1}{\|\vec{v}\|} \vec{v}$ has length 1.

\item We say that two non-zero vectors $\vec{v}$ and $\vec{w}$ are {\bf parallel}\index{vector ! parallel}\index{parallel vectors} if they have same or opposite directions.  That is, $\vec{v} \neq \vec{0}$ and $\vec{w} \neq \vec{0}$ are parallel if either $\hat{v} = \hat{w}$ or $\hat{v} = -\hat{w}$.  Show that this means $\vec{v} = k\vec{w}$ for some non-zero scalar $k$ and that $k > 0$ if the vectors have the same direction and $k < 0$ if they point in opposite directions.
\label{parallelvectorexercise}

\item The goal of this exercise is to use vectors to describe non-vertical lines in the plane.  To that end, consider the line $y = 2x - 4$. Let $\vec{v}_{\text{\tiny $0$}} = \langle 0, -4 \rangle$ and let $\vec{s} = \langle 1, 2 \rangle$.  Let $t$ be any real number.  Show that the vector defined by $\vec{v} = \vec{v}_{\text{\tiny $0$}} + t\vec{s}$, when drawn in standard position, has its terminal point on the line $y = 2x - 4$.  (Hint: Show that $\vec{v}_{\text{\tiny $0$}} + t\vec{s} = \langle t, 2t - 4 \rangle$ for any real number $t$.)  Now consider the non-vertical line $y = mx + b$.  Repeat the previous analysis with  $\vec{v}_{\text{\tiny $0$}} = \langle 0, b \rangle$ and let $\vec{s} = \langle 1, m \rangle$.  Thus any non-vertical line can be thought of as a collection of terminal points of the vector sum of $\langle 0, b \rangle$ (the position vector of the $y$-intercept) and a scalar multiple of the slope vector $\vec{s} = \langle 1, m \rangle$.
\label{2dvectorsgiveuslines} 

\item Prove the associative and identity properties of vector addition in Theorem \ref{vectoradditionprops}.

\item Prove the properties of scalar multiplication in Theorem \ref{vectorscalarmultprops}. 

\end{enumerate}

\newpage

\subsection{Answers}

\begin{enumerate}

\item  

\begin{multicols}{2}

\begin{itemize}

\item  $\vec{v} + \vec{w} = \left<15,-1 \right> $, vector
\item  $\vec{w}  - 2\vec{v}  = \left<-21,14 \right>$, vector

\end{itemize}

\end{multicols}

\begin{multicols}{2}

\begin{itemize}

\item $\| \vec{v} + \vec{w} \| = \sqrt{226}$, scalar
\item  $\| \vec{v} \| + \| \vec{w}\| = 18$, scalar

\end{itemize}

\end{multicols}

\begin{multicols}{2}

\begin{itemize}

\item $\| \vec{v} \| \vec{w} - \| \vec{w} \| \vec{v}  = \left<-21,77\right>$, vector
\item $\|w\| \hat{v}= \left<\frac{60}{13}, -\frac{25}{13} \right>$, vector

\end{itemize}

\end{multicols}

\item  

\begin{multicols}{2}

\begin{itemize}

\item  $\vec{v} + \vec{w} = \left<-12,12 \right> $, vector
\item  $\vec{w}  - 2\vec{v}  = \left<9,-60 \right>$, vector

\end{itemize}

\end{multicols}

\begin{multicols}{2}

\begin{itemize}

\item $\| \vec{v} + \vec{w} \| = 12\sqrt{2}$, scalar
\item  $\| \vec{v} \| + \| \vec{w}\| = 38$, scalar

\end{itemize}

\end{multicols}

\begin{multicols}{2}

\begin{itemize}

\item $\| \vec{v} \| \vec{w} - \| \vec{w} \| \vec{v}  = \left<-34,-612\right>$, vector
\item $\|w\| \hat{v}= \left<-\frac{91}{25}, \frac{312}{25} \right>$, vector

\end{itemize}

\end{multicols}

\item  

\begin{multicols}{2}

\begin{itemize}

\item  $\vec{v} + \vec{w} = \left<0,3\right> $, vector
\item  $\vec{w}  - 2\vec{v}  = \left<-6,6 \right>$, vector

\end{itemize}

\end{multicols}

\begin{multicols}{2}

\begin{itemize}

\item $\| \vec{v} + \vec{w} \| = 3$, scalar
\item  $\| \vec{v} \| + \| \vec{w}\| = 3\sqrt{5}$, scalar

\end{itemize}

\end{multicols}

\begin{multicols}{2}

\begin{itemize}

\item $\| \vec{v} \| \vec{w} - \| \vec{w} \| \vec{v}  = \left<-6\sqrt{5},6\sqrt{5}\right>$, vector
\item $\|w\| \hat{v}= \left<4, -2 \right>$, vector

\end{itemize}

\end{multicols}

\item  

\begin{multicols}{2}

\begin{itemize}

\item  $\vec{v} + \vec{w} = \left<8,9\right> $, vector
\item  $\vec{w}  - 2\vec{v}  = \left<-22, -3 \right>$, vector

\end{itemize}

\end{multicols}

\begin{multicols}{2}

\begin{itemize}

\item $\| \vec{v} + \vec{w} \| = \sqrt{145}$, scalar
\item  $\| \vec{v} \| + \| \vec{w}\| = 3\sqrt{29}$, scalar

\end{itemize}

\end{multicols}

\begin{multicols}{2}

\begin{itemize}

\item $\| \vec{v} \| \vec{w} - \| \vec{w} \| \vec{v}  = \left<-14\sqrt{29},6\sqrt{29}\right>$, vector
\item $\|w\| \hat{v}= \left<5, 2 \right>$, vector

\end{itemize}

\end{multicols}

\item  

\begin{multicols}{2}

\begin{itemize}

\item  $\vec{v} + \vec{w} = \left<\sqrt{3},3\right> $, vector
\item  $\vec{w}  - 2\vec{v}  = \left<4\sqrt{3}, 0 \right>$, vector

\end{itemize}

\end{multicols}

\begin{multicols}{2}

\begin{itemize}

\item $\| \vec{v} + \vec{w} \| = 2\sqrt{3}$, scalar
\item  $\| \vec{v} \| + \| \vec{w}\| = 6$, scalar

\end{itemize}

\end{multicols}

\begin{multicols}{2}

\begin{itemize}

\item $\| \vec{v} \| \vec{w} - \| \vec{w} \| \vec{v}  = \left<8\sqrt{3},0\right>$, vector
\item $\|w\| \hat{v}= \left<-2\sqrt{3}, 2 \right>$, vector

\end{itemize}

\end{multicols}

\item  

\begin{multicols}{2}

\begin{itemize}

\item  $\vec{v} + \vec{w} = \left<-\frac{1}{5},\frac{7}{5}\right> $, vector
\item  $\vec{w}  - 2\vec{v}  = \left<-2, -1 \right>$, vector

\end{itemize}

\end{multicols}

\begin{multicols}{2}

\begin{itemize}

\item $\| \vec{v} + \vec{w} \| = \sqrt{2}$, scalar
\item  $\| \vec{v} \| + \| \vec{w}\| = 2$, scalar

\end{itemize}

\end{multicols}

\begin{multicols}{2}

\begin{itemize}

\item $\| \vec{v} \| \vec{w} - \| \vec{w} \| \vec{v}  = \left<-\frac{7}{5},-\frac{1}{5}\right>$, vector
\item $\|w\| \hat{v}= \left<\frac{3}{5}, \frac{4}{5} \right>$, vector

\end{itemize}

\end{multicols}

\item  

\begin{multicols}{2}

\begin{itemize}

\item  $\vec{v} + \vec{w} = \left<0,0\right> $, vector
\item  $\vec{w}  - 2\vec{v}  = \left<-\frac{3\sqrt{2}}{2}, \frac{3\sqrt{2}}{2} \right>$, vector

\end{itemize}

\end{multicols}

\begin{multicols}{2}

\begin{itemize}

\item $\| \vec{v} + \vec{w} \| = 0$, scalar
\item  $\| \vec{v} \| + \| \vec{w}\| = 2$, scalar

\end{itemize}

\end{multicols}

\begin{multicols}{2}

\begin{itemize}

\item $\| \vec{v} \| \vec{w} - \| \vec{w} \| \vec{v}  = \left<-\sqrt{2},\sqrt{2}\right>$, vector
\item $\|w\| \hat{v}= \left<\frac{\sqrt{2}}{2}, -\frac{\sqrt{2}}{2} \right>$, vector

\end{itemize}

\end{multicols}

\pagebreak

\item  

\begin{multicols}{2}

\begin{itemize}

\item  $\vec{v} + \vec{w} = \left<-\frac{1}{2}, -\frac{\sqrt{3}}{2}\right> $, vector
\item  $\vec{w}  - 2\vec{v}  = \left<-2, -2\sqrt{3} \right>$, vector

\end{itemize}

\end{multicols}

\begin{multicols}{2}

\begin{itemize}

\item $\| \vec{v} + \vec{w} \| = 1$, scalar
\item  $\| \vec{v} \| + \| \vec{w}\| = 3$, scalar

\end{itemize}

\end{multicols}

\begin{multicols}{2}

\begin{itemize}

\item $\| \vec{v} \| \vec{w} - \| \vec{w} \| \vec{v}  = \left<-2,-2\sqrt{3}\right>$, vector
\item $\|w\| \hat{v}= \left<1, \sqrt{3} \right>$, vector

\end{itemize}

\end{multicols}

\item  

\begin{multicols}{2}

\begin{itemize}

\item  $\vec{v} + \vec{w} = \left<3,2\right> $, vector
\item  $\vec{w}  - 2\vec{v}  = \left<-6, -10 \right>$, vector

\end{itemize}

\end{multicols}

\begin{multicols}{2}

\begin{itemize}

\item $\| \vec{v} + \vec{w} \| = \sqrt{13}$, scalar
\item  $\| \vec{v} \| + \| \vec{w}\| = 7$, scalar

\end{itemize}

\end{multicols}

\begin{multicols}{2}

\begin{itemize}

\item $\| \vec{v} \| \vec{w} - \| \vec{w} \| \vec{v}  = \left<-6,-18\right>$, vector
\item $\|w\| \hat{v}= \left<\frac{6}{5}, \frac{8}{5}\right>$, vector

\end{itemize}

\end{multicols}

\item  

\begin{multicols}{2}

\begin{itemize}

\item  $\vec{v} + \vec{w} = \left<1,0\right> $, vector
\item  $\vec{w}  - 2\vec{v}  = \left<-\frac{1}{2}, -\frac{3}{2} \right>$, vector

\end{itemize}

\end{multicols}

\begin{multicols}{2}

\begin{itemize}

\item $\| \vec{v} + \vec{w} \| = 1$, scalar
\item  $\| \vec{v} \| + \| \vec{w}\| = \sqrt{2}$, scalar

\end{itemize}

\end{multicols}

\begin{multicols}{2}

\begin{itemize}

\item $\| \vec{v} \| \vec{w} - \| \vec{w} \| \vec{v}  = \left<0,-\frac{\sqrt{2}}{2}\right>$, vector
\item $\|w\| \hat{v}= \left<\frac{1}{2}, \frac{1}{2}\right>$, vector

\end{itemize}

\end{multicols}

\setcounter{HW}{\value{enumi}}

\end{enumerate}

\begin{multicols}{3}

\begin{enumerate}

\setcounter{enumi}{\value{HW}}

\item $\vec{v} = \left<3,3\sqrt{3}\right>$
\item $\vec{v} = \left<\frac{3\sqrt{2}}{2},\frac{3\sqrt{2}}{2}\right>$
\item $\vec{v} = \left< \frac{\sqrt{3}}{3}, \frac{1}{3}\right>$

\setcounter{HW}{\value{enumi}}

\end{enumerate}

\end{multicols}

\begin{multicols}{3}

\begin{enumerate}

\setcounter{enumi}{\value{HW}}

\item $\vec{v} = \left<0,12\right>$
\item $\vec{v} = \left<-2\sqrt{3}, 2\right>$
\item $\vec{v} = \left<-\sqrt{3}, 3\right>$

\setcounter{HW}{\value{enumi}}

\end{enumerate}

\end{multicols}

\begin{multicols}{3}

\begin{enumerate}

\setcounter{enumi}{\value{HW}}

\item $\vec{v} = \left<-\frac{7}{2}, 0\right>$
\item $\vec{v} = \left<-5\sqrt{3}, -5\sqrt{3}\right>$
\item $\vec{v} = \left<0, -6.25\right>$

\setcounter{HW}{\value{enumi}}

\end{enumerate}

\end{multicols}

\begin{multicols}{3}

\begin{enumerate}

\setcounter{enumi}{\value{HW}}

\item $\vec{v} = \left<6, -2\sqrt{3}\right>$
\item $\vec{v} = \left<5, -5\right>$
\item $\vec{v} = \left<2,4\right>$

\setcounter{HW}{\value{enumi}}

\end{enumerate}

\end{multicols}

\begin{multicols}{3}

\begin{enumerate}

\setcounter{enumi}{\value{HW}}

\item $\vec{v} = \left<-1, 3\right>$
\item $\vec{v} = \left<-3, -4\right>$
\item $\vec{v} = \left<24, -10\right>$

\setcounter{HW}{\value{enumi}}

\end{enumerate}

\end{multicols}

\begin{multicols}{3}

\begin{enumerate}

\setcounter{enumi}{\value{HW}}

\item $\vec{v} \approx \left<-177.96, 349.27\right>$
\item $\vec{v} \approx \left<12.96, 62.59\right>$
\item $\vec{v} \approx \left<5164.62, 1097.77\right>$

\setcounter{HW}{\value{enumi}}

\end{enumerate}

\end{multicols}

\begin{multicols}{3}

\begin{enumerate}

\setcounter{enumi}{\value{HW}}

\item $\vec{v} \approx \left<-386.73, -230.08\right>$
\item $\vec{v} \approx \left<-52.13, -160.44\right>$
\item $\vec{v} \approx \left<14.73, -21.43\right>$

\setcounter{HW}{\value{enumi}}

\end{enumerate}

\end{multicols}

\begin{multicols}{3}

\begin{enumerate}

\setcounter{enumi}{\value{HW}}

\item  $\|\vec{v}\| = 2$, $\theta = 60^{\circ}$
\item $\|\vec{v}\| = 5\sqrt{2}$, $\theta = 45^{\circ}$
\item $\|\vec{v}\| = 4$, $\theta = 150^{\circ}$

\setcounter{HW}{\value{enumi}}

\end{enumerate}

\end{multicols}

\begin{multicols}{3}

\begin{enumerate}

\setcounter{enumi}{\value{HW}}

\item $\|\vec{v}\| = 2$, $\theta = 135^{\circ}$
\item $\|\vec{v}\| = 1$, $\theta = 225^{\circ}$
\item $\|\vec{v}\| = 1$, $\theta = 240^{\circ}$

\setcounter{HW}{\value{enumi}}

\end{enumerate}

\end{multicols}

\begin{multicols}{3}

\begin{enumerate}

\setcounter{enumi}{\value{HW}}

\item  $\|\vec{v}\| = 6$, $\theta = 0^{\circ}$
\item $\|\vec{v}\| = 2.5$, $\theta = 180^{\circ}$
\item  $\|\vec{v}\| = \sqrt{7}$, $\theta = 90^{\circ}$

\setcounter{HW}{\value{enumi}}

\end{enumerate}

\end{multicols}

\begin{multicols}{3}

\begin{enumerate}

\setcounter{enumi}{\value{HW}}

\item  $\|\vec{v}\| = 10$, $\theta = 270^{\circ}$
\item $\|\vec{v}\| = 5$, $\theta \approx 53.13^{\circ}$
\item $\|\vec{v}\| = 13$, $\theta \approx 22.62^{\circ}$

\setcounter{HW}{\value{enumi}}

\end{enumerate}

\end{multicols}

\begin{multicols}{3}

\begin{enumerate}

\setcounter{enumi}{\value{HW}}

\item $\|\vec{v}\| = 5$, $\theta \approx 143.13^{\circ}$
\item $\|\vec{v}\| = 25$, $\theta \approx 106.26^{\circ}$
\item $\|\vec{v}\| = \sqrt{5}$, $\theta \approx 206.57^{\circ}$

\setcounter{HW}{\value{enumi}}

\end{enumerate}

\end{multicols}

\begin{multicols}{3}

\begin{enumerate}

\setcounter{enumi}{\value{HW}}

\item  $\|\vec{v}\| = 2\sqrt{10}$, $\theta \approx 251.57^{\circ}$
\item  $\|\vec{v}\| = \sqrt{2}$, $\theta \approx 45^{\circ}$
\item $\|\vec{v}\| = \sqrt{17}$, $\theta \approx 284.04^{\circ}$

\setcounter{HW}{\value{enumi}}

\end{enumerate}

\end{multicols}

\begin{multicols}{3}

\begin{enumerate}

\setcounter{enumi}{\value{HW}}

\item \small $\|\vec{v}\| \approx 145.48$, $\theta \approx 328.02^{\circ}$ \normalsize
\item \small $\|\vec{v}\| \approx 1274.00$, $\theta \approx 40.75^{\circ}$ \normalsize
\item \small $\|\vec{v}\| \approx 121.69$, $\theta \approx 159.66^{\circ}$ \normalsize

\setcounter{HW}{\value{enumi}}

\end{enumerate}

\end{multicols}

\begin{enumerate}

\setcounter{enumi}{\value{HW}}

\item The boat's true speed is about 10 miles per hour at a heading of S$50.6^{\circ}$W.

\item  The HMS Sasquatch's true speed is about 41 miles per hour at a heading of S$26.8^{\circ}$E.

\item  She should maintain a speed of about 35 miles per hour at a heading of S$11.8^{\circ}$E.

\item She should fly at 83.46 miles per hour with a heading of N$22.1^{\circ}$E

\item  The current is moving at about 10 miles per hour bearing N$54.6^{\circ}$W.

\item  The tension on each of the cables is about $346$ pounds.

\item  The maximum weight that can be held by the cables in that configuration is about $133$ pounds.

\item The tension on the left hand cable is $285.317$ lbs. and on the right hand cable is $92.705$ lbs.

\item The weaker student should pull about 60 pounds.  The net force on the keg is about 153 pounds.

\item The resultant force is only about 296 pounds so the couch doesn't budge.  Even if it did move, the stronger force on the third rope would have made the couch drift slightly to the south as it traveled down the street.  

\end{enumerate}


\end{document}
