
\makeatletter
\@ifclassloaded{xourse}{%
    \typeout{Start loading xmPreamble.tex (in a XOURSE)}%
    \def\isXourse{true}   % automatically defined; pre 112022 it had to be set 'manually' in a xourse
}{%
    \typeout{Start loading xmPreamble.tex (NOT in a XOURSE)}%
}
\makeatother

\usepackage{currfile}

% 201908/202301: PAS OP: babel en doclicense lijken problemen te veroorzaken in .jax bestand
% (wegens syntax error met toegevoegde \newcommands ...)
\pdfOnly{
    %\usepackage[type={CC},modifier={by-nc-sa},version={4.0}]{doclicense}
    \usepackage[hyperxmp=false,type={CC},modifier={by-nc-sa},version={4.0}]{doclicense}
}


\providecommand{\blue}[1]{{\color{blue}#1}}    
\providecommand{\red}[1]{{\color{red}#1}}

\newcommand{\onlineChoice}[1]{\pdfOnly{\wordchoicegiventrue}\wordChoice{#1}\pdfOnly{\wordchoicegivenfalse}}


% Omdat multicols niet werkt in html: enkel in pdf  (in html zijn langere pagina's misschien ook minder storend)
\newenvironment{xmmulticols}[1][2]{
 \pdfOnly{\begin{multicols}{#1}}%
}{ \pdfOnly{\end{multicols}}}


% Aanpassen printversie
%  (hier gedefinieerd, zodat ze in xourse kunnen worden gezet/overschreven)
\providebool{parttoc}
\providebool{printpartfrontpage}
\providebool{printactivitytitle}
\providebool{printactivityqrcode}
\providebool{printactivityurl}
\providebool{printcontinuouspagenumbers}
\providebool{numberactivitiesbysubpart}
\providebool{addtitlenumber}
\providebool{addsectiontitlenumber}
\addtitlenumbertrue
\addsectiontitlenumbertrue

% The following three commands are hardcoded in xake, you can't create other commands like these, without adding them to xake as well
%  ( gebruikt in xourses om juiste soort titelpagina te krijgen voor verschillende ximera's )
\newcommand{\activitychapter}[2][]{
    {    
    \ifstrequal{#1}{notnumbered}{
        \addtitlenumberfalse
    }{}
    \typeout{ACTIVITYCHAPTER #2}   % logging
	\chapterstyle
	\activity{#2}
    }
}
\newcommand{\activitysection}[2][]{
    {
    \ifstrequal{#1}{notnumbered}{
        \addsectiontitlenumberfalse
    }{}
	\typeout{ACTIVITYSECTION #2}   % logging
	\sectionstyle
	\activity{#2}
    }
}
% Practices worden als activity getoond om de grote blokken te krijgen online
\newcommand{\practicesection}[2][]{
    {
    \ifstrequal{#1}{notnumbered}{
        \addsectiontitlenumberfalse
    }{}
    \typeout{PRACTICESECTION #2}   % logging
	\sectionstyle
	\activity{#2}
    }
}

%  
% references: Ximera heeft adhoc logica	 om online labels te doen werken over verschillende files heen
% met \hyperref kan de getoonde tekst toch worden opgegeven, in plaats van af te hangen van de label-text
\ifdefined\HCode
% Link to standard \labels, but give your own description
% Usage:  Volg \hyperref[my_very_verbose_label]{deze link} voor wat tijdverlies
%   (01/2020: Ximera-server aangepast om bij class reference-keeptext de link-text NIET te vervangen door de label-text !!!) 
\renewcommand{\hyperref}[2][]{\HCode{<a class="reference reference-keeptext" href="\##1">}#2\HCode{</a>}}
%
%  Link to specific targets  (not tested ?)
\renewcommand{\hypertarget}[1]{\HCode{<a class="ximera-label" id="#1"></a>}}
\renewcommand{\hyperlink}[2]{\HCode{<a class="reference reference-keeptext" href="\##1">}#2\HCode{</a>}}
\fi



%% Make solution 'expandable'   (HACK: copied 'oplossing' in KULEuven preamble...; needs some settings in global.css to work properly) 
\ifhandout%
    \RenewEnviron{solution}[1][onzichtbaar]%
    {%
    \ifthenelse{\equal{\detokenize{#1}}{\detokenize{toon}}}
    {
    \def\PH@Command{#1}% Use PH@Command to hold the content and be a target for "\expandafter" to expand once.

    \begin{trivlist}% Begin the trivlist to use formating of the "Feedback" label.
    \item[\hskip \labelsep\small\slshape\bfseries Solution% Format the "Feedback" label. Don't forget the space.
    %(\texttt{\detokenize\expandafter{\PH@Command}}):% Format (and detokenize) the condition for feedback to trigger
    \hspace{2ex}]\small%\slshape% Insert some space before the actual feedback given.
    \BODY
    \end{trivlist}
    }
    {  % \begin{feedback}[solution]   \BODY     \end{feedback}  }
    }
    }    
\else
% ONLY for HTML; xmoplossing is styled with css, and is not, and need not be a LaTeX environment
% THUS: it does NOT use feedback anymore ...
%    \NewEnviron{oplossing}{\begin{expandable}{xmoplossing}{\nlen{Toon uitwerking}{Show solution}}{\BODY}\end{expandable}}
    \renewenvironment{solution}[1][onzichtbaar]
   {%
       \begin{expandable}{xmoplossing}{}
   }
   {%
   	   \end{expandable}
   } 
%     \newenvironment{oplossing}[1][onzichtbaar]
%    {%
%        \begin{feedback}[solution]   	
%    }
%    {%
%    	   \end{feedback}
%    } 
\fi





%
% FROM preamble.tex (ie diff eqs !)

\usepackage{tikz}
%\usepackage{tkz-euclide}
\usepackage{tikz-3dplot}
\usepackage{tikz-cd}
\usetikzlibrary{shapes.geometric}
\usetikzlibrary{arrows}
\usetikzlibrary{decorations.pathmorphing,patterns}
\usetikzlibrary{backgrounds} % added by Felipe
% \usetkzobj{all}   % NOT ALLOWED IN RECENT TeX's ...
\pgfplotsset{compat=1.13} % prevents compile error.

\pdfOnly{\renewcommand{\theHsection}{\thepart.section.\thesection}}  %% MAKES LINKS WORK should be added to CLS
\pdfOnly{\renewcommand{\part}[1]{\chapterstyle\title{#1}\begin{abstract}\end{abstract}\maketitle\def\thechaptertitle{#1}}}


\renewcommand{\vec}[1]{\mathbf{#1}}
\newcommand{\RR}{\mathbb{R}}
\providecommand{\dfn}{\textit}
\renewcommand{\dfn}{\textit}
\newcommand{\dotp}{\cdot}
\newcommand{\id}{\text{id}}
\newcommand\norm[1]{\left\lVert#1\right\rVert}
\newcommand{\dst}{\displaystyle}
 
\newtheorem{general}{Generalization}
\newtheorem{initprob}{Exploration Problem}

\tikzstyle geometryDiagrams=[ultra thick,color=blue!50!black]

\usepackage{mathtools}


%
% ADDED 8/2025
%
% \providecommand\theabstract{} % otherwise complaint Undefined control sequence.  <recently read> \theabstract  ????
\newcommand{\xmtitle}[2][]{
	\title{#2}
	\begin{abstract}
        #1
	\end{abstract}
	\maketitle
}

\oddsidemargin 0in
\evensidemargin 0in
\marginparwidth 0in
\textheight 8in
\textwidth 6.5in
\topmargin 0in
\headheight 14pt
\usepackage{amssymb,amsmath,amsthm,fancyhdr,supertabular,longtable,hhline,mathtools}
\usepackage{colortbl}
\usepackage{import, multicol,boxedminipage}
\usepackage{chapterfolder}
\usepackage[metapost,truebbox]{mfpic}
% \usepackage[pdflatex]{graphicx}
\usepackage{makeidx}
% \usepackage[colorlinks, hyperindex, plainpages=false, linkcolor=blue, urlcolor=blue, pdfpagelabels]{hyperref}
\usepackage[all]{hypcap}
\usepackage{bm}
\definecolor{ResultColor}{gray}{0.9}
\theoremstyle{definition}  % this prevents the text in definitions, theorems, and corollaries from being italicized
%replaced \newtheorem{defn}{\bf Definition}[section]
\newtheorem*{defnrecall}{\bf Definition}
%replaced \newtheorem{thm}{\bf Theorem}[section]
%notused \newtheorem{cor}[thm]{\bf Corollary}
\newtheorem{eqn}{\bf Equation}[section]
%replaced \newtheorem{ex}{\bf Example}[section]
%notused \newtheorem{fig}{\bf Figure}[section]
\setlength{\parindent}{0in}
\newcommand{\bbm}{\relax}
\newcommand{\ebm}{\relax}
%notneeded \newcommand{\bbm}{\begin{boxedminipage}{6.41in}}
%notneeded \newcommand{\ebm}{\end{boxedminipage}}
\newcommand{\ds}{\displaystyle}
\usepackage{array}
\setlength{\extrarowheight}{2pt}
\allowdisplaybreaks[2]
\usepackage{cancel}
\usepackage{sectsty}
%\usepackage{appendix}
\usepackage{textcomp}
\usepackage{multirow}
\usepackage[nottoc]{tocbibind}
\DeclareSymbolFont{AMSb}{U}{msb}{m}{n}
\DeclareMathSymbol{\C}{\mathbin}{AMSb}{"43}
\DeclareMathSymbol{\N}{\mathbin}{AMSb}{"4E}
\DeclareMathSymbol{\I}{\mathbin}{AMSb}{"5A}
\DeclareMathSymbol{\Q}{\mathbin}{AMSb}{"51}
\DeclareMathSymbol{\R}{\mathbin}{AMSb}{"52}
\DeclareMathSymbol{\W}{\mathbin}{AMSb}{"57}

\allsectionsfont{\mdseries \scshape}
\makeatletter
\renewcommand\l@section{\@dottedtocline{1}{1.5em}{3em}}
\renewcommand\l@subsection{\@dottedtocline{2}{4.5em}{3.5em}}
\makeatother
\pagestyle{fancy}
\newcounter{HW}
\newcounter{HWindent}

\renewcommand{\textinterrobang}{$! \! \! ?$}

%Below is for Iowna Font
%\renewcommand*\sfdefault{iwona}
%\usepackage[math]{iwona}

%Below is for Helvetica (scaled): 
\usepackage[scaled=.92]{helvet}   
\renewcommand{\familydefault}{\sfdefault}  %makes the text of the book sans serif
\usepackage[helvet]{sfmath}  %makes the math in the book sans serif
\allsectionsfont{\sffamily}  %makes the chapter and section titles sans serif

\makeatletter
\newcases{mycases}{\quad}{%
  \hfil$\m@th\displaystyle{##}$}{$\m@th\displaystyle{##}$\hfil}{\lbrace}{.}
\makeatother

% \let\ref\textbf

% \usepackage{enumitem}
% \setlistdepth{9} % allow up to 9 nested levels

%% OTHERWITE 'Counter too large' errors   (Ximera sets this ot \Alph ...?)
\renewcommand{\theenumi}{\arabic{enumi}}
\renewcommand{\labelenumi}{\theenumi.}



\tikzset{
 flabel/.style={
    circle, fill=blue, inner sep=2pt,
   }
}

\pgfplotsset{
  fplot/.style={
    xlabel={$x$},  % x-axis label
    ylabel={$y$},  % y-axis label	
	axis lines=middle,
	axis equal image,
	axis line style={->},
	% ticks=none,
    samples=200,
    smooth,
    % line width=1.25pt,
	mark=none,
	domain=-2:2,
	clip=false,
	% postaction={decorate},
    % decoration={
    %   markings,
    %   mark=at position 1 with {\arrow{>}}
    % }
  },
   fpplot/.style={
    fplot,
    variable=t,
    parametric,
   },
   fgraph/.style={
    thick, blue
   },
   flabel/.style={
    circle, fill=blue, inner sep=2pt,
   }
}





%% Default \begin{tikzpicture} options
\tikzset{
 xmGraphTikz/.style={},
 panel/.style={},
 }

 %% Default \begin{axis} options
\pgfplotsset{
 xmGraphPlot/.style={},
 xmGraphPlotAxis/.style={
  xlabel={$x$},  % x-axis label
  ylabel={$y$},  % y-axis label	
	axis lines=middle,
	axis equal image,
	axis line style={->},
%   panel,
 },
  panel/.code={\def\xmGraphShowPanel{true}}, % sets a macro when applied
  % xmGraphShowDesmos/.code={\def\xmGraphShowDesmos{true}}, % sets a macro when applied
  % % xmGraphShowDesmos/.default={true}, 
  % xmGraphNoDesmos/.code={\let\xmGraphShowDesmos\undefined}, 
  % % xmGraphShowDesmos,
  xmGraphShowCommand/.code={\def\xmGraphShowCommand{true}}, % sets a macro when applied
  % xmGraphShowTikZ/.code={\def\xmGraphShowTikZ{true}}, % sets a macro when applied
  % xmGraphNoTikZ/.code={\let\xmGraphShowTikZ\undDefInEd}, % sets a macro when applied
  xmDesmosWidth/.store in=\xmDesmosWidth,
  xmDesmosWidth/.default=600px,
  xmDesmosHeight/.store in=\xmDesmosHeight,
  xmDesmosHeight/.default=300px,
}

\def\xmGraphShowDesmos{true}  % default NOT inside the style: DOES NOT WORK

\ExplSyntaxOn
% Helper function to process each item
%%% The 'multi'-part, with f(x),g(x,f(x), does NOT yet work )
\cs_new_protected:Nn \__multiplot_get_function:n {
  % Check if the item contains slashes (legend and/or options separator)
  % IN 2025 ? \seq_set_split_with_nested:nnnN { / } { | | } { #1 } \l_tmpa_seq
  \seq_set_split:Nnn \l_tmpa_seq { // } { #1 }
  \int_case:nn { \seq_count:N \l_tmpa_seq } {
    { 1 } {
      % No legend, no options: just plot function
         #1
    }
    { 2 } {
      \seq_pop_left:NN \l_tmpa_seq \l_tmpa_tl  % function
      \l_tmpa_tl
    }
    { 3 } {
      \seq_pop_left:NN \l_tmpa_seq \l_tmpa_tl  % function
      \l_tmpa_tl
    }
  }
}

\cs_new_protected:Nn \__multiplot_process:nn {
  % Check if the item contains slashes (legend and/or options separator)
  % IN 2025 ? \seq_set_split_with_nested:nnnN { / } { | | } { #1 } \l_tmpa_seq
  \seq_set_split:Nnn \l_tmpa_seq { // } { #2 }
  \int_case:nn { \seq_count:N \l_tmpa_seq } {
    { 1 } {
      % No legend, no options: just plot function
      \addplot[xmGraphPlot, #1] { #2 };
    }
    { 2 } {
      % Has legend: plot function and add legend entry
      \seq_pop_left:NN \l_tmpa_seq \l_tmpa_tl  % function
      \seq_pop_left:NN \l_tmpa_seq \l_tmpb_tl  % legend
      \addplot[xmGraphPlot, #1] { \l_tmpa_tl };
      % Only add legend if not empty
      %\tl_if_empty:NF \l_tmpb_tl {
        \addlegendentry{\l_tmpb_tl}
      %}
    }
    { 3 } {
      % Has legend and options: plot function with custom options
      \seq_pop_left:NN \l_tmpa_seq \l_tmpa_tl  % function
      \seq_pop_left:NN \l_tmpa_seq \l_tmpb_tl  % legend
      \seq_pop_left:NN \l_tmpa_seq \l_tmpc_tl  % options
      \addplot[xmGraphPlot, #1,  \l_tmpc_tl] { \l_tmpa_tl };
      % Only add legend if not empty
      \tl_if_empty:NF \l_tmpb_tl {
        \addlegendentry{\l_tmpb_tl}
        }
    }
  }
}

\cs_new_protected:Nn \xmultiplot:nn {
  \clist_map_inline:nn { #2 } {
    \__multiplot_process:nn { #1 } { ##1 }
  }
}

\NewDocumentCommand \xmGraphPlot { o m } { \xmultiplot:nn {#1} {#2} }
\NewDocumentCommand \xmGraphGetFunction { m } { \__multiplot_get_function:n {#1} }
\ProvideDocumentCommand \xmGraphLegend { m } {}
\ProvideDocumentCommand \xmGraphExtra { m } {}

\ExplSyntaxOff

% \newcommand\GetKeyFromStyle[2]{%
%   \pgfkeys{/pgfplots/.cd,#1}%
%   \pgfkeysvalueof{/pgfplots/#2}%
% }

\newcommand\GetPGFplotsKeyFromStyle[2]{%
  % #1 = style name
  % #2 = key name
  \pgfkeys{/pgfplots/.cd,#1}% activate style
  %\typeout{KEY #2 in #1}
  \pgfkeysifdefined{/pgfplots/#2}{%
    \pgfkeysvalueof{/pgfplots/#2}%
   % \typeout{KEY found: #2 in #1: \pgfkeysvalueof{/pgfplots/#2}}
  }{%
    \pgfkeysifdefined{/pgfplots/axis/#2}{%
      \pgfkeysvalueof{/pgfplots/axis/#2}%
      %\typeout{KEY axis found: #2 in #1: \pgfkeysvalueof{/pgfplots/#2}}
    }{%
    %\pgfkeysifdefined{/pgfplots/.@cmd/#2}{TRUE}{FALSE}%
      % undefined → return empty
    }%
  }%
}


\newcommand{\xmgraph}[2][]{
  \begingroup
  \ifdefined\xmGraphShowDesmos    % generic tikz-options confuse the pgfkeys ...
    \typeout{SETTING tmpstyle (\xmGraphShowDesmos\  for #1)}
    \pgfplotsset{   %% The #1 will process the additional options, so that \xmGraphShowTikZ works !!!
      tmpstyle/.style={
        /pgfplots/xmGraphPlot,        % apply plot keys
        /pgfplots/xmGraphPlotAxis,    % apply axis keys
        #1                            % allow overrides
      }
    }
    \pgfkeys{/pgfplots/.cd,#1}        % activate style
  \else
    \typeout{SKIPPING tmpstyle for #1}
  \fi
%
  \ifdefined\HCode
    \HCode{\Hnewline<div class="xmdesmos">}
  \else
    \def\xmGraphShowTikZ{true}  %% ALWAYS show TikZ in PDF !!!
  \fi
%
  % Debugging/convenience : Automatically add the command to the output
  \ifdefined\xmGraphShowCommand
  \ifdefined\HCode
    \HCode{<h1> Command: \string\graph[#1] for  }\detokenize{#2} \HCode{</h1> }
  \else
     {\Large\bf With options #1 }
  \fi
  \fi
%
  % Do the DESMOS part
  \ifdefined\HCode\ifdefined\xmGraphShowDesmos
    \HCode{<div class='desmos-placeholder' data-options='}
        %%% still an issue with $x$ vs x   (Desmos prints the $'s !)
        % \def\myval{\GetPGFplotsKeyFromStyle}tmpstyle}{xlabel}
        % xlabel=\HCode{\myval},
        xmax=\GetPGFplotsKeyFromStyle{tmpstyle}{xmax},
        xmin=\GetPGFplotsKeyFromStyle{tmpstyle}{xmin},  
        ymin=\GetPGFplotsKeyFromStyle{tmpstyle}{ymin},
        ymax=\GetPGFplotsKeyFromStyle{tmpstyle}{ymax},
        \ifdefined\xmGraphShowPanel    panel=true, \fi
        \ifdefined\xmDesmosWidth       xmDesmosWidth=\xmDesmosWidth, \fi
        \ifdefined\xmDesmosHeight      xmDesmosHeight=\xmDesmosHeight, \fi
        \if\relax\detokenize\expandafter{\GetPGFplotsKeyFromStyle{tmpstyle}{axis equal image}}\relax
          axis equal image=true,
        \fi
    \HCode{' data-graph='}\detokenize{#2}\HCode{'"></div>}%
  \fi\fi%
  % Do the TIKZ part
  \ifdefined\xmGraphShowTikZ%
    \begin{tikzpicture}[xmGraphTikz]
    \begin{axis}[xmGraphPlotAxis,#1] 
      % \xmGraphPlot[]{#2}    
      \addplot[xmGraphPlot] { #2 };
      % \xmGraphPlot[#1]{#2} 
      \xmGraphLegend{}
      \xmGraphExtra{}
    \end{axis}
    \end{tikzpicture}
  \fi
  \ifdefined\HCode
    \HCode{\Hnewline</div> <!-- class desmos --> }  
  \fi
\endgroup
}

% \ExplSyntaxOn
% \bool_set_true:N \l__trace_errors_bool
% \ExplSyntaxOff


\ifdefined\xmGraphNoRedefine    % This would give the ximera.4ht version
\else
  \AtBeginDocument{\let\graph\xmgraph}
\fi