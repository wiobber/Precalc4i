\documentclass{ximera}

\begin{document}
	\author{Stitz-Zeager}
	\xmtitle{Exercises for Graphsof Polynomials}{}

\mfpicnumber{1} \opengraphsfile{ExercisesforGraphsofPolynomials} % mfpic settings added 


\begin{problem}\label{polytransfirst}
     Starting with the graph of $f(x) = x^3$, use Theorem~\ref{linearmononialgraphs} to sketch the graph of $F(x) = (x + 2)^{3} + 1$. Track at least three points of your choice through the transformations. State the domain and range of $F$.

\end{problem}

\begin{problem}
     Starting with the graph of $f(x) = x^4$, use Theorem~\ref{linearmononialgraphs} to sketch the graph of $F(x) = (x + 2)^{4} + 1$. Track at least three points of your choice through the transformations. State the domain and range of $F$.
\end{problem}

\begin{problem}
     Starting with the graph of $f(x) = x^4$, use Theorem~\ref{linearmononialgraphs} to sketch the graph of $F(x) = 2 - 3(x - 1)^{4}$. Track at least three points of your choice through the transformations. State the domain and range of $F$.
\end{problem}

\begin{problem}
     Starting with the graph of $f(x) = x^5$, use Theorem~\ref{linearmononialgraphs} to sketch the graph of $F(x) = -x^5 - 3$. Track at least three points of your choice through the transformations. State the domain and range of $F$.
\end{problem}

\begin{problem}
     Starting with the graph of $f(x) = x^5$, use Theorem~\ref{linearmononialgraphs} to sketch the graph of $F(x) = (x + 1)^{5} + 10$. Track at least three points of your choice through the transformations. State the domain and range of $F$.
\end{problem}

\begin{problem}\label{polytranslast}
     Starting with the graph of $f(x) = x^6$, use Theorem~\ref{linearmononialgraphs} to sketch the graph of $F(x) = 8 - x^6$. Track at least three points of your choice through the transformations. State the domain and range of $F$.
\end{problem}





\begin{problem}\label{findformulaforcubicgraphfirst}
    Find a formula for the function below in the form $F(x) = a(x-h)^3+k$.


% \begin{mfpic}[15]{-5}{5}{-5}{5}
% \axes
% \tlabel[cc](5,-0.5){\scriptsize $x$}
% \tlabel[cc](0.5,5){\scriptsize $y$}
% \tlabel[cc](-1.25, -3){\scriptsize $(0,-3)$}
% \tlabel[cc](1.25,-2.75){\scriptsize $(1,-2)$}
% \xmarks{-4,-3,-2,-1,1,2,3,4}
% \ymarks{-4,-3,-2, -1, 1,2,3,4}
% \tlpointsep{4pt}
% \scriptsize
% \axislabels {x}{ {$-4 \hspace{7pt}$} -4, {$-3 \hspace{7pt}$} -3, {$-2 \hspace{7pt}$} -2, {$-1 \hspace{7pt}$} -1, {$1$} 1, {$2$} 2, {$3$} 3, {$4$} 4}
% \axislabels {y}{{$-1$} -1,{$1$} 1, {$2$} 2, {$3$} 3, {$4$} 4, {$-2$} -2}
% \penwd{1.25pt}
% \arrow \reverse \arrow \function{-0.4,2.9,0.1}{(x-1)**3-2}
% \point[4pt]{(1,-2), (0,-3)}
% \tcaption{ \scriptsize$y = F(x)$}
% \normalsize
% \end{mfpic} 

\begin{tikzpicture}
\begin{axis}[
    axis lines=middle,
    xmin=-5, xmax=5,
    ymin=-5, ymax=5,
    xtick={-4,-3,-2,-1,1,2,3,4},
    ytick={-4,-3,-2,-1,1,2,3,4},
    xlabel={$x$},
    ylabel={$y$},
    enlargelimits=true,
    axis line style={thick,->},
    ticklabel style={font=\scriptsize},
    label style={font=\scriptsize},
    every axis x label/.style={at={(ticklabel* cs:1)},anchor=west},
    every axis y label/.style={at={(ticklabel* cs:1)},anchor=south},
    width=10cm, height=10cm
]

% Function: y = (x-1)^3 - 2
\addplot[
    domain=-0.4:2.9,
    samples=200,
    thick,
] {(x-1)^3 - 2};

% Points (1,-2) and (0,-3)
\addplot[only marks, mark=*] coordinates {(1,-2) (0,-3)};

% Labels near points
\node[anchor=south,font=\scriptsize] at (axis cs:1,-2) {$(1,-2)$};
\node[anchor=west,font=\scriptsize] at (axis cs:0,-3) {$(0,-3)$};

\end{axis}
\end{tikzpicture}

$F(x) = \answer{(x-1)^3-2}$

\end{problem}

\begin{problem}\label{findformulaforcubicgraphlast}
    Find a formula for the function below in the form $F(x) = a(x-h)^3+k$.

% \begin{mfpic}[15]{-5}{5}{-5}{5}
% \axes
% \tlabel[cc](5,-0.5){\scriptsize $x$}
% \tlabel[cc](0.5,5){\scriptsize $y$}
% \tlabel[cc](1,-1){\scriptsize $(0,-1)$}
% \tlabel[cc](-2.25,2.25){\scriptsize $(-2,3)$}
% \xmarks{-4,-3,-2,-1,1,2,3,4}
% \ymarks{-4,-3,-2, -1, 1,2,3,4}
% \tlpointsep{4pt}
% \scriptsize
% \axislabels {x}{ {$-4 \hspace{7pt}$} -4, {$-3 \hspace{7pt}$} -3, {$-2 \hspace{7pt}$} -2, {$-1 \hspace{7pt}$} -1, {$1$} 1, {$2$} 2, {$3$} 3, {$4$} 4}
% \axislabels {y}{{$-1$} -1, {$2$} 2, {$3$} 3, {$4$} 4, {$-2$} -2, {$-3$} -3, {$-4$} -4}
% \penwd{1.25pt}
% \arrow \reverse \arrow \function{-3.5,0.5,0.1}{3-(0.5*((x+2)**3))}
% \point[4pt]{(-2,3), (0,-1)}
% \tcaption{ \scriptsize$y = F(x)$}
% \normalsize
% \end{mfpic} 

\begin{tikzpicture}
\begin{axis}[
    axis lines=middle,
    xmin=-5, xmax=5,
    ymin=-5, ymax=5,
    xtick={-4,-3,-2,-1,1,2,3,4},
    ytick={-4,-3,-2,-1,1,2,3,4},
    xlabel={$x$},
    ylabel={$y$},
    enlargelimits=true,
    axis line style={thick,->},
    ticklabel style={font=\scriptsize},
    label style={font=\scriptsize},
    every axis x label/.style={at={(ticklabel* cs:1)},anchor=west},
    every axis y label/.style={at={(ticklabel* cs:1)},anchor=south},
    width=10cm, height=10cm
]

% Function: y = 3 - 0.5*(x+2)^3
\addplot[
    domain=-3.5:0.5,
    samples=200,
    thick,
] {3 - 0.5*(x+2)^3};

% Points (-2,3) and (0,-1)
\addplot[only marks, mark=*] coordinates {(-2,3) (0,-1)};

% Labels near points
\node[anchor=south,font=\scriptsize] at (axis cs:-2,3) {$(-2,3)$};
\node[anchor=west,font=\scriptsize] at (axis cs:0,-1) {$(0,-1)$};

\end{axis}
\end{tikzpicture}


\end{problem}



%%\begin{multicols}{2}

\begin{problem}\label{findformulaforquartgraphfirst}

Find a formula for the function below in the form $F(x) = a(x-h)^4+k$.

% \begin{mfpic}[15]{-5}{5}{-5}{5}
% \axes
% \tlabel[cc](5,-0.5){\scriptsize $x$}
% \tlabel[cc](0.5,5){\scriptsize $y$}
% \tlabel[cc](-1.25,-4.5){\scriptsize $(-1,-4)$}
% \tlabel[cc](1,-2){\scriptsize $(0,-2)$}
% \xmarks{-4,-3,-2,-1,1,2,3,4}
% \ymarks{-4,-3,-2, -1, 1,2,3,4}
% \tlpointsep{4pt}
% \scriptsize
% \axislabels {x}{ {$-4 \hspace{7pt}$} -4, {$-3 \hspace{7pt}$} -3, {$-1 \hspace{7pt}$} -1, {$1$} 1, {$2$} 2,{$3$} 3, {$4$} 4}
% \axislabels {y}{{$-1$} -1,{$1$} 1, {$2$} 2, {$3$} 3,  {$-2$} -2, {$-3$} -3, {$4$} 4}
% \penwd{1.25pt}
% \arrow \reverse \arrow \function{-2.41,0.41,0.1}{2*((x+1)**4)-4}
% \point[4pt]{(-1,-4), (0,-2)}
% \tcaption{ \scriptsize$y = F(x)$}
% \normalsize
% \end{mfpic} 

\begin{tikzpicture}
\begin{axis}[
    axis lines=middle,
    xmin=-5, xmax=5,
    ymin=-5, ymax=5,
    xtick={-4,-3,-2,-1,1,2,3,4},
    ytick={-4,-3,-2,-1,1,2,3,4},
    xlabel={$x$},
    ylabel={$y$},
    enlargelimits=true,
    axis line style={thick,->},
    ticklabel style={font=\scriptsize},
    label style={font=\scriptsize},
    every axis x label/.style={at={(ticklabel* cs:1)},anchor=west},
    every axis y label/.style={at={(ticklabel* cs:1)},anchor=south},
    width=10cm, height=10cm
]

% Function: y = 2*(x+1)^4 - 4
\addplot[
    domain=-2.41:0.41,
    samples=200,
    thick,
] {2*(x+1)^4 - 4};

% Points (-1,-4) and (0,-2)
\addplot[only marks, mark=*] coordinates {(-1,-4) (0,-2)};

% Labels near points
\node[anchor=north east,font=\scriptsize] at (axis cs:-1,-4) {$(-1,-4)$};
\node[anchor=west,font=\scriptsize] at (axis cs:0,-2) {$(0,-2)$};

\end{axis}
\end{tikzpicture}
\begin{solution}
 $F(x) = 2(x+1)^4-4$   
\end{solution}

\end{problem}

\begin{problem}\label{findformulaquartgraphlast}

Find a formula for the function below in the form $F(x) = a(x-h)^4+k$.


% \begin{mfpic}[15]{-5}{5}{-5}{5}
% \axes
% \tlabel[cc](5,-0.5){\scriptsize $x$}
% \tlabel[cc](0.5,5){\scriptsize $y$}
% \tlabel[cc](-3,0.5){\scriptsize $(-2,0)$}
% \tlabel[cc](2.5,0.5){\scriptsize $(2,0)$}
% \tlabel[cc](-2,2.5){\scriptsize $(0,2.5)$}
% \xmarks{-4,-3,-2,-1,1,2,3,4}
% \ymarks{-4,-3,-2, -1, 1,2,3,4}
% \tlpointsep{4pt}
% \scriptsize
% \axislabels {x}{ {$-4 \hspace{7pt}$} -4, {$-3 \hspace{7pt}$} -3, {$-1 \hspace{7pt}$} -1,  {$1$} 1, {$3$} 3, {$4$} 4}
% \axislabels {y}{{$-1$} -1,{$-2$} -2,{$-3$} -3,{$-4$} -4,{$1$} 1, {$2$} 2, {$3$} 3,  {$4$} 4}
% \penwd{1.25pt}
% \arrow \reverse \arrow \function{-2.5,2.5,0.1}{2.5-(0.1565*(x**4))}
% \point[4pt]{(-2,0), (2,0), (0, 2.5)}
% \tcaption{ \scriptsize$y = F(x)$}
% \normalsize
% \end{mfpic} 


\begin{tikzpicture}
\begin{axis}[
    axis lines=middle,
    xmin=-5, xmax=5,
    ymin=-5, ymax=5,
    xtick={-4,-3,-2,-1,1,2,3,4},
    ytick={-4,-3,-2,-1,1,2,3,4},
    xlabel={$x$},
    ylabel={$y$},
    enlargelimits=true,
    axis line style={thick,->},
    ticklabel style={font=\scriptsize},
    label style={font=\scriptsize},
    every axis x label/.style={at={(ticklabel* cs:1)},anchor=west},
    every axis y label/.style={at={(ticklabel* cs:1)},anchor=south},
    width=10cm, height=10cm
]

% Function: y = 2.5 - 0.1565*x^4
\addplot[
    domain=-2.5:2.5,
    samples=200,
    thick,
] {2.5 - 0.1565*x^4};

% Points (-2,0), (2,0), (0,2.5)
\addplot[only marks, mark=*] coordinates {(-2,0) (2,0) (0,2.5)};

% Labels near points
\node[anchor=south east,font=\scriptsize] at (axis cs:-2,0) {$(-2,0)$};
\node[anchor=south west,font=\scriptsize] at (axis cs:2,0) {$(2,0)$};
\node[anchor=south west,font=\scriptsize] at (axis cs:0,2.5) {$(0,2.5)$};

\end{axis}
\end{tikzpicture}
\end{problem}



%%\end{multicols}

\begin{problem}\label{polyfactsfirst}
Find the degree, the leading term, the leading coefficient, the constant term and the end behavior of the polynomial function $f(x) = 4-x-3x^2$.  

$\text{Degree} = \answer{2}$

$\text{Leading term} = \answer{-3x^{2}}$

$\text{Leading coefficient} = \answer{-3}$

$\text{Constant term} = \answer{4}$

$\lim_{x \rightarrow - \infty} f(x) = $ \wordChoice{\choice[correct]{$-\infty$}, \choice{$\infty$}}

$\lim_{x \rightarrow  \infty} f(x)  = $ \wordChoice{\choice[correct]{$-\infty$}, \choice{$\infty$}}


\end{problem}

\begin{problem}
Find the degree, the leading term, the leading coefficient, the constant term and the end behavior of the polynomial function $g(x) = 3x^5 - 2x^2 + x + 1$.  
\end{problem}

\begin{problem}
Find the degree, the leading term, the leading coefficient, the constant term and the end behavior of the polynomial function $q(r) = 1 - 16r^{4}$.  


$\text{Degree} = \answer{2}$

$\text{Leading term} = \answer{-16r^{4}}$

$\text{Leading coefficient} = \answer{-16}$

$\text{Constant term} = \answer{1}$

$\lim_{r \rightarrow - \infty} q(r) = $ \wordChoice{\choice[correct]{$-\infty$}, \choice{$\infty$}}

$\lim_{r \rightarrow  \infty} q(r)  = $ \wordChoice{\choice[correct]{$-\infty$}, \choice{$\infty$}}
\end{problem}

\begin{problem}
Find the degree, the leading term, the leading coefficient, the constant term and the end behavior of the polynomial function $Z(b) = 42b - b^{3}$.  
\end{problem}

\begin{problem}
Find the degree, the leading term, the leading coefficient, the constant term and the end behavior of the polynomial function $f(x) = \sqrt{3}x^{17} + 22.5x^{10} - \pi x^{7} + \frac{1}{3}$.

$\text{Degree} = \answer{17}$

$\text{Leading term} = \answer{\sqrt{3}x^{17}}$

$\text{Leading coefficient} = \answer{\sqrt{3}}$

$\text{Constant term} = \answer{1/3}$

$\lim_{x \rightarrow - \infty} f(x) = $ \wordChoice{\choice[correct]{$-\infty$}, \choice{$\infty$}}

$\lim_{x \rightarrow  \infty} f(x)  = $ \wordChoice{\choice{$-\infty$}, \choice[correct]{$\infty$}}

\end{problem}

\begin{problem}
Find the degree, the leading term, the leading coefficient, the constant term and the end behavior of the polynomial function $s(t) = -4.9t^{2} + v_{\mbox{\tiny $0$}}t + s_{\mbox{\tiny $0$}}$.  
\end{problem}

\begin{problem}
Find the degree, the leading term, the leading coefficient, the constant term and the end behavior of the polynomial function $P(x) = (x - 1)(x - 2)(x - 3)(x - 4)$.  

$\text{Degree} = \answer{4}$

$\text{Leading term} = \answer{x^{4}}$

$\text{Leading coefficient} = \answer{1}$

$\text{Constant term} = \answer{12}$

$\lim_{x \rightarrow - \infty} f(x) = $ \wordChoice{\choice{$-\infty$}, \choice[correct]{$\infty$}}

$\lim_{x \rightarrow  \infty} f(x)  = $ \wordChoice{\choice{$-\infty$}, \choice[correct]{$\infty$}}
\end{problem}

\begin{problem}
Find the degree, the leading term, the leading coefficient, the constant term and the end behavior of the polynomial function $p(t) = -t^2(3 - 5t)(t^{2} + t + 4)$.  
\end{problem}

\begin{problem}
Find the degree, the leading term, the leading coefficient, the constant term and the end behavior of the polynomial function $f(x) = -2x^3(x+1)(x+2)^2$.  

$\text{Degree} = \answer{6}$

$\text{Leading term} = \answer{-2x^{6}}$

$\text{Leading coefficient} = \answer{-2}$

$\text{Constant term} = \answer{0}$

$\lim_{x \rightarrow - \infty} f(x) = $ \wordChoice{\choice[correct]{$-\infty$}, \choice{$\infty$}}

$\lim_{x \rightarrow  \infty} f(x)  = $ \wordChoice{\choice[correct]{$-\infty$}, \choice{$\infty$}}
\end{problem}

\begin{problem}\label{polyfactslast}
Find the degree, the leading term, the leading coefficient, the constant term and the end behavior of the polynomial function $G(t) = 4(t-2)^2\left(t+\frac{1}{2}\right)$ . 
\end{problem}




\begin{problem}\label{zeromultgraphfirst}
Find the real zeros of the polynomial $a(x) = x(x + 2)^{2}$ and their corresponding multiplicities.  Use this information along with end behavior to provide a rough sketch of the graph of the polynomial function.  Compare your answer with the result from a graphing utility.
\end{problem}

\begin{problem}
Find the real zeros of the polynomial $g(t) = t(t + 2)^{3}$ and their corresponding multiplicities.  Use this information along with end behavior to provide a rough sketch of the graph of the polynomial function.  Compare your answer with the result from a graphing utility.
\end{problem}

\begin{problem}
Find the real zeros of the polynomial $f(z) = -2(z-2)^2(z+1)$ and their corresponding multiplicities.  Use this information along with end behavior to provide a rough sketch of the graph of the polynomial function.  Compare your answer with the result from a graphing utility.
\end{problem}

\begin{problem}
Find the real zeros of the polynomial $g(x) = (2x+1)^2(x-3)$ and their corresponding multiplicities.  Use this information along with end behavior to provide a rough sketch of the graph of the polynomial function.  Compare your answer with the result from a graphing utility.
\end{problem}


\begin{problem}
Find the real zeros of the polynomial $F(t) = t^{3}(t+ 2)^{2}$ and their corresponding multiplicities.  Use this information along with end behavior to provide a rough sketch of the graph of the polynomial function.  Compare your answer with the result from a graphing utility.
\end{problem}

\begin{problem}
Find the real zeros of the polynomial $P(z) = (z- 1)(z - 2)(z - 3)(z - 4)$ and their corresponding multiplicities.  Use this information along with end behavior to provide a rough sketch of the graph of the polynomial function.  Compare your answer with the result from a graphing utility.
\end{problem}


\begin{problem}
Find the real zeros of the polynomial $Q(x) = (x + 5)^{2}(x - 3)^{4}$ and their corresponding multiplicities.  Use this information along with end behavior to provide a rough sketch of the graph of the polynomial function.  Compare your answer with the result from a graphing utility.
\end{problem}

\begin{problem}
Find the real zeros of the polynomial $h(t) = t^2(t-2)^2(t+2)^2$ and their corresponding multiplicities.  Use this information along with end behavior to provide a rough sketch of the graph of the polynomial function.  Compare your answer with the result from a graphing utility.
\end{problem}

\begin{problem}
Find the real zeros of the polynomial $H(z) = (3-z)(z^2+1)$ and their corresponding multiplicities.  Use this information along with end behavior to provide a rough sketch of the graph of the polynomial function.  Compare your answer with the result from a graphing utility.
\end{problem}

\begin{problem}\label{zeromultgraphlast}
Find the real zeros of the polynomial $Z(x) = x(42 - x^{2})$ and their corresponding multiplicities.  Use this information along with end behavior to provide a rough sketch of the graph of the polynomial function.  Compare your answer with the result from a graphing utility.
\end{problem}

\begin{problem}\label{evenoddornotpolyfirst}
Determine analytically if the function $f(x) = 7x$ is even, odd, or neither.  Confirm your answer using a graphing utility.  

\wordChoice{\choice{even},\choice[correct]{odd}, \choice{neither}}
\end{problem}

\begin{problem}
Determine analytically if the function $g(t) = 7t + 2$ is even, odd, or neither.  Confirm your answer using a graphing utility.  
\end{problem}
 
\begin{problem}
Determine analytically if the function $p(z) = 7$ is even, odd, or neither.  Confirm your answer using a graphing utility.  

\wordChoice{\choice[correct]{even},\choice{odd}, \choice{neither}}
\end{problem}

\begin{problem}
Determine analytically if the function $F(s) = 3s^2 - 4$ is even, odd, or neither.  Confirm your answer using a graphing utility.  
\end{problem}

\begin{problem}
Determine analytically if the function $h(t) = 4-t^2$ is even, odd, or neither.  Confirm your answer using a graphing utility. 

\wordChoice{\choice[correct]{even},\choice{odd}, \choice{neither}}
\end{problem} 

\begin{problem}
Determine analytically if the function $g(x) = x^2-x-6$ is even, odd, or neither.  Confirm your answer using a graphing utility.  
\end{problem} 

\begin{problem}
Determine analytically if the function $f(x) = 2x^3 - x$ is even, odd, or neither.  Confirm your answer using a graphing utility.  

\wordChoice{\choice{even},\choice[correct]{odd}, \choice{neither}}
\end{problem} 

\begin{problem}
Determine analytically if the function $p(z) = -z^5 + 2z^3 - z$ is even, odd, or neither.  Confirm your answer using a graphing utility.  
\end{problem} 

\begin{problem}
Determine analytically if the function $G(t) = t^{6} - t^{4} + t^{2} + 9$ is even, odd, or neither.  Confirm your answer using a graphing utility.  

\wordChoice{\choice[correct]{even},\choice{odd}, \choice{neither}}
\end{problem} 
 
\begin{problem}
Determine analytically if the function $G(s) = s(s^2 - 1)$ is even, odd, or neither.  Confirm your answer using a graphing utility.  
\end{problem} 

\begin{problem}
Determine analytically if the function $f(x) = (x^2+1)(x-1)$ is even, odd, or neither.  Confirm your answer using a graphing utility.  

\wordChoice{\choice{even},\choice{odd}, \choice[correct]{neither}}
\end{problem}

\begin{problem}
Determine analytically if the function $H(t) = (t^2-1)(t^4+t^2+3)$ is even, odd, or neither.  Confirm your answer using a graphing utility.  
\end{problem}

\begin{problem}
Determine analytically if the function $g(t) = t(t-2)(t+2)$ is even, odd, or neither.  Confirm your answer using a graphing utility. 

\wordChoice{\choice{even},\choice[correct]{odd}, \choice{neither}}
\end{problem}

\begin{problem}
Determine analytically if the function $P(z) = (2z^{5} - 3z)(5z^3+z)$ is even, odd, or neither.  Confirm your answer using a graphing utility.  
\end{problem}
 
\begin{problem}\label{evenoddornotpolylast}
Determine analytically if the function $f(x) =0$ is even, odd, or neither.  Confirm your answer using a graphing utility.  

\wordChoice{\choice[correct]{even},\choice{odd}, \choice{neither}}
\end{problem}

\begin{problem}\label{evenoddpolynomialexercise}
Suppose $p(x)$ is a polynomial function written in the form of  Definition \ref{polynomialfunction}.

\begin{enumerate}

\item  If the nonzero terms of $p(x)$ consist of even powers of $x$ (or a constant), explain why $p$ is even.

\item   If the nonzero terms of $p(x)$ consist of odd powers of $x$, explain why $p$ is odd.

\item  If $p(x)$ the nonzero terms of $p(x)$  contain at least one odd power of $x$ and one even power of $x$ (or a constant term), then $p$ is neither even nor odd.

\end{enumerate}
\end{problem}


\begin{problem}
Use the results of Exercise \ref{evenoddpolynomialexercise} to determine whether the following functions are even, odd, or neither.

\begin{enumerate}
    \item $p(x) = 3x^4 + x^2 - 1$

    \item $F(s) = s^3 - 14s$

    \item $f(t) = 2t^5 - t^2 + 1$

    \item $g(x) =x^3(x^2+1)$
\end{enumerate}
\end{problem}



\begin{problem}
Show $f(x) = |x|$ is an even function.
\end{problem}

\begin{problem}
Rework Example \ref{boxnotopex} assuming the box is to be made from an 8.5 inch by 11 inch sheet of paper. Using scissors and tape, construct the box.  Are you surprised?\footnote{Consider decorating the box and presenting it to your instructor. If done well enough, maybe your instructor will issue you some bonus points.  Or maybe not.}
\end{problem}

\begin{problem}
For each function $f(x)$ listed below, compute the average rate of change over the indicated interval.\footnote{See Definition \ref{arc} in Section \ref{AverageRateofChange} for a review of this concept, as needed.}  What trends do you observe?  How do your answers manifest themselves graphically?

\[ \begin{array}{|r||c|c|c|c|c|c|}  \hline

 f(x) &  [-0.1, 0] & [0, 0.1] &[0.9, 1] & [1, 1.1] & [1.9, 2] & [2, 2.1]  \\ \hline
 1 &&&&&& \\  \hline
 x  &&&&&& \\  \hline
 x^2 &&&&&&  \\  \hline
 x^3 &&&&&& \\  \hline
 x^4 &&&&&& \\ \hline
 x^5 &&&&&& \\ \hline

\end{array} \]

\end{problem}


\begin{problem}\label{monomialarcexercise}
    For each function $f(x)$ listed below, compute the average rate of change over the indicated interval.\footnote{See Definition \ref{arc} in Section \ref{AverageRateofChange} for a review of this concept, as needed.}  What trends do you observe?  How do your answers manifest themselves graphically?

\[ \begin{array}{|r||c|c|c|c|}  \hline

 f(x) &  [0.9, 1.1] & [0.99, 1.01] &[0.999, 1.001] & [0.9999, 1.0001]  \\ \hline
 1 &&&&   \\  \hline
 x &&&&    \\  \hline
 x^2 &&&&   \\  \hline
 x^3 &&&&   \\  \hline
 x^4 &&&&  \\ \hline
 x^5 &&&&  \\ \hline

\end{array} \]

\end{problem}
 
\begin{problem}\label{lcdmaxprofitexerfirst}
Suppose the revenue $R$, in \textit{thousands} of dollars, from producing and selling $x$ \textit{hundred} LCD TVs is given by $R(x) = -5x^3+35x^2+155x$ for $0 \leq x \leq 10.07$.
Use a graphing utility to graph $y = R(x)$ and determine the number of TVs which should be sold to maximize revenue.  What is the maximum revenue?  

\begin{solution}
    The calculator gives the location  of the absolute maximum (rounded to three decimal places) as $x \approx 6.305$ and $y \approx 1115.417$.  Since $x$ represents the number of TVs sold in hundreds, $x = 6.305$ corresponds to $630.5$ TVs.  Since we can't sell half of a TV, we compare $R(6.30) \approx 1115.415$ and $R(6.31) \approx 1115.416$, so selling $631$ TVs results in a (slightly) higher revenue.  Since $y$ represents the revenue in \textit{thousands} of dollars, the maximum revenue is $\$ 1,\!115,\!416$.
\end{solution}
\end{problem}

\begin{problem}
Suppose the revenue $R$, in \textit{thousands} of dollars, from producing and selling $x$ \textit{hundred} LCD TVs is given by $R(x) = -5x^3+35x^2+155x$ for $0 \leq x \leq 10.07$.
Assume the cost, in \textit{thousands} of dollars, to produce $x$ \textit{hundred} LCD TVs is given by the function $C(x) = 200x + 25$ for $x \geq 0$. Find and simplify an expression for the profit function $P(x)$.  

(Remember: Profit = Revenue - Cost.)

\begin{solution}
    $P(x) = R(x) - C(x) = -5x^3+35x^2-45x-25$, $0 \leq x \leq 10.07$.
\end{solution}
\end{problem}

\begin{problem}\label{lcdmaxprofitexerlast}
Suppose the revenue $R$, in \textit{thousands} of dollars, from producing and selling $x$ \textit{hundred} LCD TVs is given by $R(x) = -5x^3+35x^2+155x$ for $0 \leq x \leq 10.07$.
Use a graphing utility to graph $y = P(x)$ and determine the number of TVs which should be sold to maximize profit.  What is the maximum  profit?

\begin{solution}
    The calculator gives the location  of the absolute maximum (rounded to three decimal places) as $x \approx 3.897$ and $y \approx 35.255$.  Since $x$ represents the number of TVs sold in hundreds, $x = 3.897$ corresponds to $389.7$ TVs.  Since we can't sell $0.7$ of a TV, we compare $P(3.89) \approx 35.254$ and $P(3.90) \approx 35.255$, so selling $390$ TVs results in a (slightly) higher revenue.  Since $y$ represents the revenue in \textit{thousands} of dollars, the maximum revenue is $\$ 35,\!255$.
\end{solution}
\end{problem}

\begin{problem}\label{newportaboycost}
While developing their newest game, Sasquatch Attack!, the makers of the PortaBoy (from Example \ref{PortaBoyCost}) revised their cost function and now use $C(x) = .03x^{3} - 4.5x^{2} + 225x + 250$, for $x \geq 0$. As before, $C(x)$ is the cost to make $x$ PortaBoy Game Systems.  Market research indicates that the demand function $p(x) = -1.5x + 250$ remains unchanged.  Use a graphing utility to find the production level $x$ that maximizes the \textit{profit} made by producing and selling $x$ PortaBoy game systems.
\end{problem}

\begin{problem}
According to US Postal regulations, a rectangular shipping box must satisfy the following inequality: ``Length + Girth $\leq$ 130 inches'' for Parcel Post and ``Length + Girth $\leq$ 108 inches'' for other services. 

Let's assume we have a closed rectangular box with a square face of side length $x$ as drawn below.  The length is the longest side and is clearly labeled.  The girth is the distance around the box in the other two dimensions so in our case it is the sum of the four sides of the square, $4x$. 

\begin{enumerate}

\item \label{girthbox1} Assuming that we'll be mailing a box via Parcel Post where Length + Girth $=$ 130 inches, express the length of the box in terms of $x$ and then express the volume $V$ of the box in terms of $x$.

\item \label{girthbox2} Find the dimensions of the box of maximum volume that can be shipped via Parcel Post.

\item Repeat parts \ref{girthbox1} and \ref{girthbox2} if the box is shipped using ``other services''.

\end{enumerate}

\begin{center}

% \begin{mfpic}[8]{-6}{12}{-1}{17}
% \polyline{(0,0),(-4,3)}
% \polyline{(-4,3), (-4,8)}
% \polyline{(-4,8),(0,5)}
% \polyline{(0,5),(0,0)}
% \polyline{(0,0),(12,9)}
% \polyline{(0,5),(12,14)}
% \polyline{(-4,8),(8,17)}
% \polyline{(8,17),(12,14)}
% \polyline{(12,14),(12,9)}
% \arrow \reverse \arrow \polyline{(0,-0.5),(12,8.5)}
% \tlabel[cc](8,4){\tiny length}
% \arrow \reverse \arrow \polyline{(-4, 2.5), (-0.5,-0.125)}
% \tlabel[cc](-3,0.5){\tiny $x$}
% \arrow \reverse \arrow \polyline{(-4.5, 3), (-4.5,8)}
% \tlabel[cc](-5,5){\tiny $x$}
% \end{mfpic}

\begin{tikzpicture}[scale=0.5]

% Polylines for the shape
\draw (0,0) -- (-4,3);
\draw (-4,3) -- (-4,8);
\draw (-4,8) -- (0,5);
\draw (0,5) -- (0,0);

\draw (0,0) -- (12,9);
\draw (0,5) -- (12,14);
\draw (-4,8) -- (8,17);
\draw (8,17) -- (12,14);
\draw (12,14) -- (12,9);

% Double arrow for "length"
\draw[<->] (0,-0.5) -- (12,8.5);
\node at (8,4) {length};

% Double arrow for bottom "x"
\draw[<->] (-4,2.5) -- (-0.5,-0.125);
\node at (-3,0.5) {$x$};

% Double arrow for vertical "x"
\draw[<->] (-4.5,3) -- (-4.5,8);
\node at (-5,5) {$x$};

\end{tikzpicture}
\end{center}
\end{problem}


\begin{problem}\label{sunlighthigherorder} 
This exercise revisits the data set from Exercise \ref{regsunlight} in Section \ref{QuadraticFunctions}.  In that exercise, you were given a chart of the number of hours of daylight they get on the $21^{\mbox{st}}$ of each month in Fairbanks, Alaska based on the 2009 sunrise and sunset data found on the  \href{http://aa.usno.navy.mil/data/docs/RS_OneYear.php}{\underline{U.S. Naval Observatory}} website.  Here  $x = 1$ represents January 21, 2009, $x = 2$ represents February 21, 2009, and so on.  


\small

\noindent \begin{tabular}{|l|r|r|r|r|r|r|r|r|r|r|r|r|} \hline
Month  & & & & & & & & & & & & \\
Number & 1 & 2 & 3 & 4 & 5 & 6 & 7 & 8 & 9 & 10 & 11 & 12\\ 
\hline 
Hours of  & & & & & & & & & & & & \\
Daylight & 5.8 & 9.3 & 12.4 & 15.9 & 19.4 & 21.8 & 19.4 & 15.6 & 12.4 & 9.1 & 5.6 & 3.3 \\ \hline
\end{tabular}

\begin{enumerate}

\item Find cubic (third degree) and quartic (fourth degree) polynomials which model this data and comment on the goodness of fit for each.  What can we say about using either model to make predictions about the year 2020?  (Hint: Think about the end behavior of polynomials.)  

\item Use the models to see how many hours of daylight they got on your birthday and then check the website to see how accurate the models are.  

\item Sasquatch are largely nocturnal, so what days of the year according to your models  allow for at least 14 hours of darkness for field research on the elusive creatures? 

\end{enumerate}
\end{problem}

\begin{problem}\label{circuitexercisepoly}
An electric circuit is built with a variable resistor installed.  For each of the following resistance values (measured in kilo-ohms, $k \Omega$),  the corresponding power to the load (measured in milliwatts, $mW$) is given in the table below. \footnote{The authors wish to thank Don Anthan and Ken White of Lakeland Community College for devising this problem and generating the accompanying data set.}

\noindent \begin{tabular}{|l|r|r|r|r|r|r|} \hline
Resistance: ($k \Omega$) & 1.012 & 2.199 & 3.275 & 4.676 & 6.805 & 9.975 \\ \hline
Power: ($mW$) & 1.063 & 1.496 & 1.610 & 1.613 & 1.505 & 1.314 \\ \hline
\end{tabular}

\begin{enumerate}

\item Make a scatter diagram of the data using the Resistance as the independent variable and Power as the dependent variable.

\item Use your calculator to find quadratic (2nd degree), cubic (3rd degree) and quartic (4th degree) regression models for the data and judge the reasonableness of each.

\item For each of the models found above, find the predicted maximum power that can be delivered to the load.  What is the corresponding resistance value?

\item Discuss with your classmates the limitations of these models - in particular, discuss the end behavior of each.

\end{enumerate}
\end{problem}

\begin{problem}
Below is a graph of  a polynomial function $y = p(x)$ as generated by a graphing utility.   Answer the following questions about $p$ based on the graph provided.

\centerline{\includegraphics[width=3in]{./GraphsofPolynomialsGraphics/GraphsofPolyExercise.jpg}}

\begin{enumerate}

\item  Describe the end behavior of $y = p(x)$.

\item  List the real zeros of $p$ along with their respective multiplicities.  

\item  List the local minimums and local maximums of the graph of $y = p(x)$.

\item  What can be said about the degree of and leading coefficient $p(x)$?

\item  It turns out that $p(x)$ is a seventh degree polynomial.\footnote{to be exact, $p(x) = -0.1\left(x+1.5\right)^2\left(3x\right)\left(x-1\right)^3\left(x+5\right)$.}  How can this be?

\end{enumerate}
\end{problem}

\begin{problem}\label{comparegraphfromtheoryexample}  (This Exercise is a follow up to Example \ref{graphfromtheory}.)  Use a graphing utility to  compare and contrast the graphs of $f(x) = (2x-1)(x+1)^2(1-x)(x^2+1)$ and $g(x) = (2x-1)(x+1)^2(1-x)$.
\end{problem}

\begin{problem}  Use the graph of $y= p(x) = (2x-1)(x+1)(1-x^4)$ on page \pageref{localmaxminexample} to estimate the largest open interval containing $x = -0.235$ which satisfies the the criteria for `local minimum'  in Definition \ref{localmaxmindefn}.
\end{problem}

\begin{problem}  In light of Definition \ref{localmaxmindefn}, explain why \textit{every} point on the graph of a constant function is both a local maximum and a local minimum.
\end{problem}

\begin{problem}
This exercise involves the greatest integer function, $f(x) = \lfloor x \rfloor$,  introduced in Example \ref{greatestintegerdefn}.  Explain why the points $(k,k)$ for integers $k$ are local maximums but not local minimums.
\end{problem}

\begin{problem}
Use Theorems  \ref{EBPolynomials}  and \ref{polynomialbehaviornearzeros} prove Theorem \ref{roleofmultiplicity}.
\end{problem}

\begin{problem}
Here are a few other questions for you to discuss with your classmates.  

\begin{enumerate}

\item How many and how few local extrema could a polynomial of degree $n$ have?  
\item Could a polynomial have two local maxima but no local minima?  
\item If a polynomial has two local maxima and two local minima, can it be of odd degree?  Can it be of even degree?
\item Can a polynomial have local extrema without having any real zeros?
\item Why must every polynomial of odd degree have at least one real zero?
\item Can a polynomial have two distinct real zeros and no local extrema?
\item Can an $x$-intercept yield a local extrema?  Can it yield an absolute extrema?
\item If the $y$-intercept yields an absolute minimum, what can we say about the degree of the polynomial and the sign of the leading coefficient?   

\end{enumerate}
\end{problem}

\begin{problem}\label{LagrangePolyExercise}
(This is a follow-up to Exercises \ref{LagrangeLinearExercise} in Section \ref{ConstantandLinearFunctions} and \ref{LagrangeQuadExercise} in Section \ref{QuadraticFunctions}.) The  \href{https://en.wikipedia.org/wiki/Lagrange_polynomial}{\underline{Lagrange Interpolate}} function $L$  for four  points:  $(x_{0}, y_{0})$, $(x_{1}, y_{1})$,  $(x_{2}, y_{2})$,   $(x_{3}, y_{3})$ where $x_{0}$,  $x_{1}$, $x_{2}$, and $x_{3}$ are four distinct real numbers is given by the formula: 

 \[ \begin{array}{rcl}
 
 L(x) & = &  y_{0}  \dfrac{(x - x_{1}) (x - x_{2}) (x-x_{3})}{(x_{0} - x_{1})(x_{0} - x_{2})(x_{0} - x_{3})}+ y_{1}  \dfrac{(x - x_{0}) (x - x_{2}) (x-x_{3})}{(x_{1} - x_{0})(x_{1} - x_{2})(x_{1} - x_{3})} \\ [15pt]
         &&  +y_{2}  \dfrac{(x - x_{0}) (x - x_{1}) (x-x_{3})}{(x_{2} - x_{0})(x_{2} - x_{1})(x_{2} - x_{3})}+ y_{3}  \dfrac{(x - x_{0}) (x - x_{1}) (x-x_{2})}{(x_{3} - x_{0})(x_{3} - x_{1})(x_{3} - x_{2})} \\ \end{array}\]

\begin{enumerate}

\item Choose four points with different $x$-values and construct the Lagrange Interpolate for those points.  Verify each of the points lies on the polynomial.  

\item  Verify that, in general, $L(x_{0}) = y_{0}$,  $L(x_{1}) = y_{1}$, $L(x_{2}) = y_{2}$, and  $L(x_{3}) = y_{3}$.

\item  Find $L(x)$ for the points $(-1,1)$, $(0,0)$,  $(1,1)$ and $(2,4)$.  What happens?

\item  Find $L(x)$ for the points $(-1,0)$, $(0,1)$,  $(1,2)$ and $(2,3)$.  What happens?

\item  Generalize the formula for $L(x)$ to five points.  What's the pattern?

\end{enumerate}
    
\end{problem}
 




\end{document}
