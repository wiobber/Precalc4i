\documentclass{ximera}

\begin{document}
	\author{Stitz-Zeager}
	\xmtitle{Exercises}
\mfpicnumber{1} \opengraphsfile{ExercisesforComplexZeros} % mfpic settings added 




\begin{problem}\label{compfactpolyfirst}
Find all of the zeros of the polynomial then completely factor it over the real numbers and completely factor it over the complex numbers.

$f(x) = x^{2} - 4x + 13$ 
\end{problem}

\begin{problem}
Find all of the zeros of the polynomial then completely factor it over the real numbers and completely factor it over the complex numbers.

$f(x) = x^2 - 2x + 5$
\end{problem}

\begin{problem}
Find all of the zeros of the polynomial then completely factor it over the real numbers and completely factor it over the complex numbers.

$p(z) = 3z^{2} + 2z + 10$
\end{problem}

\begin{problem}
Find all of the zeros of the polynomial then completely factor it over the real numbers and completely factor it over the complex numbers.

$p(z) = z^3-2z^2+9z-18$
\end{problem}

\begin{problem}
Find all of the zeros of the polynomial then completely factor it over the real numbers and completely factor it over the complex numbers.

$g(t) = t^{3} + 6t^{2} + 6t + 5$
\end{problem}

\begin{problem}
Find all of the zeros of the polynomial then completely factor it over the real numbers and completely factor it over the complex numbers.

$g(t) = 3t^{3} - 13t^{2} + 43t - 13$
\end{problem}

\begin{problem}
Find all of the zeros of the polynomial then completely factor it over the real numbers and completely factor it over the complex numbers.

$f(x) = x^3 + 3x^2 + 4x + 12$
\end{problem}

\begin{problem}
Find all of the zeros of the polynomial then completely factor it over the real numbers and completely factor it over the complex numbers.

$f(x) = 4x^3-6x^2-8x+15$
\end{problem}

\begin{problem}
Find all of the zeros of the polynomial then completely factor it over the real numbers and completely factor it over the complex numbers.

$p(z) = z^3 + 7z^2+9z-2$
\end{problem}

\begin{problem}
Find all of the zeros of the polynomial then completely factor it over the real numbers and completely factor it over the complex numbers.

$p(z) = 9z^3+2z+1$
\end{problem}

\begin{problem}
Find all of the zeros of the polynomial then completely factor it over the real numbers and completely factor it over the complex numbers.

$g(t) = 4t^{4} - 4t^{3} + 13t^{2} - 12t + 3$
\end{problem}

\begin{problem}
Find all of the zeros of the polynomial then completely factor it over the real numbers and completely factor it over the complex numbers.

$g(t) = 2t^4-7t^3+14t^2-15t+6$
\end{problem}

\begin{problem}
Find all of the zeros of the polynomial then completely factor it over the real numbers and completely factor it over the complex numbers.

$f(x) = x^4+x^3+7x^2+9x-18$
\end{problem}

\begin{problem}
Find all of the zeros of the polynomial then completely factor it over the real numbers and completely factor it over the complex numbers.

$f(x) = 6x^4+17x^3-55x^2+16x+12$
\end{problem}

\begin{problem}
Find all of the zeros of the polynomial then completely factor it over the real numbers and completely factor it over the complex numbers.

$p(z) = -3z^4-8z^3-12z^2-12z-5$
\end{problem}

\begin{problem}
Find all of the zeros of the polynomial then completely factor it over the real numbers and completely factor it over the complex numbers.

$p(z) = 8z^4+50z^3+43z^2+2z-4$
\end{problem}

\begin{problem}
Find all of the zeros of the polynomial then completely factor it over the real numbers and completely factor it over the complex numbers.

$g(t) = t^4+9t^2+20$
\end{problem}

\begin{problem}
Find all of the zeros of the polynomial then completely factor it over the real numbers and completely factor it over the complex numbers.

$g(t) = t^4 + 5t^2 - 24$
\end{problem}

\begin{problem}
Find all of the zeros of the polynomial then completely factor it over the real numbers and completely factor it over the complex numbers.

$f(x) = x^5 - x^4+7x^3-7x^2+12x-12$
\end{problem}

\begin{problem}
Find all of the zeros of the polynomial then completely factor it over the real numbers and completely factor it over the complex numbers.

$f(x) = x^6-64$
\end{problem}

\begin{problem}
Find all of the zeros of the polynomial then completely factor it over the real numbers and completely factor it over the complex numbers.

$f(x) = x^{4} - 2x^{3} + 27x^{2} - 2x + 26$

\begin{hint}
$x = i$ is one of the zeros.  
\end{hint}
\end{problem}

\begin{problem}\label{compfactpolylast}
Find all of the zeros of the polynomial then completely factor it over the real numbers and completely factor it over the complex numbers.

$p(z) = 2z^4+5z^3+13z^2+7z+5$ 

\begin{hint}
$z = -1+2i$ is a zero.  
\end{hint}
\end{problem} 

\begin{problem}\label{buildcomppolyfirst}
Use Theorem \ref{complexfactorization} to create a polynomial function with real number coefficients which has all of the desired characteristics.  You may leave the polynomial in factored form.

\begin{itemize}

\item The zeros of $f$ are $c = \pm 2$ and $c = \pm 1$.
\item The leading term of $f(x)$ is $117x^4$.

\end{itemize}
\end{problem}

\begin{problem}
Use Theorem \ref{complexfactorization} to create a polynomial function with real number coefficients which has all of the desired characteristics.  You may leave the polynomial in factored form.

\begin{itemize}

\item The zeros of $p$ are $c=1$ and $c = 3$.
\item $c=3$ is a zero of multiplicity 2.
\item The leading term of $p(z)$ is $-5z^3$.

\end{itemize}
\end{problem}

\begin{problem}
Use Theorem \ref{complexfactorization} to create a polynomial function with real number coefficients which has all of the desired characteristics.  You may leave the polynomial in factored form.

\begin{itemize}

\item The solutions to $g(t) = 0$ are $t = \pm 3$ and $t=6$.
\item The leading term of $g(t)$ is $7t^4$.
\item The point $(-3,0)$ is a local minimum on the graph of $y=g(t)$.

\end{itemize}
\end{problem}

\begin{problem}
Use Theorem \ref{complexfactorization} to create a polynomial function with real number coefficients which has all of the desired characteristics.  You may leave the polynomial in factored form.

\begin{itemize}

\item The solutions to $f(x) =0$ are $x = \pm 3$, $x=-2$, and $x=4$.
\item The leading term of $f(x)$ is $-x^5$.
\item The point $(-2, 0)$ is a local maximum on the graph of $y=f(x)$.

\end{itemize}
\end{problem}

\begin{problem}
Use Theorem \ref{complexfactorization} to create a polynomial function with real number coefficients which has all of the desired characteristics.  You may leave the polynomial in factored form.

\begin{itemize}

\item $p$ is degree 4.
\item $\ds{\lim_{z \rightarrow \infty} p(z) =  -\infty}$.
\item $p$ has exactly three $z$-intercepts:  $(-6,0)$, $(1,0)$ and $(117,0)$.
\item  The graph of $y=p(z)$ crosses through the $z$-axis at $(1,0)$.

\end{itemize}
\end{problem}

\begin{problem}
Use Theorem \ref{complexfactorization} to create a polynomial function with real number coefficients which has all of the desired characteristics.  You may leave the polynomial in factored form.

\begin{itemize}

\item The zeros of $g$ are $c=\pm 1$ and $c = \pm i$.
\item The leading term of $g(t)$ is $42t^4$.

\end{itemize}
\end{problem}

\begin{problem}
Use Theorem \ref{complexfactorization} to create a polynomial function with real number coefficients which has all of the desired characteristics.  You may leave the polynomial in factored form.

\begin{itemize}

\item $c=2i$ is a zero.
\item the point $(-1,0)$ is a local minimum on the graph of $y=f(x)$.
\item the leading term of $f(x)$ is $117x^4$.

\end{itemize}
\end{problem}

\begin{problem}
Use Theorem \ref{complexfactorization} to create a polynomial function with real number coefficients which has all of the desired characteristics.  You may leave the polynomial in factored form.

\begin{itemize}

\item The solutions to $p(z) = 0$ are $z = \pm 2$ and $z=\pm 7i$.
\item The leading term of $p(z)$ is $-3z^5$.
\item The point $(2,0)$ is a local maximum on the graph of $y=p(z)$.

\end{itemize}
\end{problem}

\begin{problem}
Use Theorem \ref{complexfactorization} to create a polynomial function with real number coefficients which has all of the desired characteristics.  You may leave the polynomial in factored form.

\begin{itemize}

\item $g$ is degree $5$.
\item $t=6$, $t = i$ and $t = 1-3i$ are zeros of $g$.
\item $\ds{\lim_{t \rightarrow -\infty} g(t) = \infty}$ 

\end{itemize}
\end{problem}

\begin{problem}
Use Theorem \ref{complexfactorization} to create a polynomial function with real number coefficients which has all of the desired characteristics.  You may leave the polynomial in factored form.

\begin{itemize}

\item The leading term of $f(x)$ is $-2x^3$.
\item $c=2i$ is a zero.
\item $f(0) = -16$.

\end{itemize}
\end{problem}

\begin{problem}\label{polyfromgraphfirst}
Find a possible formula for the polynomial function given its graph.  You may leave the polynomial in factored form. 

$y=f(x)$.  %$f(x) = x(x+6)(x-6)

\begin{mfpic}[10]{-7}{7}{-6}{6}
\axes
\tlabel[cc](7,-0.5){\scriptsize $x$}
\tlabel[cc](0.5,6){\scriptsize $y$}
\tlabel[cc](-4.5, 0.75){\scriptsize $(-6,0)$}
\tlabel[cc](5, 0.75){\scriptsize $(6,0)$}
\tlabel[cc](1, 0.75){\scriptsize $(0,0)$}
\tlabel[cc](-3, 5){\scriptsize $(-3,81)$}
\point[4pt]{(-6,0),(0,0),(6,0), (-3,4.05) }
\xmarks{-6 step 1 until 6}
\tiny
\tlpointsep{4pt}
\axislabels {x}{{$-6 \hspace{6pt}$} -6, {$-5 \hspace{6pt}$} -5, {$-4 \hspace{6pt}$} -4, {$-3 \hspace{6pt}$} -3, {$-2 \hspace{6pt}$} -2, {$-1 \hspace{6pt}$} -1, {$1$} 1, {$2$} 2, {$3$} 3, {$4$} 4, {$5$} 5, {$6$} 6}
\normalsize
\penwd{1.25pt}
\arrow \reverse \arrow \function{-7,7,0.1}{((x**3) - 36*x)/20}
\end{mfpic}
\end{problem}

\begin{problem}
Find a possible formula for the polynomial function given its graph.  You may leave the polynomial in factored form. 

$y=g(t)$  %$g(t) = t(t+2)^3$

\begin{mfpic}[20]{-3}{3}{-2}{5}
\axes
\tlabel[cc](3,-0.5){\scriptsize $t$}
\tlabel[cc](0.25,5){\scriptsize $y$}
\tlabel[cc](-1.75, 0.3){\scriptsize $(-2,0)$}
\tlabel[cc](0.5, 0.3){\scriptsize $(0,0)$}
\tlabel[cc](-2, -1){\scriptsize $(-1,-1)$}
\point[4pt]{(-2,0), (0,0), (-1,-1)}
\xmarks{-2,-1, 1, 2}
\tiny
\tlpointsep{4pt}
\axislabels {x}{{$-2 \hspace{6pt}$} -2, {$-1 \hspace{6pt}$} -1, {$1$} 1, {$2$} 2}
\normalsize
\penwd{1.25pt}
\arrow \reverse \arrow \function{-3,0.3,0.1}{x*((x + 2)**3)}

\end{mfpic}
\end{problem}

\begin{problem}
Find a possible formula for the polynomial function given its graph.  You may leave the polynomial in factored form. 

$y = p(z)$  %$p(z) = -2(z+1)(z-2)^2$

\begin{mfpic}[30][15]{-3}{3}{-4}{4}
\axes
\tlabel[cc](3,-0.5){\scriptsize $z$}
\tlabel[cc](0.25,4){\scriptsize $y$}
\tlabel[cc](-2, 0.75){\scriptsize $(-1,0)$}
\tlabel[cc](2, 0.75){\scriptsize $(2,0)$}
\tlabel[cc](0.75, -2.25){\scriptsize $(0, -8)$}
\point[4pt]{(2,0), (-1,0), (0, -1.6)}
\xmarks{-2,-1,1,2}
\tiny
\tlpointsep{4pt}
\axislabels {x}{{$-2 \hspace{6pt}$} -2, {$-1 \hspace{6pt}$} -1, {$1$} 1, {$2$} 2}
\normalsize
\penwd{1.25pt}
\arrow \reverse \arrow \function{-1.70,3.45,0.1}{(-0.4)*((x-2)**2)*(x+1)}
\end{mfpic}
\end{problem}

\begin{problem}
Find a possible formula for the polynomial function given its graph.  You may leave the polynomial in factored form. 

$y = f(x)$  %$f(x) = 4\left(x+ \frac{1}{2}\right)^2 (x-3)$

\begin{mfpic}[30][15]{-2}{4}{-4}{4}
\axes
\tlabel[cc](4,-0.5){\scriptsize $x$}
\tlabel[cc](0.25,4){\scriptsize $y$}
\tlabel[cc](-0.75, 0.75){\scriptsize $\left(-\frac{1}{2},0 \right)$}
\tlabel[cc](2.5, 0.75){\scriptsize $(3,0)$}
\tlabel[cc](3.25, -3.125){\scriptsize $(2,-25)$}
\point[4pt]{(-0.5,0), (3,0), (2, -3.125)}
\xmarks{-1,1,2,3}
\tiny
\tlpointsep{4pt}
\axislabels {x}{ {$1$} 1, {$2$} 2, {$3$} 3}
\normalsize
\penwd{1.25pt}
\arrow \reverse \arrow \function{-1.5,3.3,0.1}{(0.5)*((x+0.5)**2)*(x-3)}
\end{mfpic}
\end{problem}

\begin{problem}
Find a possible formula for the polynomial function given its graph.  You may leave the polynomial in factored form. 

$y = F(s)$  %$F(s)  =-s(s+2)^2$

\begin{mfpic}[30][15]{-3}{3}{-5}{5}
\axes
\tlabel[cc](3,-0.5){\scriptsize $s$}
\tlabel[cc](0.25,5){\scriptsize $y$}
\tlabel[cc](-2, -0.75){\scriptsize $(-2,0)$}
\tlabel[cc](0.5, 0.75){\scriptsize $(0,0)$}
\tlabel[cc](-1.25, 1.75){\scriptsize $(-1,1)$}
\point[4pt]{(-2,0), (0,0), (-1,1)}
\xmarks{-2,-1,1,2}
\tiny
\tlpointsep{4pt}
\axislabels {x}{ {$1$} 1, {$2$} 2}
\normalsize
\penwd{1.25pt}
\arrow \reverse \arrow \function{-3,0.65,0.1}{0-x*((x + 2)**2)}
\end{mfpic}
\end{problem}

\begin{problem}
Find a possible formula for the polynomial function given its graph.  You may leave the polynomial in factored form. 

$y = G(t)$  %$G(t) = t^3(t+2)^2$

\begin{mfpic}[30][15]{-3}{3}{-5}{5}
\axes
\tlabel[cc](-2, 0.75){\scriptsize $(-2,0)$}
\tlabel[cc](0.5, -0.75){\scriptsize $(0,0)$}
\tlabel[cc](-1, -2){\scriptsize $(-1,-1)$}
\tlabel[cc](3,-0.5){\scriptsize $t$}
\tlabel[cc](0.25,5){\scriptsize $y$}
\point[4pt]{(-2,0), (0,0), (-1,-1)}
\xmarks{-2,-1,1,2}
\tiny
\tlpointsep{4pt}
\axislabels {x}{  {$2$} 2}
\normalsize
\penwd{1.25pt}
\arrow \reverse \arrow \function{-2.45,0.85,0.1}{(x**3)*((x + 2)**2)}
\end{mfpic}
\end{problem}

\begin{problem}\label{cmpgeoalgexfirst}
With help from your classmates, choose several nonzero complex numbers $z$, find their complex conjugates $\overline{z}$.  Plot each pair $z$ and $\overline{z}$ in the Complex Plane.  What appears to be the relationship between these numbers geometrically?  State and prove a general result. 
\end{problem}

\begin{problem}
With help from your classmates, choose several nonzero complex numbers $z$ and  find $-z$.  Plot each pair $z$ and $-z$ in the Complex Plane.  What appears to be the relationship between these numbers geometrically?  State and prove a general result.
\end{problem}

\begin{problem}
With help from your classmates, choose several different complex numbers $z$ and find the product of $i$ and $z$,  $iz$.  Plot each pair of $z$ and $iz$ in the Complex Plane.  In each case, show the line containing the origin and the point corresponding to $z$ is perpendicular\footnote{See Theorem \ref{parallelperpendicularslopetheorem} in Section \ref{AppLines} of you need a refresher on how to do this.} to the line containing the origin and the point corresponding to $iz$.  Show this result holds in general for every nonzero complex number.
\end{problem}

\begin{problem}\label{cmpgeoalgexlast}
Given a complex number $z = a+bi$, we define the \index{modulus ! of a complex number}\textbf{modulus} of $z$, $|z|$, by $|z| = \sqrt{a^2+b^2}$.  With help from your classmates, calculate $|z|$ for several different complex numbers, $z$.  What does $z$ measure geometrically?  Show that if $x$ is a real number, then the modulus of $x$ is the same as the absolute value of $x$, and comment how all this relates to Definition \ref{absvaldistdefn} in Section \ref{AppAbsValEqIneq}.
\end{problem}

\begin{problem}\label{zbarexercise}
Let $z$ and $w$ be arbitrary complex numbers.  Show that  $\overline{z} \, \overline{w}  = \overline{zw}$ and $\overline{\overline{z}} = z$.  
\end{problem}  

\end{document}
