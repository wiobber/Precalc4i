\documentclass{ximera}

\begin{document}
	\author{Stitz-Zeager}
	\xmtitle{Exercises for Polydivision}{}

\mfpicnumber{1} \opengraphsfile{ExercisesforPolydivision} % mfpic settings added 

\begin{problem}\label{synthdivreviewfirst}
Use synthetic division to compute $\left(3x^2-2x+1 \right) \div \left(x-1\right)$.  Identify the quotient and remainder. Write the divisor, quotient and remainder in the form $p(x) = d(x) \, q(x) + r(x),\,$, as in Theorem \ref{polydivthm}.

$3x^2-2x+1 = \answer{\left(x-1\right) (3x+1)+2}$

\end{problem}

\begin{problem}
Use synthetic division to compute $\left(x^2-5 \right) \div \left(x-5\right)$.  Identify the quotient and remainder. Write the divisor, quotient and remainder in the form $p(x) = d(x) \, q(x) + r(x),\,$, as in Theorem \ref{polydivthm}.

$x^2-5 = \left(\answer{x-5}\right)\left(\answer{x+5}\right) + \answer{20}$


\end{problem}

\begin{problem}
Use synthetic division to compute $\left(3-4t-2t^2 \right) \div \left(t+1\right)$.  Identify the quotient and remainder. Write the divisor, quotient and remainder in the form $p(t) = d(t) \, q(t) + r(t),\,$, as in Theorem \ref{polydivthm}.

$(3-4t-2t^2 = \answer{\left(t+1\right)(-2t-2)+5}$

\end{problem}

\begin{problem}
Use synthetic division to compute $\left(4t^2-5t +3\right) \div \left(t+3\right)$.  Identify the quotient and remainder. Write the divisor, quotient and remainder in the form $p(t) = d(t) \, q(t) + r(t),\,$, as in Theorem \ref{polydivthm}.

$4t^2-5t +3 = \left(\answer{t+3}\right)\left(\answer{4t-17}\right) + \answer{54}$
\end{problem}

\begin{problem}
Use synthetic division to compute $\left(z^3 + 8 \right) \div \left(z+2\right)$.  Identify the quotient and remainder. Write the divisor, quotient and remainder in the form $p(z) = d(z) \, q(z) + r(z),\,$, as in Theorem \ref{polydivthm}.

$z^3 + 8 = \answer{\left(z+2\right) \left(z^2-2z+4\right) + 0}$

\end{problem}

\begin{problem}
Use synthetic division to compute $\left(4z^3 +2z-3 \right) \div \left(z -3\right)$.  Identify the quotient and remainder. Write the divisor, quotient and remainder in the form $p(z) = d(z) \, q(z) + r(z),\,$, as in Theorem \ref{polydivthm}.

$4z^3 +2z-3 = \left(\answer{z-3}\right)\left(\answer{4z^2+12z+38}\right) + \answer{111}$
\end{problem}

\begin{problem}
Use synthetic division to compute $\left(18x^2-15x-25\right) \div \left(x - \frac{5}{3} \right)$.  Identify the quotient and remainder. Write the divisor, quotient and remainder in the form $p(x) = d(x) \, q(x) + r(x),\,$, as in Theorem \ref{polydivthm}.

$18x^2-15x-25 = \answer{\left(x - \frac{5}{3} \right)(18x+15)+0}$

\end{problem}

\begin{problem}
Use synthetic division to compute $\left(4x^2-1 \right) \div \left(x - \frac{1}{2} \right)$.  Identify the quotient and remainder. Write the divisor, quotient and remainder in the form $p(x) = d(x) \, q(x) + r(x),\,$, as in Theorem \ref{polydivthm}.

$4x^2-1 = \left(\answer{x - \frac{1}{2}}\right)\left(\answer{4x+2}\right) + \answer{0}$
\end{problem}

\begin{problem}
Use synthetic division to compute $\left(2t^3+t^2+2t+1 \right) \div \left(t + \frac{1}{2} \right)$.  Identify the quotient and remainder. Write the divisor, quotient and remainder in the form $p(t) = d(t) \, q(t) + r(t),\,$, as in Theorem \ref{polydivthm}.

$2t^3+t^2+2t+1 = \answer{\left(t + \frac{1}{2} \right)\left(2t^2+2\right)+0}$

\end{problem}

\begin{problem}
Use synthetic division to compute $\left(3t^3 - t + 4 \right) \div \left(t - \frac{2}{3} \right)$.  Identify the quotient and remainder. Write the divisor, quotient and remainder in the form $p(t) = d(t) \, q(t) + r(t),\,$, as in Theorem \ref{polydivthm}.

$3t^3 - t + 4 = \left(\answer{t - \frac{2}{3}}\right)\left(\answer{3t^2+2t+\frac{1}{3}}\right) + \answer{\frac{38}{9}}$
\end{problem}

\begin{problem}
Use synthetic division to compute $\left(2z^3 - 3z +1 \right) \div \left(z - \frac{1}{2} \right)$.  Identify the quotient and remainder. Write the divisor, quotient and remainder in the form $p(z) = d(z) \, q(z) + r(z),\,$, as in Theorem \ref{polydivthm}.

$2z^3 - 3z +1 = \answer{\left(z - \frac{1}{2} \right) \left(2z^2+z-\frac{5}{2}\right)-\frac{1}{4}}$

\end{problem}

\begin{problem}
Use synthetic division to compute $\left(4z^4-12z^3+13z^2 -12z+9\right) \div \left(z - \frac{3}{2} \right)$.  Identify the quotient and remainder. Write the divisor, quotient and remainder in the form $p(z) = d(z) \, q(z) + r(z),\,$, as in Theorem \ref{polydivthm}.

$4z^4-12z^3+13z^2 -12z+9 = \left(\answer{z - \frac{3}{2}}\right)\left(\answer{4z^3-6z^2+4z-6}\right) + \answer{0}$
\end{problem}

\begin{problem}
Use synthetic division to compute $\left(x^4-6x^2+9 \right) \div \left(x -\sqrt{3} \right)$.  Identify the quotient and remainder. Write the divisor, quotient and remainder in the form $p(x) = d(x) \, q(x) + r(x),\,$, as in Theorem \ref{polydivthm}.

$x^4-6x^2+9 = \answer{\left(x -\sqrt{3} \right) \left(x^3+\sqrt{3} \,x^2-3x-3\sqrt{3}\right) + 0}$

\end{problem}

\begin{problem}\label{synthdivreviewlast}
Use synthetic division to compute $\left(x^6-6x^4+12x^2-8\right) \div \left(x +\sqrt{2} \right)$.  Identify the quotient and remainder. Write the divisor, quotient and remainder in the form $p(x) = d(x) \, q(x) + r(x),\,$, as in Theorem \ref{polydivthm}.

\begin{solution}
$\left(x^6-6x^4+12x^2-8\right) = \left(x +\sqrt{2} \right) \left(x^5-\sqrt{2} \, x^4-4x^3+4\sqrt{2} \, x^2+4x-4\sqrt{2}\right) + 0$
\end{solution}
\end{problem}

\begin{problem}\label{remainderexerfirst}
Find $p(c)$ using the Remainder Theorem.  If $p(c) = 0$, use the Factor Theorem to partially factor the polynomial function.

$p(x) = 2x^2 - x + 1$, $c = 4$

$p(4) = \answer{29}$

\end{problem}

\begin{problem}
Find $p(c)$ using the Remainder Theorem.  If $p(c) = 0$, use the Factor Theorem to partially factor the polynomial function.

$p(x) = 4x^2-33x-180$, $c = 12$

$p(12) =\answer{0}$, $p(x) = \answer{(x-12)(4x+15)}$
\end{problem}

\begin{problem}
Find $p(c)$ using the Remainder Theorem.  If $p(c) = 0$, use the Factor Theorem to partially factor the polynomial function.

$p(t) = 2t^3 - t + 6$, $c=-3$

$p(-3)= \answer{-45}$

\end{problem}

\begin{problem}
Find $p(c)$ using the Remainder Theorem.  If $p(c) = 0$, use the Factor Theorem to partially factor the polynomial function.

$p(t) = t^3+2t^2+3t+4$, $c =-1$

$p(-1)=\answer{2}$
\end{problem}

\begin{problem}
Find $p(c)$ using the Remainder Theorem.  If $p(c) = 0$, use the Factor Theorem to partially factor the polynomial function.

$p(z) =3z^3-6z^2+4z-8$, $c=2$

$p(2) = \answer{0}$, $p(z)= \answer{(z-2) \left(3z^2+4\right)}$

\end{problem}

\begin{problem}
Find $p(c)$ using the Remainder Theorem.  If $p(c) = 0$, use the Factor Theorem to partially factor the polynomial function.

$p(z) = 8z^3+12z^2+6z+1$, $c =-\frac{1}{2}$

\begin{solution}
$p\left(-\frac{1}{2}\right) = 0$, $p(z)  = \left(z+\frac{1}{2}\right)\left(8z^2+8z+2\right)$   
\end{solution}
\end{problem}

\begin{problem}
Find $p(c)$ using the Remainder Theorem.  If $p(c) = 0$, use the Factor Theorem to partially factor the polynomial function.

$p(x) = x^4 - 2x^2+4$, $c=\frac{3}{2}$

\begin{solution}
$p\left(\frac{3}{2}\right) = \frac{73}{16}$
\end{solution}


\end{problem}

\begin{problem}
Find $p(c)$ using the Remainder Theorem.  If $p(c) = 0$, use the Factor Theorem to partially factor the polynomial function.

$p(x) = 6x^4-x^2+2$, $c =-\frac{2}{3}$

\begin{solution}
$p\left(-\frac{2}{3}\right) = \frac{74}{27}$
\end{solution}
\end{problem}

\begin{problem}
Find $p(c)$ using the Remainder Theorem.  If $p(c) = 0$, use the Factor Theorem to partially factor the polynomial function.

$p(t) = t^4 +t^3-6t^2-7t-7$, $c=-\sqrt{7}$

\begin{solution}
$p(-\sqrt{7}) = 0$, $p(t) = (t+\sqrt{7})\left(t^3+(1-\sqrt{7}) t^2+(1-\sqrt{7})t-\sqrt{7}  \right)$
\end{solution}


\end{problem}

\begin{problem}\label{remainderexerlast}
Find $p(c)$ using the Remainder Theorem.  If $p(c) = 0$, use the Factor Theorem to partially factor the polynomial function.

$p(t) = t^2-4t+1$, $c =2-\sqrt{3}$

\begin{solution}
$p(2-\sqrt{3}) =0$, $p(t) = (t-(2-\sqrt{3}))(t-(2+\sqrt{3})) $
\end{solution}
\end{problem}

\begin{problem}\label{factorpolyzerofirst}
In this exercise you are given a polynomial function and one of its zeros.  Find the remaining real zeros and factor the polynomial. 

$x^{3} - 6x^{2} + 11x - 6, \;\; c = 1$ 

\begin{solution}
$x^{3} - 6x^{2} + 11x - 6 = (x - 1)(x - 2)(x - 3)$
\end{solution}

\end{problem}

\begin{problem}
In this exercise you are given a polynomial function and one of its zeros.  Find the remaining real zeros and factor the polynomial. 

$x^{3} - 24x^{2} + 192x - 512, \;\; c = 8$

When we factor the polynomial we get $x^{3} - 24x^{2} + 192x - 512 = \answer{(x - 8)^{3}}$
\end{problem}

\begin{problem}
In this exercise you are given a polynomial function and one of its zeros.  Find the remaining real zeros and factor the polynomial. 

$3t^{3} + 4t^{2} - t - 2, \;\; c = \frac{2}{3}$

\begin{solution}
$3t^{3} + 4t^{2} - t - 2 = 3\left(t - \frac{2}{3}\right)(t + 1)^{2}$
\end{solution}

\end{problem}

\begin{problem}
In this exercise you are given a polynomial function and one of its zeros.  Find the remaining real zeros and factor the polynomial. 

$2t^3-3t^2-11t+6, \;\; c=\frac{1}{2}$

\begin{solution}
$2t^3-3t^2-11t+6 = 2\left(t-\frac{1}{2}\right)(t+2)(t-3)$
\end{solution}
\end{problem}

\begin{problem}
In this exercise you are given a polynomial function and one of its zeros.  Find the remaining real zeros and factor the polynomial. 

$z^3+2z^2-3z-6, \;\; c = -2$

\begin{solution}
$z^3+2z^2-3z-6 = (z+2)(z+\sqrt{3})(z-\sqrt{3})$
\end{solution}


\end{problem}

\begin{problem}
In this exercise you are given a polynomial function and one of its zeros.  Find the remaining real zeros and factor the polynomial. 

$2z^3-z^2-10z+5, \;\; c=\frac{1}{2}$

\begin{solution}
$2z^3-z^2-10z+5=2\left(z-\frac{1}{2}\right)(z+\sqrt{5})(z-\sqrt{5})$
\end{solution}
\end{problem}

\begin{problem}
In this exercise you are given a polynomial function and one of its zeros.  Find the remaining real zeros and factor the polynomial. 

$4x^{4} - 28x^{3} + 61x^{2} - 42x + 9$, $c = \frac{1}{2}$ is a zero of multiplicity 2 

\begin{solution}
$4x^{4} - 28x^{3} + 61x^{2} - 42x + 9 = 4\left(x - \frac{1}{2} \right)^{2}(x - 3)^{2}$
\end{solution}

\end{problem}

\begin{problem}
In this exercise you are given a polynomial function and one of its zeros.  Find the remaining real zeros and factor the polynomial. 

$t^5+2t^4-12t^3-38t^2-37t-12$, $c=-1$ is a zero of multiplicity 3

When we factor the polynomial we get $t^5+2t^4-12t^3-38t^2-37t-12 = \answer{(t+1)^3(t+3)(t-4)}$
\end{problem}

\begin{problem}
In this exercise you are given a polynomial function and one of its zeros.  Find the remaining real zeros and factor the polynomial. 

$125z^{5} - 275z^{4} - 2265z^{3} - 3213z^{2} - 1728z - 324$, $c = -\frac{3}{5}$ is a zero of multiplicity 3

\begin{solution}
$125z^{5} - 275z^{4} - 2265z^{3} - 3213z^{2} - 1728z - 324$ = $125\left(z + \frac{3}{5} \right)^{3}(z + 2)(z - 6)$
\end{solution}

\end{problem}

\begin{problem}\label{factorpolyzerolast}
In this exercise you are given a polynomial function and one of its zeros.  Find the remaining real zeros and factor the polynomial. 

$x^{2} - 2x - 2, \;\; c = 1 - \sqrt{3}$

\begin{solution}
$x^{2} - 2x - 2 = (x - (1 - \sqrt{3}))(x - (1 + \sqrt{3}))$
\end{solution}
\end{problem}

\begin{problem}
Find a quadratic polynomial with \underline{integer} coefficients which has $x = \dfrac{3}{5} \pm \dfrac{\sqrt{29}}{5}$ as its real zeros. 

$p(x) = 5x^{2} - 6x - 4$

\end{problem}

\begin{problem}\label{verifyrootsex}

For $f(x) = x^3 + 4x^2-5x-14$, show $f(-3-\sqrt{2}) = 0$ and $f(-3+\sqrt{2}) = 0$ two ways:

\begin{enumerate}

\item  By direct substitution.

\item  Using synthetic division and the Factor Theorem

\end{enumerate}
\end{problem}

\begin{problem}\label{oneisazeroex}
Let $f(x) = a_{n} x^{n} + a_{n-1} x^{n-1} + \ldots + a_{2} x^{2} + a_{1} x + a_{0}$ be a polynomial function with the property that $ a_{n}+a_{n-1} + \ldots + a_{1} + a_{0} = 0$.  (That is, the sum of the coefficients and the constant term is $0$.)  

Prove that $(x-1)$ is a factor of $f(x)$.

\begin{hint}
Show $f(1) = 0$ and invoke the Factor Theorem.
\end{hint}
\end{problem}

\begin{problem}
Verify the result in number \ref{oneisazeroex} with the functions: $f(x) = x^3 - 2x + 1$ and  $f(x) = 3x^4-x-2$.

\begin{solution}
\begin{itemize}

\item For $f(x) = x^3 - 2x + 1$, the coefficients $1+(-2) + 1 = 0$ and $f(x) = (x-1)(x^2+x-1)$.

\item  For $f(x) = 3x^4-x-2$ the coefficients $3+(-1)+(-2) = 0$ and $f(x) = (x-1)(3x^3+3x^2+3x+2)$.

\end{itemize}
\end{solution}
\end{problem}

\begin{problem}\label{monomialdiffquotex} 
Suppose $a$ is a nonzero real number.  Find the quotients below, using synthetic division as required. 

\begin{itemize}
    \item $\dfrac{x - a}{x-a}$
    \item $\dfrac{x^2 - a^2}{x-a}$ 
    \item $\dfrac{x^3 - a^3}{x-a}$ 
    \item $\dfrac{x^4 - a^4}{x-a}$ 
    \item $\dfrac{x^5 - a^5}{x-a}$ 
\end{itemize}

Based on the pattern that evolves, find the quotient: $\dfrac{x^{10} - a^{10}}{x-a}$.  What about  $\dfrac{x^{n} - a^{n}}{x-a}$?
\end{problem}

\begin{problem}\label{geoseriespreview}
Use your result from number \ref{monomialdiffquotex} to rewrite the sum: $1 + r + r^2 + \dots + r^{n-2} + r^{n-1}$ as a quotient. What assumptions need to be made about $r$?
\end{problem}



\end{document}
