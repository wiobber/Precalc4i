\documentclass{ximera}

\begin{document}
	\author{Stitz-Zeager}
	\xmtitle{Exercises}
\mfpicnumber{1} \opengraphsfile{ExercisesforRealZeros} % mfpic settings added 

\begin{problem}\label{prelimpolystufffirst}
For the polynomial $f(x) = x^{3} - 2x^{2} - 5x + 6$:

\begin{itemize}
\item  Use Cauchy's Bound to find an interval containing all of the real zeros.
\item  Use the Rational Zeros Theorem to make a list of possible rational zeros.
\item  Use Descartes' Rule of Signs to list the possible number of positive and negative real zeros, counting multiplicities.
\end{itemize}
\end{problem}

\begin{problem}
For the polynomial $f(x) = x^{4} + 2x^{3} - 12x^{2} - 40x - 32$:

\begin{itemize}
\item  Use Cauchy's Bound to find an interval containing all of the real zeros.
\item  Use the Rational Zeros Theorem to make a list of possible rational zeros.
\item  Use Descartes' Rule of Signs to list the possible number of positive and negative real zeros, counting multiplicities.
\end{itemize}
\end{problem}

\begin{problem}
For the polynomial $p(z) = z^{4} - 9z^{2} - 4z + 12$:

\begin{itemize}
\item  Use Cauchy's Bound to find an interval containing all of the real zeros.
\item  Use the Rational Zeros Theorem to select the possible rational zeros.
\begin{selectAll}
    \choice[correct]{$\pm 1$}
    \choice[correct]{$\pm 2$}
    \choice[correct]{$\pm 3$}
    \choice[correct]{$\pm 4$}
    \choice{$\pm 5$}
    \choice[correct]{$\pm 6$}
    \choice{$\pm 9$}
    \choice[correct]{$\pm 12$}
  \end{selectAll}
\item  Use Descartes' Rule of Signs to list the possible number of positive and negative real zeros, counting multiplicities.
\end{itemize}
\end{problem}

\begin{problem}
For the polynomial $p(z) = z^{3} + 4z^{2} - 11z + 6$:

\begin{itemize}
\item  Use Cauchy's Bound to find an interval containing all of the real zeros.
\item  Use the Rational Zeros Theorem to make a list of possible rational zeros.
\item  Use Descartes' Rule of Signs to list the possible number of positive and negative real zeros, counting multiplicities.
\end{itemize}
\end{problem}

\begin{problem}
For the polynomial $g(t) = t^{3} - 7t^{2} + t - 7$:

\begin{itemize}
\item  Use Cauchy's Bound to find an interval containing all of the real zeros.
\item  Use the Rational Zeros Theorem to make a list of possible rational zeros.
\item  Use Descartes' Rule of Signs to list the possible number of positive and negative real zeros, counting multiplicities.
\end{itemize}
\end{problem}

\begin{problem}
For the polynomial $g(t) = -2t^{3} + 19t^{2} - 49t + 20$:

\begin{itemize}
\item  Use Cauchy's Bound to find an interval containing all of the real zeros.
\item  Use the Rational Zeros Theorem to select the possible rational zeros.
\begin{selectAll}
    \choice[correct]{$\pm 1$}
    \choice[correct]{$\pm 2$}
    \choice{$\pm 3$}
    \choice[correct]{$\pm 4$}
    \choice[correct]{$\pm 5$}
    \choice[correct]{$\pm 10$}
    \choice[correct]{$\pm 20$}
    \choice[correct]{$\pm \frac{1}{2}$}
    \choice{$\pm \frac{3}{2}$} 
    \choice[correct]{$\pm \frac{5}{2}$}  
  \end{selectAll}
\item  Use Descartes' Rule of Signs to list the possible number of positive and negative real zeros, counting multiplicities.
\end{itemize}
\end{problem}

\begin{problem}
For the polynomial $f(x) = -17x^{3} + 5x^{2} + 34x - 10$:

\begin{itemize}
\item  Use Cauchy's Bound to find an interval containing all of the real zeros.
\item  Use the Rational Zeros Theorem to make a list of possible rational zeros.
\item  Use Descartes' Rule of Signs to list the possible number of positive and negative real zeros, counting multiplicities.
\end{itemize}
\end{problem}

\begin{problem}
For the polynomial $f(x) = 36x^{4} - 12x^{3} - 11x^{2} + 2x + 1$:

\begin{itemize}
\item  Use Cauchy's Bound to find an interval containing all of the real zeros.
\item  Use the Rational Zeros Theorem to make a list of possible rational zeros.
\item  Use Descartes' Rule of Signs to list the possible number of positive and negative real zeros, counting multiplicities.
\end{itemize}
\end{problem}

\begin{problem}
For the polynomial $p(z) = 3z^{3} + 3z^{2} - 11z - 10$:

\begin{itemize}
\item  Use Cauchy's Bound to find an interval containing all of the real zeros.
\item  Use the Rational Zeros Theorem to select the possible rational zeros.
\begin{selectAll}
    \choice[correct]{$\pm 1$}
    \choice[correct]{$\pm 2$}
    \choice{$\pm 3$}
    \choice{$\pm 4$}
    \choice[correct]{$\pm 5$}
    \choice[correct]{$\pm 10$}
    \choice[correct]{$\pm \frac{1}{3}$}
    \choice[correct]{$\pm \frac{2}{3}$}
    \choice{$\pm \frac{4}{3}$} 
    \choice[correct]{$\pm \frac{5}{3}$}
    \choice[correct]{$\pm \frac{5}{3}$} 
  \end{selectAll}
\item  Use Descartes' Rule of Signs to list the possible number of positive and negative real zeros, counting multiplicities.
\end{itemize}
\end{problem}

\begin{problem}\label{prelimpolystufflast}
For the polynomial $p(z) = 2z^4+z^3-7z^2-3z+3$:

\begin{itemize}
\item  Use Cauchy's Bound to find an interval containing all of the real zeros.
\item  Use the Rational Zeros Theorem to make a list of possible rational zeros.
\item  Use Descartes' Rule of Signs to list the possible number of positive and negative real zeros, counting multiplicities.
\end{itemize}
\end{problem}

\begin{problem}\label{findrealzerosexerfirst}
Find the real zeros of the polynomial using the techniques specified by your instructor.  State the multiplicity of each real zero.

$f(x) = x^{3} - 2x^{2} - 5x + 6$
\end{problem}

\begin{problem}
Find the real zeros of the polynomial using the techniques specified by your instructor.  State the multiplicity of each real zero.

$f(x) = x^{4} + 2x^{3} - 12x^{2} - 40x - 32$

\begin{solution}
$x = -2$ (mult. 3), $x = 4$ (mult. 1)   
\end{solution}
\end{problem}

\begin{problem}
Find the real zeros of the polynomial using the techniques specified by your instructor.  State the multiplicity of each real zero.

$p(z) = z^{4} - 9z^{2} - 4z + 12$
\end{problem}

\begin{problem}
Find the real zeros of the polynomial using the techniques specified by your instructor.  State the multiplicity of each real zero.

$p(z) = z^{3} + 4z^{2} - 11z + 6$
\end{problem}

\begin{problem}
Find the real zeros of the polynomial using the techniques specified by your instructor.  State the multiplicity of each real zero.

$g(t) = t^{3} - 7t^{2} + t - 7$

\begin{solution}
$t = 7$ (mult. 1) 
\end{solution}
\end{problem}

\begin{problem}
Find the real zeros of the polynomial using the techniques specified by your instructor.  State the multiplicity of each real zero.

$g(t) = -2t^{3} + 19t^{2} - 49t + 20$
\end{problem}

\begin{problem}
Find the real zeros of the polynomial using the techniques specified by your instructor.  State the multiplicity of each real zero.

$f(x) = -17x^{3} + 5x^{2} + 34x - 10$
\end{problem}

\begin{problem}
Find the real zeros of the polynomial using the techniques specified by your instructor.  State the multiplicity of each real zero.

$f(x) = 36x^{4} - 12x^{3} - 11x^{2} + 2x + 1$

\begin{solution}
$x = \frac{1}{2}$ (mult. 2), $x = -\frac{1}{3}$ (mult. 2) 
\end{solution}
\end{problem}

\begin{problem}
Find the real zeros of the polynomial using the techniques specified by your instructor.  State the multiplicity of each real zero.

$p(z) = 3z^{3} + 3z^{2} - 11z - 10$
\end{problem}

\begin{problem}
Find the real zeros of the polynomial using the techniques specified by your instructor.  State the multiplicity of each real zero.

$p(z) = 2z^4+z^3-7z^2-3z+3$
\end{problem}

\begin{problem}
Find the real zeros of the polynomial using the techniques specified by your instructor.  State the multiplicity of each real zero.

$g(t) = 9t^{3} - 5t^{2} - t$

\begin{solution}
$t = 0$, $t = \frac{5 \pm \sqrt{61}}{18}$ (each has mult. 1)
\end{solution}
\end{problem}

\begin{problem}
Find the real zeros of the polynomial using the techniques specified by your instructor.  State the multiplicity of each real zero.

$g(t) = 6t^{4} - 5t^{3} - 9t^{2}$
\end{problem}

\begin{problem}
Find the real zeros of the polynomial using the techniques specified by your instructor.  State the multiplicity of each real zero.

$f(x) = x^4+2x^2 - 15$
\end{problem}

\begin{problem}
Find the real zeros of the polynomial using the techniques specified by your instructor.  State the multiplicity of each real zero.

$f(x) = x^4-9x^2+14$

$x = \pm \sqrt{2}$, $x = \pm \sqrt{7}$ (each has mult. 1)
\end{problem}

\begin{problem}
Find the real zeros of the polynomial using the techniques specified by your instructor.  State the multiplicity of each real zero.

$p(z) = 3z^4-14z^2-5$
\end{problem}

\begin{problem}
Find the real zeros of the polynomial using the techniques specified by your instructor.  State the multiplicity of each real zero.

$p(z) = 2z^4-7z^2+6$
\end{problem}

\begin{problem}
Find the real zeros of the polynomial using the techniques specified by your instructor.  State the multiplicity of each real zero.

$g(t) = t^6-3t^3-10$

\begin{solution}
$t = \sqrt[3]{-2} = -\sqrt[3]{2}$, $t = \sqrt[3]{5}$ (each has mult. 1)
\end{solution}
\end{problem}

\begin{problem}
Find the real zeros of the polynomial using the techniques specified by your instructor.  State the multiplicity of each real zero.

$g(t) = 2t^6-9t^3+10$
\end{problem}

\begin{problem}
Find the real zeros of the polynomial using the techniques specified by your instructor.  State the multiplicity of each real zero.

$f(x) = x^5-2x^4-4x+8$
\end{problem}

\begin{problem}\label{findrealzerosexerlast}
Find the real zeros of the polynomial using the techniques specified by your instructor.  State the multiplicity of each real zero.

$f(x) = 2x^5+3x^4-18x-27$ 

$x = -\frac{3}{2}$, $x = \pm \sqrt{3}$ (each has mult. 1)
\end{problem}

\begin{problem}\label{realzeroswcalcfirst}
Use your calculator \footnote{You \textit{can} do these without your calculator, but it may test your mettle!} to help you find the real zeros of the polynomial.  State the multiplicity of each real zero.

$f(x) = x^{5} - 60x^{3} - 80x^{2} + 960x + 2304$
\end{problem}

\begin{problem}
Use your calculator \footnote{You \textit{can} do these without your calculator, but it may test your mettle!} to help you find the real zeros of the polynomial.  State the multiplicity of each real zero.

$f(x) = 25x^{5} - 105x^{4} + 174x^{3} - 142x^{2} + 57x - 9$
\end{problem}

\begin{problem}\label{realzeroswcalclast}
Use your calculator \footnote{You \textit{can} do these without your calculator, but it may test your mettle!} to help you find the real zeros of the polynomial.  State the multiplicity of each real zero.

$f(x) = 90x^{4} - 399x^{3} + 622x^{2} - 399x + 90$

\begin{solution}
$x = \frac{2}{3}$, $x = \frac{3}{2}$, $x = \frac{5}{3}$, $x = \frac{3}{5}$ (each has mult. 1)
\end{solution}
\end{problem}

\begin{problem}
Find the real zeros of $f(x) = x^{3} - \frac{1}{12}x^{2} - \frac{7}{72}x + \frac{1}{72}$ by first finding a polynomial $q(x)$ with integer coefficients such that $q(x) = N \cdot f(x)$ for some integer $N$.  (Recall that the Rational Zeros Theorem required the polynomial in question to have integer coefficients.) Show that $f$ and $q$ have the same real zeros.
\end{problem}

\begin{problem}\label{polyequexerfirst}
Find the real solutions of the polynomial equation.  (See Example \ref{polyeqineqexample}.)

$9x^{3} = 5x^{2} + x$ 
\end{problem}
  
\begin{problem}
Find the real solutions of the polynomial equation.  (See Example \ref{polyeqineqexample}.)

$9x^{2}+5x^{3}= 6x^{4}$ 

\begin{solution}
$x = 0, \frac{5 \pm \sqrt{241}}{12}$
\end{solution}
\end{problem} 

\begin{problem}
Find the real solutions of the polynomial equation.  (See Example \ref{polyeqineqexample}.)

$z^{3} + 6 = 2z^{2} + 5z$ 
\end{problem} 

\begin{problem}
Find the real solutions of the polynomial equation.  (See Example \ref{polyeqineqexample}.)

$z^{4} + 2z^{3} = 12z^{2} + 40z + 32$ 
\end{problem} 

\begin{problem}
Find the real solutions of the polynomial equation.  (See Example \ref{polyeqineqexample}.)

$t^{3} - 7t^{2} = 7-t$

\begin{solution}
$t=7$
\end{solution}
\end{problem} 

\begin{problem}
Find the real solutions of the polynomial equation.  (See Example \ref{polyeqineqexample}.)

$2t^{3} = 19t^{2} - 49t + 20$ 
\end{problem}  

\begin{problem}
Find the real solutions of the polynomial equation.  (See Example \ref{polyeqineqexample}.)

$x^{3} + x^{2} = \dfrac{11x + 10}{3}$ 
\end{problem} 

\begin{problem}
Find the real solutions of the polynomial equation.  (See Example \ref{polyeqineqexample}.)

$x^4+2x^2 = 15$

\begin{solution}
$z = \pm \sqrt{3}$
\end{solution}
\end{problem}  

\begin{problem}
Find the real solutions of the polynomial equation.  (See Example \ref{polyeqineqexample}.)

$14z^{2}+5=3z^{4}$ 
\end{problem}   

\begin{problem}\label{polyequexerlast} 
Find the real solutions of the polynomial equation.  (See Example \ref{polyeqineqexample}.)

$2z^5+3z^4 = 18z + 27$ 
\end{problem}  

\begin{problem}\label{polyinequexerfirst}
Solve the polynomial inequality and state your answer using interval notation.  (Hint: type the word ”infinity” to get the $\infty$ symbol.)

$-2x^{3} + 19x^{2} - 49x + 20 > 0$   

$\answer{(-\infty,\frac{1}{2})} \cup \answer{(4,5)}$  
\end{problem}
 
\begin{problem}
Solve the polynomial inequality and state your answer using interval notation.

$x^{4} - 9x^{2} \leq 4x - 12$
\end{problem}

\begin{problem}
Solve the polynomial inequality and state your answer using interval notation.

$(z - 1)^{2} \geq 4$
\end{problem}

\begin{problem}
Solve the polynomial inequality and state your answer using interval notation.

$4z^3 \geq 3z+1$

\begin{solution}
\item $\left\{ -\dfrac{1}{2} \right\} \cup [1, \infty)$
\end{solution} 
\end{problem}

\begin{problem}
Solve the polynomial inequality and state your answer using interval notation.

$t^4 \leq 16+4t-t^3$
\end{problem}

\begin{problem}
Solve the polynomial inequality and state your answer using interval notation.

$3t^2 + 2t < t^4$
\end{problem}

\begin{problem}
Solve the polynomial inequality and state your answer using interval notation.

$\dfrac{x^3+2 x^2}{2} < x+2$

\begin{solution}
\item $(-\infty, -2) \cup \left(-\sqrt{2}, \sqrt{2} \right)$
\end{solution}
\end{problem}

\begin{problem}
Solve the polynomial inequality and state your answer using interval notation.

$\dfrac{x^3+20x}{8} \geq x^2 + 2$
\end{problem}

\begin{problem}
Solve the polynomial inequality and state your answer using interval notation.

$2z^4>5z^2+3$
\end{problem}

\begin{problem}\label{polyinequexerlast}
Solve the polynomial inequality and state your answer using interval notation.

$z^6 + z^3 \geq 6$

\begin{solution}
$(-\infty, -\sqrt[3]{3}\,) \cup (\sqrt[3]{2}, \infty)$
\end{solution}
\end{problem} 

\begin{problem}\label{polyineqfromgraphfirst}
Use the the graph of the given polynomial function to  solve the stated inequality.

\begin{mfpic}[10]{-7}{7}{-6}{6}
\axes
\tlabel[cc](7,-0.5){\scriptsize $x$}
\tlabel[cc](0.5,6){\scriptsize $y$}
\tlabel[cc](-4.5, 0.75){\scriptsize $(-6,0)$}
\tlabel[cc](5, 0.75){\scriptsize $(6,0)$}
\tlabel[cc](1, 0.75){\scriptsize $(0,0)$}
\point[4pt]{(-6,0),(0,0),(6,0)}
\xmarks{-6 step 1 until 6}
\tiny
\tlpointsep{4pt}
\axislabels {x}{{$-6 \hspace{6pt}$} -6, {$-5 \hspace{6pt}$} -5, {$-4 \hspace{6pt}$} -4, {$-3 \hspace{6pt}$} -3, {$-2 \hspace{6pt}$} -2, {$-1 \hspace{6pt}$} -1, {$1$} 1, {$2$} 2, {$3$} 3, {$4$} 4, {$5$} 5, {$6$} 6}
\normalsize
\penwd{1.25pt}
\arrow \reverse \arrow \function{-7,7,0.1}{((x**3) - 36*x)/20}
\tcaption{$y = f(x)$}
\end{mfpic}

Solve $f(x) < 0$. 
\end{problem}

\begin{problem}
Use the the graph of the given polynomial function to  solve the stated inequality.

\begin{mfpic}[20][20]{-3}{3}{-2}{5}
\axes
\tlabel[cc](3,-0.5){\scriptsize $t$}
\tlabel[cc](0.25,5){\scriptsize $y$}
\tlabel[cc](-1.75, 0.3){\scriptsize $(-2,0)$}
\tlabel[cc](0.5, 0.3){\scriptsize $(0,0)$}
\point[4pt]{(-2,0), (0,0)}
\xmarks{-2,-1, 1, 2}
\tiny
\tlpointsep{4pt}
\axislabels {x}{{$-2 \hspace{6pt}$} -2, {$-1 \hspace{6pt}$} -1, {$1$} 1, {$2$} 2}
\normalsize
\penwd{1.25pt}
\arrow \reverse \arrow \function{-3,0.3,0.1}{x*((x + 2)**3)}
\tcaption{$y = g(t)$ }
\end{mfpic}

Solve $g(t) > 0$. 
\end{problem}

\begin{problem}
Use the the graph of the given polynomial function to  solve the stated inequality.

\begin{mfpic}[20][10]{-3}{3}{-4}{4}
\axes
\tlabel[cc](3,-0.5){\scriptsize $z$}
\tlabel[cc](0.25,4){\scriptsize $y$}
\tlabel[cc](-2, 0.75){\scriptsize $(-1,0)$}
\tlabel[cc](2, 0.75){\scriptsize $(2,0)$}
\point[4pt]{(2,0), (-1,0)}
\xmarks{-2,-1,1,2}
\tiny
\tlpointsep{4pt}
\axislabels {x}{{$-2 \hspace{6pt}$} -2, {$-1 \hspace{6pt}$} -1, {$1$} 1, {$2$} 2}
\normalsize
\penwd{1.25pt}
\arrow \reverse \arrow \function{-1.70,3.45,0.1}{(-0.4)*((x-2)**2)*(x+1)}
\tcaption{$y = p(z)$}
\end{mfpic}

Solve $p(z) \geq 0$ . 
\end{problem}  

\begin{problem}
Use the the graph of the given polynomial function to  solve the stated inequality.

\begin{mfpic}[20][10]{-2}{4}{-4}{4}
\axes
\tlabel[cc](4,-0.5){\scriptsize $x$}
\tlabel[cc](0.25,4){\scriptsize $y$}
\tlabel[cc](-0.75, 0.75){\scriptsize $\left(-\frac{1}{2},0 \right)$}
\tlabel[cc](2.5, 0.75){\scriptsize $(3,0)$}
\point[4pt]{(-0.5,0), (3,0)}
\xmarks{-1,1,2,3}
\tiny
\tlpointsep{4pt}
\axislabels {x}{ {$1$} 1, {$2$} 2, {$3$} 3}
\normalsize
\penwd{1.25pt}
\arrow \reverse \arrow \function{-1.5,3.3,0.1}{(0.5)*((x+0.5)**2)*(x-3)}
\tcaption{$y = f(x)$ }
\end{mfpic}

Solve $f(x) < 0$. 
\end{problem}  

\begin{problem}
Use the the graph of the given polynomial function to  solve the stated inequality.

\begin{mfpic}[20][10]{-3}{3}{-5}{5}
\axes
\tlabel[cc](3,-0.5){\scriptsize $s$}
\tlabel[cc](0.25,5){\scriptsize $y$}
\tlabel[cc](-2, -0.75){\scriptsize $(-2,0)$}
\tlabel[cc](0.5, 0.75){\scriptsize $(0,0)$}
\point[4pt]{(-2,0), (0,0)}
\xmarks{-2,-1,1,2}
\tiny
\tlpointsep{4pt}
\axislabels {x}{ {$1$} 1, {$2$} 2}
\normalsize
\penwd{1.25pt}
\arrow \reverse \arrow \function{-3,0.65,0.1}{0-x*((x + 2)**2)}
\tcaption{$y = F(s)$}
\end{mfpic}

Solve $F(s) \leq 0$. 
\end{problem}
   
\begin{problem}\label{polyineqfromgraphlast} 
Use the the graph of the given polynomial function to  solve the stated inequality.

\begin{mfpic}[20][10]{-3}{3}{-5}{5}
\axes
\tlabel[cc](-2, 0.75){\scriptsize $(-2,0)$}
\tlabel[cc](0.5, -0.75){\scriptsize $(0,0)$}
\tlabel[cc](3,-0.5){\scriptsize $t$}
\tlabel[cc](0.25,5){\scriptsize $y$}
\point[4pt]{(-2,0), (0,0)}
\xmarks{-2,-1,1,2}
\tiny
\tlpointsep{4pt}
\axislabels {x}{  {$2$} 2}
\normalsize
\penwd{1.25pt}
\arrow \reverse \arrow \function{-2.45,0.85,0.1}{(x**3)*((x + 2)**2)}
\tcaption{$y = G(t)$}
\end{mfpic}

Solve $G(t) \geq 0$. 
\end{problem}

\begin{problem}
Use the Intermediate Value Theorem, Theorem \ref{IVT}, to prove that $f(x) = x^{3} - 9x + 5$ has a real zero in each of the following intervals: $[-4, -3], [0, 1]$ and $[2, 3]$.  
\end{problem}

\begin{problem}
Use the concepts of End Behavior and the Intermediate Value Theorem to prove any odd-degree polynomial function with real number coefficients has at least one real zero.
\end{problem}

\begin{problem}
Find an even-degree polynomial function with real number coefficients which has no real zeros.
\end{problem}

\begin{problem}\label{bisectionexercise} 
Continue  the Bisection Method as introduced on  \pageref{bisectionmethod} to approximate the real zero of $f(x) = x^5-x-1$ to three decimal places.
\end{problem}

\begin{problem}\label{sqrt2isirrationalexercise} 
In this exercise, we prove $\sqrt{2}$ is an irrational number and approximate its value.  Let $f(x) = x^2-2$.

\begin{enumerate} 

\item Use Decartes' Rule of Signs to prove $f$ has exactly one positive real zero.

\item Use the Intermediate Value Theorem to prove $f$ has a zero in $[1,2]$.

\item\label{sqrt2isirrationalexercise}  
Use the Rational Zeros Theorem to prove $f$ has no rational zeros.

\item  Use the Bisection Method (see  \pageref{bisectionmethod}) to approximate the zero of $f$ on $[1,2]$ to three decimal places.
\end{enumerate}
\end{problem}

\begin{problem}
Generalize the argument given in Exercise \ref{sqrt2isirrationalexercise} to prove:

\begin{enumerate}

\item If $N$ is not the perfect square of an integer, then $\sqrt{N}$ is irrational.
\begin{hint}
Consider $f(x) = x^2-N$.
\end{hint}

\item  For natural numbers $n \geq 2$, if $N$ is not the perfect $n^{\text{th}}$ power of an integer, then $\sqrt[n]{N}$ is irrational.
\begin{hint}
Consider $f(x) = x^n-N$.
\end{hint}

\end{enumerate}
\end{problem}

\begin{problem}
In Example \ref{boxnotopex} in Section \ref{GraphsofPolynomials}, a box with no top is constructed from a $10$ inch $\times$ $12$ inch piece of cardboard by cutting out congruent squares from each corner of the cardboard and then folding the resulting tabs.  We determined the volume of that box (in cubic inches) is given by  the function$V(x) = 4x^3-44x^2+120x$, where $x$ denotes the length of the side of the square which is removed from each corner (in inches), $0 < x < 5$.  Solve the inequality $V(x) \geq 80$ analytically and interpret your answer in the context of that example. 
\end{problem}

\begin{problem}
From Exercise \ref{newportaboycost} in Section \ref{GraphsofPolynomials}, $C(x) = .03x^{3} - 4.5x^{2} + 225x + 250$, for $x \geq 0$ models the cost, in dollars, to produce $x$ PortaBoy game systems. If the production budget is $\$5000$, find the number of game systems which can be produced and still remain under budget.
\end{problem}

\begin{problem}
Let $f(x) = 5x^{7} - 33x^{6} + 3x^{5} - 71x^{4} - 597x^{3} + 2097x^{2} - 1971x + 567$.  With the help of your classmates, find the $x$- and $y$- intercepts of the graph of $f$.  Find the intervals on which the function is increasing, the intervals on which it is decreasing and the local extrema.  Sketch the graph of $f$, using more than one picture if necessary to show all of the important features of the graph.  
\end{problem}

\begin{problem}
With the help of your classmates, create a list of five polynomials with different degrees whose real zeros cannot be found using any of the techniques in this section.
\end{problem}

\end{document}
