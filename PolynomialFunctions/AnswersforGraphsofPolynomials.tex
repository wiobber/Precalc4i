\documentclass{ximera}

\begin{document}
	\author{Stitz-Zeager}
	\xmtitle{Answers}
\mfpicnumber{1} \opengraphsfile{ExercisesforGraphsofPolynomials} % mfpic settings added 




\subsection{Answers}

%\begin{multicols}{2}
\begin{enumerate}


\item $F(x) = (x + 2)^{3} + 1$ \\ 
domain: $(-\infty, \infty)$ \\ 
range: $(-\infty, \infty)$ \\

\begin{mfpic}[20][8]{-5}{1}{-11}{13}
\axes
\tlabel[cc](1,-0.75){\scriptsize $x$}
\tlabel[cc](0.5,13){\scriptsize $y$}
\point[4pt]{(-4,-7),(-3,0),(-2,1),(-1,2),(0,9)}
\xmarks{-4,-3,-2,-1}
\ymarks{-10 step 1 until 12}
\tiny
\tlpointsep{4pt}
\axislabels {x}{{$-4 \hspace{6pt}$} -4, {$-3 \hspace{6pt}$} -3, {$-2 \hspace{6pt}$} -2, {$-1 \hspace{6pt}$} -1}
\axislabels {y}{{$-10$} -10, {$-9$} -9, {$-8$} -8, {$-7$} -7, {$-6$} -6, {$-5$} -5, {$-4$} -4, {$-3$} -3, {$-2$} -2, {$-1$} -1, {$1$} 1, {$2$} 2, {$3$} 3, {$4$} 4, {$5$} 5, {$6$} 6, {$7$} 7, {$8$} 8, {$9$} 9, {$10$} 10, {$11$} 11, {$12$} 12}
\normalsize
\penwd{1.25pt}
\arrow \reverse \arrow \function{-4.25,0.25,0.1}{((x + 2)**3) + 1}
\end{mfpic}

\vfill

%\columnbreak

\item $F(x) = (x + 2)^{4} + 1$\\
domain: $(-\infty, \infty)$\\
range: $[1, \infty)$\\

\begin{mfpic}[20][8]{-5}{1}{-1}{22}
\axes
\tlabel[cc](1,-0.75){\scriptsize $x$}
\tlabel[cc](0.5,22){\scriptsize $y$}
\point[4pt]{(-4,17),(-3,2),(-2,1),(-1,2),(0,17)}
\xmarks{-4,-3,-2,-1}
\ymarks{1 step 1 until 21}
\tiny
\tlpointsep{4pt}
\axislabels {x}{{$-4 \hspace{6pt}$} -4, {$-3 \hspace{6pt}$} -3, {$-2 \hspace{6pt}$} -2, {$-1 \hspace{6pt}$} -1}
\axislabels {y}{{$1$} 1, {$2$} 2, {$3$} 3, {$4$} 4, {$5$} 5, {$6$} 6, {$7$} 7, {$8$} 8, {$9$} 9, {$10$} 10, {$11$} 11, {$12$} 12, {$13$} 13, {$14$} 14, {$15$} 15, {$16$} 16, {$17$} 17, {$18$} 18, {$19$} 19, {$20$} 20, {$21$} 21}
\normalsize
\penwd{1.25pt}
\arrow \reverse \arrow \function{-4.12,0.12,0.1}{((x + 2)**4) + 1}
\end{mfpic}

\setcounter{HW}{\value{enumi}}
\end{enumerate}
%\end{multicols}

%\begin{multicols}{2}
\begin{enumerate}
\setcounter{enumi}{\value{HW}}


\item $F(x) = 2 - 3(x - 1)^{4}$\\
domain: $(-\infty, \infty)$\\
range: $(-\infty, 2]$\\

\begin{mfpic}[20][8]{-1}{3}{-14}{3}
\axes
\tlabel[cc](3,-0.75){\scriptsize $x$}
\tlabel[cc](0.5,3){\scriptsize $y$}
\point[4pt]{(1,2),(0,-1),(2,-1)}
\xmarks{1,2}
\ymarks{-13 step 1 until 2}
\tiny
\tlpointsep{4pt}
\axislabels {x}{{$1$} 1, {$2$} 2}
\axislabels {y}{{$-13$} -13, {$-12$} -12, {$-11$} -11, {$-10$} -10, {$-9$} -9, {$-8$} -8, {$-7$} -7, {$-6$} -6, {$-5$} -5, {$-4$} -4, {$-3$} -3, {$-2$} -2, {$-1$} -1, {$1$} 1, {$2$} 2}
\normalsize
\penwd{1.25pt}
\arrow \reverse \arrow \function{-0.5,2.5,0.1}{2 - 3*((x - 1)**4)}
\end{mfpic}


\vfill

%\columnbreak

\item $F(x) = -x^{5} - 3$\\
domain: $(-\infty, \infty)$\\
range: $(-\infty, \infty)$\\

\begin{mfpic}[20][8]{-2}{2}{-11}{11}
\axes
\tlabel[cc](2,-0.75){\scriptsize $x$}
\tlabel[cc](0.5,11){\scriptsize $y$}
\point[4pt]{(-1,-2),(0,-3),(1,-4)}
\xmarks{-1,1}
\ymarks{-10 step 1 until 10}
\tiny
\tlpointsep{4pt}
\axislabels {x}{{$-1 \hspace{6pt}$} -1, {$1$} 1}
\axislabels {y}{{$-10$} -10, {$-9$} -9, {$-8$} -8, {$-7$} -7, {$-6$} -6, {$-5$} -5, {$-4$} -4, {$-3$} -3, {$-2$} -2, {$-1$} -1, {$1$} 1, {$2$} 2, {$3$} 3, {$4$} 4, {$5$} 5, {$6$} 6, {$7$} 7, {$8$} 8, {$9$} 9, {$10$} 10}
\normalsize
\penwd{1.25pt}
\arrow \reverse \arrow \function{-1.68,1.5,0.1}{-(x**5) - 3}
\end{mfpic}


\setcounter{HW}{\value{enumi}}
\end{enumerate}
%\end{multicols}


\pagebreak

%\begin{multicols}{2}
\begin{enumerate}

\setcounter{enumi}{\value{HW}}
\item $F(x) = (x+1)^5+10$\\
domain: $(-\infty, \infty)$\\
range: $(-\infty, \infty)$\\

\begin{mfpic}[20][8]{-5}{1}{-1}{22}
\axes
\tlabel[cc](1,-0.75){\scriptsize $x$}
\tlabel[cc](0.5,22){\scriptsize $y$}
\point[4pt]{(0,11), (-1,10), (-2,9)}
\xmarks{-4,-3,-2,-1}
\ymarks{1 step 1 until 21}
\tiny
\tlpointsep{4pt}
\axislabels {x}{{$-4 \hspace{6pt}$} -4, {$-3 \hspace{6pt}$} -3, {$-2 \hspace{6pt}$} -2, {$-1 \hspace{6pt}$} -1}
\axislabels {y}{{$1$} 1, {$2$} 2, {$3$} 3, {$4$} 4, {$5$} 5, {$6$} 6, {$7$} 7, {$8$} 8, {$9$} 9, {$10$} 10, {$11$} 11, {$12$} 12, {$13$} 13, {$14$} 14, {$15$} 15, {$16$} 16, {$17$} 17, {$18$} 18, {$19$} 19, {$20$} 20, {$21$} 21}
\normalsize
\penwd{1.25pt}
\arrow \reverse \arrow \function{-2.64,0.58,0.1}{((x + 1)**5) + 10}
\end{mfpic}


\vfill

%\columnbreak

\item $F(x) = 8-x^{6}$\\
domain: $(-\infty, \infty)$\\
range: $(-\infty, 8]$\\

\begin{mfpic}[20][8]{-2}{2}{-11}{11}
\axes
\tlabel[cc](2,-0.75){\scriptsize $x$}
\tlabel[cc](0.5,11){\scriptsize $y$}
\point[4pt]{(-1,7),(0,8),(1,7)}
\xmarks{-1,1}
\ymarks{-10 step 1 until 10}
\tiny
\tlpointsep{4pt}
\axislabels {x}{{$-1 \hspace{6pt}$} -1, {$1$} 1}
\axislabels {y}{{$-10$} -10, {$-9$} -9, {$-8$} -8, {$-7$} -7, {$-6$} -6, {$-5$} -5, {$-4$} -4, {$-3$} -3, {$-2$} -2, {$-1$} -1, {$1$} 1, {$2$} 2, {$3$} 3, {$4$} 4, {$5$} 5, {$6$} 6, {$7$} 7, {$8$} 8, {$9$} 9, {$10$} 10}
\normalsize
\penwd{1.25pt}
\arrow \reverse \arrow \function{-1.6,1.6,0.1}{8-(x**6)}
\end{mfpic}


\setcounter{HW}{\value{enumi}}
\end{enumerate}
%\end{multicols}


%\begin{multicols}{2}
\begin{enumerate}
\setcounter{enumi}{\value{HW}}

\item  $F(x) = (x-1)^3-2$ \vphantom{$F(x) = -\frac{1}{2} (x+2)^3+3$}
\item $F(x) = -\frac{1}{2} (x+2)^3+3$

\setcounter{HW}{\value{enumi}}
\end{enumerate}
%\end{multicols}

%\begin{multicols}{2}
\begin{enumerate}
\setcounter{enumi}{\value{HW}}

\item  $F(x) = 2(x+1)^4-4$ \
\item $F(x) = -0.15625x^4+2.5$

\setcounter{HW}{\value{enumi}}
\end{enumerate}
%\end{multicols}


%\begin{multicols}{2}
\begin{enumerate}
\setcounter{enumi}{\value{HW}}
\item $f(x) = 4-x-3x^2$ \\
Degree 2 \\
Leading term $-3x^{2}$\\
Leading coefficient $-3$\\
Constant term $4$\\
$\ds{\lim_{x \rightarrow - \infty} f(x)  = -\infty}$ \\
$\ds{\lim_{x \rightarrow  \infty} f(x)  = -\infty}$ \\


\item  $g(x) = 3x^5 - 2x^2 + x + 1$ \\
Degree 5 \\
Leading term $3x^5$\\
Leading coefficient $3$\\
Constant term $1$\\
$\ds{\lim_{x \rightarrow - \infty} g(x)  = -\infty}$ \\
$\ds{\lim_{x \rightarrow \infty} g(x)  = \infty}$ \\



\setcounter{HW}{\value{enumi}}
\end{enumerate}
%\end{multicols}

%\begin{multicols}{2}
\begin{enumerate}
\setcounter{enumi}{\value{HW}}

\item $q(r) = 1 - 16r^{4}$\\
Degree 4 \\
Leading term $-16r^{4}$\\
Leading coefficient $-16$\\
Constant term $1$\\
$\ds{\lim_{r \rightarrow - \infty} q(r)  = -\infty}$ \\
$\ds{\lim_{r \rightarrow \infty} q(r)  = -\infty}$ \\


\item $Z(b) = 42b - b^{3}$\\
Degree 3 \\
Leading term $-b^{3}$\\
Leading coefficient $-1$\\
Constant term $0$\\
$\ds{\lim_{b \rightarrow - \infty} Z(b)  = \infty}$ \\
$\ds{\lim_{b \rightarrow  \infty} Z(b)  =  - \infty}$ \\

\setcounter{HW}{\value{enumi}}
\end{enumerate}
%\end{multicols}

%\begin{multicols}{2}
\begin{enumerate}
\setcounter{enumi}{\value{HW}}

\item $f(x) = \sqrt{3}x^{17} + 22.5x^{10} - \pi x^{7} + \frac{1}{3}$\\
Degree 17 \\
Leading term $\sqrt{3}x^{17}$\\
Leading coefficient $\sqrt{3}$\\
Constant term $\frac{1}{3}$\\
$\ds{\lim_{x \rightarrow - \infty} f(x)  = -\infty}$ \\
$\ds{\lim_{x \rightarrow  \infty} f(x)  = \infty}$ \\



\item $s(t) = -4.9t^{2} + v_{\mbox{\tiny $0$}}t + s_{\mbox{\tiny $0$}}$\\
Degree 2 \\
Leading term $-4.9t^{2}$\\
Leading coefficient $-4.9$\\
Constant term $s_{\mbox{\tiny $0$}}$\\
$\ds{\lim_{t \rightarrow - \infty} s(t)  = -\infty}$ \\
$\ds{\lim_{t \rightarrow  \infty} s(t)  = -\infty}$ \\



\setcounter{HW}{\value{enumi}}
\end{enumerate}
%\end{multicols}

%\begin{multicols}{2}
\begin{enumerate}
\setcounter{enumi}{\value{HW}}


\item $P(x) = (x - 1)(x - 2)(x - 3)(x - 4)$\\
Degree 4 \\
Leading term $x^{4}$\\
Leading coefficient $1$\\
Constant term $24$\\
$\ds{\lim_{x \rightarrow - \infty} P(x)  = \infty}$ \\
$\ds{\lim_{x \rightarrow  \infty} P(x)  = -\infty}$ \\


\item $p(t) = -t^2(3 - 5t)(t^{2} + t + 4)$\\
Degree 5 \\
Leading term $5t^{5}$\\
Leading coefficient $5$\\
Constant term $0$\\
$\ds{\lim_{t \rightarrow - \infty} p(t)  = -\infty}$ \\
$\ds{\lim_{t \rightarrow  \infty} p(t)  = \infty}$ \\

\setcounter{HW}{\value{enumi}}
\end{enumerate}
%\end{multicols}



%\begin{multicols}{2}
\begin{enumerate}
\setcounter{enumi}{\value{HW}}

\item $f(x) = -2x^3(x+1)(x+2)^2$ \\
Degree 6 \\
Leading term $-2x^{6}$\\
Leading coefficient $-2$\\
Constant term $0$\\
$\ds{\lim_{x \rightarrow - \infty} f(x)  = -\infty}$ \\
$\ds{\lim_{x \rightarrow  \infty} f(x)  = -\infty}$ \\

\item $G(t) = 4(t-2)^2\left(t+\frac{1}{2}\right)$ \\
Degree 3 \\
Leading term $4t^3$\\
Leading coefficient $4$\\
Constant term $8$\\
$\ds{\lim_{t \rightarrow - \infty} G(t)  = -\infty}$ \\
$\ds{\lim_{t \rightarrow  \infty} G(t)  = \infty}$ \\


\setcounter{HW}{\value{enumi}}
\end{enumerate}
%\end{multicols}

%\begin{multicols}{2}
\begin{enumerate}
\setcounter{enumi}{\value{HW}}

\item $a(x) = x(x + 2)^{2}$\\
$x = 0$ multiplicity 1\\
$x = -2$ multiplicity 2\\

\begin{mfpic}[20][10]{-3}{1}{-3}{5}
\axes
\tlabel[cc](1,-0.5){\scriptsize $x$}
\tlabel[cc](0.25,5){\scriptsize $y$}
\point[4pt]{(-2,0), (0,0)}
\xmarks{-2,-1}
\tiny
\tlpointsep{4pt}
\axislabels {x}{{$-2 \hspace{6pt}$} -2, {$-1 \hspace{6pt}$} -1}
\normalsize
\penwd{1.25pt}
\arrow \reverse \arrow \function{-3,0.65,0.1}{x*((x + 2)**2)}
\end{mfpic}

\vfill

%\columnbreak

\item $g(t) = t(t + 2)^{3}$\\
$t = 0$ multiplicity 1\\
$t = -2$ multiplicity 3\\

\begin{mfpic}[20][20]{-3}{1}{-2}{5}
\axes
\tlabel[cc](1,-0.5){\scriptsize $t$}
\tlabel[cc](0.25,5){\scriptsize $y$}
\point[4pt]{(-2,0), (0,0)}
\xmarks{-2,-1}
\tiny
\tlpointsep{4pt}
\axislabels {x}{{$-2 \hspace{6pt}$} -2, {$-1 \hspace{6pt}$} -1}
\normalsize
\penwd{1.25pt}
\arrow \reverse \arrow \function{-3,0.3,0.1}{x*((x + 2)**3)}
\end{mfpic}


\setcounter{HW}{\value{enumi}}
\end{enumerate}
%\end{multicols}

%\begin{multicols}{2}
\begin{enumerate}
\setcounter{enumi}{\value{HW}}

\item $f(z) = -2(z-2)^2(z+1)$\\
$z=2$ multiplicity 2 \\
$z=-1$ multiplicity 1\\

\begin{mfpic}[20][10]{-3}{3}{-4}{4}
\axes
\tlabel[cc](3,-0.5){\scriptsize $z$}
\tlabel[cc](0.25,4){\scriptsize $y$}
\point[4pt]{(2,0), (-1,0)}
\xmarks{-2,-1,1,2}
\tiny
\tlpointsep{4pt}
\axislabels {x}{{$-2 \hspace{6pt}$} -2, {$-1 \hspace{6pt}$} -1, {$1$} 1, {$2$} 2}
\normalsize
\penwd{1.25pt}
\arrow \reverse \arrow \function{-1.70,3.45,0.1}{(-0.4)*((x-2)**2)*(x+1)}
\end{mfpic}



\item $g(x) = (2x+1)^2(x-3)$\\
$x=-\frac{1}{2}$ multiplicity 2 \\
$x=3$ multiplicity 1\\

\begin{mfpic}[20][10]{-2}{4}{-4}{4}
\axes
\tlabel[cc](4,-0.5){\scriptsize $x$}
\tlabel[cc](0.25,4){\scriptsize $y$}
\point[4pt]{(-0.5,0), (3,0)}
\xmarks{-1,1,2,3}
\tiny
\tlpointsep{4pt}
\axislabels {x}{{$-1 \hspace{6pt}$} -1, {$1$} 1, {$2$} 2, {$3$} 3}
\normalsize
\penwd{1.25pt}
\arrow \reverse \arrow \function{-1.5,3.3,0.1}{(0.5)*((x+0.5)**2)*(x-3)}
\end{mfpic}



\setcounter{HW}{\value{enumi}}
\end{enumerate}
%\end{multicols}



%\begin{multicols}{2}
\begin{enumerate}
\setcounter{enumi}{\value{HW}}

\item $F(t) = t^{3}(t + 2)^{2}$\\
$t = 0$ multiplicity 3\\
$t = -2$ multiplicity 2\\

\begin{mfpic}[20][10]{-3}{1}{-3}{5}
\axes
\tlabel[cc](1,-0.5){\scriptsize $t$}
\tlabel[cc](0.25,5){\scriptsize $y$}
\point[4pt]{(-2,0), (0,0)}
\xmarks{-2,-1}
\tiny
\tlpointsep{4pt}
\axislabels {x}{{$-2 \hspace{6pt}$} -2, {$-1 \hspace{6pt}$} -1}
\normalsize
\penwd{1.25pt}
\arrow \reverse \arrow \function{-2.45,0.85,0.1}{(x**3)*((x + 2)**2)}
\end{mfpic}

\vfill

%\columnbreak

\item $P(z) = (z - 1)(z - 2)(z - 3)(z - 4)$\\
$z = 1$ multiplicity 1\\
$z = 2$ multiplicity 1\\
$z = 3$ multiplicity 1\\
$z = 4$ multiplicity 1\\

\begin{mfpic}[20][10]{0}{5}{-1}{5}
\axes
\tlabel[cc](5,-0.5){\scriptsize $z$}
\tlabel[cc](0.25,5){\scriptsize $y$}
\point[4pt]{(1,0),(2,0),(3,0),(4,0)}
\xmarks{1,2,3,4}
\tiny
\tlpointsep{4pt}
\axislabels {x}{{$1$} 1, {$2$} 2, {$3$} 3, {$4$} 4}
\normalsize
\penwd{1.25pt}
\arrow \reverse \arrow \function{0.6,4.4,0.1}{(x - 1)*(x - 2)*(x - 3)*(x - 4)}
\end{mfpic}

\setcounter{HW}{\value{enumi}}
\end{enumerate}
%\end{multicols}


%\begin{multicols}{2}
\begin{enumerate}
\setcounter{enumi}{\value{HW}}


\item $Q(x) = (x + 5)^{2}(x - 3)^{4}$\\
$x = -5$ multiplicity 2\\
$x = 3$ multiplicity 4\\

\begin{mfpic}[10][20]{-6}{6}{-1}{3}
\axes
\tlabel[cc](6,-0.5){\scriptsize $x$}
\tlabel[cc](0.5,3){\scriptsize $y$}
\point[4pt]{(-5,0),(3,0)}
\xmarks{-5 step 1 until 5}
\tiny
\tlpointsep{4pt}
\axislabels {x}{{$-5 \hspace{6pt}$} -5, {$-4 \hspace{6pt}$} -4, {$-3 \hspace{6pt}$} -3, {$-2 \hspace{6pt}$} -2, {$-1 \hspace{6pt}$} -1, {$1$} 1, {$2$} 2, {$3$} 3, {$4$} 4, {$5$} 5}
\normalsize
\penwd{1.25pt}
\arrow \reverse \arrow \function{-5.9,5.6,0.1}{(((x + 5)**2)*((x - 3)**4))/2000}
\end{mfpic}

\vfill

%\columnbreak

\item $f(t) = t^2(t-2)^2(t+2)^2$\\
$t = -2$ multiplicity 2\\
$t = 0$ multiplicity 2\\
$t = 2$ multiplicity 2\\

\begin{mfpic}[20][10]{-3}{3}{-1}{5}
\axes
\tlabel[cc](3,-0.5){\scriptsize $t$}
\tlabel[cc](0.5,5){\scriptsize $y$}
\point[4pt]{(-2,0), (0,0), (2,0)}
\xmarks{-2 step 1 until 2}
\tiny
\tlpointsep{4pt}
\axislabels {x}{{$-2 \hspace{6pt}$} -2, {$-1 \hspace{6pt}$} -1, {$1$} 1, {$2$} 2}
\normalsize
\penwd{1.25pt}
\arrow \reverse \arrow \function{-2.45,2.45,0.1}{(0.2)*(x**2)*((x-2)**2)*((x+2)**2)}
\end{mfpic}

\setcounter{HW}{\value{enumi}}
\end{enumerate}
%\end{multicols}

%\begin{multicols}{2}
\begin{enumerate}
\setcounter{enumi}{\value{HW}}

\item $H(z) = (3-z)\left(z^2+1\right)$\\
$z=3$ multiplicity 1\\

\begin{mfpic}[20][10]{-1}{4}{-4}{4}
\axes
\tlabel[cc](4,-0.5){\scriptsize $z$}
\tlabel[cc](0.5,3){\scriptsize $y$}
\point[4pt]{(3,0)}
\xmarks{1 step 1 until 3}
\tiny
\tlpointsep{4pt}
\axislabels {x}{{$1$} 1, {$2$} 2, {$3$} 3}
\normalsize
\penwd{1.25pt}
\arrow \reverse \arrow \function{-0.75,3.3,0.1}{(0.5)*(3-x)*((x**2)+1)}
\end{mfpic}

\vfill

%\columnbreak

\item $Z(x) = x(42 - x^{2})$\\
$x = -\sqrt{42}$  multiplicity 1\\
$x = 0$ multiplicity 1\\
$x = \sqrt{42}$ multiplicity 1\\

\begin{mfpic}[10]{-7}{7}{-6}{6}
\axes
\tlabel[cc](7,-0.5){\scriptsize $x$}
\tlabel[cc](0.5,6){\scriptsize $y$}
\point[4pt]{(-6.4807,0),(0,0),(6.4807,0)}
\xmarks{-6 step 1 until 6}
\tiny
\tlpointsep{4pt}
\axislabels {x}{{$-6 \hspace{6pt}$} -6, {$-5 \hspace{6pt}$} -5, {$-4 \hspace{6pt}$} -4, {$-3 \hspace{6pt}$} -3, {$-2 \hspace{6pt}$} -2, {$-1 \hspace{6pt}$} -1, {$1$} 1, {$2$} 2, {$3$} 3, {$4$} 4, {$5$} 5, {$6$} 6}
\normalsize
\penwd{1.25pt}
\arrow \reverse \arrow \function{-7,7,0.1}{(42*x - x**3)/20}
\end{mfpic}

\setcounter{HW}{\value{enumi}}
\end{enumerate}
%\end{multicols}

%\begin{multicols}{3}
\begin{enumerate}
\setcounter{enumi}{\value{HW}}

\item odd
\item neither
\item even

\setcounter{HW}{\value{enumi}}
\end{enumerate}
%\end{multicols}

%\begin{multicols}{3}
\begin{enumerate}
\setcounter{enumi}{\value{HW}}

\item even
\item even
\item neither

\setcounter{HW}{\value{enumi}}
\end{enumerate}
%\end{multicols}


%\begin{multicols}{3}
\begin{enumerate}
\setcounter{enumi}{\value{HW}}

\item odd
\item odd
\item even

\setcounter{HW}{\value{enumi}}
\end{enumerate}
%\end{multicols}

%\begin{multicols}{3}
\begin{enumerate}
\setcounter{enumi}{\value{HW}}

\item odd
\item neither
\item even

\setcounter{HW}{\value{enumi}}
\end{enumerate}
%\end{multicols}

%\begin{multicols}{3}
\begin{enumerate}
\setcounter{enumi}{\value{HW}}

\item odd
\item even
\item even  \textbf{and} odd

\setcounter{HW}{\value{enumi}}
\end{enumerate}
%\end{multicols}

\begin{enumerate}
\setcounter{enumi}{\value{HW}}

\addtocounter{enumi}{1}

\item  %\begin{multicols}{4}
\begin{enumerate}

\item even

\item  odd

\item  neither

\item  odd\footnote{You need to first multiply out the expression for $g(x)$ so it is in the form prescribed by Definition \ref{polynomialfunction}.}

\end{enumerate}

%\end{multicols}

\item For $f(x) = |x|$, $f(-x) = |-x| = |(-1) x|  = |-1| |x| = (1) |x| = |x|$.  Hence, $f(-x) = f(x)$.

\item  $V(x) = x(8.5-2x)(11-2x) = 4x^3-39x^2+93.5x$, $0 < x < 4.25$.  Volume is maximized when $x \approx 1.58$, so we get the dimensions of the box with maximum volume are: height $\approx$ 1.58 inches, width $\approx$ 5.34 inches, and depth $\approx$ 7.84 inches.  The maximum volume is $\approx$ 66.15 cubic inches.

\item  Each of these average rates of change indicate slope of the curve over the given interval.  Smaller slopes correspond to `flatter' curves and higher slopes correspond to `steeper' curves.

\[ \begin{array}{|r||c|c|c|c|c|c|}  \hline

 f(x) &  [-0.1, 0] & [0, 0.1] &[0.9, 1] & [1, 1.1] & [1.9, 2] & [2, 2.1]  \\ \hline
 1 &  0 &   0  & 0   & 0   & 0  & 0 \\  \hline
 x &  1 &   1  & 1   & 1   & 1  & 1 \\  \hline
 x^2 & -0.1 & 0.1 & 1.9 & 2.1 & 3.9 & 4.1  \\  \hline
 x^3 & 0.01  & 0.01 & 2.71 & 3.31 & 11.41 & 12.61 \\  \hline
 x^4&  -0.001 & 0.001 & 3.439 & 4.641 & 29.679 & 34.481 \\ \hline
 x^5 & 0.0001 & 0.0001 & 4.0951 & 6.1051 & 72.3901 & 88.4101 \\ \hline

\end{array} \]


\item As we sample points closer to $x=1$, the slope of the curve approaches the exponent on $x$.

\[ \begin{array}{|r||c|c|c|c|}  \hline

 f(x) &  [0.9, 1.1] & [0.99, 1.01] &[0.999, 1.001] & [0.9999, 1.0001]  \\ \hline
 1 &  0 &   0  & 0   & 0   \\  \hline
 x &  1 &   1  & 1   & 1   \\  \hline
 x^2 & 2 & 2 & 2 & 2  \\  \hline
 x^3 & 3.01  & 3.0001 & \approx 3 & \approx 3  \\  \hline
 x^4&  4.04 & 4.0004 & \approx 4 & \approx 4 \\ \hline
 x^5 & 5.1001 & \approx 5.001 & \approx 5 & \approx 5  \\ \hline

\end{array} \]


\item The calculator gives the location  of the absolute maximum (rounded to three decimal places) as $x \approx 6.305$ and $y \approx 1115.417$.  Since $x$ represents the number of TVs sold in hundreds, $x = 6.305$ corresponds to $630.5$ TVs.  Since we can't sell half of a TV, we compare $R(6.30) \approx 1115.415$ and $R(6.31) \approx 1115.416$, so selling $631$ TVs results in a (slightly) higher revenue.  Since $y$ represents the revenue in \textit{thousands} of dollars, the maximum revenue is $\$ 1,\!115,\!416$.

\item $P(x) = R(x) - C(x) = -5x^3+35x^2-45x-25$, $0 \leq x \leq 10.07$.

\item  The calculator gives the location  of the absolute maximum (rounded to three decimal places) as $x \approx 3.897$ and $y \approx 35.255$.  Since $x$ represents the number of TVs sold in hundreds, $x = 3.897$ corresponds to $389.7$ TVs.  Since we can't sell $0.7$ of a TV, we compare $P(3.89) \approx 35.254$ and $P(3.90) \approx 35.255$, so selling $390$ TVs results in a (slightly) higher revenue.  Since $y$ represents the revenue in \textit{thousands} of dollars, the maximum revenue is $\$ 35,\!255$.

\item Making and selling 71 PortaBoys yields a maximized profit of \$5910.67.


\item \begin{enumerate}

\item To maximize the volume, we assume we start with the maximum Length $+$ Girth of $130$,  so the length is $130 - 4x$.  The volume of a rectangular box is  `length $\times$ width $\times$ height' so we get $V(x) = x^{2}(130 - 4x) = -4x^{3} + 130x^{2}$.  

\item Using a graphing utility, we get a (local) maximum of  $y = V(x)$ at $(21.67, 20342.59)$.  Hence, the maximum volume is $20342.59\mbox{in.}^{3}$ using a box with dimensions $21.67\mbox{in.} \times 21.67\mbox{in.} \times 43.32\mbox{in.}$.

\item If we start with Length $+$ Girth $= 108$ then the length is $108 - 4x$ so  $V(x) = -4x^{3} + 108x^{2}$.  Graphing $y = V(x)$  shows a (local) maximum at $(18.00, 11664.00)$ so the dimensions of the box with maximum volume are $18.00\mbox{in.} \times 18.00\mbox{in.} \times 36\mbox{in.}$ for a volume of $11664.00\mbox{in.}^{3}$.  (Calculus will confirm that the measurements which maximize the volume are \underline{exactly} 18in. by 18in. by 36in., however, as I'm sure you are aware by now, we treat all numerical results as approximations and list them as such.)

\end{enumerate}

\setcounter{HW}{\value{enumi}}
\end{enumerate}




\begin{enumerate}
\setcounter{enumi}{\value{HW}}


\item \begin{itemize} \item The cubic regression model is $p_{\mbox{\tiny $3$}}(x) = 0.0226x^{3} - 0.9508x^{2} + 8.615x - 3.446$.  It has $R^{2} = 0.9377$ which isn't bad.  The graph of $y = p_{\mbox{\tiny $3$}}(x)$ along with the data is shown below on the left.  Note $p_{\mbox{\tiny $3$}}$ hits the $x$-axis at about $x = 12.45$ making this a bad model for future predictions.  

\item To use the model to approximate the number of hours of sunlight on your birthday, you'll have to figure out what decimal value of $x$ is close enough to your birthday and then plug it into the model.  Jeff's birthday is July 31 which is 10 days after July 21 ($x = 7$).  Assuming 30 days in a month, I think $x = 7.33$ should work for my birthday and $p_{\mbox{\tiny $3$}}(7.33) \approx 17.5$.  The website says there will be about $18.25$ hours of daylight that day. 

\item  To have 14 hours of darkness we need 10 hours of daylight.  We see that $p_{\mbox{\tiny $3$}}(1.96) \approx 10$ and $p_{\mbox{\tiny $3$}}(10.05) \approx 10$ so it seems reasonable to say that we'll have at least 14 hours of darkness from December 21, 2008 ($x = 0$) to February 21, 2009 ($x = 2$) and then again from October 21,2009 ($x = 10$) to December 21, 2009 ($x = 12$).

\end{itemize}

\begin{itemize}

\item The quartic regression model is $p_{\mbox{\tiny $4$}}(x) = 0.0144x^{4} - 0.3507x^{3} + 2.259x^{2} - 1.571x + 5.513$.  It has $R^{2} = 0.9859$ which is good.  The graph of $y = p_{\mbox{\tiny $4$}}(x)$  along with data is shown below on the right.  Note  $p_{\mbox{\tiny $4$}}(15)$ is above $24$ making this a bad model as well for future predictions.  

\item Here, $p_{\mbox{\tiny $4$}}(7.33) \approx 18.71$ so this model more accurately predicts the number of hours of daylight on Jeff's birthday.  

\item This model says we'll have at least 14 hours of darkness from December 21, 2008 ($x = 0$) to about March 1, 2009 ($x = 2.30$) and then again from October 10, 2009 ($x = 9.667$) to December 21, 2009 ($x = 12$).

\end{itemize}

\begin{center}

\begin{tabular}{cc}

\includegraphics[width=2.5in]{./GraphsofPolynomialsGraphics/DaylightRegCubic.jpg} \hspace{.25in} & \includegraphics[width=2.5in]{./GraphsofPolynomialsGraphics/DaylightRegQuartic.jpg} \\

$y = p_{\mbox{\tiny $3$}}(x)$ \hspace{.25in} & $y = p_{\mbox{\tiny $4$}}(x)$ \\

\end{tabular}

\end{center}

\newpage

\item \begin{enumerate}

\item The scatter plot is shown below with each of the three regression models.

\item The quadratic model is $P_{\mbox{\tiny $2$}}(x) = -0.021x^{2} + 0.241x + 0.956$, $R^{2} = 0.7771$. \\
The cubic model is $P_{\mbox{\tiny $3$}}(x) = 0.005x^{3} - 0.103x^{2} + 0.602x + 0.573$,  $R^{2} = 0.9815$. \\
The quartic model is $P_{\mbox{\tiny $4$}}(x) = -0.000969x^{4} + 0.0253x^{3} - 0.240x^{2} + 0.944x + 0.330$,  $R^{2} = 0.9993$.

\item The models give maximums: $P_{\mbox{\tiny $2$}}(5.737) \approx 1.648$, $P_{\mbox{\tiny $3$}}(4.232) \approx 1.657$ and $P_{\mbox{\tiny $4$}}(3.784) \approx 1.630$.
\end{enumerate}



\hspace{-.1in} \begin{tabular}{ccc}

\includegraphics[width=1.8in]{./GraphsofPolynomialsGraphics/CircuitRegQuadratic.jpg} \hspace{.1in} &
\includegraphics[width=1.8in]{./GraphsofPolynomialsGraphics/CircuitRegCubic.jpg} \hspace{.1in} &
\includegraphics[width=1.8in]{./GraphsofPolynomialsGraphics/CircuitRegQuartic.jpg} \\

$y = P_{\mbox{\tiny $2$}}(x)$ \hspace{.1in} & $y = P_{\mbox{\tiny $3$}}(x)$ & $y = P_{\mbox{\tiny $4$}}(x)$\\

\end{tabular}

\item \begin{enumerate}

\item   $\ds{\lim_{x \rightarrow - \infty} p(x)  = -\infty}$  and $\ds{\lim_{x \rightarrow  \infty} p(x)  = -\infty}$

\item The zeros appear to be: $x=-1.5$, even multiplicity - probably $2$ since it doesn't `look like' the graph is very flat near $x = 2$;  $x=0$, odd multiplicity - probably $1$ since the graph seems fairly linear as it passes through the origin;  $x=1$ odd multiplicity - probably $3$ or higher since the graph seems fairly `flat' near $x = 1$.

\item  local minimum:  approximately $(-0.773, -2.888)$;  local maximums:  approximately $(-1.5,0)$, and $(0.32, 0.532)$

\item  Based on the graph, even degree (at least $6$ based on multiplicities) with a negative leading coefficient based on the end behavior.

\item  We only have a \textit{portion} of the graph represented here.

\end{enumerate}

\addtocounter{enumi}{1}

\item We are looking for the largest open interval containing $x = -0.235$ for which the graph of $y = p(x)$ is at or above $y=-1.121$.  Since each of the gridlines on the $x$-axis correspond to $0.2$ units, we approximate this interval as  $(-1.25 \, \text{ish}, 1.1 \, \text{ish})$.

\addtocounter{enumi}{4}

\item 

%\begin{multicols}{2}
\begin{enumerate} \addtocounter{enumii}{2} 
\item $L(x) = x^2$


\item $L(x) = x+1$

\end{enumerate}
%\end{multicols}

\end{enumerate}


\end{document}
